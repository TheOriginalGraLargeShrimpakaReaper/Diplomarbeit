%! Author = itgramic
%! Date = 29.09.23

% Preamble
%\newpage
%\begin{landscape}
%\begin{flushleft}
\clearpage
%    \KOMAoptions{paper=A3,paper=landscape,pagesize}
\KOMAoptions{paper=A3,paper=landscape,pagesize, DIV=20}
\recalctypearea
\section{Zieldefinition}
%\begin{flushleft}
Das administrieren einer \Gls{PostgreSQL} Datenbank umfasst i.d.R. \cite{5TA9H3I4,IFLSFQU4} folgende zehn Tasks die zum täglichen Alltag gehören:
%\end{flushleft}
\begin{table}[H]
\resizebox{\columnwidth}{!}{%
\begin{tabular}{@{}llll@{}}
\toprule
\textbf{Nr.} & \textbf{Aufgabe}                                                                   & \textbf{Beschreibung}                                                                                                                                                                                                                                                                                                                                                                                                                                                 & \textbf{Wichtigkeit} \\ \midrule
1            & Failover                                                                           & In einem Fehlerfall soll die DB-Node auf einen Standby-Node übergeben werden.                                                                                                                                                                                                                                                                                                                                                                                         & Hoch                 \\
2            & Failover Restore                                                                   & \begin{tabular}[c]{@{}l@{}}Nach einem Failover muss der DB-Node wieder vom Standby-Node auf den primären Node zurückgesetzt werden.\\ Dabei darf es zu keinem Datenverlust kommen, also alle Daten die auf dem Standby-Node erfasst wurden, \\ müssen beim Failover Restore auf den primären DB-Node zurückgeschrieben werden.\end{tabular}                                                                                                                            & Hoch                 \\
3            & Filesystem Management                                                              & \begin{tabular}[c]{@{}l@{}}Die Datenmenge in Datenbanken wachsen in der Regel beständig.\\ Die Belegung von Tablespaces und Filesystem muss deshalb überwacht und ggf. erweitert werden.\\ Läuft eine Disk voll, kommt es im besten Fall zu einem Stillstand der DB, im schlimmsten Fall zu Inkonsistenzen und Datenverlust\end{tabular}                                                                                                                              & Hoch                 \\
4            & Monitoring                                                                         & \begin{tabular}[c]{@{}l@{}}Nebst den allgemeinen Metriken wie CPU / Memory Usage und der Port-Verfügbarkeit gibt es noch eine Reihe weiterer Aspekte die observiert werden müssen.\\ Zum Beispiel, ob es zu verzögerungen bei der Replikation kommt oder die Tablespaces genügend Platz haben.\\ Dazu gehört auch das Überwachen des Logs und entsprechende Schritte im Fehlerfall.\end{tabular}                                                                        & Mittel               \\
5            & Statistiken / Cleanup Jobs justieren                                               & \begin{tabular}[c]{@{}l@{}}\Gls{PostgreSQL} sammelt Statistiken um SQL Queries optimaler ausführen zu können. \\ Zudem wird im Rahmen des gleichen Sheduled Tasks ein Cleanup vorgenommen, \\ so dass z.B. gelöschte Datensätze den Disk Space nicht sinnlos belegen.\\ Die Konfiguration dieser Jobs muss an der Metrik der Datenbank angepasst werden, \\ weil gewisse Tasks dann entweder viel zu oft oder viel zu wenig bis gar nicht mehr ausgeführt werden.\end{tabular} & Mittel               \\
6            & SQL optimierungen                                                                  & \begin{tabular}[c]{@{}l@{}}In \Gls{PostgreSQL} können inperfomante SQL Statements ausgelesen werden und zum Teil werden auch informationen zum Tuning geliefert\cite{RJFW5WUH}.\\ Diese müssen regelmässig ausgelesen werden\end{tabular}                                                                                                                                                                                                                                                  & Tief                 \\
7            & \begin{tabular}[c]{@{}l@{}}Health Checks und Aktionen\\ (Maintenance)\end{tabular} & \begin{tabular}[c]{@{}l@{}}Regelmässig muss die Gesundheit der DBs überprüft werden, etwa ob Tabellen und/oder Indizes sich aufgebläht haben oder ob Locks vorhanden sind\cite{DPBK2HT5}.\\ Während der Hauptarbeitszeit muss dies mindestens alle 90 Minuten geprüft und ggf. reagiert werden.\end{tabular}                                                                                                                                                                         & Hoch                 \\
8            & Housekeeping                                                                       & \begin{tabular}[c]{@{}l@{}}Mit Housekeeping Jobs werden regelmässig Trace- und Alertlogfiles aufgeräumt, \\ um Platz auf den Disken zu sparen, aber auch, um die Übersichtlichkeit zu wahren.\end{tabular}                                                                                                                                                                                                                                                              & Mittel               \\
9            & Verwalten von DB Objekten                                                          & \begin{tabular}[c]{@{}l@{}}Regelmässig müssen DB Objekte wie Datenbanken, Tabellen, Trigger, Views etc. angepasst oder erstellt werden.\\ Dies richtet sich nach den Bedürfnis der Kunden resp. deren Applikationen.\end{tabular}                                                                                                                                                                                                                                     & Tief                 \\
10           & User Management                                                                    & \begin{tabular}[c]{@{}l@{}}Die Zugriffe der User müssen Überwacht, angepasst, erfasst oder gesperrt werden.\\ Auch diese Aufgabe richtet sich nach den Bedürfnissen der Kunden.\end{tabular}                                                                                                                                                                                                                                                                          & Tief                 \\ \bottomrule
\end{tabular}%
}
\caption{Administrative Aufgaben}
\label{tab:administrative_aufgaben}
\end{table}
%\begin{flushleft}
\clearpage
%    \KOMAoptions{paper=A3,paper=landscape,pagesize}
\KOMAoptions{paper=A3,paper=landscape,pagesize, DIV=20}
\recalctypearea
Von diesen Tasks müssen Teile davon zu 50\% automatisiert werden wobei alle Muss-Aufgaben automatisiert werden müssen.
Diese wären nachfolgende Tasks die automatisiert werden können.
%\end{flushleft}

%\newpage
%\begin{flushleft}
%\clearpage
%%    \KOMAoptions{paper=A3,paper=landscape,pagesize}
%\KOMAoptions{paper=A3,paper=landscape,pagesize, DIV=20}
%\recalctypearea
%\end{landscape}
%\begin{landscape}
\begin{table}[H]
\resizebox{\columnwidth}{!}{%
\begin{tabular}{@{}lllllll@{}}
\toprule
\textbf{Nr.} & \textbf{Aufgabe}                                                                   & \textbf{Wichtigkeit} & \textbf{Zu automatisierender Task}                                                                                                                                                                                                                                                                                                                                                              & \textbf{Priorität} & \textbf{Muss / Kann} & \textbf{Spätester Termin} \\ \midrule
1            & Failover                                                                           & Hoch                 & Automatisierter Failover auf mindestens einen sekundären DB-Node                                                                                                                                                                                                                                                                                                                                & 1                  & Muss                 & Abgabe                    \\
2            & Failover Restore                                                                   & Hoch                 & \begin{tabular}[c]{@{}l@{}}Sobald der Primäre DB-Node wieder vorhanden ist,\\ muss automatisch auf den primären DB-Node zurückgesetzt werden.\end{tabular}                                                                                                                                                                                                                                      & 1                  & Muss                 &                           \\
3            & Filesystem Management                                                              & Hoch                 & \begin{tabular}[c]{@{}l@{}}Das Filesystem muss beim erreichen von 95\% Usage automatisiert vergrössert werden.\\ Die Vergrösserung muss anhand der Wachstumsrate (die mittels Linux Commands zu ermitteln ist),\\ vergrössert werden\end{tabular}                                                                                                                                               & 4                  & Kann                 &                           \\
4            & Monitoring                                                                         & Mittel               & Der Status der Clusterumgebung und der Replikation muss im \Gls{PRTG} überwacht werden                                                                                                                                                                                                                                                                                                                & 2                  & Muss                 &                           \\
5            & Statistiken / Cleanup Jobs justieren                                               & Mittel               & \begin{tabular}[c]{@{}l@{}}Regelmässig müssen die Parameter für den \Gls{AUTOVACUUM} Job berechnet werden und\\ das Configfile postgresql.conf automatisch angepasst werden\end{tabular}                                                                                                                                                                                                              & 2                  & Muss                 &                           \\
6            & SQL-Optimierungen                                                                  & Tief                 & \begin{tabular}[c]{@{}l@{}}Es gibt SQL Abfragen, mit dem fehlende Indizes ermittelt werden können.\\ Diese Indizes sollen automatisiert erstellt werden.\\ \\ Im gleichen Zug sollen aber auch Indizes, welche nicht verwendet werden, entfernt werden.\\ Sie tragen nicht nur nichts zu performanteren Abfragen bei\\ sondern beziehen unnötige Ressourcen bei Datenmanipulationen\cite{RJFW5WUH}.\end{tabular} & 2                  & Kann                 &                           \\
7            & \begin{tabular}[c]{@{}l@{}}Health Checks und Aktionen\\ (Maintenance)\end{tabular} & Hoch                 & \begin{tabular}[c]{@{}l@{}}Tabellen und Indizes können sich aufblähen (bloated table / bloated index)\\ Ist ein Index aufgebläht, kann dies mittels eines REINDEX\\ mit geringem Impact auf die Datenbank gelöst werden\cite{DPBK2HT5}.\end{tabular}                                                                                                                                                            & 2                  & Muss                 &                           \\
8            & Housekeeping                                                                       & Mittel               & Log Rotation muss aktiviert werden und alte Logs regelmässig gelöscht werden.                                                                                                                                                                                                                                                                                                                   & 3                  & Kann                 &                           \\
9            & Verwalten von DB Objekten                                                          & Tief                 & Keine Automatisierung möglich                                                                                                                                                                                                                                                                                                                                                                   & 5                  &                      &                           \\
10           & User Management                                                                    & Tief                 & Regelmässige Reports sollen User aufzeigen, die seit mehr als einer Woche nicht mehr aktiv waren.                                                                                                                                                                                                                                                                                               & 4                  & Kann                 &                           \\ \bottomrule
\end{tabular}%
}
\caption{Automatisierung Administrativer Aufgaben}
\label{tab:automatgisierung_administrative_aufgaben}
\end{table}
%\end{flushleft}
%\end{landscape}
%\begin{flushleft}
Mit dieser Arbeit sollen folgende Ergebnisse und Resultate erzielt werden:
\begin{itemize}
    \item \textbf{Ergebnisse} \\ Mindestens drei Methoden einen \Gls{PostgreSQL Cluster} aufzubauen müssen analysiert und evaluiert werden
    \item \textbf{Resultate} \\ Aus den mindestens drei Methoden muss die optimale Methode ermittelt werden. \\ Am Ende muss zudem ein funktionierendes Testsystem bestehen.
\end{itemize}
%\end{flushleft}
%\begin{flushleft}

%\end{flushleft}
%\begin{flushleft}
\clearpage
%    \KOMAoptions{paper=A3,paper=landscape,pagesize}
\KOMAoptions{paper=A3,paper=landscape,pagesize, DIV=15}
\recalctypearea
Daraus ergeben sich folgende Ziele:
%\begin{landscape}
\begin{table}[H]
\resizebox{\columnwidth}{!}{%
\begin{tabular}{@{}llll@{}}
\toprule
\textbf{Nr.} & \textbf{Ziel}                                     & \textbf{Beschreibung}                                                                                                                                                                                                                                                                                                                                                                                                                                                                                                                                                                                                                                                                                                                                                                                                                                                                & \textbf{Priorität} \\ \midrule
1            & Evaluation                                        & \begin{tabular}[c]{@{}l@{}}Am Ende der Evaluationsphase müssen mindestens drei Methoden für einen \Gls{PostgreSQL HA Cluster} müssen evaluirt werden.\\ Innerhalb der Evaluation muss analysiert werden, welche Methode oder welches Tool sich hierfür eignen würde.\end{tabular}                                                                                                                                                                                                                                                                                                                                                                                                                                                                                                                                                                                   & Hoch               \\
2            & Testsystem                                        & Am Ende der Diplomarbeit muss ein funktionierendes Testsystem installiert sein.                                                                                                                                                                                                                                                                                                                                                                                                                                                                                                                                                                                                                                                                                                                                                                                                      & Hoch               \\
3            & Automatisierter Failover                          & \begin{tabular}[c]{@{}l@{}}Ein \Gls{PostgreSQL Cluster} muss im Fehlerfall auf mindestens einen Standby-Node umschwenken.\\ Dabei muss das Timeout so niedrig sein, dass Applikationen nicht auf ein Timeout laufen.\end{tabular}                                                                                                                                                                                                                                                                                                                                                                                                                                                                                                                                                                                                                                   & Hoch               \\
4            & Automatisierter Failover Restore                  & Nach einem Failover muss es zu einem Fallback oder Failover Restore kommen, sobald der Primary-Node wieder verfügbar ist.                                                                                                                                                                                                                                                                                                                                                                                                                                                                                                                                                                                                                                                                                                                                                            & Hoch               \\
5            & Monitoring - Cluster Healthcheck                  & \begin{tabular}[c]{@{}l@{}}Die wichtigsten Parameter für das Monitoring des \Gls{PostgreSQL Cluster}s (isready, Locks, bloated Tables), \\ der Replikation (Replay Lag, Standby alive) und des \Gls{PostgreSQL HA Cluster}s müssen überwacht werden.\end{tabular}                                                                                                                                                                                                                                                                                                                                                                                                                                                                                                                                                                                  & Mittel             \\
6            & \Gls{AUTOVACUUM} - Parameter verwalten                  & \begin{tabular}[c]{@{}l@{}}Täglich müssen die Parameter für den \Gls{AUTOVACUUM} Job berechnet werden und\\ das Configfile postgresql.conf automatisch angepasst werden\end{tabular}                                                                                                                                                                                                                                                                                                                                                                                                                                                                                                                                                                                                                                                                                                       & Mittel             \\
7            & SQL optimierungen - Indizes tracken und verwalten & Täglich fehlende Indizes automatisiert erstellen und nicht mehr verwendete Indizes automatisiert entfernen                                                                                                                                                                                                                                                                                                                                                                                                                                                                                                                                                                                                                                                                                                                                                                           & Mittel             \\
8            & Maintenance - Indizes säubern                     & Täglich bloated Indices, also aufgeblähte Indizes, automatisiert erkennen und mittels REINDEX bereinigen                                                                                                                                                                                                                                                                                                                                                                                                                                                                                                                                                                                                                                                                                                                                                                             & Hoch               \\
9            & Housekeeping - Log Rotation                       & Die Log Rotation muss aktiviert werden. Die Logs müssen aber auch in das KSGR-Log Repository geschrieben werden                                                                                                                                                                                                                                                                                                                                                                                                                                                                                                                                                                                                                                                                                                                                                                      & Hoch               \\
10           & User Management - Monitoring                      & Nicht verwendete User sollen einmal pro Woche automatisiert erkannt und in einem Report gemeldet werden.                                                                                                                                                                                                                                                                                                                                                                                                                                                                                                                                                                                                                                                                                                                                                                             & Tief               \\
11           & Evaluationsziel                                   & Am Ende der Evaluationsphase muss ein Entscheid getroffen worden sein, welche Methode verwendet wird.                                                                                                                                                                                                                                                                                                                                                                                                                                                                                                                                                                                                                                                                                                                                                                                & Hoch               \\
12           & Installationsziel                                 & Die Testinstallation muss lauffähig sein und zudem alle Anforderungen und Ziele (3 und 4) erfüllen                                                                                                                                                                                                                                                                                                                                                                                                                                                                                                                                                                                                                                                                                                                                                                                   & Hoch               \\
13           & Testziele                                         & \begin{tabular}[c]{@{}l@{}}Folgende Testziele müssen erreicht werden:\\ 1. Der \Gls{PostgreSQL Cluster} muss immer lauffähig sein solange noch ein Node up ist, unabhängig davon welche Nodes des \Gls{PostgreSQL HA Cluster}s down ist\\ 2. Ein Switchover auf alle sekundären Nodes muss möglich sein\\ 3. Der Fallback auf den primären Node muss erfolgreich sein, unabhängig davon, ob ein Failover oder \Gls{Switchover} stattgefunden hat\\ 4. Das Timeout bei einem \Gls{Failover} / \Gls{Switchover} muss unterhalb der Default Timeouts der Applikationen \Gls{GitLab} und \Gls{Harbor} liegen.\\ 5. Das Replay Lag zwischen Primary und Secondary darf beim Initialen Start nicht über eine Minute dauern oder 1KiB nicht überschreiten\end{tabular} & Hoch               \\ \bottomrule
\end{tabular}%
}
\caption{Ziele}
\label{tab:ziele}
\end{table}
%\end{landscape}
%\clearpage
%\KOMAoptions{paper=A4,paper=portrait,pagesize}
%\recalctypearea
%\end{flushleft}
%\clearpage
%\KOMAoptions{paper=A4,paper=portrait,pagesize}
%\recalctypearea
