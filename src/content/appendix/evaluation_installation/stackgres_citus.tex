%! Author = ibw
%! Date = 24.01.24

% Preamble
\subsection{Stackgres mit Citus}
\label{subsec:evaluation_installation_stackgres}
\subsubsection{Prerequisites}
\paragraph{StorageClass setzen}
Zuerst muss die StorageClass und das PersistentVolume gesetzt werden:
\lstset{style=gra_codestyle}
\begin{lstlisting}[language=yaml, caption=StackGres-Citus - StorageClass setzen,captionpos=b,label={lst:stackgres-citus-storageclass-yaml},breaklines=true]
# https://docs.yugabyte.com/preview/yugabyte-platform/install-yugabyte-platform/prepare-environment/kubernetes/#configure-storage-class
# https://github.com/rancher/local-path-provisioner
apiVersion: storage.k8s.io/v1
kind: StorageClass
metadata:
  name: stackgres-storage
provisioner: rancher.io/local-path
parameters:
  nodePath: /srv/data/local-path-provisioner
volumeBindingMode: WaitForFirstConsumer
reclaimPolicy: Delete
---
apiVersion: v1
kind: PersistentVolume
metadata:
  name: stackgres-storage-pv
  labels:
    type: local
spec:
  accessModes:
    - ReadWriteOnce
  capacity:
    storage: 1Gi
  storageClassName: "stackgres-storage"
  hostPath:
    path: /srv/data/local-path-provisioner
  nodeAffinity:
    required:
      nodeSelectorTerms:
      - matchExpressions:
        - key: kubernetes.io/hostname
          operator: In
          values:
          - sks1183
          - sks1184
          - sks1185
\end{lstlisting}

Die Storage Class und das PercistenVolume muss aktiviert werden:
\lstset{style=gra_codestyle}
\begin{lstlisting}[language=bash, caption=StackGres-Citus - StorageClass / PersistentVolume aktivieren,captionpos=b,label={lst:stackgres_citus-storageclass-apply},breaklines=true]
gramic@cks4040:~$ kubectl apply -f /home/gramic/PycharmProjects/rke2_settings/stackgres_citus/stackgres_citus/storageclass.yaml
storageclass.storage.k8s.io/stackgres-storage created
persistentvolume/stackgres-storage-pv created
\end{lstlisting}
\subsubsection{Installation}
\lstset{style=gra_codestyle}
\begin{lstlisting}[language=bash, caption=StackGres-Citus - Namespace,captionpos=b,label={lst:stackgres-citus-namespace},breaklines=true]
kubectl create namespace sg-platform
\end{lstlisting}

Das Manifest beinhaltet nur die wichtigsten Parameter.\\
Der Grossteil wird über das Cluster-Deployment gesetzt:
\lstset{style=gra_codestyle}
\begin{lstlisting}[language=yaml, caption=StackGres-Citus - Helm Chart Manifest,captionpos=b,label={lst:stackgres_citus-values.yaml},breaklines=true]
# -- The container registry host (and port) where the images will be pulled from.
containerRegistry: quay.io
# -- Image pull policy used for images loaded by the Operator
imagePullPolicy: "IfNotPresent"
# Section to configure Operator Installation ServiceAccount
serviceAccount:
  # -- If `true` the Operator Installation ServiceAccount will be created
  create: true
  # -- Section to configure Operator ServiceAccount annotations
  annotations: {}
  # -- Repositories credentials Secret names to attach to ServiceAccounts and Pods
  repoCredentials: []

# Section to configure Operator Pod
operator:
  # Section to configure Operator image
  image:
    # -- Operator image name
    name: "stackgres/operator"
    # -- Operator image tag
    tag: "1.9.0"
    # -- Operator image pull policy
    pullPolicy: "IfNotPresent"
  # -- Operator Pod annotations
  annotations: {}
  # -- Operator Pod resources. See https://kubernetes.io/docs/reference/generated/kubernetes-api/v1.27/#resourcerequirements-v1-core
#  resources: {}
  resources:
     requests:
       cpu: "1"
       memory: 1Gi
     limits:
       cpu: "1"
       memory: 1Gi
  # -- Operator Pod node selector
  nodeSelector: {}
  # -- Operator Pod tolerations. See https://kubernetes.io/docs/reference/generated/kubernetes-api/v1.27/#toleration-v1-core
  tolerations: []
  # -- Operator Pod affinity. See https://kubernetes.io/docs/reference/generated/kubernetes-api/v1.27/#affinity-v1-core
  affinity: {}
  # Section to configure Operator ServiceAccount
  serviceAccount:
    # -- Section to configure Operator ServiceAccount annotations
    annotations: {}
    # -- Repositories credentials Secret names to attach to ServiceAccounts and Pods
    repoCredentials: []
  # Section to configure Operator Service
  service:
    # -- Section to configure Operator Service annotations
    annotations: {}

# Section to configure REST API Pod
restapi:
  # -- REST API Pod name
  name: stackgres-restapi
  # Section to configure REST API image
  image:
    # -- REST API image name
    name: "stackgres/restapi"
    # -- REST API image tag
    tag: "1.9.0"
    # -- REST API image pull policy
    pullPolicy: "IfNotPresent"
  # -- REST API Pod annotations
  annotations: {}
  resources:
     requests:
       cpu: "1"
       memory: 1Gi
     limits:
       cpu: "1"
       memory: 1Gi
  # -- REST API Pod node selector
  nodeSelector: {}
  # -- REST API Pod tolerations. See https://kubernetes.io/docs/reference/generated/kubernetes-api/v1.27/#toleration-v1-core
  tolerations: []
  # -- REST API Pod affinity. See https://kubernetes.io/docs/reference/generated/kubernetes-api/v1.27/#affinity-v1-core
  affinity: {}
  # Section to configure REST API ServiceAccount
  serviceAccount:
    # -- REST API ServiceAccount annotations
    annotations: {}
    # -- Repositories credentials Secret names to attach to ServiceAccounts and Pods
    repoCredentials: []
  # Section to configure REST API Service
  service:
    # -- REST API Service annotations
    annotations: {}

# Section to configure Web Console container
adminui:
  # Section to configure Web Console image
  image:
    # -- Web Console image name
    name: "stackgres/admin-ui"
    # -- Web Console image tag
    tag: "1.9.0"
    # -- Web Console image pull policy
    pullPolicy: "IfNotPresent"
  resources:
     requests:
       cpu: "1"
       memory: 1Gi
     limits:
       cpu: "1"
       memory: 1Gi
  # Section to configure Web Console service.
  service:
    # -- When set to `true` the HTTP port will be exposed in the Web Console Service
    exposeHTTP: true
    # -- The type used for the service of the UI:
    type: ClusterIP
    # -- (string) LoadBalancer will get created with the IP specified in
    loadBalancerIP:          # gramic, load-balancer-IP
    # -- (array) If specified and supported by the platform,
    loadBalancerSourceRanges:
    # -- (integer) The HTTPS port used to expose the Service on Kubernetes nodes
    nodePort:
    # -- (integer) The HTTP port used to expose the Service on Kubernetes nodes
    nodePortHTTP:

# Section to configure Operator Installation Jobs
jobs:
  # Section to configure Operator Installation Jobs image
  image:
    # -- Operator Installation Jobs image name
    name: "stackgres/jobs"
    # -- Operator Installation Jobs image tag
    tag: "1.9.0"
    # -- Operator Installation Jobs image pull policy
    pullPolicy: "IfNotPresent"
  # -- Operator Installation Jobs annotations
  annotations: {}
  # -- Operator Installation Jobs resources. See https://kubernetes.io/docs/reference/generated/kubernetes-api/v1.27/#resourcerequirements-v1-core
  resources: {}
  # -- Operator Installation Jobs node selector
  nodeSelector: {}
  # -- Operator Installation Jobs tolerations. See https://kubernetes.io/docs/reference/generated/kubernetes-api/v1.27/#toleration-v1-core
  tolerations: []
  # -- Operator Installation Jobs affinity. See https://kubernetes.io/docs/reference/generated/kubernetes-api/v1.27/#affinity-v1-core
  affinity: {}

# Section to configure deployment aspects.
deploy:
  # -- When set to `true` the Operator will be deployed.
  operator: true
  # -- When set to `true` the Web Console / REST API will be deployed.
  restapi: true

# Section to configure the Operator, REST API and Web Console certificates and JWT RSA key-pair.
cert:
  # -- If set to `true` the CertificateSigningRequest used to generate the certificate used by
  #   Webhooks will be approved by the Operator Installation Job.
  autoapprove: true
  # -- When set to `true` the Operator certificate will be created.
  createForOperator: true
  # -- When set to `true` the Web Console / REST API certificate will be created.
  createForWebApi: true
  # -- (string) The Secret name with the Operator Webhooks certificate issued by the Kubernetes cluster CA
  #   of type kubernetes.io/tls. See https://kubernetes.io/docs/concepts/configuration/secret/#tls-secrets
  secretName:
  # -- When set to `true` the Operator certificates will be regenerated if `createForOperator` is set to `true`, and the certificate is expired or invalid.
  regenerateCert: true
  # -- (integer) The duration in days of the generated certificate for the Operator after which it will expire and be regenerated.
  #   If not specified it will be set to 730 (2 years) by default.
  certDuration: 730
  # -- (string) The Secret name with the Web Console / REST API certificate
  #   of type kubernetes.io/tls. See https://kubernetes.io/docs/concepts/configuration/secret/#tls-secrets
  webSecretName:
  # -- When set to `true` the Web Console / REST API certificates will be regenerated if `createForWebApi` is set to `true`, and the certificate is expired or invalid.
  regenerateWebCert: true
  # -- When set to `true` the Web Console / REST API RSA key pair will be regenerated if `createForWebApi` is set to `true`, and the certificate is expired or invalid.
  regenerateWebRsa: true
  # -- (integer) The duration in days of the generated certificate for the Web Console / REST API after which it will expire and be regenerated.
  #   If not specified it will be set to 730 (2 years) by default.
  webCertDuration:
  # -- (integer) The duration in days of the generated RSA key pair for the Web Console / REST API after which it will expire and be regenerated.
  #   If not specified it will be set to 730 (2 years) by default.
  webRsaDuration:
  # -- (string) The private RSA key used to create the Operator Webhooks certificate issued by the
  #   Kubernetes cluster CA.
  key:
  # -- (string) The Operator Webhooks certificate issued by Kubernetes cluster CA.
  crt:
  # -- (string) The private RSA key used to generate JWTs used in REST API authentication.
  jwtRsaKey:
  # -- (string) The public RSA key used to verify JWTs used in REST API authentication.
  jwtRsaPub:
  # -- (string) The private RSA key used to create the Web Console / REST API certificate
  webKey:
  # -- (string) The Web Console / REST API certificate
  webCrt:
  # Section to configure cert-manager integration to generate Operator certificates
  certManager:
    # -- When set to `true` then Issuer and Certificate for Operator and Web Console / REST API
    #   Pods will be generated
    autoConfigure: false
    # -- The requested duration (i.e. lifetime) of the Certificates. See https://cert-manager.io/docs/reference/api-docs/#cert-manager.io%2fv1
    duration: "2160h"
    # -- How long before the currently issued certificate’s expiry cert-manager should renew the certificate. See https://cert-manager.io/docs/reference/api-docs/#cert-manager.io%2fv1
    renewBefore: "360h"
    # -- The private key cryptography standards (PKCS) encoding for this certificate’s private key to be encoded in. See https://cert-manager.io/docs/reference/api-docs/#cert-manager.io/v1.CertificatePrivateKey
    encoding: PKCS1
    # -- Size is the key bit size of the corresponding private key for this certificate. See https://cert-manager.io/docs/reference/api-docs/#cert-manager.io/v1.CertificatePrivateKey
    size: 2048

# Section to configure RBAC for Web Console admin user
rbac:
  # -- When set to `true` the admin user is assigned the `cluster-admin` ClusterRole by creating
  #   ClusterRoleBinding.
  create: true

# Section to configure Web Console authentication
authentication:
  # -- Specify the authentication mechanism to use. By default is `jwt`, see https://stackgres.io/doc/latest/api/rbac#local-secret-mechanism.
  #   If set to `oidc` then see https://stackgres.io/doc/latest/api/rbac/#openid-connect-provider-mechanism.
  type: jwt
  # -- (boolean) When `true` will create the secret used to store the `admin` user credentials to access the UI.
  createAdminSecret: true
  # -- The admin username that will be required to access the UI
  user: admin
  # -- (string) The admin password that will be required to access the UI
  #password: "TES2&Daggerfall"
  password:
  # Section to configure Web Console OIDC authentication
  oidc:
    # tlsVerification -- (string) Can be one of `required`, `certificate-validation` or `none`
# Section to configure Prometheus integration.
prometheus:

  allowAutobind: true
# Section to configure Grafana integration
grafana:
  # -- When set to `true` embed automatically Grafana into the Web Console by creating the
  #   StackGres dashboards and the read-only role used to read it from the Web Console
  autoEmbed: false
  # -- The schema to access Grafana. By default http. (used to embed manually and
  #   automatically grafana)
  schema: http
  # -- (string) The service host name to access grafana (used to embed manually and
  #   automatically Grafana).
  webHost:
  # -- The datasource name used to create the StackGres Dashboards into Grafana
  datasourceName: Prometheus
  # -- The username to access Grafana. By default admin. (used to embed automatically
  #   Grafana)
  user: admin
  # -- The password to access Grafana. By default prom-operator (the default in for
  #   kube-prometheus-stack helm chart). (used to embed automatically Grafana)
  password: prom-operator
  # -- Use follwing fields to indicate a secret where the grafana admin credentials are stored (replace user/password)

  # -- (string) The namespace of secret with credentials to access Grafana. (used to
  #   embed automatically Grafana, alternative to use `user` and `password`)
  secretNamespace:
  # -- (string) The name of secret with credentials to access Grafana. (used to embed
  #   automatically Grafana, alternative to use `user` and `password`)
  secretName:
  # -- (string) The key of secret with username used to access Grafana. (used to embed
  #   automatically Grafana, alternative to use `user` and `password`)
  secretUserKey:
  # -- (string) The key of secret with password used to access Grafana. (used to
  #   embed automatically Grafana, alternative to use `user` and `password`)
  secretPasswordKey:
  # -- (string) The ConfigMap name with the dashboard JSONs
  #   that will be created in Grafana. If not set the default
  #   StackGres dashboards will be created. (used to embed automatically Grafana)
  dashboardConfigMap:
  # -- (array) The URLs of the PostgreSQL dashboards created in Grafana (used to embed manually
  urls:
  # Create and copy/paste grafana API token:
  # - Grafana > Configuration > API Keys > Add API key (for viewer) > Copy key value
  token:

# Section to configure extensions
extensions:
  repositoryUrls:
  - https://extensions.stackgres.io/postgres/repository?proxyUrl=http%3A%2F%2Fsproxy.sivc.first-it.ch%3A8080?skipHostnameVerification:true&setHttpScheme:true
  cache:
    # -- When set to `true` enable the extensions cache.
    enabled: false
    # -- An array of extensions pattern used to pre-loaded estensions into the extensions cache
    preloadedExtensions:
    - x86_64/linux/timescaledb-1\.7\.4-pg12
    # Section to configure the extensions cache PersistentVolume
    persistentVolume:
      size: 60Gi
      storageClass: "stackgres-storage"
    hostPath:
developer:
  version:
  # -- (string) Set `quarkus.log.level`. See https://quarkus.io/guides/logging#root-logger-configuration
  logLevel:
  # -- If set to `true` add extra debug to any script controlled by the reconciliation cycle of the operator configuration
  showDebug: false
  # -- Set `quarkus.log.console.format` to `%d{yyyy-MM-dd HH:mm:ss,SSS} %-5p [%c{4.}] (%t) %s%e%n`. See https://quarkus.io/guides/logging#logging-format
  showStackTraces: false
  # -- Only work with JVM version and allow connect
  # on port 8000 of operator Pod with jdb or similar
  enableJvmDebug: false
  # -- Only work with JVM version and if `enableJvmDebug` is `true`
  #   suspend the JVM until a debugger session is started
  enableJvmDebugSuspend: false
  # -- (string) Set the external Operator IP
  externalOperatorIp:
  # -- (integer) Set the external Operator port
  externalOperatorPort:
  # -- (string) Set the external REST API IP
  externalRestApiIp:
  # -- (integer) Set the external REST API port
  externalRestApiPort:
  # -- If set to `true` and `extensions.cache.enabled` is also `true`
  #   it will try to download extensions from images (experimental)
  allowPullExtensionsFromImageRepository: false
  # -- It set to `true` disable arbitrary user that is set for OpenShift clusters
  disableArbitraryUser: false
  # Section to define patches for some StackGres Pods
  patches:
    # Section to define volumes to be used by the operator container
    operator:
      # -- Pod volumes. See https://kubernetes.io/docs/reference/generated/kubernetes-api/v1.27/#volume-v1-core
      volumes: []
      # -- Pod's container volume mounts. See https://kubernetes.io/docs/reference/generated/kubernetes-api/v1.27/#volumemount-v1-core
      volumeMounts: []
    # Section to define volumes to be used by the restapi container
    restapi:
      # -- Pod volumes. See https://kubernetes.io/docs/reference/generated/kubernetes-api/v1.27/#volume-v1-core
      volumes: []
      # -- Pod's container volume mounts. See https://kubernetes.io/docs/reference/generated/kubernetes-api/v1.27/#volumemount-v1-core
      volumeMounts: []
    # Section to define volumes to be used by the adminui container
    adminui:
      # -- Pod volumes. See https://kubernetes.io/docs/reference/generated/kubernetes-api/v1.27/#volume-v1-core
      volumes: []
      # -- Pod's container volume mounts. See https://kubernetes.io/docs/reference/generated/kubernetes-api/v1.27/#volumemount-v1-core
      volumeMounts: []
    # Section to define volumes to be used by the jobs container
    jobs:
      # -- Pod volumes. See https://kubernetes.io/docs/reference/generated/kubernetes-api/v1.27/#volume-v1-core
      volumes: []
      # -- Pod's container volume mounts. See https://kubernetes.io/docs/reference/generated/kubernetes-api/v1.27/#volumemount-v1-core
      volumeMounts: []
    # Section to define volumes to be used by the cluster controller container
    clusterController:
      # -- Pod volumes. See https://kubernetes.io/docs/reference/generated/kubernetes-api/v1.27/#volume-v1-core
      volumes: []
      # -- Pod's container volume mounts. See https://kubernetes.io/docs/reference/generated/kubernetes-api/v1.27/#volumemount-v1-core
      volumeMounts: []
    # Section to define volumes to be used by the distributedlogs controller container
    distributedlogsController:
      # -- Pod volumes. See https://kubernetes.io/docs/reference/generated/kubernetes-api/v1.27/#volume-v1-core
      volumes: []
      # -- Pod's container volume mounts. See https://kubernetes.io/docs/reference/generated/kubernetes-api/v1.27/#volumemount-v1-core
      volumeMounts: []
\end{lstlisting}

Die Installation erfolgt dann wie folgt:
\lstset{style=gra_codestyle}
\begin{lstlisting}[language=bash, caption=StackGres-Citus - Installation,captionpos=b,label={lst:stackgres_citus-values-apply},breaklines=true]
helm install -n sg-platform stackgres-operator stackgres-charts/stackgres-operator -f /home/gramic/PycharmProjects/rke2_settings/stackgres_citus/stackgres_citus/values.yaml
\end{lstlisting}

Damit nun aber der Cluster sauber läuft und auf den Cluster zugegriffen werden kann, müssen folgende Steps ebenfalls nachträglich ausgeführt werden:
\lstset{style=gra_codestyle}
\begin{lstlisting}[language=bash, caption=StackGres-Citus - Post-Installation,captionpos=b,label={lst:stackgres_citus-post-installation},breaklines=true]
# Check if the operator was successfully deployed and is available:
kubectl describe deployment -n sg-platform stackgres-operator
kubectl wait -n sg-platform deployment/stackgres-operator --for condition=Available
# Check if the restapi was successfully deployed and is available:
kubectl describe deployment -n sg-platform stackgres-restapi
kubectl wait -n sg-platform deployment/stackgres-restapi --for condition=Available
# To access StackGres Operator UI from localhost, run the below commands:
POD_NAME=$(kubectl get pods --namespace sg-platform -l "stackgres.io/restapi=true" -o jsonpath="{.items[0].metadata.name}")
kubectl port-forward "$POD_NAME" 8443:9443 --namespace sg-platform
\end{lstlisting}
Jetzt kann das GUI unter folgendem Link geöffnet werden:
\url{https://localhost:8443}

\begin{warning}
  \textbf{Port Forwarding}\\
  Wenn das Port Forwarding so gemacht wird, muss die CLI offenbleiben.\\
  Sonst wird die Verbindung umgehend gekappt.\\
  Daher empfiehlt es sich, mehrere Terminals offen zu halten.
\end{warning}

Der Username ist \texttt{admin} aber das Passwort ist generisch.\\
Der Username liesse sich mit folgendem Command auslesen:
\lstset{style=gra_codestyle}
\begin{lstlisting}[language=bash, caption=StackGres-Citus - System Username,captionpos=b,label={lst:stackgres_citus-system-username},breaklines=true]
kubectl get secret -n sg-platform stackgres-restapi-admin --template '{{ printf "username = %s\n" (.data.k8sUsername | base64decode) }}'
\end{lstlisting}
Dieses lässt sich mit folgendem Command auslesen:
\lstset{style=gra_codestyle}
\begin{lstlisting}[language=bash, caption=StackGres-Citus - System Passwort,captionpos=b,label={lst:stackgres_citus-system-password},breaklines=true]
kubectl get secret -n sg-platform stackgres-restapi-admin --template '{{ printf "password = %s\n" (.data.clearPassword | base64decode) }}'
\end{lstlisting}

Am Schluss sollte das Passwort aber noch gesäubert werden:
\lstset{style=gra_codestyle}
\begin{lstlisting}[language=bash, caption=StackGres-Citus - System Passwort Cleanup,captionpos=b,label={lst:stackgres_citus-system-password-cleanup},breaklines=true]
kubectl patch secret --namespace sg-platform stackgres-restapi-admin --type json -p '[{"op":"remove","path":"/data/clearPassword"}]'
\end{lstlisting}

\subsubsection{Deployment - Benchmarking}
Zuerts wurde das Instanz-Profil für den Coordinator und die Shards deployt:
\lstset{style=gra_codestyle}
\begin{lstlisting}[language=yaml, caption=StackGres-Citus - Benchmarking - SGInstanceProfile Coordinator,captionpos=b,label={lst:SGInstanceProfile_pgbench_coord.yaml},breaklines=true]
apiVersion: stackgres.io/v1
kind: SGInstanceProfile
metadata:
  namespace: sg-platform
  name: sg-pgbench-coordinator
spec:
  cpu: "4"
  memory: "4Gi"
\end{lstlisting}
\lstset{style=gra_codestyle}
\begin{lstlisting}[language=yaml, caption=StackGres-Citus - Benchmarking - SGInstanceProfile Shard,captionpos=b,label={lst:SGInstanceProfile_pgbench_shard.yaml},breaklines=true]
apiVersion: stackgres.io/v1
kind: SGInstanceProfile
metadata:
  namespace: sg-platform
  name: sg-pgbench-shard
spec:
  cpu: "4"
  memory: "8Gi"
\end{lstlisting}

Deployt wird ebenfalls via \texttt{kubectl}:
\lstset{style=gra_codestyle}
\begin{lstlisting}[language=bash, caption=StackGres-Citus - Benchmarking - Instanz-Profil Deploy,captionpos=b,label={lst:stackgres_citus-deploy-benchmarking-instance-profiles},breaklines=true]
kubectl apply -f /home/gramic/PycharmProjects/rke2_settings/stackgres_citus/stackgres_citus/SGInstanceProfile_pgbench_coord.yaml
kubectl apply -f /home/gramic/PycharmProjects/rke2_settings/stackgres_citus/stackgres_citus/SGInstanceProfile_pgbench_shard.yaml
\end{lstlisting}

Auf ein Deployment einer \texttt{SGPostgresConfig} wurde verzichtet, da für Patroni in den ersten drei Benchmarks keine spezifischen Anpassungen erfuhr.\\
Das Manifest für den Benchmark-Cluster sieht entsprechend den vorgaben wie folgt aus:
\lstset{style=gra_codestyle}
\begin{lstlisting}[language=yaml, caption=StackGres-Citus - Benchmarking - SGShardedCluster,captionpos=b,label={lst:SGShardedCluster_pgbench.yaml},breaklines=true]
apiVersion: stackgres.io/v1alpha1
kind: SGShardedCluster
metadata:
  name: sg-pgbench
  namespace: sg-platform
spec:
  type: citus
  database: pgbench_eval_bench
  postgres:
    version: '16'
  coordinator:
    instances: 1
    pods:
      persistentVolume:
#        size: '75Gi'
        size: '230Gi'
        storageClass: "stackgres-storage"
      disableConnectionPooling: true
    sgInstanceProfile: "sg-pgbench-coordinator"
  shards:
    clusters: 3
    instancesPerCluster: 1
    pods:
      persistentVolume:
        size: '75Gi'
        storageClass: "stackgres-storage"
    sgInstanceProfile: "sg-pgbench-shard"
  postgresServices:
    coordinator:
      primary:
        type: LoadBalancer
      any:
        type: LoadBalancer
    shards:
      primaries:
        type: LoadBalancer
  metadata:
    annotations:
      primaryService:
        metallb.universe.tf/loadBalancerIPs: 10.0.20.106
      replicasService:
        metallb.universe.tf/loadBalancerIPs: 10.0.20.153
        externalTrafficPolicy: "Cluster"
  profile: "testing"
\end{lstlisting}
Der Deploy:
\lstset{style=gra_codestyle}
\begin{lstlisting}[language=bash, caption=StackGres-Citus - Benchmark - Cluster Deploy,captionpos=b,label={lst:stackgres_citus-benchmnarking-deploy-cluster},breaklines=true]
kubectl apply -f /home/gramic/PycharmProjects/rke2_settings/stackgres_citus/stackgres_citus/SGShardedCluster_pgbench.yaml
\end{lstlisting}

Damit nun aber eine Verbindung auf die DB (IP 10.0.20.106) gemacht werden kann, muss das Passwort ausgelesen werden:
\lstset{style=gra_codestyle}
\begin{lstlisting}[language=bash, caption=StackGres-Citus - Benchmark DB Passwort,captionpos=b,label={lst:stackgres_citus-get-benchmark-cluster-passwd},breaklines=true]
kubectl get secrets -n sg-platform sg-pgbench -o jsonpath='{.data.superuser-password}' | base64 -d
\end{lstlisting}

\subsubsection{Deployment - Self Healing Tests}
Auch hier wurde zuerst das Instanz-Profil für den Coordinator und die Shards deployt:
\lstset{style=gra_codestyle}
\begin{lstlisting}[language=yaml, caption=StackGres-Citus - Self Healing Testing - SGInstanceProfile Coordinator,captionpos=b,label={lst:SGInstanceProfile_self_healing_coord.yaml},breaklines=true]
#Not needed actualy
apiVersion: stackgres.io/v1
kind: SGInstanceProfile
metadata:
  namespace: sg-platform
  name: sg-self-healing-coordinator
spec:
  cpu: "1"
  memory: "2Gi"
\end{lstlisting}
\lstset{style=gra_codestyle}
\begin{lstlisting}[language=yaml, caption=StackGres-Citus - Self Healing Testing - SGInstanceProfile Shard,captionpos=b,label={lst:SGInstanceProfile_self_healing_shard.yaml},breaklines=true]
#Not needed actualy
apiVersion: stackgres.io/v1
kind: SGInstanceProfile
metadata:
  namespace: sg-platform
  name: sg-self-healing-shard
spec:
  cpu: "1"
  memory: "2Gi"
\end{lstlisting}

Deployt wird ebenfalls via \texttt{kubectl}:
\lstset{style=gra_codestyle}
\begin{lstlisting}[language=bash, caption=StackGres-Citus - Self Healing Testing - Instanz-Profil Deploy,captionpos=b,label={lst:stackgres_citus-self-healing-testing-deploy-instance-profiles},breaklines=true]
kubectl apply -f /home/gramic/PycharmProjects/rke2_settings/stackgres_citus/stackgres_citus/SGInstanceProfile_self_healing_coord.yaml
kubectl apply -f /home/gramic/PycharmProjects/rke2_settings/stackgres_citus/stackgres_citus/SGInstanceProfile_self_healing_shard.yaml
\end{lstlisting}

Auch beim Self Healing Testing wurde auf ein Deployment einer \texttt{SGPostgresConfig} verzichtet.\\
Dafür wurden drei Coordinator-Instanzen und drei Shard-Instanzen pro Cluster deklariert.\\
Das Manifest für den Benchmark-Cluster sieht entsprechend den vorgaben wie folgt aus:
\lstset{style=gra_codestyle}
\begin{lstlisting}[language=yaml, caption=StackGres-Citus - Self Healing Testing - SGShardedCluster,captionpos=b,label={lst:SGShardedCluster_self_healing_test.yaml},breaklines=true]
apiVersion: stackgres.io/v1alpha1
kind: SGShardedCluster
metadata:
  name: sg-healing-test
  namespace: sg-platform
spec:
  type: citus
  database: self_healing_test
  postgres:
    version: '16'
  coordinator:
    syncInstances: 3
    replication: "strict-sync"
    pods:
      persistentVolume:
        size: '2Gi'
        storageClass: "stackgres-storage"
    sgInstanceProfile: "sg-self-healing-coordinator"
  shards:
    clusters: 3
    instancesPerCluster: 3
    pods:
      persistentVolume:
        size: '5Gi'
        storageClass: "stackgres-storage"
    sgInstanceProfile: "sg-self-healing-shard"
  postgresServices:
    coordinator:
      primary:
        type: LoadBalancer
      any:
        type: LoadBalancer
    shards:
      primaries:
        type: LoadBalancer
  metadata:
    annotations:
      primaryService:
        metallb.universe.tf/loadBalancerIPs: 10.0.20.152
      replicasService:
        metallb.universe.tf/loadBalancerIPs: 10.0.20.151
        externalTrafficPolicy: "Cluster"
  profile: "testing"
\end{lstlisting}
Der Deploy:
\lstset{style=gra_codestyle}
\begin{lstlisting}[language=bash, caption=StackGres-Citus - Self Healing Testing - Cluster Deploy,captionpos=b,label={lst:stackgres_citus-self-healing-test-deploy-cluster},breaklines=true]
kubectl apply -f /home/gramic/PycharmProjects/rke2_settings/stackgres_citus/stackgres_citus/SGShardedCluster_self_healing_test.yaml
\end{lstlisting}

Das Passwort für den Cluster (IP 10.0.20.152) muss ebenfalls ausgelesen werden:
\lstset{style=gra_codestyle}
\begin{lstlisting}[language=bash, caption=StackGres-Citus - Self Healing Testing DB Passwort,captionpos=b,label={lst:stackgres_citus-get-self-healing-testing-cluster-passwd},breaklines=true]
kubectl get secrets -n sg-platform sg-healing-test -o jsonpath='{.data.superuser-password}' | base64 -d
\end{lstlisting}

\subsubsection{Rekonfiguration mit 250GiB Storage}
\paragraph{Bereinigen}
Zuerst wurde auch hier der komplette Namespace bereinigt, wobei es zwingend ist, vorher im GUI die Cluster zu löschen und mittels CLI den Operator zu deinstallieren.\\
Andernfalls wird nicht alles sauber entfernt was zu schwer Nachvollziehbaren Sideeffects führt:
\lstset{style=gra_codestyle}
\begin{lstlisting}[language=bash, caption=StackGres-Citus - Deinstallieren,captionpos=b,label={lst:stackgres_citus-deinstall-cleanup},breaklines=true]
helm delete stackgres-operator --namespace sg-platform
kubectl delete namespace sg-platform
kubectl delete pv stackgres-storage-pv
kubectl delete storageclass stackgres-storage
kubectl delete pvc --namespace sg-platform
kubectl delete pvc --namespace sg-platform
\end{lstlisting}

\paragraph{StorageClass setzen}
\lstset{style=gra_codestyle}
\begin{lstlisting}[language=yaml, caption=StackGres-Citus - StorageClass setzen,captionpos=b,label={lst:storageclass_big.yaml},breaklines=true]
# https://docs.yugabyte.com/preview/yugabyte-platform/install-yugabyte-platform/prepare-environment/kubernetes/#configure-storage-class
# https://github.com/rancher/local-path-provisioner
apiVersion: storage.k8s.io/v1
kind: StorageClass
metadata:
  name: stackgres-storage-big
provisioner: rancher.io/local-path
parameters:
  nodePath: /srv/data/local-path-provisioner
volumeBindingMode: WaitForFirstConsumer
reclaimPolicy: Delete
---
apiVersion: v1
kind: PersistentVolume
metadata:
  name: stackgres-storage-pv
  labels:
    type: local
spec:
  accessModes:
    - ReadWriteOnce
  capacity:
    storage: 340Gi
  storageClassName: "stackgres-storage"
  hostPath:
    path: /srv/data/local-path-provisioner
\end{lstlisting}

\lstset{style=gra_codestyle}
\begin{lstlisting}[language=bash, caption=StackGres-Citus - StorageClass / PersistentVolume Grosse Volumes aktivieren,captionpos=b,label={lst:stackgres_citus-storageclass-big-apply},breaklines=true]
kubectl apply -f /home/gramic/PycharmProjects/rke2_settings/stackgres_citus/stackgres_citus/storageclass_big.yaml
\end{lstlisting}

\paragraph{Installation}
\lstset{style=gra_codestyle}
\begin{lstlisting}[language=bash, caption=StackGres-Citus - Namespace 250GiB,captionpos=b,label={lst:stackgres-citus-namespace-250gib},breaklines=true]
kubectl create namespace sg-platform
\end{lstlisting}

Natürlich muss auch wieder das \gls{helm}-Manifest deployt werden:
\lstset{style=gra_codestyle}
\begin{lstlisting}[language=bash, caption=StackGres-Citus - Installation  250GiB,captionpos=b,label={lst:stackgres_citus-values-apply-250gib},breaklines=true]
helm install -n sg-platform stackgres-operator stackgres-charts/stackgres-operator -f /home/gramic/PycharmProjects/rke2_settings/stackgres_citus/stackgres_citus/values.yaml
\end{lstlisting}
\paragraph{Deployment - Benchmarking mit 250GiB}
Die Ressourcen für die Shards mussten erhöht werden:
\lstset{style=gra_codestyle}
\begin{lstlisting}[language=yaml, caption=StackGres-Citus - Benchmarking - SGInstanceProfile Coordinator 250GiB,captionpos=b,label={lst:SGInstanceProfile_pgbench_coord.yaml-250gib},breaklines=true]
apiVersion: stackgres.io/v1
kind: SGInstanceProfile
metadata:
  namespace: sg-platform
  name: sg-pgbench-coordinator
spec:
  cpu: "4"
  memory: "4Gi"
\end{lstlisting}
\lstset{style=gra_codestyle}
\begin{lstlisting}[language=yaml, caption=StackGres-Citus - Benchmarking - SGInstanceProfile Shard 250GiB,captionpos=b,label={lst:SGInstanceProfile_pgbench_shard.yaml-250gib},breaklines=true]
apiVersion: stackgres.io/v1
kind: SGInstanceProfile
metadata:
  namespace: sg-platform
  name: sg-pgbench-shard
spec:
  cpu: "4"
  memory: "12Gi"
\end{lstlisting}

\lstset{style=gra_codestyle}
\begin{lstlisting}[language=bash, caption=StackGres-Citus - Benchmarking - Instanz-Profil Deploy 250GiB,captionpos=b,label={lst:stackgres_citus-deploy-benchmarking-instance-profiles-250gib},breaklines=true]
kubectl apply -f /home/gramic/PycharmProjects/rke2_settings/stackgres_citus/stackgres_citus/SGInstanceProfile_pgbench_coord.yaml
kubectl apply -f /home/gramic/PycharmProjects/rke2_settings/stackgres_citus/stackgres_citus/SGInstanceProfile_pgbench_shard.yaml
\end{lstlisting}

Das Manifest für den Benchmark-Cluster wurde die Shard-Cluster Anzahl auf 2 reduziert und die PVC entsprechend dimensioniert:
\lstset{style=gra_codestyle}
\begin{lstlisting}[language=yaml, caption=StackGres-Citus - Benchmarking - SGShardedCluster 250GiB,captionpos=b,label={lst:SGShardedCluster_pgbench.yaml-250gib},breaklines=true]
apiVersion: stackgres.io/v1alpha1
kind: SGShardedCluster
metadata:
  name: sg-pgbench
  namespace: sg-platform
spec:
  type: citus
  database: pgbench_eval_bench
  postgres:
    version: '16'
  coordinator:
    instances: 1
    pods:
      persistentVolume:
        size: '230Gi'
        storageClass: "stackgres-storage"

      disableConnectionPooling: true
    sgInstanceProfile: "sg-pgbench-coordinator"
  shards:
    clusters: 2 # gramic, 19.04.2024: 250GiB Tabelle
    instancesPerCluster: 1
    pods:
      persistentVolume:
        size: '230Gi'
        storageClass: "stackgres-storage"
    sgInstanceProfile: "sg-pgbench-shard"
  postgresServices:
    coordinator:
      primary:
        type: LoadBalancer
      any:
        type: LoadBalancer
    shards:
      primaries:
        type: LoadBalancer
  metadata:
    annotations:
      primaryService:
        metallb.universe.tf/loadBalancerIPs: 10.0.20.106
      replicasService:
        metallb.universe.tf/loadBalancerIPs: 10.0.20.153
        externalTrafficPolicy: "Cluster"
  profile: "testing"
\end{lstlisting}

Der Deploy:
\lstset{style=gra_codestyle}
\begin{lstlisting}[language=bash, caption=StackGres-Citus - Benchmark - Cluster Deploy 250GiB,captionpos=b,label={lst:stackgres_citus-benchmnarking-deploy-cluster-250gib},breaklines=true]
kubectl apply -f /home/gramic/PycharmProjects/rke2_settings/stackgres_citus/stackgres_citus/SGShardedCluster_pgbench.yaml
\end{lstlisting}

\subsubsection{SQL Statements - Benchmarking}
\label{subsubsec:stackgres_citus_benchmarking_sql}
Für das Benchmarking wird die Tabelle \texttt{pgbench\_eval\_bench} bereits beim Deployment erstellt.\\
Allerdings ohne Tablespaces.

Daher sind keine weiteren Schritte an dieser Stelle notwendig, die Tabellen werden von \texttt{pgbench} beim Initialisieren erstellt.
\subsubsection{SQL Statements - Testing}
\label{subsubsec:stackgres_citus_self-healing-testing_sql}
Auch hier wird die Tabelle bereits beim Deployment des Clusters erstellt.\\
Es müssen aber natürlich noch gemäss ERD die Tabellen erstellt werden:
\hyperref[subsubsec:erd_self_healing_test]{Evaluation - ERD self\_healing\_test}
Auch hier wird auf Tablespaces verzichtet.\\
Erst werden die Rollen erstellt, gefolgt von den Usern und Schemas.\\
Die Schemas müssen entsprechend mittels \texttt{GRANT} berechtigt werden.
Die Tabellen müssen entsprechend den Schemas erstellt werden, es werden Reference Table Shards erzeugt.\\
Das gesamte \texttt{CREATE}-Skript:
\lstset{style=gra_codestyle}
\begin{lstlisting}[language=sql, caption=StackGres-Citus - Self Healing Tests - CREATE-SQL,captionpos=b,label={stackgres_citus-self-healing-create-sql},breaklines=true]
--  Rollen erstellen
drop role if exists hrm;
create role hrm;
drop role if exists accountands;
create role accountands;
drop role if exists customer_service_officers;
create role customer_service_officers;
drop role if exists legal_affairs;
create role legal_affairs;

--  User erstellen
drop user if exists hrm_1;
drop user if exists hrm_2;
create user hrm_1 with password 'hrm1' role hrm;
create user hrm_2 with password 'hrm2' role hrm;

drop user if exists cso_1;
drop user if exists cso_2;
create user cso_1 with password 'cso1'role customer_service_officers;
create user cso_2 with password 'cso2' role customer_service_officers;

drop user if exists la_1;
drop user if exists la_2;
create user la_1 with password 'la1' role legal_affairs;
create user la_2 with password 'la2' role legal_affairs;

--  Schemas erstellen
drop schema if exists hrm;
create schema hrm authorization hrm;
drop schema if exists accountands;
create schema accountands authorization accountands;
drop schema if exists customer_service_officers;
create schema customer_service_officers authorization customer_service_officers;
drop schema if exists generell;
create schema generell;

--  GRANTS erstellen
grant all on all tables in schema hrm to legal_affairs;
grant all on all tables in schema accountands to legal_affairs;
grant all on all tables in schema customer_service_officers to legal_affairs;
grant all on all tables in schema generell to legal_affairs;
grant all on all tables in schema hrm to postgres;
grant all on all tables in schema accountands to postgres;
grant all on all tables in schema customer_service_officers to postgres;
grant all on all tables in schema generell to postgres;

 -- self_healing_accounts für Schema customer_service_officers
drop table if exists customer_service_officers.self_healing_accounts;
create table customer_service_officers.self_healing_accounts (
    account_id int primary key,
    firstname varchar(255) not null,
    lastname varchar(255) not null,
    birthday date not null,
    postal_code varchar(50),
    street varchar(255),
    country_code varchar(2),
    phone varchar(25),
    mail varchar(255) check (mail like '%@%')
);
create unique index accounts_personal_mark on customer_service_officers.self_healing_accounts(firstname, lastname, birthday);
SELECT create_reference_table('customer_service_officers.self_healing_accounts');

--  self_healing_employees für Schema hrm
drop table if exists hrm.self_healing_employees;
create table hrm.self_healing_employees (
    employees_id int primary key,
    firstname varchar(255) not null,
    lastname varchar(255) not null,
    birthday date not null,
    postal_code varchar(50),
    street varchar(255),
    country_code varchar(2),
    phone varchar(25),
    mail varchar(255) check (mail like '%@%')
);
create unique index employees_personal_mark on hrm.self_healing_employees(firstname, lastname, birthday);
SELECT create_reference_table('hrm.self_healing_employees');

--  self_healing_accountand_protocol für Schema accountands
drop table if exists accountands.self_healing_accountand_protocol;
create table accountands.self_healing_accountand_protocol (
       acc_protocol_id int primary key,
       description varchar(100) not null,
       protocol_date date not null,
       employees_id int not null,
       rapport TEXT,
       foreign key (employees_id) references hrm.self_healing_employees(employees_id) on update restrict on delete restrict
);
SELECT create_reference_table('accountands.self_healing_accountand_protocol');

--  self_healing_intranet für public Schema
drop table if exists generell.self_healing_intranet;
create table generell.self_healing_intranet (
       intranet_id int primary key,
       content text
);
SELECT create_reference_table('generell.self_healing_intranet');

--  self_healing_intranet für public Schema
drop table if exists generell.self_healing_intranet_users;
create table generell.self_healing_intranet_users (
       intranet_user_id int primary key,
       employees_id int not null,
       foreign key (employees_id) references hrm.self_healing_employees(employees_id) on update restrict on delete restrict
);
create unique index intranet_unique_combi on generell.self_healing_intranet_users(intranet_user_id, employees_id);
SELECT create_reference_table('generell.self_healing_intranet_users');
\end{lstlisting}

Es sollen aber auch gleich Daten initial geschrieben werden:
\lstset{style=gra_codestyle}
\begin{lstlisting}[language=sql, caption=StackGres-Citus - Self Healing Tests - Init Data,captionpos=b,label={lst:stackgres_citus-self-healing-init-data},breaklines=true]
insert into customer_service_officers.self_healing_accounts (account_id, firstname, lastname, birthday) VALUES (100, 'a', 'b', '01.01.2000');
insert into customer_service_officers.self_healing_accounts (account_id, firstname, lastname, birthday) VALUES (200, 'c', 'd', '01.01.2000');
insert into customer_service_officers.self_healing_accounts (account_id, firstname, lastname, birthday) VALUES (300, 'f', 'g', '01.01.2000');

insert into hrm.self_healing_employees (employees_id, firstname, lastname, birthday) VALUES (100, 'a', 'b', '01.01.2000');
insert into hrm.self_healing_employees (employees_id, firstname, lastname, birthday) VALUES (200, 'c', 'd', '01.01.2000');
insert into hrm.self_healing_employees (employees_id, firstname, lastname, birthday) VALUES (300, 'f', 'g', '01.01.2000');

insert into accountands.self_healing_accountand_protocol (acc_protocol_id, description, protocol_date, employees_id, rapport)  values (100, 'bla', '07.04.2024', 100, 'blabla');
insert into accountands.self_healing_accountand_protocol (acc_protocol_id, description, protocol_date, employees_id, rapport)  values (200, 'yada', '07.04.2024', 100, 'ydayadyada');
insert into accountands.self_healing_accountand_protocol (acc_protocol_id, description, protocol_date, employees_id, rapport)  values (300, 'something', '07.04.2024', 300, 'something');

insert into generell.self_healing_intranet(intranet_id, content) VALUES (100, 'yadada');
insert into generell.self_healing_intranet(intranet_id, content) VALUES (500, 'bla bla');
insert into generell.self_healing_intranet(intranet_id, content) VALUES (1000, 'talking and talking');

insert into generell.self_healing_intranet_users(intranet_user_id, employees_id) values(100, 100);
insert into generell.self_healing_intranet_users(intranet_user_id, employees_id) values(200, 200);
insert into generell.self_healing_intranet_users(intranet_user_id, employees_id) values(300, 300);

select * from customer_service_officers.self_healing_accounts;
select * from hrm.self_healing_employees;
select * from accountands.self_healing_accountand_protocol;
select * from generell.self_healing_intranet_users;
\end{lstlisting}

Während dem Failover-Test müssen Daten beschrieben werden:
\lstset{style=gra_codestyle}
\begin{lstlisting}[language=sql, caption=StackGres-Citus - Self Healing Tests - Failover Data,captionpos=b,label={lst:stackgres_citus-self-healing-failover-data},breaklines=true]
insert into customer_service_officers.self_healing_accounts (account_id, firstname, lastname, birthday) VALUES (400, 'i', 'j', '01.01.2005');
insert into customer_service_officers.self_healing_accounts (account_id, firstname, lastname, birthday) VALUES (500, 'k', 'l', '01.01.2003');
insert into customer_service_officers.self_healing_accounts (account_id, firstname, lastname, birthday) VALUES (600, 'm', 'n', '01.01.2001');

insert into hrm.self_healing_employees (employees_id, firstname, lastname, birthday) VALUES (400, 'i', 'j', '01.01.2005');
insert into hrm.self_healing_employees (employees_id, firstname, lastname, birthday) VALUES (500, 'k', 'l', '01.01.2003');
insert into hrm.self_healing_employees (employees_id, firstname, lastname, birthday) VALUES (600, 'm', 'n', '01.01.2001');

insert into accountands.self_healing_accountand_protocol (acc_protocol_id, description, protocol_date, employees_id, rapport)  values (400, 'bla', '07.04.2024', 200, 'blabla');
insert into accountands.self_healing_accountand_protocol (acc_protocol_id, description, protocol_date, employees_id, rapport)  values (500, 'yada', '07.04.2024', 600, 'ydayadyada');
insert into accountands.self_healing_accountand_protocol (acc_protocol_id, description, protocol_date, employees_id, rapport)  values (1000, 'something', '07.04.2024', 300, 'something');

insert into generell.self_healing_intranet(intranet_id, content) VALUES (200, 'yadada');
insert into generell.self_healing_intranet(intranet_id, content) VALUES (600, 'bla bla');
insert into generell.self_healing_intranet(intranet_id, content) VALUES (900, 'talking and talking');

insert into generell.self_healing_intranet_users(intranet_user_id, employees_id) values(400, 400);
insert into generell.self_healing_intranet_users(intranet_user_id, employees_id) values(500, 500);
insert into generell.self_healing_intranet_users(intranet_user_id, employees_id) values(600, 600);

select * from customer_service_officers.self_healing_accounts;
select * from hrm.self_healing_employees;
select * from accountands.self_healing_accountand_protocol;
select * from generell.self_healing_intranet;
select * from generell.self_healing_intranet_users;
\end{lstlisting}

Nach dem Recovery müssen die Daten entsprechend vorhanden sein und es müssen weitere Daten beschrieben werden können:
\lstset{style=gra_codestyle}
\begin{lstlisting}[language=sql, caption=StackGres-Citus - Self Healing Tests - Recovery Data,captionpos=b,label={lst:stackgres_citus-self-healing-recovery-data},breaklines=true]
insert into generell.self_healing_intranet_users(intranet_user_id, employees_id) values(1000, 400);
insert into generell.self_healing_intranet_users(intranet_user_id, employees_id) values(2000, 500);
insert into generell.self_healing_intranet_users(intranet_user_id, employees_id) values(3000, 600);

select count(*) from customer_service_officers.self_healing_accounts;
select count(*) from hrm.self_healing_employees;
select count(*) from accountands.self_healing_accountand_protocol;
select count(*) from generell.self_healing_intranet;
select count(*) from generell.self_healing_intranet_users;
\end{lstlisting}
