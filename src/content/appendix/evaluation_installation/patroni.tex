%! Author = gra
%! Date = 05.04.24

% Preamble
\begin{flushleft}
    \subsection{Patroni}
    \label{subsec:evaluation_installation_patroni}
    \subsubsection{Prerequisites}
    Ganz am Anfang steht die Firewall.\\
    Diese muss auf allen drei Patroni-Servern wie folgt gesetzt werden:
\lstset{style=gra_codestyle}
\begin{lstlisting}[language=bash, caption=Patroni - Firewall Settings,captionpos=b,label={lst:patroni-firewall-settings},breaklines=true]
# sks1232 / sks1233 / sks1234 / sks9016(10.0.28.16)
nano /etc/iptables/rules.v4
*filter
:INPUT ACCEPT [0:0]
:FORWARD ACCEPT [0:0]
:OUTPUT ACCEPT [0:0]
-A INPUT -s 10.0.0.0/8 -p tcp -m tcp --dport 22 -j ACCEPT
-A INPUT -s 10.0.9.115/32 -p udp -m udp --dport 161 -m comment --comment "Allow SNMP for probe 10.0.9.115" -j ACCEPT
-A INPUT -s 10.0.9.76/32 -p udp -m udp --dport 161 -m comment --comment "Allow SNMP for probe 10.0.9.76" -j ACCEPT
-A INPUT -s 10.0.36.147/32 -p udp -m udp --dport 161 -m comment --comment "Allow SNMP for probe 10.0.36.147" -j ACCEPT
-A INPUT -s 10.0.9.35/32 -p udp -m udp --dport 161 -m comment --comment "Allow SNMP for probe 10.0.9.35" -j ACCEPT
-A INPUT -s 10.0.9.37/32 -p udp -m udp --dport 161 -m comment --comment "Allow SNMP for probe 10.0.9.37" -j ACCEPT
-A INPUT -s 10.0.9.74/32 -p udp -m udp --dport 161 -m comment --comment "Allow SNMP for probe 10.0.9.74" -j ACCEPT
-A INPUT -s 10.0.9.75/32 -p udp -m udp --dport 161 -m comment --comment "Allow SNMP for probe 10.0.9.75" -j ACCEPT
-A INPUT -s 10.0.9.36/32 -p udp -m udp --dport 161 -m comment --comment "Allow SNMP for probe 10.0.9.36" -j ACCEPT
-A INPUT -s 10.0.9.14/32 -p udp -m udp --dport 161 -m comment --comment "Allow SNMP for probe 10.0.9.14" -j ACCEPT
-A INPUT -s 10.0.0.0/8 -p icmp -m icmp --icmp-type 8 -j ACCEPT
# generell
-A INPUT -s 10.0.0.0/8 -p tcp -m tcp --dport 443 -j ACCEPT
# postgres
-A INPUT -s 10.0.0.0/8 -p tcp -m tcp --dport 5432 -j ACCEPT
# patroni
-A INPUT -s 10.0.0.0/8 -p tcp -m tcp --dport 2379 -j ACCEPT
-A INPUT -s 10.0.0.0/8 -p tcp -m tcp --dport 2380 -j ACCEPT
-A INPUT -s 10.0.0.0/8 -p tcp -m tcp --dport 2376 -j ACCEPT
-A INPUT -s 10.0.0.0/8 -p tcp -m tcp --dport 6432 -j ACCEPT
-A INPUT -s 10.0.0.0/8 -p tcp -m tcp --dport 8008 -j ACCEPT
-A INPUT -s 10.0.0.0/8 -p tcp -m tcp --dport 7000 -j ACCEPT
-A INPUT -s 10.0.0.0/8 -p tcp -m tcp --dport 8080 -j ACCEPT
COMMIT
# Completed

systemctl restart iptables
systemctl status iptables
\end{lstlisting}

\end{flushleft}
\begin{flushleft}
    Danach muss der Proxy gesetzt werden:
\lstset{style=gra_codestyle}
\begin{lstlisting}[language=bash, caption=Patroni - Proxy Settings,captionpos=b,label={lst:patroni-proxy-settings},breaklines=true]
# sks1232 / sks1233 / sks1234
# Proxy setzen
# nano /etc/profile.d/proxy.sh
export https_proxy=http://sproxy.sivc.first-it.ch:8080
export HTTPS_PROXY=http://sproxy.sivc.first-it.ch:8080
export http_proxy=http://sproxy.sivc.first-it.ch:8080
export HTTP_PROXY=http://sproxy.sivc.first-it.ch:8080
export no_proxy=localhost,127.0.0.0/8,::1,10.0.0.0/8,172.16.0.0/12,192.168.0.0/16
export NO_PROXY=localhost,127.0.0.0/8,::1,10.0.0.0/8,172.16.0.0/12,192.168.0.0/16
# source /etc/profile.d/proxy.sh
\end{lstlisting}

\end{flushleft}
\begin{flushleft}
    Damit das PostgreSQL-Repository eingebunden werden kann,\\
    muss dem apt-Proxy gesetzt werden.\\
    Da via \Gls{Foreman} Installiert wurde, muss dieser ausgenommen werden:
\lstset{style=gra_codestyle}
\begin{lstlisting}[language=bash, caption=Patroni - apt-Proxy Settings,captionpos=b,label={lst:patroni-apt-proxy-settings},breaklines=true]
# sks1232 / sks1233 / sks1234
# apt-Proxy setzen
# nano /etc/apt/apt.conf.d/proxy.conf
Acauire::http::Proxy "http://sproxy.sivc.first-it.ch:8080";
Acauire::https::Proxy "http://sproxy.sivc.first-it.ch:8080";
Acquire::http::proxy::foreman.ksgr.ch "DIRECT";
\end{lstlisting}

\end{flushleft}
\begin{flushleft}
    Im nächsten Schritt kann das PostgreSQL-Repository eingebunden werden.\\
    \begin{warning}
    Achtung, die von PostgreSQL beschriebene Variante wurde in Debian 10 als Deprecated gesetzt,
    mit Debian 13 wird diese Repository-Integration einen Fehler werden.
    \end{warning}
\lstset{style=gra_codestyle}
\begin{lstlisting}[language=bash, caption=Patroni - PostgreSQL einbinden,captionpos=b,label={lst:patroni-include-repository},breaklines=true]
# sks1232 / sks1233 / sks1234
# PostgreSQL Repository einbinden
sudo sh -c 'echo "deb https://apt.postgresql.org/pub/repos/apt $(lsb_release -cs)-pgdg main" > /etc/apt/sources.list.d/pgdg.list'
wget --quiet -O - https://www.postgresql.org/media/keys/ACCC4CF8.asc | sudo apt-key add -

# Ausloggen und wieder einloggen
apt update
\end{lstlisting}

\end{flushleft}
\begin{flushleft}
    Nun muss der \Gls{PostgreSQL Cluster}, Patroni, python3-etcd und python3-psycopg2 installiert werden:
\lstset{style=gra_codestyle}
\begin{lstlisting}[language=bash, caption=Patroni - Prerequisites installieren,captionpos=b,label={lst:patroni-prerequisites-install},breaklines=true]
apt install postgresql-16 postgresql-server-dev-16 patroni python3-etcd python3-psycopg2
\end{lstlisting}
    Im nächsten Schritt müssen Patroni und der \Gls{PostgreSQL Cluster} gestoppt werden:
\lstset{style=gra_codestyle}
\begin{lstlisting}[language=bash, caption=Patroni - Stop Patroni und PostgreSQL,captionpos=b,label={lst:patroni-stop-postgresql-patroni},breaklines=true]
systemctl stop postgresql patroni
\end{lstlisting}
    Anschliessend muss noch vom PostgreSQL-Verzeichnis \texttt{/usr/lib/postgresql/16/bin/} ein Symlink nach \texttt{/usr/sbin/} gesetzt werden:
\lstset{style=gra_codestyle}
\begin{lstlisting}[language=bash, caption=Patroni - Symlink bin,captionpos=b,label={lst:patroni-symlink-bin},breaklines=true]
ln -s /usr/lib/postgresql/16/bin/* /usr/sbin/
\end{lstlisting}

\end{flushleft}
\begin{flushleft}
    Zu guter Letzt sollte geprüft werden, ob alle Versionen passen und am richtigen Ort sind:
\lstset{style=gra_codestyle}
\begin{lstlisting}[language=bash, caption=Patroni - Checks,captionpos=b,label={lst:patroni-checks},breaklines=true]
which patroni
which psql
patroni --version
\end{lstlisting}
    Damit kann zum \gls{etcd} übergegangen werden.
\end{flushleft}

%\begin{flushleft}
    \subsubsection{etcd}
    Auf \texttt{sks9016} sollte erst das Repository angepasst werden und anschliessend der \texttt{etcd-server} installiert werden:
\lstset{style=gra_codestyle}
\begin{lstlisting}[language=bash, caption=Patroni - etcd-server konfigurieren=b,label={lst:patroni-etcd-server-config},breaklines=true]
# nano /etc/default/etcd
ETCD_LISTEN_PEER_URLS="http://10.0.28.16:2380"
ETCD_LISTEN_CLIENT_URLS="http://localhost:2379,http://10.0.28.16:2379"
ETCD_INITIAL_ADVERTISE_PEER_URLS="http://10.0.28.16:2380"
ETCD_INITIAL_CLUSTER="default=http://10.0.28.16:2380,"
ETCD_ADVERTISE_CLIENT_URLS="http://10.0.28.16:2379"
ETCD_INITIAL_CLUSTER_TOKEN="etcd-cluster"
ETCD_INITIAL_CLUSTER_STATE="new"
\end{lstlisting}
    Der Service sollte rebooted werden und seine lauffähigkeit getestet werden:
\lstset{style=gra_codestyle}
\begin{lstlisting}[language=bash, caption=Patroni - etcd-server reboot=b,label={lst:patroni-etcd-server-reboot},breaklines=true]
systemctl restart etcd
systemctl is-enabled etcd
systemctl status etcd
\end{lstlisting}
    Es sollte nun ein Member gelistet werden:
\lstset{style=gra_codestyle}
\begin{lstlisting}[language=bash, caption=Patroni - etcd-server list menber=b,label={lst:patroni-etcd-server-list-member},breaklines=true]
etcdctl member list
\end{lstlisting}
    Damit kann nun das Bootstrapping gestartet werden.
%\end{flushleft}
%\begin{flushleft}
    \subsubsection{Bootstrapping}
%\end{flushleft}
%\begin{flushleft}
    \subsubsection{HAproxy}
%\end{flushleft}
