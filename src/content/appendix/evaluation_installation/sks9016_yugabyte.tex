%! Author = gramic
%! Date = 03.04.24

% Preamble
\begin{flushleft}
    \subsection{sks9016 - yugabyteDB}
    \subsubsection{yugabyteDB - Download und Installation yugabyteDB}
    Ohne yugabyteDB zu installieren, lässt sich \texttt{ysql\_bench} nicht ausführen.
    Daher muss das ganze Package erst heruntergeladen werden:
    \lstset{style=gra_codestyle}
    \begin{lstlisting}[language=bash, caption=sks9016 - Download yugabyteDB On-Premise,captionpos=b,label={lst:sks9016-yugabytedb-download-on-premise},breaklines=true]
root@sks9016:~# wget https://downloads.yugabyte.com/releases/2.21.0.0/yugabyte-2.21.0.0-b545-linux-x86_64.tar.gz
    \end{lstlisting}
\end{flushleft}
\begin{flushleft}
    Im nächsten Schritt wird es im \texttt{/opt} entpackt und das \texttt{post\_install.sh}-Skript ausgeführt:
    \lstset{style=gra_codestyle}
    \begin{lstlisting}[language=bash, caption=sks9016 - Installation yugabyteDB On-Premise,captionpos=b,label={lst:sks9016-yugabytedb-install-on-premise},breaklines=true]
root@sks9016:/opt# tar xvfz yugabyte-2.21.0.0-b545-linux-x86_64.tar.gz && cd yugabyte-2.21.0.0/
...
root@sks9016:/opt/yugabyte-2.21.0.0# ./bin/post_install.sh
    \end{lstlisting}
\end{flushleft}
\begin{flushleft}
    Um nun zu Testen, ob das ganze Funktioniert, kann eine Verbindung zum Evaluationssystem hergestellt werden:
    \lstset{style=gra_codestyle}
    \begin{lstlisting}[language=bash, caption=sks9016 - Check yugabyteDB On-Premise,captionpos=b,label={lst:sks9016-yugabytedb-check-on-premise},breaklines=true]
root@sks9016:/opt/yugabyte-2.21.0.0# cd /opt/yugabyte-2.21.0.0/postgres/bin/
root@sks9016:/opt/yugabyte-2.21.0.0/postgres/bin# ./ysqlsh "host=10.0.20.106 user=yadmin"
Password for user yadmin:
ysqlsh (11.2-YB-2.21.0.0-b0)
Type "help" for help.

No entry for terminal type "xterm-256color";
using dumb terminal settings.
yugabyte=# exit
    \end{lstlisting}
    Damit ist der Benchmarking-Server ready.
\end{flushleft}