%! Author = gramic
%! Date = 11.05.24

% Preamble
\begin{flushleft}
    \subsection{Maintenance-Tool - \Gls{AUTOVACUUM}}
    \label{subsec:maintenance_autovacuum}
    Das Python-Skript:
    \lstset{style=gra_codestyle}
    \begin{lstlisting}[language=python, caption=Maintenance-Tool - \Gls{AUTOVACUUM} - ksgr\_postgresql\_maintenance\_autovacuum\_calculation.py,captionpos=b,label={lst:maintenannce-tool-autovacuum-python},breaklines=true]
import os
import sys
from datetime import date

import pandas as pd
import psycopg2
import yaml


#   Pass Excption Class
class OneException(Exception):
    pass

#   Read the ConfigMap
#
#   Get the ConfigMap from the Namespace
def read_config(confimap):
    # config = os.environ['config-map-connections.yaml']
    script_dir = os.path.dirname(os.path.abspath(__file__))
    configfile = os.path.join(script_dir, confimap)
    config = dict()
    with open(configfile, "r") as file:
        config = yaml.load(file, Loader=yaml.FullLoader)
    print(config)
    return config


#   Open Database Connection
#
#   Opens a Database Connection to the over given Database.
#   Returns the opened Connection.
def database_connection(config, database, user, password):
    # user = config['connectiondata']['username']
    host = config['host']
    port = config['port_ro']
    conn = psycopg2.connect(database=database,
                            user=user,
                            host=host,
                            password=password,
                            port=port)
    return conn


#   Get the Secret
#
#   Get the Secret from the Namespace
def get_secret(secret_name):
    secret = open(secret_name, "r").read()
    return secret

#   Get the Username
#
#   Get the Username from the Secret from the Namespace
def get_username(secret_name):
    username = open(secret_name, "r").read()
    print("username", username)
    return username

#   Get PostgreSQL Parameters
#
#   Get autovacuum_vacuum_scale_factor and autovacuum_vacuum_threshold from pg_settings
#   Get sum of reltuples from pg_class and get the sender_host (Primary) IP Address from pg_stat_wal_receiver
def get_paramaters(connection):
    cur = connection.cursor()
    sql = """select
    (
        select
            setting as autovacuum_vacuum_scale_factor
        from pg_settings
        where
            name = 'autovacuum_vacuum_scale_factor'
    ) as autovacuum_vacuum_scale_factor,
    (
        select
            setting as autovacuum_vacuum_threshold
        from pg_settings
        where
            name = 'autovacuum_vacuum_threshold'
    ) as autovacuum_vacuum_threshold,
    (
        select
            sum(reltuples) as reltuples
        from pg_class
    ) as reltuples,
    (
        select
            sender_host
        from pg_stat_wal_receiver
    ) as sender_host
;"""
    cur.execute(sql)
    postgresql_parameters = cur.fetchone()
    autovacuum_vacuum_scale_factor = postgresql_parameters[0]
    autovacuum_vacuum_threshold = postgresql_parameters[1]
    reltuples = postgresql_parameters[2]
    sender_host = postgresql_parameters[3]

    postgresql_parameter = dict(autovacuum_vacuum_scale_factor=autovacuum_vacuum_scale_factor,
                                autovacuum_vacuum_threshold=autovacuum_vacuum_threshold, reltuples=reltuples,
                                sender_host=sender_host)

    return postgresql_parameter

#   Calculate and Check AUTOVACUUM
#
#   Calculate needed autovacuum_vacuum_scale_factor and check with the configured ones.
#   Create HTML Table if values are different
def calculate_check_autovacuum(postgres_params, configmap_autovacuum):
    autovacuum_vacuum_scale_factor_actual = float(postgres_params.get('autovacuum_vacuum_scale_factor'))
    autovacuum_vacuum_threshold = float(postgres_params.get('autovacuum_vacuum_threshold'))
    reltuples = float(postgres_params.get('reltuples'))
    sender_host = postgres_params.get('sender_host')
    max_death_tuples = configmap_autovacuum.get('max_death_tuples')
    autovacuum_vacuum_scale_factor = round(((max_death_tuples - autovacuum_vacuum_threshold) / reltuples), 2)

    if autovacuum_vacuum_scale_factor != autovacuum_vacuum_scale_factor_actual:
        result_table = create_pandas_dataframe(autovacuum_vacuum_scale_factor_actual, autovacuum_vacuum_scale_factor, ['autovacuum_vacuum_scale_factor'])
        result_path = configmap_autovacuum.get('calc_result')
        filename_prefix = configmap_autovacuum.get('filename_prefix')
        filename = filename_prefix + str(date.today()) + '.html'
        write_calculating_result(result_table, result_path, filename)

#   Pandas DataFrame
#
#   Create a Pandas DataFrame from the overgiven Parameters
def create_pandas_dataframe(autovacuum_vacuum_scale_factor_actual, autovacuum_vacuum_scale_factor, index):
    data = {'Bestehend': autovacuum_vacuum_scale_factor_actual, 'Berechnet': autovacuum_vacuum_scale_factor}
    result_df = pd.DataFrame(data, index=index)
    return result_df


#   Write HTML
#
#   Write the Result Pandas DataFrame to HTML Table
def write_calculating_result(result_table, path, filename):
    # html = result_table.to_html(classes='table table-stripped')
    html = result_table.to_html()
    filepath = path + '/' + filename
    text_file = open(filepath, "w")
    text_file.write(html)
    text_file.close()

#   Maintenance Main
#
#   Calculate the AUTOVACUUM Parameters
def maintenance_autovacuum_calculation(confimap_conn, configmap_autovacuum):
    try:
        #   Read the Configs
        config = read_config(confimap_conn)
        config_autovac = read_config(configmap_autovacuum)

        #   Get Configuration
        database = config['database']
        secret_username = config['secret_username']
        secret_password = config['secret_password']

        #   Get Secret of the Database
        secret = get_secret(secret_password)

        #   Get Username for the Database
        user = get_username(secret_username)

        #   Open Database Connection
        connection = database_connection(config, database, user, secret)

        #   Get the PostgreSQL Parameters
        postgres_params = get_paramaters(connection)

        #   Calculate and Check the Autovacuum Vacuum Scale Factor
        calculate_check_autovacuum(postgres_params, config_autovac)

        #   Close Database Connection
        connection.close()

        # cleanup_bloated_database(config)
    except OneException as e:
        print(repr(e))

arg1 = sys.argv[1]
arg2 = sys.argv[2]
maintenance_autovacuum_calculation(arg1, arg2)
    \end{lstlisting}
\end{flushleft}
\begin{flushleft}
    Das Python-Skript \texttt{ksgr\_postgresql\_maintenance\_autovacuum\_calculation.py} kann in ein ConfigMap umgewandelt werden:
    \lstset{style=gra_codestyle}
    \begin{lstlisting}[language=bash, caption=Maintenance-Tool - \Gls{AUTOVACUUM} - Python > ConfigMap,captionpos=b,label={lst:maintenannce-tool-autovacuum-python-to-configmap},breaklines=true]
kubectl create configmap ksgr-postgresql-maintenance-autovacuum-calculation --from-file /home/gramic/PycharmProjects/ksgr_postgresql_maintenance/ksgr_postgresql_maintenance_autovacuum_calculation.py --dry-run=client -o yaml > /home/gramic/PycharmProjects/ksgr_postgresql_maintenance/ksgr_postgresql_maintenance_autovacuum_calculation.yml -n ksgr-postgresql-maintenance
    \end{lstlisting}
    Entsprechend lässt es sich deployen:
    \lstset{style=gra_codestyle}
    \begin{lstlisting}[language=bash, caption=Maintenance-Tool - Bloated Tables / Indices - Python - ConfigMap Deploy,captionpos=b,label={lst:maintenannce-tool-autovacuum-python-configmap-deploy},breaklines=true]
kubectl create -f /home/gramic/PycharmProjects/ksgr_postgresql_maintenance/ksgr_postgresql_maintenance_autovacuum_calculation.yml -n ksgr-postgresql-maintenance
    \end{lstlisting}
\end{flushleft}
\begin{flushleft}
    Ein neues Secret muss nicht deployt werden.
\end{flushleft}
\begin{flushleft}
    Im ConfigMap \texttt{configmap-vks0041-postgres} wird der Hostname, den Read-Write Port, der Read-Only Port, die Datenbank und die Secret-Pfade gespeichert:
    \lstset{style=gra_codestyle}
    \begin{lstlisting}[language=yaml, caption=Maintenance-Tool - \Gls{AUTOVACUUM} - configmap-vks0041-postgres,captionpos=b,label={lst:maintenannce-tool-autovacuum-configmap-vks0041-postgres},breaklines=true]
apiVersion: v1
kind: ConfigMap
metadata:
  name: configmap-vks0041-postgres
  namespace: ksgr-postgresql-maintenance
data:
  configmap-vks0041-postgres.yaml: |-
    host: "10.0.22.178"
    port_rw: "5000"
    port_ro: "5001"
    database: "postgres"
    secret_username: "/mnt/etc/secret-volume/secret-vks0041/user"
    secret_password: "/mnt/etc/secret-volume/secret-vks0041/password"
    \end{lstlisting}
    Das deployment sieht entsprechend aus:
    \lstset{style=gra_codestyle}
    \begin{lstlisting}[language=bash, caption=Maintenance-Tool - \Gls{AUTOVACUUM} - configmap-vks0041-postgres Deploy,captionpos=b,label={lst:maintenannce-tool-configmap-vks0041-postgres-deploy},breaklines=true]
kubectl create -f /home/gramic/PycharmProjects/ksgr_postgresql_maintenance/configmap-vks0041-postgres.yaml
    \end{lstlisting}
    Für das \Gls{AUTOVACUUM} wird zudem die ConfigMap \texttt{configmap-vks0041-autovacuum} erstellt.\\
    In diesem werden der Pfad zum Node (Host) Filesystem, die Anzahl Toter Tupels und dem Filenamen-Präfix:
    \lstset{style=gra_codestyle}
    \begin{lstlisting}[language=yaml, caption=Maintenance-Tool - \Gls{AUTOVACUUM} - configmap-vks0041-autovacuum,captionpos=b,label={lst:maintenannce-tool-autovacuum-configmap-vks0041-autovacuum},breaklines=true]
apiVersion: v1
kind: ConfigMap
metadata:
  name: configmap-vks0041-autovacuum
  namespace: ksgr-postgresql-maintenance
data:
  configmap-vks0041-autovacuum.yaml: |-
    calc_result: "/mnt/data"
    max_death_tuples: 1000
    filename_prefix: "autovacuum-parameter-"
    \end{lstlisting}
    Das deployment sieht entsprechend aus:
    \lstset{style=gra_codestyle}
    \begin{lstlisting}[language=bash, caption=Maintenance-Tool - \Gls{AUTOVACUUM} - configmap-vks0041-autovacuum Deploy,captionpos=b,label={lst:maintenannce-tool-configmap-vks0041-autovacuum-deploy},breaklines=true]
kubectl create -f /home/gramic/PycharmProjects/ksgr_postgresql_maintenance/configmap-vks0041-autovacuum.yaml
    \end{lstlisting}
\end{flushleft}
\begin{flushleft}
    Das finale Deployment erfolgt mittels einem CronJob.\\
    Jeden Tag um Mitternacht wird ein Pod deployt, welcher Skript ausführt.\\
    Dabei müssen erst folgende Dependencis installiert werden:
    \begin{itemize}
        \item psycopg2-binary
        \item PyYAML
        \item kubernetes
        \item pandas
    \end{itemize}
    Normalerweise würden diese PyPi-Packages in einem \gls{helm} Chart als Ressource mitgegeben werden.\\
    Dann wird das Skript mit dem beiden ConfigMaps ausgeführt.\\
    Dazu wird das Secret und die ConfigMaps in das Pod-Filesystem gemouted.
    \lstset{style=gra_codestyle}
    \begin{lstlisting}[language=yaml, caption=Maintenance-Tool - \Gls{AUTOVACUUM} - ksgr-maintenance-autovacuum-caluclation,captionpos=b,label={lst:maintenannce-tool-autovacuum-ksgr-maintenance-autovacuum-caluclation},breaklines=true]
apiVersion: batch/v1
kind: CronJob
metadata:
  name: ksgr-maintenance-autovacuum-caluclation
  namespace: ksgr-postgresql-maintenance
spec:
  schedule: "0 0 * * *"
  jobTemplate:
    spec:
      template:
        metadata:
          labels:
            workload: cronjob
            type: ksgr-postgresql-maintenance
        spec:
          containers:
            - name: ksgr-maintenance-autovacuum-caluclation-container
              image: python:3.11-bookworm
              command: ["sh", "-c"]
              args:
                [
                  "python -m pip install --proxy http://sproxy.sivc.first-it.ch:8080 psycopg2-binary PyYAML kubernetes pandas;
                  python /tmp/python/ksgr_postgresql_maintenance_autovacuum_calculation.py /tmp/conf/postgres/configmap-vks0041-postgres.yaml /tmp/conf/autovacuum/configmap-vks0041-autovacuum.yaml"
                ]
              volumeMounts:
              - name: ksgr-postgresql-maintenance-autovacuum-calculation
                mountPath: /tmp/python
              - name: configmap-vks0041-postgres
                mountPath: /tmp/conf/postgres
              - name: configmap-vks0041-autovacuum
                mountPath: /tmp/conf/autovacuum
              - name: secret-volume
                readOnly: true
                mountPath: /mnt/etc/secret-volume/secret-vks0041
              - name: ksgr-data
                mountPath: /mnt/data
          restartPolicy: Never
          volumes:
            - name: ksgr-postgresql-maintenance-autovacuum-calculation
              configMap:
                name: ksgr-postgresql-maintenance-autovacuum-calculation
            - name: configmap-vks0041-postgres
              configMap:
                name: configmap-vks0041-postgres
            - name: configmap-vks0041-autovacuum
              configMap:
                name: configmap-vks0041-autovacuum
            - name: ksgr-data
              hostPath:
                path: /mnt/data
            - name: secret-volume
              secret:
                secretName: secret-vks0041
    \end{lstlisting}
    Das Deployment sieht entsprechend aus:
    \lstset{style=gra_codestyle}
    \begin{lstlisting}[language=bash, caption=Maintenance-Tool - \Gls{AUTOVACUUM} - ksgr-maintenance-autovacuum-caluclation Deploy,captionpos=b,label={lst:maintenannce-tool-autovacuum-ksgr-maintenance-autovacuum-caluclation-deploy},breaklines=true]
kubectl apply -f /home/gramic/PycharmProjects/ksgr_postgresql_maintenance/ksgr_postgresql_maintenance_autovacuum_calculation_cronjob.yaml
    \end{lstlisting}
\end{flushleft}