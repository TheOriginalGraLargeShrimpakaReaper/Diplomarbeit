%! Author = gramic
%! Date = 11.05.24

% Preamble
\begin{flushleft}
    \subsection{Maintenance - Bloated Tables}
    Das Python-Skript Connected
    \lstset{style=gra_codestyle}
    \begin{lstlisting}[language=python, caption=Maintenance-Tool - Bloated Tables / Indices - ksgr\_postgresql\_maintenance\_bloated\_tables.py,captionpos=b,label={lst:maintenannce-tool-bloated-tables-python},breaklines=true]
import os
import psycopg2
from psycopg2.extensions import ISOLATION_LEVEL_AUTOCOMMIT
import yaml
from kubernetes import client, config
import base64
import sys

#   Pass Excption Class
class OneException(Exception):
    pass

#   Read the ConfigMap
#
#   Get the ConfigMap from the Namespace
def read_config(confimap):
    # config = os.environ['config-map-connections.yaml']
    script_dir = os.path.dirname(os.path.abspath(__file__))
    configfile = os.path.join(script_dir, confimap)
    config = dict()
    with open(configfile, "r") as file:
        config = yaml.load(file, Loader=yaml.FullLoader)
    print(config)
    return config

#   Open Database Connection
#
#   Opens a Database Connection to the over given Database.
#   Returns the opened Connection.
def database_connection(config, database, user, password):
    # user = config['connectiondata']['username']
    host = config['host']
    port = config['port']
    conn = psycopg2.connect(database=database,
                            user=user,
                            host=host,
                            password=password,
                            port=port)
    return conn

#   Get the Secret
#
#   Get the Secret from the Namespace
def get_secret(secret_name):
    secret = open(secret_name, "r").read()
    print("secret",secret)
    return secret

#   Get the Username
#
#   Get the Username from the Secret from the Namespace
def get_username(secret_name):
    username = open(secret_name, "r").read()
    print("username", username)
    return username

#   Cleanup Bloated Database
#
#   Connected to the Database, check the AUTOVACUUM and VACUUM the Table and
#   REINDEX all Indices of the Table
def cleanup_bloated_database(config):
    #   Get Configuration
    dead_tuple_percent = config['dead_tuple_percent']
    database = config['database']
    secret_username = config['secret_username']
    secret_password = config['secret_password']

    #   Get Secret of the Database
    secret = get_secret(secret_password)

    #   Get Username for the Database
    user = get_username(secret_username)

    #   Open Database Connection
    connection = database_connection(config, database, user, secret)

    #   Get the AUTOVACUUM State
    active_autovaacum = check_autovacuum(connection)

    #   Ckeck AUTOVACUUM State
    #
    #   Run Maintenance Job if no AUTOVACUUM is running
    if active_autovaacum == 0:
        #   Get bloated Tables
        bloated_tables = get_bloated_tables(connection, dead_tuple_percent)
        #   Cleanup bloated Tables
        cleanup_bloated_tables(connection, bloated_tables)
    else:
        print('autovacuum maintenance active!')

    #   Close Database Connection
    connection.close()

#   Check AUTOVACUUM
#
#   Check if an AUTOVACUUM Job is running.
#   For this, a Select on the Table pg_stat_activity on all Queries like autovacuum:% is used.
def check_autovacuum(connection):
    cur = connection.cursor()
    sql = "select count(*) as active from pg_stat_activity where query like 'autovacuum:%';"
    cur.execute(sql)
    active_autovaacum = cur.fetchone()
    active_autovaacum = active_autovaacum[0]

    return active_autovaacum

#   Get Bloated Tables
#
#   Select the Bloated Tables.
def get_bloated_tables(connection, dead_tuple_percent):
    cur = connection.cursor()
    sql = """select
    relname as table,
    (pgstattuple(oid)).dead_tuple_percent
from pg_class
where
    relkind = 'r' and
    (pgstattuple(oid)).dead_tuple_percent > %s
order by dead_tuple_percent desc;
        """
    cur.execute(sql, (dead_tuple_percent,))
    return cur

#   Cleanup Bloated Tables
#
#   Run through the Bloated Tables and VACUUM and REINDEX
def cleanup_bloated_tables(connection, cur):
    bloated_tables = cur.fetchall()
    for bloated_table in bloated_tables:
        table = bloated_table[0]
        vacuum_table(connection, table)
        reindex_table(connection, table)
    print("finished")

#   VACUUM Table
#
#   VACUUM the Table
def vacuum_table(connection, table):
    old_isolation = connection.isolation_level
    connection.set_isolation_level(ISOLATION_LEVEL_AUTOCOMMIT)
    cur = connection.cursor()
    sql = "vacuum " + table + ";"
    cur.execute(sql)
    connection.set_isolation_level(old_isolation)
    print("vaccum table " + table + " successfully")

#   REINDEX
#
#   REINDEX the whole Table
def reindex_table(connection, table):
    cur = connection.cursor()
    sql = "reindex table " + table + ";"
    cur.execute(sql)
    connection.commit()
    print("reindex table " + table + " successfully")

#   Main Method
def maintenance_bloated_databases(confimap):
    # confimap = confimap_Arg[0]
    try:
        # print(confimap)
        config = read_config(confimap)
        cleanup_bloated_database(config)
    except OneException as e:
        print(repr(e))

arg1 = sys.argv[1]
maintenance_bloated_databases( arg1 )
    \end{lstlisting}
\end{flushleft}
\begin{flushleft}
    Zuerst muss der Namespace \texttt{ksgr-postgresql-maintenance} erstellt werden:
    \lstset{style=gra_codestyle}
    \begin{lstlisting}[language=bash, caption=Maintenance-Tool - Bloated Tables / Indices - Namespace,captionpos=b,label={lst:maintenannce-tool-bloated-tables-namespace},breaklines=true]
kubectl create namespace ksgr-postgresql-maintenance
    \end{lstlisting}
\end{flushleft}
\begin{flushleft}
    Das Python-Skript \texttt{ksgr\_postgresql\_maintenance\_bloated\_tables.py} kann in ein ConfigMap umgewandelt werden:
    \lstset{style=gra_codestyle}
    \begin{lstlisting}[language=bash, caption=Maintenance-Tool - Bloated Tables / Indices - Python > ConfigMap,captionpos=b,label={lst:maintenannce-tool-bloated-tables-python-to-configmap},breaklines=true]
kubectl create configmap ksgr-postgresql-maintenance-bloated-tables --from-file /home/gramic/PycharmProjects/ksgr_postgresql_maintenance/cleaned/ksgr_postgresql_maintenance_bloated_tables.py --dry-run=client -o yaml > /home/gramic/PycharmProjects/ksgr_postgresql_maintenance/cleaned/ksgr_postgresql_maintenance_bloated_tables.yml -n ksgr-postgresql-maintenance
    \end{lstlisting}
    Entsprechend lässt es sich deployen:
    \lstset{style=gra_codestyle}
    \begin{lstlisting}[language=bash, caption=Maintenance-Tool - Bloated Tables / Indices - Python - ConfigMap Deploy,captionpos=b,label={lst:maintenannce-tool-bloated-tables-python-configmap-deploy},breaklines=true]
kubectl create -f /home/gramic/PycharmProjects/ksgr_postgresql_maintenance/cleaned/ksgr_postgresql_maintenance_bloated_tables.yml -n ksgr-postgresql-maintenance
    \end{lstlisting}
\end{flushleft}
\begin{flushleft}
    Der Username und das Passwort für den Cluster wird als Secret gespeichert.\\
    Dazu werden erst die Daten in Base64 Strings umgewandelt:
    \lstset{style=gra_codestyle}
    \begin{lstlisting}[language=bash, caption=Maintenance-Tool - Bloated Tables / Indices - Base64,captionpos=b,label={lst:maintenannce-tool-bloated-tables-base64},breaklines=true]
USER=postgres
PASSWORD=<Password Secure / Safe>
echo -n ${USER} | base64
echo -n ${PASSWORD} | base64
    \end{lstlisting}
    Diese werden nun in ein Secret gegeben:
    \lstset{style=gra_codestyle}
    \begin{lstlisting}[language=yaml, caption=Maintenance-Tool - Bloated Tables / Indices - Secret,captionpos=b,label={lst:maintenannce-tool-bloated-tables-secret},breaklines=true]
apiVersion: v1
kind: Secret
metadata:
  name: secret-vks0041
  namespace: ksgr-postgresql-maintenance
type: Opaque
data:
  user: <Base64 User>
  password: <Base64 Password>
    \end{lstlisting}
    Das Secret kann deployt werden:
    \lstset{style=gra_codestyle}
    \begin{lstlisting}[language=yaml, caption=Maintenance-Tool - Bloated Tables / Indices - Secret Deploy,captionpos=b,label={lst:maintenannce-tool-bloated-tables-secret-deploy},breaklines=true]
kubectl create -f /home/gramic/PycharmProjects/ksgr_postgresql_maintenance/secret_vks0041.yaml
    \end{lstlisting}
\end{flushleft}
\begin{flushleft}

\end{flushleft}