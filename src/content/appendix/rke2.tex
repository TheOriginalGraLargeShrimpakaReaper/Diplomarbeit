%! Author = itgramic
%! Date = 26.01.24

% Preamble
\section{rke2}
\subsection{Vorbereitung}
\subsection{Installation}
\subsubsection{server}
Es gibt kein apt-Package.
Daher muss zuerst das tarball-Package heruntergeladen werden:
\lstset{style=gra_codestyle}
\begin{lstlisting}[language=bash, caption=Downlaod rke2,captionpos=b,label={lst:download-rke2},breaklines=true]
sudo curl -sfL https://get.rke2.io | sh -
\end{lstlisting}
Anschliessend muss das Package installiert werden:
\lstset{style=gra_codestyle}
\begin{lstlisting}[language=bash, caption=Downlaod rke2,captionpos=b,label={lst:download-rke2},breaklines=true]
sudo curl -sfL https://get.rke2.io | sh -
\end{lstlisting}


\subsubsection{agents}
\subsection{Cluster Konfiguration}
\subsubsection{server}
\subsubsection{agents}