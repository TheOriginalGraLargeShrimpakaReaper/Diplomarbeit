%! Author = itgramic
%! Date = 26.01.24

% Preamble
\section{rke2}
\subsection{Vorbereitung}
Da Package aus WAN-Repositories geladen werden, muss eine Proxy-Connection nach aussen gemacht werden können:
\lstset{style=gra_codestyle}
\begin{lstlisting}[language=bash, caption=Proxy Settings,captionpos=b,label={lst:proxy-settings},breaklines=true]
sudo nano /etc/profile.d/proxy.sh

export https_proxy=http://sproxy.sivc.first-it.ch:8080
export HTTPS_PROXY=http://sproxy.sivc.first-it.ch:8080
export http_proxy=http://sproxy.sivc.first-it.ch:8080
export HTTP_PROXY=http://sproxy.sivc.first-it.ch:8080
export no_proxy=localhost,127.0.0.0/8,::1,10.0.0.0/8,172.16.0.0/12,192.168.0.0/16
export NO_PROXY=localhost,127.0.0.0/8,::1,10.0.0.0/8,172.16.0.0/12,192.168.0.0/16

source /etc/profile.d/proxy.sh
\end{lstlisting}

\subsection{Installation}
\subsubsection{server}
Es gibt kein apt-Package.
Daher muss zuerst das tarball-Package heruntergeladen werden:
\lstset{style=gra_codestyle}
\begin{lstlisting}[language=bash, caption=Downlaod rke2 server,captionpos=b,label={lst:download-rke2-server},breaklines=true]
sudo curl -sfL https://get.rke2.io | sh -
\end{lstlisting}
Anschliessend muss das Package installiert werden:
\lstset{style=gra_codestyle}
\begin{lstlisting}[language=bash, caption=rke2 server installieren,captionpos=b,label={lst:install-rke2-server},breaklines=true]
sudo curl -sfL https://get.rke2.io | sh -
\end{lstlisting}

\subsubsection{agents}
Der Agent muss direkt heruntergeladen werden:
\lstset{style=gra_codestyle}
\begin{lstlisting}[language=bash, caption=Downlaod rke2 agent,captionpos=b,label={lst:download-rke2-agent},breaklines=true]
curl -sfL https://get.rke2.io | INSTALL_RKE2_TYPE="agent" sh -
\end{lstlisting}

Anschliessend muss der Dienst aktiviert werden:
\lstset{style=gra_codestyle}
\begin{lstlisting}[language=bash, caption=rke2 agent aktivieren,captionpos=b,label={lst:rke2-agent-enable},breaklines=true]
systemctl enable rke2-agent.service
\end{lstlisting}

\subsection{Cluster Konfiguration}
\subsubsection{server}
Auch für Kubernetes und die Pots müssen die Proxy-Einstellungen gemacht werden:
\lstset{style=gra_codestyle}
\begin{lstlisting}[language=bash, caption=rke2 server proxy,captionpos=b,label={lst:rke2-server-proxy},breaklines=true]
nano /etc/default/rke2-server
HTTPS_PROXY=http://sproxy.sivc.first-it.ch:8080
HTTP_PROXY=http://sproxy.sivc.first-it.ch:8080
NO_PROXY=localhost,127.0.0.0/8,::1,10.0.0.0/8,172.16.0.0/12,192.168.0.0/16

CONTAINERD_HTTPS_PROXY=http://sproxy.sivc.first-it.ch:8080
CONTAINERD_HTTP_PROXY=http://sproxy.sivc.first-it.ch:8080
CONTAINERD_NO_PROXY=localhost,127.0.0.0/8,::1,10.0.0.0/8,172.16.0.0/12,192.168.0.0/16
\end{lstlisting}

Dieses File muss entsprechend in das Homeverzeichnis gespeichert werden:
\lstset{style=gra_codestyle}
\begin{lstlisting}[language=bash, caption=rke2 server proxy kopieren,captionpos=b,label={lst:rke2-server-proxy-copy},breaklines=true]
mkdir /home/itgramic/.kube
cp /etc/rancher/rke2/rke2.yaml /home/itgramic/.kube/config
\end{lstlisting}

Für den Netzwerkteil muss nun Cilium installiert werden:
\lstset{style=gra_codestyle}
\begin{lstlisting}[language=bash, caption=rke2 server cilium installieren,captionpos=b,label={lst:rke2-server-cilium-install},breaklines=true]
nano /var/lib/rancher/rke2/server/manifests/rke2-cilium-config.yaml
---
apiVersion: helm.cattle.io/v1
kind: HelmChartConfig
metadata:
  name: rke2-cilium
  namespace: kube-system
spec:
  valuesContent: |-
    eni:
      enabled: true

\end{lstlisting}

Cilium muss nun aktiviert werden:
\begin{lstlisting}[language=bash, caption=rke2 server cilium aktivieren,captionpos=b,label={lst:rke2-server-cilium-apply},breaklines=true]
/var/lib/rancher/rke2/bin/kubectl apply -f /var/lib/rancher/rke2/server/manifests/rke2-cilium-config.yaml
\end{lstlisting}

Der rke2-Server muss nun mit der entsprechenden Config gestartet werden, anschliessend muss Cilium noch in die Conig und diese mittels Service reboot aktiviert werden:
\lstset{style=gra_codestyle}
\begin{lstlisting}[language=bash, caption=rke2 server starten,captionpos=b,label={lst:rke2-server-start},breaklines=true]
/var/lib/rancher/rke2/bin/kubectl cluster-info --kubeconfig /etc/rancher/rke2/rke2.yaml
nano /etc/rancher/rke2/config.yaml
cni:
- cilium

systemctl restart rke2-server.service
\end{lstlisting}

Entsprechend muss die Firewall gesetzt werden:
\lstset{style=gra_codestyle}
\begin{lstlisting}[language=bash, caption=iptables entries server,captionpos=b,label={lst:iptables-server-entries},breaklines=true]
nano /etc/iptables/rules.v4

# Generated by iptables-save v1.8.9 (nf_tables)
*filter
:INPUT DROP [0:0]
:FORWARD ACCEPT [0:0]
:OUTPUT ACCEPT [0:0]
-A INPUT -m state --state RELATED,ESTABLISHED -j ACCEPT
-A INPUT -p udp -m udp --sport 53 -j ACCEPT
-A INPUT -p icmp -j ACCEPT
-A INPUT -i lo -j ACCEPT
-A INPUT -s 10.0.0.0/8 -p tcp -m tcp --dport 22 -j ACCEPT
-A INPUT -s 10.0.9.115/32 -p udp -m udp --dport 161 -m comment --comment "Allow SNMP for probe 10.0.9.115" -j ACCEPT
-A INPUT -s 10.0.9.76/32 -p udp -m udp --dport 161 -m comment --comment "Allow SNMP for probe 10.0.9.76" -j ACCEPT
-A INPUT -s 10.0.36.147/32 -p udp -m udp --dport 161 -m comment --comment "Allow SNMP for probe 10.0.36.147" -j ACCEPT
-A INPUT -s 10.0.9.35/32 -p udp -m udp --dport 161 -m comment --comment "Allow SNMP for probe 10.0.9.35" -j ACCEPT
-A INPUT -s 10.0.9.37/32 -p udp -m udp --dport 161 -m comment --comment "Allow SNMP for probe 10.0.9.37" -j ACCEPT
-A INPUT -s 10.0.9.74/32 -p udp -m udp --dport 161 -m comment --comment "Allow SNMP for probe 10.0.9.74" -j ACCEPT
-A INPUT -s 10.0.9.75/32 -p udp -m udp --dport 161 -m comment --comment "Allow SNMP for probe 10.0.9.75" -j ACCEPT
-A INPUT -s 10.0.9.36/32 -p udp -m udp --dport 161 -m comment --comment "Allow SNMP for probe 10.0.9.36" -j ACCEPT
-A INPUT -s 10.0.9.14/32 -p udp -m udp --dport 161 -m comment --comment "Allow SNMP for probe 10.0.9.14" -j ACCEPT
-A INPUT -s 10.0.0.0/8 -p icmp -m icmp --icmp-type 8 -j ACCEPT
-A INPUT -s 10.0.0.0/8 -p tcp -m tcp --dport 6443 -j ACCEPT
-A INPUT -s 10.0.0.0/8 -p tcp -m tcp --dport 9345 -j ACCEPT
COMMIT
# Completed

systemctl restart iptables
\end{lstlisting}

Für den Connect der Agents muss noch ein Token generiert werden:
\begin{lstlisting}[language=bash, caption=rke2 server token,captionpos=b,label={lst:rke2-server-token},breaklines=true]
cni:
- cilium
token: <password safe>
\end{lstlisting}

\subsubsection{agents}

