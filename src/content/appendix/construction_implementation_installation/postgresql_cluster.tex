%! Author = gramic
%! Date = 09.05.24

% Preamble
\begin{flushleft}
    \subsection{Prequenteries}
    Auf allen Hosts müssen die Firewalls angepasst werden:
    \lstset{style=gra_codestyle}
    \begin{lstlisting}[language=bash, caption=Testsystem - Firewall Settings,captionpos=b,label={lst:testsystem-installation-firewall},breaklines=true]
# nano /etc/iptables/rules.v4
# Generated by iptables-save v1.8.9 (nf_tables)
*filter
:INPUT ACCEPT [0:0]
:FORWARD ACCEPT [0:0]
:OUTPUT ACCEPT [0:0]
-A INPUT -s 10.0.0.0/8 -p tcp -m tcp --dport 22 -j ACCEPT
-A INPUT -s 10.0.9.115/32 -p udp -m udp --dport 161 -m comment --comment "Allow SNMP for probe 10.0.9.115" -j ACCEPT
-A INPUT -s 10.0.9.76/32 -p udp -m udp --dport 161 -m comment --comment "Allow SNMP for probe 10.0.9.76" -j ACCEPT
-A INPUT -s 10.0.36.147/32 -p udp -m udp --dport 161 -m comment --comment "Allow SNMP for probe 10.0.36.147" -j ACCEPT
-A INPUT -s 10.0.9.35/32 -p udp -m udp --dport 161 -m comment --comment "Allow SNMP for probe 10.0.9.35" -j ACCEPT
-A INPUT -s 10.0.9.37/32 -p udp -m udp --dport 161 -m comment --comment "Allow SNMP for probe 10.0.9.37" -j ACCEPT
-A INPUT -s 10.0.9.74/32 -p udp -m udp --dport 161 -m comment --comment "Allow SNMP for probe 10.0.9.74" -j ACCEPT
-A INPUT -s 10.0.9.75/32 -p udp -m udp --dport 161 -m comment --comment "Allow SNMP for probe 10.0.9.75" -j ACCEPT
-A INPUT -s 10.0.9.36/32 -p udp -m udp --dport 161 -m comment --comment "Allow SNMP for probe 10.0.9.36" -j ACCEPT
-A INPUT -s 10.0.9.14/32 -p udp -m udp --dport 161 -m comment --comment "Allow SNMP for probe 10.0.9.14" -j ACCEPT
-A INPUT -s 10.0.0.0/8 -p icmp -m icmp --icmp-type 8 -j ACCEPT
# generell
-A INPUT -s 10.0.0.0/8 -p tcp -m tcp --dport 443 -j ACCEPT
-A INPUT -s 10.0.0.0/8 -p tcp -m tcp --dport 8080 -j ACCEPT
# postgres
-A INPUT -s 10.0.0.0/8 -p tcp -m tcp --dport 5432 -j ACCEPT
# pgbounder
-A INPUT -s 10.0.0.0/8 -p tcp -m tcp --dport 6432 -j ACCEPT
# etcd
-A INPUT -s 10.0.0.0/8 -p tcp -m tcp --dport 2379 -j ACCEPT
-A INPUT -s 10.0.0.0/8 -p tcp -m tcp --dport 2380 -j ACCEPT
# patroni
-A INPUT -s 10.0.0.0/8 -p tcp -m tcp --dport 8008 -j ACCEPT
# haproxy
-A INPUT -s 10.0.0.0/8 -p tcp -m tcp --dport 5000 -j ACCEPT
-A INPUT -s 10.0.0.0/8 -p tcp -m tcp --dport 5001 -j ACCEPT
-A INPUT -s 10.0.0.0/8 -p tcp -m tcp --dport 5002 -j ACCEPT
-A INPUT -s 10.0.0.0/8 -p tcp -m tcp --dport 5003 -j ACCEPT
-A INPUT -s 10.0.0.0/8 -p tcp -m tcp --dport 7000 -j ACCEPT
COMMIT
# Completed
# systemctl restart iptables
    \end{lstlisting}
    Als nächstes muss der Proxy gesetzt werden:
    \lstset{style=gra_codestyle}
    \begin{lstlisting}[language=bash, caption=Testsystem - Proxy Settings,captionpos=b,label={lst:testsystem-installation-proxy-settings},breaklines=true]
# nano /etc/profile.d/proxy.sh
export http_proxy="http://xksgr_vks0041_inet:<Password Secure / Safe>@webproxy.sivc.first-it.ch:9090";
export https_proxy="http://xksgr_vks0041_inet:<Password Secure / Safe>@webproxy.sivc.first-it.ch:9090";
export no_proxy="127.0.0.1,localhost,10.0.0.0/8,.ksgr.ch,.sivc.first-it.ch"

export HTTP_PROXY="http://xksgr_vks0041_inet:<Password Secure / Safe>@webproxy.sivc.first-it.ch:9090";
export HTTPS_PROXY="http://xksgr_vks0041_inet:<Password Secure / Safe>@webproxy.sivc.first-it.ch:9090";
export NO_PROXY="127.0.0.1,localhost,10.0.0.0/8,.ksgr.ch,.sivc.first-it.ch"
# source /etc/profile.d/proxy.sh
    \end{lstlisting}
    Auch der apt Proxy muss gesetzt werden:
    \lstset{style=gra_codestyle}
    \begin{lstlisting}[language=bash, caption=Testsystem - apt-Proxy Settings,captionpos=b,label={lst:testsystem-installation-apt-proxy-settings},breaklines=true]
# nano /etc/apt/apt.conf.d/proxy.conf
Acauire::http::Proxy "http://xksgr_vks0041_inet:<Password Secure / Safe>@webproxy.sivc.first-it.ch:9090";
Acauire::https::Proxy "http://xksgr_vks0041_inet<Password Secure / Safe>@webproxy.sivc.first-it.ch:9090";
Acquire::http::proxy::foreman.ksgr.ch "DIRECT";
    \end{lstlisting}
    \subsection{Deployment}
    Zuerst muss das \texttt{Inventory} angepasst hat werden:
    \lstset{style=gra_codestyle}
    \begin{lstlisting}[language=bash, caption=Testsystem - Deployment - inventory,captionpos=b,label={lst:testsystem-deployment-inventory},breaklines=true]
# if dcs_exists: false and dcs_type: "etcd"
[etcd_cluster]  # recommendation: 3, or 5-7 nodes
10.0.22.170
10.0.22.171
10.0.22.172

# if dcs_exists: false and dcs_type: "consul"
[consul_instances]  # recommendation: 3 or 5-7 nodes

# if with_haproxy_load_balancing: true
[balancers]
10.0.22.176
10.0.22.177

# PostgreSQL nodes
[master]
10.0.22.173 postgresql_exists=false

[replica]
10.0.22.174 postgresql_exists=false
10.0.22.175 postgresql_exists=false

[postgres_cluster:children]
master
replica

# if pgbackrest_install: true and "repo_host" is set
[pgbackrest]  # optional (Dedicated Repository Host)


# Connection settings
[all:vars]
ansible_connection='ssh'
ansible_ssh_port='22'
ansible_user='itgramic'
ansible_ssh_pass='<Secret ;-)>'  # "sshpass" package is required for use "ansible_ssh_pass"
#ansible_ssh_private_key_file=
ansible_python_interpreter='/usr/bin/env python3'

[pgbackrest:vars]

    \end{lstlisting}
    Nun muss das \texttt{main.yml} dran.
    Wichtig sind folgende einstellungen.
    \lstset{style=gra_codestyle}
    \begin{lstlisting}[language=yaml, caption=Testsystem - Deployment - main.yml,captionpos=b,label={lst:testsystem-deployment-main.yml},breaklines=true]
---
# ---------------------------------------------------------------------
# Proxy variables (optional) for download packages using a proxy server
proxy_env: {}  # yamllint disable rule:braces
# ---------------------------------------------------------------------
#proxy_env:
#  http_proxy: "http://xksgr_vks0041_inet:<Password Secure / Safe>@webproxy.sivc.first-it.ch:9090"
#  https_proxy: "http://xksgr_vks0041_inet:<Password Secure / Safe>@webproxy.sivc.first-it.ch:9090"

# Cluster variables
#cluster_vip: ""  # IP address for client access to the databases in the cluster (optional).
cluster_vip: "10.0.22.178"  # IP address for client access to the databases in the cluster (optional).
vip_interface: "{{ ansible_default_ipv4.interface }}"  # interface name (e.g., "ens32").
# Note: VIP-based solutions such as keepalived or vip-manager may not function correctly in cloud environments like AWS.

#patroni_cluster_name: "postgres-cluster"  # the cluster name (must be unique for each cluster)
patroni_cluster_name: "k8s-core-psql"  # the cluster name (must be unique for each cluster)
patroni_install_version: "3.3.0"  # or 'latest'

patroni_superuser_username: "postgres"
patroni_superuser_password: "<Password Secure / Safe>"
patroni_replication_username: "replicator"
patroni_replication_password: "<Password Secure / Safe>"

#synchronous_mode: false  # or 'true' for enable synchronous database replication
synchronous_mode: true  # or 'true' for enable synchronous database replication
synchronous_mode_strict: false  # if 'true' then block all client writes to the master, when a synchronous replica is not available
#synchronous_node_count: 1  # number of synchronous standby databases
synchronous_node_count: 2  # number of synchronous standby databases

# Load Balancing
with_haproxy_load_balancing: true  # or 'true' if you want to install and configure the load-balancing
haproxy_listen_port:
  master: 5000
  replicas: 5001
  replicas_sync: 5002
  replicas_async: 5003
# The following ('_direct') ports are used for direct connections to the PostgreSQL database,
# bypassing the PgBouncer connection pool (if 'pgbouncer_install' is 'true').
# Uncomment the relevant lines if you need to set up direct connections.
  stats: 7000
haproxy_maxconn:
  global: 100000
  master: 10000
  replica: 10000
haproxy_timeout:
  client: "60m"
  server: "60m"

# keepalived (if 'cluster_vip' is specified and 'with_haproxy_load_balancing' is 'true')
keepalived_virtual_router_id: "{{ cluster_vip.split('.')[3] | int }}" # The last octet of 'cluster_vip' IP address is used by default.
# virtual_router_id - must be unique in the network (available values are 0..255).

# vip-manager (if 'cluster_vip' is specified and 'with_haproxy_load_balancing' is 'false')
vip_manager_version: "2.4.0"  # version to install
vip_manager_conf: "/etc/patroni/vip-manager.yml"
vip_manager_interval: "1000"  # time (in milliseconds) after which vip-manager wakes up and checks if it needs to register or release ip addresses.
vip_manager_iface: "{{ vip_interface }}"  # interface to which the virtual ip will be added
vip_manager_ip: "{{ cluster_vip }}"  # the virtual ip address to manage
vip_manager_mask: "24"  # netmask for the virtual ip


# DCS (Distributed Consensus Store)
dcs_exists: false  # or 'true' if you don't want to deploy a new etcd cluster
dcs_type: "etcd"  # or 'consul'

# if dcs_type: "etcd" and dcs_exists: false
etcd_version: "3.5.11"  # version for deploy etcd cluster
etcd_data_dir: "/var/lib/etcd"
etcd_cluster_name: "etcd-{{ patroni_cluster_name }}"  # ETCD_INITIAL_CLUSTER_TOKEN

# if dcs_type: "etcd" and dcs_exists: true
patroni_etcd_hosts: []  # list of servers of an existing etcd cluster

patroni_etcd_namespace: "service"  # (optional) etcd namespace (prefix)
patroni_etcd_username: "" # (optional) username for etcd authentication
patroni_etcd_password: "" # (optional) password for etcd authentication
patroni_etcd_protocol: "" # (optional) http or https, if not specified http is used

# if dcs_type: "consul"
consul_version: "1.15.8"
consul_config_path: "/etc/consul"
consul_configd_path: "{{ consul_config_path }}/conf.d"
consul_data_path: "/var/lib/consul"
consul_domain: "consul"  # Consul domain name
consul_datacenter: "dc1"  # Datacenter label (can be specified for each host in the inventory)
consul_disable_update_check: true  # Disables automatic checking for security bulletins and new version releases
consul_enable_script_checks: true  # This controls whether health checks that execute scripts are enabled on this agent
consul_enable_local_script_checks: true  # Enable them when they are defined in the local configuration files
consul_ui: false  # Enable the consul UI?
consul_syslog_enable: true  # Enable logging to syslog
consul_iface: "{{ ansible_default_ipv4.interface }}"  # specify the interface name with a Private IP (ex. "enp7s0")
# TLS
# You can enable TLS encryption by dropping a CA certificate, server certificate, and server key in roles/consul/files/
consul_tls_enable: false
consul_tls_ca_crt: "ca.crt"
consul_tls_server_crt: "server.crt"
consul_tls_server_key: "server.key"
# DNS
consul_recursors: []  # List of upstream DNS servers
consul_dnsmasq_enable: true  # Enable DNS forwarding with Dnsmasq
consul_dnsmasq_cache: 0  # dnsmasq cache-size (0 - disable caching)
consul_dnsmasq_servers: "{{ nameservers }}" # Upstream DNS servers used by dnsmasq
consul_join: []  # List of LAN servers of an existing consul cluster, to join.

# https://developer.hashicorp.com/consul/docs/discovery/services
consul_services:
  - name: "{{ patroni_cluster_name }}"
    id: "{{ patroni_cluster_name }}-master"
    tags: ['master', 'primary']
    port: "{{ pgbouncer_listen_port }}"  # or "{{ postgresql_port }}" if pgbouncer_install: false
    checks:
      - { http: "http://{{ inventory_hostname }}:{{ patroni_restapi_port }}/primary", interval: "2s" }
      - { args: ["systemctl", "status", "pgbouncer"], interval: "5s" }  # comment out this check if pgbouncer_install: false
  - name: "{{ patroni_cluster_name }}"
    id: "{{ patroni_cluster_name }}-replica"
    tags: ['replica']
    port: "{{ pgbouncer_listen_port }}"
    checks:
      - { http: "http://{{ inventory_hostname }}:{{ patroni_restapi_port }}/replica?lag={{ patroni_maximum_lag_on_replica }}", interval: "2s" }
      - { args: ["systemctl", "status", "pgbouncer"], interval: "5s" }

# PostgreSQL variables
postgresql_version: "16"
# postgresql_data_dir: see vars/Debian.yml or vars/RedHat.yml
postgresql_listen_addr: "0.0.0.0" # Listen on all interfaces. Or use "{{ inventory_hostname }},127.0.0.1" to listen on a specific IP address.
postgresql_port: "5432"
postgresql_encoding: "UTF8"  # for bootstrap only (initdb)
postgresql_locale: "en_US.UTF-8"  # for bootstrap only (initdb)
postgresql_data_checksums: true  # for bootstrap only (initdb)
postgresql_password_encryption_algorithm: "scram-sha-256"  # or "md5" if your clients do not work with passwords encrypted with SCRAM-SHA-256

# (optional) list of users to be created (if not already exists)
postgresql_users:
  - { name: "{{ pgbouncer_auth_username }}", password: "{{ pgbouncer_auth_password }}", flags: "LOGIN", role: "" }
  - { name: "xksgr_sks1160_gitlab", password: "<Password Secure / Safe>", flags: "SUPERUSER" }
  - { name: "xksgr_sks1172_harbor", password: "<Password Secure / Safe>", flags: "SUPERUSER" }
  - { name: "xksgr_sks1195_kcsso", password: "<Password Secure / Safe>", flags: "SUPERUSER" }
  - { name: "xksgr_k8s_core_psql_monitor", password: "<Password Secure / Safe>", flags: "LOGIN", role: "pg_monitor" }
  - { name: "xksgr_k8s_core_psql_backup", password: "<Password Secure / Safe>", flags: "SUPERUSER" }

# (optional) list of databases to be created (if not already exists)
#postgresql_databases: []
postgresql_databases:
  - { db: "k8s_core_gitlab_prod", encoding: "UTF8", lc_collate: "en_US.UTF-8", lc_ctype: "en_US.UTF-8", owner: "xksgr_sks1160_gitlab" }
  - { db: "k8s_core_harbor_prod", encoding: "UTF8", lc_collate: "en_US.UTF-8", lc_ctype: "en_US.UTF-8", owner: "xksgr_sks1172_harbor" }
  - { db: "k8s_core_keycloak_prod", encoding: "UTF8", lc_collate: "en_US.UTF-8", lc_ctype: "en_US.UTF-8", owner: "xksgr_sks1195_kcsso" }
  - { db: "gramic_test", encoding: "UTF8", lc_collate: "en_US.UTF-8", lc_ctype: "en_US.UTF-8", owner: "postgres" }

# (optional) list of schemas to be created (if not already exists)
postgresql_schemas: []
#  - { schema: "myschema", db: "mydatabase", owner: "mydb-user" }

# (optional) list of database extensions to be created (if not already exists)
#postgresql_extensions: []
postgresql_extensions:
  - { ext: "pgstattuple", db: "postgres" }
  - { ext: "pgstattuple", db: "k8s_core_gitlab_prod" }
  - { ext: "pgstattuple", db: "k8s_core_harbor_prod" }
  - { ext: "pgstattuple", db: "k8s_core_keycloak_prod" }
  - { ext: "pgstattuple", db: "gramic_test" }

# postgresql parameters to bootstrap dcs (are parameters for example)
postgresql_parameters:
  - { option: "max_connections", value: "500" }
  - { option: "superuser_reserved_connections", value: "5" }
  - { option: "password_encryption", value: "{{ postgresql_password_encryption_algorithm }}" }
  - { option: "max_locks_per_transaction", value: "512" }
  - { option: "max_prepared_transactions", value: "0" }
  - { option: "huge_pages", value: "try" }  # or "on" if you set "vm_nr_hugepages" in kernel parameters
  - { option: "shared_buffers", value: "{{ (ansible_memtotal_mb * 0.25) | int }}MB" }  # by default, 25% of RAM
  - { option: "effective_cache_size", value: "{{ (ansible_memtotal_mb * 0.75) | int }}MB" }  # by default, 75% of RAM
  - { option: "work_mem", value: "128MB" }  # please change this value
  - { option: "maintenance_work_mem", value: "256MB" }  # please change this value
  - { option: "checkpoint_timeout", value: "15min" }
  - { option: "checkpoint_completion_target", value: "0.9" }
  - { option: "min_wal_size", value: "2GB" }
  - { option: "max_wal_size", value: "8GB" }  # or 16GB/32GB
  - { option: "wal_buffers", value: "32MB" }
  - { option: "default_statistics_target", value: "1000" }
  - { option: "seq_page_cost", value: "1" }
  - { option: "random_page_cost", value: "1.1" }  # or "4" for HDDs with slower random access
  - { option: "effective_io_concurrency", value: "200" }  # or "2" for traditional HDDs with lower I/O parallelism
  - { option: "synchronous_commit", value: "on" }  # or 'off' if you can you lose single transactions in case of a crash
  - { option: "autovacuum", value: "on" }  # never turn off the autovacuum!
  - { option: "autovacuum_max_workers", value: "5" }
  - { option: "autovacuum_vacuum_scale_factor", value: "0.01" }  # or 0.005/0.001
  - { option: "autovacuum_analyze_scale_factor", value: "0.01" }
  - { option: "autovacuum_vacuum_cost_limit", value: "500" }  # or 1000/5000
  - { option: "autovacuum_vacuum_cost_delay", value: "2" }
  - { option: "autovacuum_naptime", value: "1s" }
  - { option: "max_files_per_process", value: "4096" }
  - { option: "archive_mode", value: "on" }
  - { option: "archive_timeout", value: "1800s" }
  - { option: "archive_command", value: "cd ." }  # not doing anything yet with WAL-s
  - { option: "wal_level", value: "logical" }
  - { option: "wal_keep_size", value: "2GB" }
  - { option: "max_wal_senders", value: "10" }
  - { option: "max_replication_slots", value: "10" }
  - { option: "hot_standby", value: "on" }
  - { option: "wal_log_hints", value: "on" }
  - { option: "wal_compression", value: "on" }
  - { option: "shared_preload_libraries", value: "pg_stat_statements,auto_explain" }
  - { option: "pg_stat_statements.max", value: "10000" }
  - { option: "pg_stat_statements.track", value: "all" }
  - { option: "pg_stat_statements.track_utility", value: "false" }
  - { option: "pg_stat_statements.save", value: "true" }
  - { option: "auto_explain.log_min_duration", value: "10s" }  # enable auto_explain for 10-second logging threshold. Decrease this value if necessary
  - { option: "auto_explain.log_analyze", value: "true" }
  - { option: "auto_explain.log_buffers", value: "true" }
  - { option: "auto_explain.log_timing", value: "false" }
  - { option: "auto_explain.log_triggers", value: "true" }
  - { option: "auto_explain.log_verbose", value: "true" }
  - { option: "auto_explain.log_nested_statements", value: "true" }
  - { option: "auto_explain.sample_rate", value: "0.01" }  # enable auto_explain for 1% of queries logging threshold
  - { option: "track_io_timing", value: "on" }
  - { option: "log_lock_waits", value: "on" }
  - { option: "log_temp_files", value: "0" }
  - { option: "track_activities", value: "on" }
  - { option: "track_activity_query_size", value: "4096" }
  - { option: "track_counts", value: "on" }
  - { option: "track_functions", value: "all" }
  - { option: "log_checkpoints", value: "on" }
  - { option: "logging_collector", value: "on" }
  - { option: "log_truncate_on_rotation", value: "on" }
  - { option: "log_rotation_age", value: "1d" }
  - { option: "log_rotation_size", value: "0" }
  - { option: "log_line_prefix", value: "'%t [%p-%l] %r %q%u@%d '" }
  - { option: "log_filename", value: "postgresql-%a.log" }
  - { option: "log_directory", value: "{{ postgresql_log_dir }}" }
  - { option: "hot_standby_feedback", value: "on" }  # allows feedback from a hot standby to the primary that will avoid query conflicts
  - { option: "max_standby_streaming_delay", value: "30s" }
  - { option: "wal_receiver_status_interval", value: "10s" }
  - { option: "idle_in_transaction_session_timeout", value: "10min" }  # reduce this timeout if possible
  - { option: "jit", value: "off" }
  - { option: "max_worker_processes", value: "24" }
  - { option: "max_parallel_workers", value: "8" }
  - { option: "max_parallel_workers_per_gather", value: "2" }
  - { option: "max_parallel_maintenance_workers", value: "2" }
  - { option: "tcp_keepalives_count", value: "10" }
  - { option: "tcp_keepalives_idle", value: "300" }
  - { option: "tcp_keepalives_interval", value: "30" }

# Set this variable to 'true' if you want the cluster to be automatically restarted
# after changing the 'postgresql_parameters' variable that requires a restart in the 'config_pgcluster.yml' playbook.
# By default, the cluster will not be automatically restarted.
pending_restart: false

# specify additional hosts that will be added to the pg_hba.conf
postgresql_pg_hba:
  - { type: "local", database: "all", user: "{{ patroni_superuser_username }}", address: "", method: "trust" }
  - { type: "local", database: "all", user: "{{ pgbouncer_auth_username }}", address: "", method: "trust" } # required for pgbouncer auth_user
  - { type: "local", database: "replication", user: "all", address: "", method: "trust" }
  - { type: "host", database: "replication", user: "all", address: "127.0.0.1/32", method: "trust" }
  - { type: "host", database: "replication", user: "all", address: "::1/128", method: "trust" }
  - { type: "host", database: "replication", user: "{{ patroni_replication_username }}", address: "10.0.22.173/24", method: "{{ postgresql_password_encryption_algorithm }}" }
  - { type: "host", database: "replication", user: "{{ patroni_replication_username }}", address: "10.0.22.174/24", method: "{{ postgresql_password_encryption_algorithm }}" }
  - { type: "host", database: "replication", user: "{{ patroni_replication_username }}", address: "10.0.22.175/24", method: "{{ postgresql_password_encryption_algorithm }}" }
  - { type: "local", database: "all", user: "all", address: "", method: "{{ postgresql_password_encryption_algorithm }}" }
  - { type: "host", database: "all", user: "all", address: "127.0.0.1/32", method: "{{ postgresql_password_encryption_algorithm }}" }
  - { type: "host", database: "all", user: "all", address: "::1/128", method: "{{ postgresql_password_encryption_algorithm }}" }
  - { type: "host", database: "k8s_core_gitlab_prod", user: "xksgr_sks1160_gitlab", address: "10.0.20.88/24", method: "{{ postgresql_password_encryption_algorithm }}" }
  - { type: "host", database: "k8s_core_harbor_prod", user: "xksgr_sks1172_harbor", address: "10.0.20.92/24", method: "{{ postgresql_password_encryption_algorithm }}" }
  - { type: "host", database: "k8s_core_keycloak_prod", user: "xksgr_sks1195_kcsso", address: "10.0.20.98/24", method: "{{ postgresql_password_encryption_algorithm }}" }
  - { type: "host", database: "all", user: "{{ patroni_superuser_username }}", address: "10.0.20.63/24", method: "{{ postgresql_password_encryption_algorithm }}" }
  - { type: "host", database: "all", user: "{{ patroni_superuser_username }}", address: "10.0.20.43/24", method: "{{ postgresql_password_encryption_algorithm }}" }
  - { type: "host", database: "all", user: "{{ patroni_superuser_username }}", address: "10.0.20.77/24", method: "{{ postgresql_password_encryption_algorithm }}" }

# list of lines that Patroni will use to generate pg_ident.conf
postgresql_pg_ident: []

# the password file (~/.pgpass)
postgresql_pgpass:
  - "localhost:{{ postgresql_port }}:*:{{ patroni_superuser_username }}:{{ patroni_superuser_password }}"
  - "{{ inventory_hostname }}:{{ postgresql_port }}:*:{{ patroni_superuser_username }}:{{ patroni_superuser_password }}"
  - "*:{{ pgbouncer_listen_port }}:*:{{ patroni_superuser_username }}:{{ patroni_superuser_password }}"
  - "*:{{ postgresql_port }}:*:{{ patroni_replication_username }}:{{ patroni_replication_password }}"
  - "10.0.20.88:5432:k8s_core_gitlab_prod:xksgr_sks1160_gitlab:<Password Secure / Safe>"
  - "10.0.20.92:5432:k8s_core_harbor_prod:xksgr_sks1172_harbor:j<Password Secure / Safe>"
  - "10.0.20.98:5432:k8s_core_keycloak_prod:xksgr_sks1195_kcsso:<Password Secure / Safe>"

# PgBouncer parameters
pgbouncer_install: true  # or 'false' if you do not want to install and configure the pgbouncer service
pgbouncer_processes: 1  # Number of pgbouncer processes to be used. Multiple processes use the so_reuseport option for better performance.
pgbouncer_conf_dir: "/etc/pgbouncer"
pgbouncer_log_dir: "/var/log/pgbouncer"
pgbouncer_listen_addr: "0.0.0.0" # Listen on all interfaces. Or use "{{ inventory_hostname }}" to listen on a specific IP address.
pgbouncer_listen_port: 6432
pgbouncer_max_client_conn: 10000
pgbouncer_max_db_connections: 1000
pgbouncer_max_prepared_statements: 1024
pgbouncer_default_pool_size: 20
pgbouncer_query_wait_timeout: 120
pgbouncer_default_pool_mode: "session"
pgbouncer_admin_users: "{{ patroni_superuser_username }}"  # comma-separated list of users, who are allowed to change settings
pgbouncer_stats_users: "{{ patroni_superuser_username }}"  # comma-separated list of users who are just allowed to use SHOW command
pgbouncer_ignore_startup_parameters: "extra_float_digits,geqo,search_path"
pgbouncer_auth_type: "{{ postgresql_password_encryption_algorithm }}"
pgbouncer_auth_user: true # or 'false' if you want to manage the list of users for authentication in the database via userlist.txt
pgbouncer_auth_username: pgbouncer # user who can query the database via the user_search function
pgbouncer_auth_password: "<Password Secure / Safe>"
pgbouncer_auth_dbname: "postgres"
pgbouncer_client_tls_sslmode: "disable"
pgbouncer_client_tls_key_file: ""
pgbouncer_client_tls_cert_file: ""
pgbouncer_client_tls_ca_file: ""
pgbouncer_client_tls_protocols: "secure" # allowed values: tlsv1.0, tlsv1.1, tlsv1.2, tlsv1.3, all, secure (tlsv1.2,tlsv1.3)
pgbouncer_client_tls_ciphers: "default" # allowed values: default, secure, fast, normal, all (not recommended)

pgbouncer_pools:
  - { name: "postgres", dbname: "postgres", pool_parameters: "" }
  - { name: "k8s_core_gitlab_prod", dbname: "k8s_core_gitlab_prod", pool_parameters: "" }
  - { name: "k8s_core_harbor_prod", dbname: "k8s_core_harbor_prod", pool_parameters: "" }
  - { name: "k8s_core_keycloak_prod", dbname: "k8s_core_keycloak_prod", pool_parameters: "" }
  - { name: "gramic_test", dbname: "gramic_test", pool_parameters: "" }

# Extended variables (optional)
patroni_restapi_listen_addr: "0.0.0.0" # Listen on all interfaces. Or use "{{ inventory_hostname }}" to listen on a specific IP address.
patroni_restapi_port: 8008
patroni_ttl: 30
patroni_loop_wait: 10
patroni_retry_timeout: 10
patroni_master_start_timeout: 300
patroni_maximum_lag_on_failover: 1048576 # (1MB) the maximum bytes a follower may lag to be able to participate in leader election.
patroni_maximum_lag_on_replica: "100MB" # the maximum of lag that replica can be in order to be available for read-only queries.

# https://patroni.readthedocs.io/en/latest/yaml_configuration.html#postgresql
patroni_callbacks: []

# https://patroni.readthedocs.io/en/latest/replica_bootstrap.html#standby-cluster
# Requirements:
# 1. the cluster name for Standby Cluster must be unique ('patroni_cluster_name' variable)
# 2. the IP addresses (or network) of the Standby Cluster servers must be added to the pg_hba.conf of the Main Cluster ('postgresql_pg_hba' variable).
patroni_standby_cluster:
  host: ""  # an address of remote master
  port: "5432"  # a port of remote master

# Permanent replication slots.
# These slots will be preserved during switchover/failover.
# https://patroni.readthedocs.io/en/latest/dynamic_configuration.html
patroni_slots: []

patroni_log_destination: stderr  # or 'logfile'
# if patroni_log_destination: logfile
patroni_log_dir: /var/log/patroni
patroni_log_level: info
patroni_log_traceback_level: error
patroni_log_format: "%(asctime)s %(levelname)s: %(message)s"
patroni_log_dateformat: ""
patroni_log_max_queue_size: 1000
patroni_log_file_num: 4
patroni_log_file_size: 25000000  # bytes
patroni_log_loggers_patroni_postmaster: warning
patroni_log_loggers_urllib3: warning  # or 'debug'

patroni_watchdog_mode: automatic  # or 'off', 'required'
patroni_watchdog_device: /dev/watchdog

patroni_postgresql_use_pg_rewind: true  # or 'false'
# try to use pg_rewind on the former leader when it joins cluster as a replica.

patroni_remove_data_directory_on_rewind_failure: false  # or 'true' (if use_pg_rewind: 'true')
# avoid removing the data directory on an unsuccessful rewind
# if 'true', Patroni will remove the PostgreSQL data directory and recreate the replica.

patroni_remove_data_directory_on_diverged_timelines: false  # or 'true'
# if 'true', Patroni will remove the PostgreSQL data directory and recreate the replica
# if it notices that timelines are diverging and the former master can not start streaming from the new master.

# https://patroni.readthedocs.io/en/latest/replica_bootstrap.html#bootstrap
patroni_cluster_bootstrap_method: "initdb"  # or "wal-g", "pgbackrest", "pg_probackup"

# https://patroni.readthedocs.io/en/latest/replica_bootstrap.html#building-replicas
patroni_create_replica_methods:
  - basebackup

pgbackrest:
  - { option: "command", value: "/usr/bin/pgbackrest --stanza={{ pgbackrest_stanza }} --delta restore" }
  - { option: "keep_data", value: "True" }
  - { option: "no_params", value: "True" }
wal_g:
  - { option: "command", value: "wal-g backup-fetch {{ postgresql_data_dir }} LATEST" }
  - { option: "no_params", value: "True" }
basebackup:
  - { option: "max-rate", value: "100M" }
  - { option: "checkpoint", value: "fast" }
#  - { option: "waldir", value: "{{ postgresql_wal_dir }}" }
pg_probackup:
  - { option: "command", value: "{{ pg_probackup_restore_command }}" }
  - { option: "no_params", value: "true" }

# "restore_command" written to recovery.conf when configuring follower (create replica)
postgresql_restore_command: ""

# pg_probackup
pg_probackup_install: false  # or 'true'
pg_probackup_install_from_postgrespro_repo: true  # or 'false'
pg_probackup_version: "{{ postgresql_version }}"
pg_probackup_instance: "pg_probackup_instance_name"
pg_probackup_dir: "/mnt/backup_dir"
pg_probackup_threads: "4"
pg_probackup_add_keys: "--recovery-target=latest --skip-external-dirs --no-validate"
# Ensure there is a space at the beginning of each part to prevent commands from concatenating.
pg_probackup_command_parts:
  - "pg_probackup-{{ pg_probackup_version }}"
  - " restore -B {{ pg_probackup_dir }}"
  - " --instance {{ pg_probackup_instance }}"
  - " -j {{ pg_probackup_threads }}"
  - " {{ pg_probackup_add_keys }}"
pg_probackup_restore_command: "{{ pg_probackup_command_parts | join('') }}"
pg_probackup_patroni_cluster_bootstrap_command: "{{ pg_probackup_command_parts | join('') }}"

# WAL-G
wal_g_install: false  # or 'true'
wal_g_version: "3.0.0"
wal_g_json:  # config https://github.com/wal-g/wal-g#configuration
  - { option: "AWS_ACCESS_KEY_ID", value: "{{ AWS_ACCESS_KEY_ID | default('') }}" }  # define values or pass via --extra-vars
  - { option: "AWS_SECRET_ACCESS_KEY", value: "{{ AWS_SECRET_ACCESS_KEY | default('') }}" }  # define values or pass via --extra-vars
  - { option: "WALG_S3_PREFIX", value: "{{ WALG_S3_PREFIX | default('') }}" } # define values or pass via --extra-vars
  - { option: "WALG_COMPRESSION_METHOD", value: "brotli" }  # or "lz4", "lzma", "zstd"
  - { option: "WALG_DELTA_MAX_STEPS", value: "6" }  # determines how many delta backups can be between full backups
  - { option: "PGDATA", value: "{{ postgresql_data_dir }}" }
  - { option: "PGHOST", value: "{{ postgresql_unix_socket_dir }}" }
  - { option: "PGPORT", value: "{{ postgresql_port }}" }
  - { option: "PGUSER", value: "{{ patroni_superuser_username }}" }
#  - { option: "AWS_S3_FORCE_PATH_STYLE", value: "true" }  # to use Minio.io S3-compatible storage
#  - { option: "AWS_ENDPOINT", value: "http://minio:9000" }  # to use Minio.io S3-compatible storage
#  - { option: "", value: "" }
wal_g_archive_command: "wal-g wal-push %p"
wal_g_patroni_cluster_bootstrap_command: "wal-g backup-fetch {{ postgresql_data_dir }} LATEST"

# Define job_parts outside of wal_g_cron_jobs
# Ensure there is a space at the beginning of each part to prevent commands from concatenating.
wal_g_backup_command:
  - "[ $(curl -s -o /dev/null -w '%{http_code}' http://{{ inventory_hostname }}:{{ patroni_restapi_port }}) = '200' ]"
  - " && wal-g backup-push {{ postgresql_data_dir }} > {{ postgresql_log_dir }}/walg_backup.log 2>&1"
wal_g_delete_command:
  - "[ $(curl -s -o /dev/null -w '%{http_code}' http://{{ inventory_hostname }}:{{ patroni_restapi_port }}) = '200' ]"
  - " && wal-g delete retain FULL 4 --confirm > {{ postgresql_log_dir }}/walg_delete.log 2>&1"

wal_g_cron_jobs:
  - name: "WAL-G: Create daily backup"
    user: "postgres"
    file: /etc/cron.d/walg
    minute: "30"
    hour: "3"
    day: "*"
    month: "*"
    weekday: "*"
    job: "{{ wal_g_backup_command | join('') }}"
  - name: "WAL-G: Delete old backups" # retain 4 full backups (adjust according to your company's backup retention policy)
    user: "postgres"
    file: /etc/cron.d/walg
    minute: "30"
    hour: "6"
    day: "*"
    month: "*"
    weekday: "*"
    job: "{{ wal_g_delete_command | join('') }}"

# pgBackRest
pgbackrest_install: false  # or 'true'
pgbackrest_install_from_pgdg_repo: false  # or 'false'
pgbackrest_stanza: "{{ patroni_cluster_name }}"  # specify your --stanza
pgbackrest_repo_type: "posix"  # or "s3", "gcs", "azure"
pgbackrest_repo_host: ""  # dedicated repository host (optional)
pgbackrest_repo_user: "postgres"
pgbackrest_conf_file: "/etc/pgbackrest/pgbackrest.conf"
# config https://pgbackrest.org/configuration.html
pgbackrest_conf:
  global:  # [global] section
    - { option: "log-level-file", value: "detail" }
    - { option: "log-path", value: "/var/log/pgbackrest" }
    - { option: "repo1-type", value: "{{ pgbackrest_repo_type | lower }}" }
    - { option: "repo1-path", value: "/var/lib/pgbackrest" }
    - { option: "repo1-retention-full", value: "4" }
    - { option: "repo1-retention-archive", value: "4" }
    - { option: "start-fast", value: "y" }
    - { option: "stop-auto", value: "y" }
    - { option: "resume", value: "n" }
    - { option: "link-all", value: "y" }
    - { option: "spool-path", value: "/var/spool/pgbackrest" }
    - { option: "archive-async", value: "y" } # Enables asynchronous WAL archiving (details: https://pgbackrest.org/user-guide.html#async-archiving)
    - { option: "archive-get-queue-max", value: "1GiB" }
  stanza:  # [stanza_name] section
    - { option: "process-max", value: "4" }
    - { option: "log-level-console", value: "info" }
    - { option: "recovery-option", value: "recovery_target_action=promote" }
    - { option: "pg1-socket-path", value: "{{ postgresql_unix_socket_dir }}" }
    - { option: "pg1-path", value: "{{ postgresql_data_dir }}" }
#    - { option: "", value: "" }
# (optional) dedicated backup server config (if "repo_host" is set)
pgbackrest_server_conf:
  global:
    - { option: "log-level-file", value: "detail" }
    - { option: "log-level-console", value: "info" }
    - { option: "log-path", value: "/var/log/pgbackrest" }
    - { option: "repo1-type", value: "{{ pgbackrest_repo_type | lower }}" }
    - { option: "repo1-path", value: "/var/lib/pgbackrest" }
    - { option: "repo1-retention-full", value: "4" }
    - { option: "repo1-retention-archive", value: "4" }
    - { option: "repo1-bundle", value: "y" }
    - { option: "repo1-block", value: "y" }
    - { option: "start-fast", value: "y" }
    - { option: "stop-auto", value: "y" }
    - { option: "resume", value: "n" }
    - { option: "link-all", value: "y" }
    - { option: "archive-check", value: "y" }
    - { option: "archive-copy", value: "n" }
    - { option: "backup-standby", value: "y" }
#    - { option: "", value: "" }
# the stanza section will be generated automatically

pgbackrest_archive_command: "pgbackrest --stanza={{ pgbackrest_stanza }} archive-push %p"

pgbackrest_patroni_cluster_restore_command:
  '/usr/bin/pgbackrest --stanza={{ pgbackrest_stanza }} --delta restore'  # restore from latest backup
#  '/usr/bin/pgbackrest --stanza={{ pgbackrest_stanza }} --type=time "--target=2020-06-01 11:00:00+03" --delta restore'  # Point-in-Time Recovery (example)

# By default, the cron jobs is created on the database server.
# If 'repo_host' is defined, the cron jobs will be created on the pgbackrest server.
pgbackrest_cron_jobs:
  - name: "pgBackRest: Full Backup"
    file: "/etc/cron.d/pgbackrest-{{ patroni_cluster_name }}"
    user: "postgres"
    minute: "30"
    hour: "6"
    day: "*"
    month: "*"
    weekday: "0"
    job: "pgbackrest --type=full --stanza={{ pgbackrest_stanza }} backup"
    # job: "if [ $(psql -tAXc 'select pg_is_in_recovery()') = 'f' ]; then pgbackrest --type=full --stanza={{ pgbackrest_stanza }} backup; fi"
  - name: "pgBackRest: Diff Backup"
    file: "/etc/cron.d/pgbackrest-{{ patroni_cluster_name }}"
    user: "postgres"
    minute: "30"
    hour: "6"
    day: "*"
    month: "*"
    weekday: "1-6"
    job: "pgbackrest --type=diff --stanza={{ pgbackrest_stanza }} backup"
    # job: "if [ $(psql -tAXc 'select pg_is_in_recovery()') = 'f' ]; then pgbackrest --type=diff --stanza={{ pgbackrest_stanza }} backup; fi"

# PITR mode (if patroni_cluster_bootstrap_method: "pgbackrest" or "wal-g"):
# 1) The database cluster directory will be cleaned (for "wal-g") or overwritten (for "pgbackrest" --delta restore).
# 2) And also the patroni cluster "{{ patroni_cluster_name }}" will be removed from the DCS (if exist) before recovery.

disable_archive_command: true  # or 'false' to not disable archive_command after restore
keep_patroni_dynamic_json: true  # or 'false' to remove patroni.dynamic.json after restore (if exists)

# Netdata - https://github.com/netdata/netdata
netdata_install: false  # or 'true' for install Netdata on postgresql cluster nodes (with kickstart.sh)
netdata_install_options: "--stable-channel --disable-telemetry --dont-wait"
netdata_conf:
  web_bind_to: "*"
  # https://learn.netdata.cloud/docs/store/change-metrics-storage
  memory_mode: "dbengine"  # The long-term metrics storage with efficient RAM and disk usage.
  page_cache_size: 64  # Determines the amount of RAM in MiB that is dedicated to caching Netdata metric values.
  dbengine_disk_space: 1024  # Determines the amount of disk space in MiB that is dedicated to storing Netdata metric values.

...

    \end{lstlisting}
    Jetzt muss geprüft werden, ob die Hosts für Ansible erreichbar sind:
    \lstset{style=gra_codestyle}
    \begin{lstlisting}[language=bash, caption=Deploy - Anhang - Ansible Ping,captionpos=b,label={lst:deploy-appendix-ansible-ping},breaklines=true]
root@sks1000:~/postgresql_cluster# ansible all -m ping
10.0.22.176 | SUCCESS => {
    "changed": false,
    "ping": "pong"
}
10.0.22.177 | SUCCESS => {
    "changed": false,
    "ping": "pong"
}
10.0.22.172 | SUCCESS => {
    "changed": false,
    "ping": "pong"
}
10.0.22.170 | SUCCESS => {
    "changed": false,
    "ping": "pong"
}
10.0.22.171 | SUCCESS => {
    "changed": false,
    "ping": "pong"
}
10.0.22.175 | SUCCESS => {
    "changed": false,
    "ping": "pong"
}
10.0.22.173 | SUCCESS => {
    "changed": false,
    "ping": "pong"
}
10.0.22.174 | SUCCESS => {
    "changed": false,
    "ping": "pong"
}
    \end{lstlisting}

    Deployt wird wie folgt:
    \lstset{style=gra_codestyle}
    \begin{lstlisting}[language=bash, caption=Deploy - Anhang - deploy\_pgcluster.yml,captionpos=b,label={lst:deploy-appendix-pgcluster},breaklines=true]
ansible-playbook deploy_pgcluster.yml
    \end{lstlisting}
    Nach dem deployment sollte folgende Ausgabe zu sehen sein:
    \lstset{style=gra_codestyle}
    \begin{lstlisting}[language=bash, caption=Deploy - Anhang - Deployt,captionpos=b,label={lst:deploy-appendix-deployt},breaklines=true]
root@sks1000:~/postgresql_cluster# ansible-playbook deploy_pgcluster.yml

PLAY [Deploy PostgreSQL HA Cluster (based on "Patroni")] **********************************************************************************************************************************************************

TASK [Gathering Facts] ********************************************************************************************************************************************************************************************
ok: [10.0.22.170]
ok: [10.0.22.172]
ok: [10.0.22.171]
ok: [10.0.22.176]
ok: [10.0.22.177]
ok: [10.0.22.173]
ok: [10.0.22.175]
ok: [10.0.22.174]

TASK [Include main variables] *************************************************************************************************************************************************************************************
ok: [10.0.22.170]
ok: [10.0.22.171]
ok: [10.0.22.172]
ok: [10.0.22.176]
ok: [10.0.22.177]
ok: [10.0.22.173]
ok: [10.0.22.174]
ok: [10.0.22.175]

TASK [Include system variables] ***********************************************************************************************************************************************************************************
ok: [10.0.22.171]
ok: [10.0.22.170]
ok: [10.0.22.176]
ok: [10.0.22.177]
ok: [10.0.22.172]
ok: [10.0.22.173]
ok: [10.0.22.174]
ok: [10.0.22.175]

TASK [Include OS-specific variables] ******************************************************************************************************************************************************************************
ok: [10.0.22.171]
ok: [10.0.22.170]
ok: [10.0.22.172]
ok: [10.0.22.176]
ok: [10.0.22.177]
ok: [10.0.22.173]
ok: [10.0.22.174]
ok: [10.0.22.175]

TASK [System information] *****************************************************************************************************************************************************************************************
ok: [10.0.22.170] => {
    "system_info": {
        "Architecture": "x86_64",
        "CPU model": "Intel(R) Xeon(R) Gold 6142 CPU @ 2.60GHz, count: 4, cores: 1",
        "Disk space total": "45.14 GB",
        "Kernel": "6.1.0-18-amd64",
        "Memory": "7.71 GB",
        "OS": "Debian 12.5",
        "Product name": "VMware Virtual Platform",
        "Virtualization type": "VMware"
    }
}
ok: [10.0.22.171] => {
    "system_info": {
        "Architecture": "x86_64",
        "CPU model": "Intel(R) Xeon(R) Gold 6142 CPU @ 2.60GHz, count: 4, cores: 1",
        "Disk space total": "45.14 GB",
        "Kernel": "6.1.0-18-amd64",
        "Memory": "7.71 GB",
        "OS": "Debian 12.5",
        "Product name": "VMware Virtual Platform",
        "Virtualization type": "VMware"
    }
}
ok: [10.0.22.176] => {
    "system_info": {
        "Architecture": "x86_64",
        "CPU model": "Intel(R) Xeon(R) Gold 6142 CPU @ 2.60GHz, count: 4, cores: 1",
        "Disk space total": "45.14 GB",
        "Kernel": "6.1.0-18-amd64",
        "Memory": "7.71 GB",
        "OS": "Debian 12.5",
        "Product name": "VMware Virtual Platform",
        "Virtualization type": "VMware"
    }
}
ok: [10.0.22.177] => {
    "system_info": {
        "Architecture": "x86_64",
        "CPU model": "Intel(R) Xeon(R) Gold 6142 CPU @ 2.60GHz, count: 4, cores: 1",
        "Disk space total": "45.14 GB",
        "Kernel": "6.1.0-18-amd64",
        "Memory": "7.71 GB",
        "OS": "Debian 12.5",
        "Product name": "VMware Virtual Platform",
        "Virtualization type": "VMware"
    }
}
ok: [10.0.22.172] => {
    "system_info": {
        "Architecture": "x86_64",
        "CPU model": "Intel(R) Xeon(R) Gold 6142 CPU @ 2.60GHz, count: 4, cores: 1",
        "Disk space total": "45.14 GB",
        "Kernel": "6.1.0-18-amd64",
        "Memory": "7.71 GB",
        "OS": "Debian 12.5",
        "Product name": "VMware Virtual Platform",
        "Virtualization type": "VMware"
    }
}
ok: [10.0.22.173] => {
    "system_info": {
        "Architecture": "x86_64",
        "CPU model": "Intel(R) Xeon(R) Gold 6142 CPU @ 2.60GHz, count: 8, cores: 1",
        "Disk space total": "45.14 GB",
        "Kernel": "6.1.0-20-amd64",
        "Memory": "7.71 GB",
        "OS": "Debian 12.5",
        "Product name": "VMware Virtual Platform",
        "Virtualization type": "VMware"
    }
}
ok: [10.0.22.174] => {
    "system_info": {
        "Architecture": "x86_64",
        "CPU model": "Intel(R) Xeon(R) Gold 6142 CPU @ 2.60GHz, count: 8, cores: 1",
        "Disk space total": "45.14 GB",
        "Kernel": "6.1.0-20-amd64",
        "Memory": "7.71 GB",
        "OS": "Debian 12.5",
        "Product name": "VMware Virtual Platform",
        "Virtualization type": "VMware"
    }
}
ok: [10.0.22.175] => {
    "system_info": {
        "Architecture": "x86_64",
        "CPU model": "Intel(R) Xeon(R) Gold 6142 CPU @ 2.60GHz, count: 8, cores: 1",
        "Disk space total": "45.14 GB",
        "Kernel": "6.1.0-20-amd64",
        "Memory": "7.71 GB",
        "OS": "Debian 12.5",
        "Product name": "VMware Virtual Platform",
        "Virtualization type": "VMware"
    }
}

TASK [pre-checks : PostgreSQL | check that data directory "/var/lib/postgresql/16/main" is not initialized] *******************************************************************************************************
ok: [10.0.22.173]
ok: [10.0.22.175]
ok: [10.0.22.174]

TASK [Update apt cache] *******************************************************************************************************************************************************************************************
ok: [10.0.22.170]
ok: [10.0.22.176]
ok: [10.0.22.172]
ok: [10.0.22.177]
ok: [10.0.22.171]
ok: [10.0.22.173]
ok: [10.0.22.175]
ok: [10.0.22.174]

TASK [Make sure the gnupg and apt-transport-https packages are present] *******************************************************************************************************************************************
changed: [10.0.22.170]
changed: [10.0.22.176]
changed: [10.0.22.172]
changed: [10.0.22.171]
changed: [10.0.22.177]
changed: [10.0.22.173]
changed: [10.0.22.175]
changed: [10.0.22.174]

TASK [Make sure that the iproute is installed] ********************************************************************************************************************************************************************
ok: [10.0.22.170]
ok: [10.0.22.176]
ok: [10.0.22.172]
ok: [10.0.22.171]
ok: [10.0.22.177]
ok: [10.0.22.173]
ok: [10.0.22.175]
ok: [10.0.22.174]

PLAY [etcd_cluster.yml | Deploy etcd Cluster] *********************************************************************************************************************************************************************

TASK [Gathering Facts] ********************************************************************************************************************************************************************************************
ok: [10.0.22.170]
ok: [10.0.22.172]
ok: [10.0.22.171]

TASK [Include main variables] *************************************************************************************************************************************************************************************
ok: [10.0.22.170]
ok: [10.0.22.171]
ok: [10.0.22.172]

TASK [Include system variables] ***********************************************************************************************************************************************************************************
ok: [10.0.22.170]
ok: [10.0.22.171]
ok: [10.0.22.172]

TASK [Include OS-specific variables] ******************************************************************************************************************************************************************************
ok: [10.0.22.170]
ok: [10.0.22.171]
ok: [10.0.22.172]

TASK [Update apt cache] *******************************************************************************************************************************************************************************************
ok: [10.0.22.170]
ok: [10.0.22.172]
ok: [10.0.22.171]

TASK [Make sure the gnupg and apt-transport-https packages are present] *******************************************************************************************************************************************
ok: [10.0.22.170]
ok: [10.0.22.172]
ok: [10.0.22.171]

TASK [sysctl : Build a sysctl_conf dynamic variable] **************************************************************************************************************************************************************
ok: [10.0.22.170] => (item=etcd_cluster)
ok: [10.0.22.171] => (item=etcd_cluster)
ok: [10.0.22.172] => (item=etcd_cluster)

TASK [etcd : Make sure the unzip/tar packages are present] ********************************************************************************************************************************************************
changed: [10.0.22.170]
changed: [10.0.22.172]
changed: [10.0.22.171]

TASK [etcd : Download "etcd" package] *****************************************************************************************************************************************************************************
changed: [10.0.22.171] => (item=https://github.com/etcd-io/etcd/releases/download/v3.5.11/etcd-v3.5.11-linux-amd64.tar.gz)
[WARNING]: Module remote_tmp /tmp/root/ansible did not exist and was created with a mode of 0700, this may cause issues when running as another user. To avoid this, create the remote_tmp dir with the correct
permissions manually
changed: [10.0.22.172] => (item=https://github.com/etcd-io/etcd/releases/download/v3.5.11/etcd-v3.5.11-linux-amd64.tar.gz)
changed: [10.0.22.170] => (item=https://github.com/etcd-io/etcd/releases/download/v3.5.11/etcd-v3.5.11-linux-amd64.tar.gz)

TASK [etcd : Extract "etcd" into /tmp] ****************************************************************************************************************************************************************************
changed: [10.0.22.170]
changed: [10.0.22.172]
changed: [10.0.22.171]

TASK [etcd : Copy "etcd" and "etcdctl" binary files to /usr/local/bin/] *******************************************************************************************************************************************
changed: [10.0.22.170] => (item=etcd)
changed: [10.0.22.172] => (item=etcd)
changed: [10.0.22.171] => (item=etcd)
changed: [10.0.22.170] => (item=etcdctl)
changed: [10.0.22.172] => (item=etcdctl)
changed: [10.0.22.171] => (item=etcdctl)

TASK [etcd : Add etcd user] ***************************************************************************************************************************************************************************************
changed: [10.0.22.170]
changed: [10.0.22.171]
changed: [10.0.22.172]

TASK [etcd : Create etcd conf directory] **************************************************************************************************************************************************************************
changed: [10.0.22.170]
changed: [10.0.22.172]
changed: [10.0.22.171]

TASK [etcd : Create etcd data directory] **************************************************************************************************************************************************************************
changed: [10.0.22.170]
changed: [10.0.22.172]
changed: [10.0.22.171]
[WARNING]: sftp transfer mechanism failed on [10.0.22.170]. Use ANSIBLE_DEBUG=1 to see detailed information
[WARNING]: sftp transfer mechanism failed on [10.0.22.172]. Use ANSIBLE_DEBUG=1 to see detailed information
[WARNING]: sftp transfer mechanism failed on [10.0.22.171]. Use ANSIBLE_DEBUG=1 to see detailed information

TASK [etcd : Generate conf file "/etc/etcd/etcd.conf"] ************************************************************************************************************************************************************
changed: [10.0.22.170]
changed: [10.0.22.172]
changed: [10.0.22.171]

TASK [etcd : Copy systemd service file] ***************************************************************************************************************************************************************************
changed: [10.0.22.170]
changed: [10.0.22.171]
changed: [10.0.22.172]

TASK [etcd : Enable and start etcd service] ***********************************************************************************************************************************************************************
changed: [10.0.22.172]
changed: [10.0.22.171]
changed: [10.0.22.170]

TASK [etcd : Wait for port 2379 to become open on the host] *******************************************************************************************************************************************************
ok: [10.0.22.170]
ok: [10.0.22.172]
ok: [10.0.22.171]

TASK [etcd : Wait until the etcd cluster is healthy] **************************************************************************************************************************************************************
ok: [10.0.22.172]
ok: [10.0.22.171]
ok: [10.0.22.170]

TASK [etcd : cluster health] **************************************************************************************************************************************************************************************
ok: [10.0.22.170] => {
    "msg": "http://10.0.22.170:2379 is healthy: successfully committed proposal: took = 2.588786ms\n"
}
ok: [10.0.22.171] => {
    "msg": "http://10.0.22.171:2379 is healthy: successfully committed proposal: took = 5.296369ms\n"
}
ok: [10.0.22.172] => {
    "msg": "http://10.0.22.172:2379 is healthy: successfully committed proposal: took = 2.024139ms\n"
}

PLAY [consul.yml | Configure Consul instances] ********************************************************************************************************************************************************************
skipping: no hosts matched

PLAY [deploy_pgcluster.yml | Postgres Cluster Configuration] ******************************************************************************************************************************************************

TASK [Gathering Facts] ********************************************************************************************************************************************************************************************
ok: [10.0.22.174]
ok: [10.0.22.173]
ok: [10.0.22.175]

TASK [Include main variables] *************************************************************************************************************************************************************************************
ok: [10.0.22.173]
ok: [10.0.22.174]
ok: [10.0.22.175]

TASK [Include system variables] ***********************************************************************************************************************************************************************************
ok: [10.0.22.173]
ok: [10.0.22.174]
ok: [10.0.22.175]

TASK [Include OS-specific variables] ******************************************************************************************************************************************************************************
ok: [10.0.22.173]
ok: [10.0.22.174]
ok: [10.0.22.175]

TASK [add-repository : Add repository apt-key] ********************************************************************************************************************************************************************
changed: [10.0.22.173] => (item={'key': 'https://www.postgresql.org/media/keys/ACCC4CF8.asc'})
changed: [10.0.22.175] => (item={'key': 'https://www.postgresql.org/media/keys/ACCC4CF8.asc'})
changed: [10.0.22.174] => (item={'key': 'https://www.postgresql.org/media/keys/ACCC4CF8.asc'})

TASK [add-repository : Add repository] ****************************************************************************************************************************************************************************
changed: [10.0.22.175] => (item={'repo': 'deb https://apt.postgresql.org/pub/repos/apt/ bookworm-pgdg main'})
changed: [10.0.22.173] => (item={'repo': 'deb https://apt.postgresql.org/pub/repos/apt/ bookworm-pgdg main'})
changed: [10.0.22.174] => (item={'repo': 'deb https://apt.postgresql.org/pub/repos/apt/ bookworm-pgdg main'})

TASK [packages : Install system packages] *************************************************************************************************************************************************************************
ok: [10.0.22.173] => (item=['python3'])
ok: [10.0.22.175] => (item=['python3'])
ok: [10.0.22.174] => (item=['python3'])
ok: [10.0.22.173] => (item=python3)
ok: [10.0.22.175] => (item=python3)
ok: [10.0.22.174] => (item=python3)
changed: [10.0.22.175] => (item=python3-dev)
changed: [10.0.22.173] => (item=python3-dev)
changed: [10.0.22.174] => (item=python3-dev)
changed: [10.0.22.175] => (item=python3-psycopg2)
changed: [10.0.22.173] => (item=python3-psycopg2)
changed: [10.0.22.175] => (item=python3-setuptools)
changed: [10.0.22.174] => (item=python3-psycopg2)
changed: [10.0.22.173] => (item=python3-setuptools)
changed: [10.0.22.174] => (item=python3-setuptools)
changed: [10.0.22.175] => (item=python3-pip)
changed: [10.0.22.173] => (item=python3-pip)
ok: [10.0.22.175] => (item=curl)
ok: [10.0.22.173] => (item=curl)
ok: [10.0.22.175] => (item=less)
ok: [10.0.22.173] => (item=less)
ok: [10.0.22.175] => (item=sudo)
ok: [10.0.22.173] => (item=sudo)
changed: [10.0.22.174] => (item=python3-pip)
ok: [10.0.22.174] => (item=curl)
ok: [10.0.22.174] => (item=less)
ok: [10.0.22.174] => (item=sudo)
changed: [10.0.22.175] => (item=vim)
changed: [10.0.22.173] => (item=vim)
ok: [10.0.22.175] => (item=gcc)
ok: [10.0.22.173] => (item=gcc)
changed: [10.0.22.175] => (item=jq)
changed: [10.0.22.173] => (item=jq)
ok: [10.0.22.175] => (item=iptables)
ok: [10.0.22.173] => (item=iptables)
changed: [10.0.22.174] => (item=vim)
ok: [10.0.22.174] => (item=gcc)
changed: [10.0.22.175] => (item=acl)
changed: [10.0.22.173] => (item=acl)
changed: [10.0.22.175] => (item=dnsutils)
changed: [10.0.22.174] => (item=jq)
changed: [10.0.22.173] => (item=dnsutils)
ok: [10.0.22.174] => (item=iptables)
changed: [10.0.22.174] => (item=acl)
changed: [10.0.22.174] => (item=dnsutils)

TASK [packages : PostgreSQL | ensure postgresql database-cluster manager package] *********************************************************************************************************************************
changed: [10.0.22.173]
changed: [10.0.22.174]
changed: [10.0.22.175]

TASK [packages : PostgreSQL | disable initializing of a default postgresql cluster] *******************************************************************************************************************************
changed: [10.0.22.175]
changed: [10.0.22.173]
changed: [10.0.22.174]

TASK [packages : PostgreSQL | disable log rotation with logrotate for postgresql] *********************************************************************************************************************************
changed: [10.0.22.173]
changed: [10.0.22.174]
changed: [10.0.22.175]

TASK [packages : Install PostgreSQL packages] *********************************************************************************************************************************************************************
changed: [10.0.22.174] => (item=postgresql-16)
changed: [10.0.22.173] => (item=postgresql-16)
changed: [10.0.22.175] => (item=postgresql-16)
ok: [10.0.22.173] => (item=postgresql-client-16)
ok: [10.0.22.174] => (item=postgresql-client-16)
ok: [10.0.22.175] => (item=postgresql-client-16)
ok: [10.0.22.173] => (item=postgresql-contrib-16)
ok: [10.0.22.175] => (item=postgresql-contrib-16)
ok: [10.0.22.174] => (item=postgresql-contrib-16)
changed: [10.0.22.174] => (item=postgresql-server-dev-16)
changed: [10.0.22.175] => (item=postgresql-server-dev-16)
changed: [10.0.22.173] => (item=postgresql-server-dev-16)
[WARNING]: sftp transfer mechanism failed on [10.0.22.174]. Use ANSIBLE_DEBUG=1 to see detailed information
[WARNING]: sftp transfer mechanism failed on [10.0.22.173]. Use ANSIBLE_DEBUG=1 to see detailed information
[WARNING]: sftp transfer mechanism failed on [10.0.22.175]. Use ANSIBLE_DEBUG=1 to see detailed information

TASK [sudo : Add user to /etc/sudoers.d/] *************************************************************************************************************************************************************************
changed: [10.0.22.174] => (item={'name': 'postgres', 'nopasswd': 'yes', 'commands': 'ALL'})
changed: [10.0.22.173] => (item={'name': 'postgres', 'nopasswd': 'yes', 'commands': 'ALL'})
changed: [10.0.22.175] => (item={'name': 'postgres', 'nopasswd': 'yes', 'commands': 'ALL'})

TASK [swap : Ensure swap exists] **********************************************************************************************************************************************************************************
ok: [10.0.22.173]
ok: [10.0.22.175]
ok: [10.0.22.174]

TASK [swap : Swap exists] *****************************************************************************************************************************************************************************************
ok: [10.0.22.173] => {
    "msg": "swap_size_mb: 3904"
}
ok: [10.0.22.174] => {
    "msg": "swap_size_mb: 3904"
}
ok: [10.0.22.175] => {
    "msg": "swap_size_mb: 3904"
}

TASK [swap : Disable all existing swaps] **************************************************************************************************************************************************************************
changed: [10.0.22.173]
changed: [10.0.22.174]
changed: [10.0.22.175]

TASK [swap : Remove swap from /etc/fstab] *************************************************************************************************************************************************************************
changed: [10.0.22.174]
changed: [10.0.22.173]
changed: [10.0.22.175]

TASK [swap : Remove swap file (if exists)] ************************************************************************************************************************************************************************
ok: [10.0.22.174]
ok: [10.0.22.175]
ok: [10.0.22.173]

TASK [swap : Create swap file] ************************************************************************************************************************************************************************************
changed: [10.0.22.175]
changed: [10.0.22.173]
changed: [10.0.22.174]

TASK [swap : Set permissions on swap file] ************************************************************************************************************************************************************************
changed: [10.0.22.173]
changed: [10.0.22.174]
changed: [10.0.22.175]

TASK [swap : Make swap file if necessary] *************************************************************************************************************************************************************************
changed: [10.0.22.175]
changed: [10.0.22.173]
changed: [10.0.22.174]

TASK [swap : Run swapon on the swap file] *************************************************************************************************************************************************************************
changed: [10.0.22.173]
changed: [10.0.22.174]
changed: [10.0.22.175]

TASK [swap : Manage swap file entry in fstab] *********************************************************************************************************************************************************************
changed: [10.0.22.173]
changed: [10.0.22.175]
changed: [10.0.22.174]

TASK [sysctl : Build a sysctl_conf dynamic variable] **************************************************************************************************************************************************************
ok: [10.0.22.173] => (item=master)
ok: [10.0.22.173] => (item=postgres_cluster)
ok: [10.0.22.174] => (item=postgres_cluster)
ok: [10.0.22.174] => (item=replica)
ok: [10.0.22.175] => (item=postgres_cluster)
ok: [10.0.22.175] => (item=replica)

TASK [sysctl : Setting kernel parameters] *************************************************************************************************************************************************************************
changed: [10.0.22.173] => (item={'name': 'vm.overcommit_memory', 'value': '2'})
changed: [10.0.22.175] => (item={'name': 'vm.overcommit_memory', 'value': '2'})
changed: [10.0.22.173] => (item={'name': 'vm.swappiness', 'value': '1'})
changed: [10.0.22.175] => (item={'name': 'vm.swappiness', 'value': '1'})
changed: [10.0.22.173] => (item={'name': 'vm.min_free_kbytes', 'value': '102400'})
changed: [10.0.22.175] => (item={'name': 'vm.min_free_kbytes', 'value': '102400'})
changed: [10.0.22.173] => (item={'name': 'vm.dirty_expire_centisecs', 'value': '1000'})
changed: [10.0.22.175] => (item={'name': 'vm.dirty_expire_centisecs', 'value': '1000'})
changed: [10.0.22.175] => (item={'name': 'vm.dirty_background_bytes', 'value': '67108864'})
changed: [10.0.22.174] => (item={'name': 'vm.overcommit_memory', 'value': '2'})
changed: [10.0.22.175] => (item={'name': 'vm.dirty_bytes', 'value': '536870912'})
changed: [10.0.22.174] => (item={'name': 'vm.swappiness', 'value': '1'})
changed: [10.0.22.175] => (item={'name': 'vm.zone_reclaim_mode', 'value': '0'})
changed: [10.0.22.174] => (item={'name': 'vm.min_free_kbytes', 'value': '102400'})
changed: [10.0.22.175] => (item={'name': 'kernel.numa_balancing', 'value': '0'})
changed: [10.0.22.174] => (item={'name': 'vm.dirty_expire_centisecs', 'value': '1000'})
changed: [10.0.22.175] => (item={'name': 'kernel.sched_autogroup_enabled', 'value': '0'})
changed: [10.0.22.174] => (item={'name': 'vm.dirty_background_bytes', 'value': '67108864'})
changed: [10.0.22.173] => (item={'name': 'vm.dirty_background_bytes', 'value': '67108864'})
changed: [10.0.22.175] => (item={'name': 'net.ipv4.ip_nonlocal_bind', 'value': '1'})
changed: [10.0.22.174] => (item={'name': 'vm.dirty_bytes', 'value': '536870912'})
changed: [10.0.22.175] => (item={'name': 'net.ipv4.ip_forward', 'value': '1'})
changed: [10.0.22.174] => (item={'name': 'vm.zone_reclaim_mode', 'value': '0'})
changed: [10.0.22.175] => (item={'name': 'net.ipv4.ip_local_port_range', 'value': '10000 65535'})
changed: [10.0.22.174] => (item={'name': 'kernel.numa_balancing', 'value': '0'})
changed: [10.0.22.175] => (item={'name': 'net.core.netdev_max_backlog', 'value': '10000'})
changed: [10.0.22.174] => (item={'name': 'kernel.sched_autogroup_enabled', 'value': '0'})
changed: [10.0.22.174] => (item={'name': 'net.ipv4.ip_nonlocal_bind', 'value': '1'})
changed: [10.0.22.173] => (item={'name': 'vm.dirty_bytes', 'value': '536870912'})
changed: [10.0.22.173] => (item={'name': 'vm.zone_reclaim_mode', 'value': '0'})
changed: [10.0.22.173] => (item={'name': 'kernel.numa_balancing', 'value': '0'})
changed: [10.0.22.173] => (item={'name': 'kernel.sched_autogroup_enabled', 'value': '0'})
changed: [10.0.22.175] => (item={'name': 'net.ipv4.tcp_max_syn_backlog', 'value': '8192'})
changed: [10.0.22.173] => (item={'name': 'net.ipv4.ip_nonlocal_bind', 'value': '1'})
changed: [10.0.22.175] => (item={'name': 'net.core.somaxconn', 'value': '65535'})
changed: [10.0.22.174] => (item={'name': 'net.ipv4.ip_forward', 'value': '1'})
changed: [10.0.22.173] => (item={'name': 'net.ipv4.ip_forward', 'value': '1'})
changed: [10.0.22.174] => (item={'name': 'net.ipv4.ip_local_port_range', 'value': '10000 65535'})
changed: [10.0.22.174] => (item={'name': 'net.core.netdev_max_backlog', 'value': '10000'})
changed: [10.0.22.175] => (item={'name': 'net.ipv4.tcp_tw_reuse', 'value': '1'})
changed: [10.0.22.173] => (item={'name': 'net.ipv4.ip_local_port_range', 'value': '10000 65535'})
changed: [10.0.22.173] => (item={'name': 'net.core.netdev_max_backlog', 'value': '10000'})
changed: [10.0.22.174] => (item={'name': 'net.ipv4.tcp_max_syn_backlog', 'value': '8192'})
changed: [10.0.22.173] => (item={'name': 'net.ipv4.tcp_max_syn_backlog', 'value': '8192'})
changed: [10.0.22.174] => (item={'name': 'net.core.somaxconn', 'value': '65535'})
changed: [10.0.22.173] => (item={'name': 'net.core.somaxconn', 'value': '65535'})
changed: [10.0.22.174] => (item={'name': 'net.ipv4.tcp_tw_reuse', 'value': '1'})
changed: [10.0.22.173] => (item={'name': 'net.ipv4.tcp_tw_reuse', 'value': '1'})

TASK [transparent_huge_pages : Create systemd service "disable-transparent-huge-pages.service"] *******************************************************************************************************************
changed: [10.0.22.174]
changed: [10.0.22.173]
changed: [10.0.22.175]

RUNNING HANDLER [transparent_huge_pages : Start disable-transparent-huge-pages service] ***************************************************************************************************************************
changed: [10.0.22.174]
changed: [10.0.22.173]
changed: [10.0.22.175]

TASK [pam_limits : Linux PAM limits | add or modify nofile limits] ************************************************************************************************************************************************
changed: [10.0.22.174] => (item={'limit_type': 'soft', 'limit_item': 'nofile', 'value': 65536})
changed: [10.0.22.175] => (item={'limit_type': 'soft', 'limit_item': 'nofile', 'value': 65536})
changed: [10.0.22.173] => (item={'limit_type': 'soft', 'limit_item': 'nofile', 'value': 65536})
changed: [10.0.22.174] => (item={'limit_type': 'hard', 'limit_item': 'nofile', 'value': 200000})
changed: [10.0.22.175] => (item={'limit_type': 'hard', 'limit_item': 'nofile', 'value': 200000})
changed: [10.0.22.173] => (item={'limit_type': 'hard', 'limit_item': 'nofile', 'value': 200000})

TASK [locales : Generate locales] *********************************************************************************************************************************************************************************
ok: [10.0.22.174] => (item={'language_country': 'en_US', 'encoding': 'UTF-8'})
ok: [10.0.22.175] => (item={'language_country': 'en_US', 'encoding': 'UTF-8'})
ok: [10.0.22.173] => (item={'language_country': 'en_US', 'encoding': 'UTF-8'})

TASK [locales : Set locale "en_US.utf-8" into /etc/default/locale] ************************************************************************************************************************************************
changed: [10.0.22.173] => (item=LANG=en_US.utf-8)
changed: [10.0.22.174] => (item=LANG=en_US.utf-8)
changed: [10.0.22.175] => (item=LANG=en_US.utf-8)
changed: [10.0.22.173] => (item=LANGUAGE=en_US.utf-8)
changed: [10.0.22.174] => (item=LANGUAGE=en_US.utf-8)
changed: [10.0.22.175] => (item=LANGUAGE=en_US.utf-8)
changed: [10.0.22.173] => (item=LC_ALL=en_US.utf-8)
changed: [10.0.22.174] => (item=LC_ALL=en_US.utf-8)
changed: [10.0.22.175] => (item=LC_ALL=en_US.utf-8)

PLAY [balancers.yml | Configure HAProxy load balancers] ***********************************************************************************************************************************************************

TASK [Gathering Facts] ********************************************************************************************************************************************************************************************
ok: [10.0.22.176]
ok: [10.0.22.177]

TASK [Include main variables] *************************************************************************************************************************************************************************************
ok: [10.0.22.176]
ok: [10.0.22.177]

TASK [Include system variables] ***********************************************************************************************************************************************************************************
ok: [10.0.22.176]
ok: [10.0.22.177]

TASK [Include OS-specific variables] ******************************************************************************************************************************************************************************
ok: [10.0.22.176]
ok: [10.0.22.177]

TASK [Update apt cache] *******************************************************************************************************************************************************************************************
ok: [10.0.22.176]
ok: [10.0.22.177]

TASK [Make sure the gnupg and apt-transport-https packages are present] *******************************************************************************************************************************************
ok: [10.0.22.176]
ok: [10.0.22.177]

TASK [sysctl : Build a sysctl_conf dynamic variable] **************************************************************************************************************************************************************
ok: [10.0.22.176] => (item=balancers)
ok: [10.0.22.177] => (item=balancers)

TASK [sysctl : Setting kernel parameters] *************************************************************************************************************************************************************************
changed: [10.0.22.176] => (item={'name': 'net.ipv4.ip_nonlocal_bind', 'value': '1'})
changed: [10.0.22.177] => (item={'name': 'net.ipv4.ip_nonlocal_bind', 'value': '1'})
changed: [10.0.22.176] => (item={'name': 'net.ipv4.ip_forward', 'value': '1'})
changed: [10.0.22.177] => (item={'name': 'net.ipv4.ip_forward', 'value': '1'})
changed: [10.0.22.176] => (item={'name': 'net.ipv4.ip_local_port_range', 'value': '10000 65535'})
changed: [10.0.22.177] => (item={'name': 'net.ipv4.ip_local_port_range', 'value': '10000 65535'})
changed: [10.0.22.176] => (item={'name': 'net.core.netdev_max_backlog', 'value': '10000'})
changed: [10.0.22.177] => (item={'name': 'net.core.netdev_max_backlog', 'value': '10000'})
changed: [10.0.22.177] => (item={'name': 'net.ipv4.tcp_max_syn_backlog', 'value': '8192'})
changed: [10.0.22.176] => (item={'name': 'net.ipv4.tcp_max_syn_backlog', 'value': '8192'})
changed: [10.0.22.176] => (item={'name': 'net.core.somaxconn', 'value': '65535'})
changed: [10.0.22.177] => (item={'name': 'net.core.somaxconn', 'value': '65535'})
changed: [10.0.22.176] => (item={'name': 'net.ipv4.tcp_tw_reuse', 'value': '1'})
changed: [10.0.22.177] => (item={'name': 'net.ipv4.tcp_tw_reuse', 'value': '1'})

TASK [haproxy : Install HAProxy package] **************************************************************************************************************************************************************************
changed: [10.0.22.176]
changed: [10.0.22.177]

TASK [haproxy : Make sure the kernel parameter "net.ipv4.ip_nonlocal_bind" are enabled] ***************************************************************************************************************************
ok: [10.0.22.177]
ok: [10.0.22.176]

TASK [haproxy : Add haproxy group] ********************************************************************************************************************************************************************************
ok: [10.0.22.176]
ok: [10.0.22.177]

TASK [haproxy : Add haproxy user] *********************************************************************************************************************************************************************************
changed: [10.0.22.176]
changed: [10.0.22.177]

TASK [haproxy : Create directories] *******************************************************************************************************************************************************************************
changed: [10.0.22.177] => (item=/etc/haproxy)
changed: [10.0.22.176] => (item=/etc/haproxy)
ok: [10.0.22.177] => (item=/run/haproxy)
ok: [10.0.22.176] => (item=/run/haproxy)
changed: [10.0.22.177] => (item=/var/lib/haproxy/dev)
changed: [10.0.22.176] => (item=/var/lib/haproxy/dev)
[WARNING]: sftp transfer mechanism failed on [10.0.22.177]. Use ANSIBLE_DEBUG=1 to see detailed information
[WARNING]: sftp transfer mechanism failed on [10.0.22.176]. Use ANSIBLE_DEBUG=1 to see detailed information

TASK [haproxy : Generate conf file "/etc/haproxy/haproxy.cfg"] ****************************************************************************************************************************************************
changed: [10.0.22.177]
changed: [10.0.22.176]

TASK [haproxy : Generate systemd service file "/etc/systemd/system/haproxy.service"] ******************************************************************************************************************************
changed: [10.0.22.177]
changed: [10.0.22.176]

TASK [confd : Download and copy "confd" binary file to /usr/local/bin/] *******************************************************************************************************************************************
changed: [10.0.22.177] => (item=https://github.com/kelseyhightower/confd/releases/download/v0.16.0/confd-0.16.0-linux-amd64)
changed: [10.0.22.176] => (item=https://github.com/kelseyhightower/confd/releases/download/v0.16.0/confd-0.16.0-linux-amd64)

TASK [confd : Create conf directories] ****************************************************************************************************************************************************************************
changed: [10.0.22.176] => (item=/etc/confd/conf.d)
changed: [10.0.22.177] => (item=/etc/confd/conf.d)
changed: [10.0.22.176] => (item=/etc/confd/templates)
changed: [10.0.22.177] => (item=/etc/confd/templates)

TASK [confd : Generate conf file "/etc/confd/confd.toml"] *********************************************************************************************************************************************************
changed: [10.0.22.177]
changed: [10.0.22.176]

TASK [confd : Generate conf file "/etc/confd/conf.d/haproxy.toml"] ************************************************************************************************************************************************
changed: [10.0.22.176]
changed: [10.0.22.177]

TASK [confd : Generate template "/etc/confd/templates/haproxy.tmpl"] **********************************************************************************************************************************************
changed: [10.0.22.177]
changed: [10.0.22.176]

TASK [confd : Copy systemd service file] **************************************************************************************************************************************************************************
changed: [10.0.22.177]
changed: [10.0.22.176]

TASK [keepalived : Install keepalived packages] *******************************************************************************************************************************************************************
changed: [10.0.22.177]
changed: [10.0.22.176]

TASK [keepalived : Make sure the kernel parameters "net.ipv4.ip_nonlocal_bind", "net.ipv4.ip_forward" are enabled] ************************************************************************************************
ok: [10.0.22.176] => (item=net.ipv4.ip_nonlocal_bind)
ok: [10.0.22.177] => (item=net.ipv4.ip_nonlocal_bind)
ok: [10.0.22.177] => (item=net.ipv4.ip_forward)
ok: [10.0.22.176] => (item=net.ipv4.ip_forward)

TASK [keepalived : Make sure the "/usr/libexec/keepalived" directory exists] **************************************************************************************************************************************
changed: [10.0.22.176]
changed: [10.0.22.177]

TASK [keepalived : Create vrrp_script "/usr/libexec/keepalived/haproxy_check.sh"] *********************************************************************************************************************************
changed: [10.0.22.177]
changed: [10.0.22.176]

TASK [keepalived : Generate conf file "/etc/keepalived/keepalived.conf"] ******************************************************************************************************************************************
changed: [10.0.22.177]
changed: [10.0.22.176]

RUNNING HANDLER [haproxy : Restart haproxy service] ***************************************************************************************************************************************************************
changed: [10.0.22.177]
changed: [10.0.22.176]

RUNNING HANDLER [haproxy : Check HAProxy is started and accepting connections] ************************************************************************************************************************************
ok: [10.0.22.177]
ok: [10.0.22.176]

RUNNING HANDLER [confd : Restart confd service] *******************************************************************************************************************************************************************
changed: [10.0.22.177]
changed: [10.0.22.176]

RUNNING HANDLER [keepalived : Restart keepalived service] *********************************************************************************************************************************************************
changed: [10.0.22.177]
changed: [10.0.22.176]

RUNNING HANDLER [keepalived : Wait for the cluster ip address (VIP) "10.0.22.178" is running] *********************************************************************************************************************
ok: [10.0.22.176]
ok: [10.0.22.177]

PLAY [deploy_pgcluster.yml | Install and configure pgBackRest] ****************************************************************************************************************************************************

TASK [Gathering Facts] ********************************************************************************************************************************************************************************************
ok: [10.0.22.173]
ok: [10.0.22.175]
ok: [10.0.22.174]

TASK [Include main variables] *************************************************************************************************************************************************************************************
ok: [10.0.22.173]
ok: [10.0.22.174]
ok: [10.0.22.175]

TASK [Include OS-specific variables] ******************************************************************************************************************************************************************************
ok: [10.0.22.173]
ok: [10.0.22.174]
ok: [10.0.22.175]

PLAY [deploy_pgcluster.yml | PostgreSQL Cluster Deployment] *******************************************************************************************************************************************************

TASK [Gathering Facts] ********************************************************************************************************************************************************************************************
ok: [10.0.22.175]
ok: [10.0.22.173]
ok: [10.0.22.174]

TASK [Include main variables] *************************************************************************************************************************************************************************************
ok: [10.0.22.173]
ok: [10.0.22.174]
ok: [10.0.22.175]

TASK [Include system variables] ***********************************************************************************************************************************************************************************
ok: [10.0.22.173]
ok: [10.0.22.174]
ok: [10.0.22.175]

TASK [Include OS-specific variables] ******************************************************************************************************************************************************************************
ok: [10.0.22.173]
ok: [10.0.22.174]
ok: [10.0.22.175]

TASK [cron : Make sure that the cron package is installed] ********************************************************************************************************************************************************
ok: [10.0.22.175]
ok: [10.0.22.174]
ok: [10.0.22.173]

TASK [pgbouncer : Install pgbouncer package] **********************************************************************************************************************************************************************
changed: [10.0.22.175]
changed: [10.0.22.173]
changed: [10.0.22.174]

TASK [pgbouncer : Ensure config directory "/etc/pgbouncer" exist] *************************************************************************************************************************************************
changed: [10.0.22.173]
changed: [10.0.22.175]
changed: [10.0.22.174]

TASK [pgbouncer : Ensure log directory "/var/log/pgbouncer" exist] ************************************************************************************************************************************************
changed: [10.0.22.173]
changed: [10.0.22.174]
changed: [10.0.22.175]

TASK [pgbouncer : Check if pgbouncer systemd service file exists] *************************************************************************************************************************************************
ok: [10.0.22.173]
ok: [10.0.22.174]
ok: [10.0.22.175]

TASK [pgbouncer : Stop and disable standard init script] **********************************************************************************************************************************************************
changed: [10.0.22.174]
changed: [10.0.22.175]
changed: [10.0.22.173]

TASK [pgbouncer : Configure pgbouncer systemd service file] *******************************************************************************************************************************************************
changed: [10.0.22.175] => (item=pgbouncer)
changed: [10.0.22.173] => (item=pgbouncer)
changed: [10.0.22.174] => (item=pgbouncer)

TASK [pgbouncer : Ensure pgbouncer service is enabled] ************************************************************************************************************************************************************
changed: [10.0.22.173] => (item=pgbouncer)
changed: [10.0.22.174] => (item=pgbouncer)
changed: [10.0.22.175] => (item=pgbouncer)

TASK [pgbouncer : Enable log rotation with logrotate] *************************************************************************************************************************************************************
changed: [10.0.22.175] => (item=pgbouncer)
changed: [10.0.22.174] => (item=pgbouncer)
changed: [10.0.22.173] => (item=pgbouncer)

TASK [pgbouncer : Configure pgbouncer.ini] ************************************************************************************************************************************************************************
changed: [10.0.22.175] => (item=pgbouncer)
changed: [10.0.22.173] => (item=pgbouncer)
changed: [10.0.22.174] => (item=pgbouncer)

TASK [pgpass : Configure a password file (/var/lib/postgresql/.pgpass)] *******************************************************************************************************************************************
changed: [10.0.22.175]
changed: [10.0.22.174]
changed: [10.0.22.173]

RUNNING HANDLER [Restart pgbouncer service] ***********************************************************************************************************************************************************************
changed: [10.0.22.174] => (item=pgbouncer)
changed: [10.0.22.173] => (item=pgbouncer)
changed: [10.0.22.175] => (item=pgbouncer)

RUNNING HANDLER [Wait for port "6432" to become open on the host] *************************************************************************************************************************************************
ok: [10.0.22.173]
ok: [10.0.22.174]
ok: [10.0.22.175]

TASK [patroni : Copy patroni requirements.txt file] ***************************************************************************************************************************************************************
changed: [10.0.22.174]
changed: [10.0.22.173]
changed: [10.0.22.175]

TASK [patroni : Install setuptools] *******************************************************************************************************************************************************************************
changed: [10.0.22.175]
changed: [10.0.22.173]
changed: [10.0.22.174]

TASK [patroni : Install requirements] *****************************************************************************************************************************************************************************
changed: [10.0.22.174]
changed: [10.0.22.173]
changed: [10.0.22.175]

TASK [patroni : Install patroni 3.3.0] ****************************************************************************************************************************************************************************
changed: [10.0.22.174]
changed: [10.0.22.173]
changed: [10.0.22.175]

TASK [patroni : Create conf directory] ****************************************************************************************************************************************************************************
changed: [10.0.22.174]
changed: [10.0.22.173]
changed: [10.0.22.175]

TASK [patroni : Generate conf file "/etc/patroni/patroni.yml"] ****************************************************************************************************************************************************
changed: [10.0.22.173]
changed: [10.0.22.174]
changed: [10.0.22.175]

TASK [patroni : Copy systemd service file "/etc/systemd/system/patroni.service"] **********************************************************************************************************************************
changed: [10.0.22.173]
changed: [10.0.22.174]
changed: [10.0.22.175]

TASK [patroni : Prepare PostgreSQL | make sure the postgresql log directory "/var/log/postgresql" exists] *********************************************************************************************************
changed: [10.0.22.173]
changed: [10.0.22.174]
changed: [10.0.22.175]

TASK [patroni : Prepare PostgreSQL | make sure PostgreSQL data directory "/var/lib/postgresql/16/main" exists] ****************************************************************************************************
changed: [10.0.22.174]
changed: [10.0.22.173]
changed: [10.0.22.175]

TASK [patroni : Prepare PostgreSQL | make sure the postgresql config files exists] ********************************************************************************************************************************
ok: [10.0.22.173]
ok: [10.0.22.174]
ok: [10.0.22.175]

TASK [patroni : Prepare PostgreSQL | generate default postgresql config files] ************************************************************************************************************************************
changed: [10.0.22.173]
changed: [10.0.22.175]
changed: [10.0.22.174]

TASK [patroni : Prepare PostgreSQL | make sure the data directory "/var/lib/postgresql/16/main" is empty] *********************************************************************************************************
changed: [10.0.22.174] => (item=absent)
changed: [10.0.22.173] => (item=absent)
changed: [10.0.22.175] => (item=absent)
changed: [10.0.22.173] => (item=directory)
changed: [10.0.22.174] => (item=directory)
changed: [10.0.22.175] => (item=directory)

TASK [patroni : Start patroni service on the Master server] *******************************************************************************************************************************************************
changed: [10.0.22.173]

TASK [patroni : Wait for port 8008 to become open on the host] ****************************************************************************************************************************************************
ok: [10.0.22.173]

TASK [patroni : Check PostgreSQL is started and accepting connections on Master] **********************************************************************************************************************************
ok: [10.0.22.173]

TASK [patroni : Wait for the cluster to initialize (master is the leader with the lock)] **************************************************************************************************************************
ok: [10.0.22.173]

TASK [patroni : Prepare PostgreSQL | generate pg_hba.conf] ********************************************************************************************************************************************************
changed: [10.0.22.173]
changed: [10.0.22.175]
changed: [10.0.22.174]

TASK [patroni : Prepare PostgreSQL | reload for apply the pg_hba.conf] ********************************************************************************************************************************************
changed: [10.0.22.173]
ok: [10.0.22.175]
ok: [10.0.22.174]

TASK [patroni : Make sure the postgresql users are present, and password does not differ from the specified] ******************************************************************************************************
failed: [10.0.22.173] (item=None) => {"ansible_loop_var": "item", "changed": false, "false_condition": "patroni_cluster_bootstrap_method != \"initdb\" and (pgbackrest_install|bool or wal_g_install|bool) and (existing_pgcluster is not defined or not existing_pgcluster|bool)", "item": null, "msg": "Failed to template loop_control.label: 'None' has no attribute 'name'. 'None' has no attribute 'name'", "skip_reason": "Conditional result was False"}
...ignoring
failed: [10.0.22.174] (item=None) => {"ansible_loop_var": "item", "changed": false, "false_condition": "patroni_cluster_bootstrap_method != \"initdb\" and (pgbackrest_install|bool or wal_g_install|bool) and (existing_pgcluster is not defined or not existing_pgcluster|bool)", "item": null, "msg": "Failed to template loop_control.label: 'None' has no attribute 'name'. 'None' has no attribute 'name'", "skip_reason": "Conditional result was False"}
...ignoring
failed: [10.0.22.175] (item=None) => {"ansible_loop_var": "item", "changed": false, "false_condition": "patroni_cluster_bootstrap_method != \"initdb\" and (pgbackrest_install|bool or wal_g_install|bool) and (existing_pgcluster is not defined or not existing_pgcluster|bool)", "item": null, "msg": "Failed to template loop_control.label: 'None' has no attribute 'name'. 'None' has no attribute 'name'", "skip_reason": "Conditional result was False"}
...ignoring

TASK [patroni : Start patroni service on Replica servers] *********************************************************************************************************************************************************
changed: [10.0.22.174]
changed: [10.0.22.175]

TASK [patroni : Wait for port 8008 to become open on the host] ****************************************************************************************************************************************************
ok: [10.0.22.174]
ok: [10.0.22.175]

TASK [patroni : Check that the patroni is healthy on the replica server] ******************************************************************************************************************************************
ok: [10.0.22.174]
ok: [10.0.22.175]

TASK [patroni : Turning off postgresql autostart from config "start.conf" (will be managed by patroni)] ***********************************************************************************************************
changed: [10.0.22.173]
changed: [10.0.22.174]
changed: [10.0.22.175]

TASK [patroni : Disable "postgresql@16-main" service] *************************************************************************************************************************************************************
ok: [10.0.22.173]
ok: [10.0.22.174]
ok: [10.0.22.175]

TASK [patroni : Add PATRONICTL_CONFIG_FILE environment variable into /etc/environment] ****************************************************************************************************************************
changed: [10.0.22.173]
changed: [10.0.22.175]
changed: [10.0.22.174]

TASK [postgresql-users : Make sure the PostgreSQL users are present] **********************************************************************************************************************************************
changed: [10.0.22.173] => (item=pgbouncer)
changed: [10.0.22.173] => (item=xksgr_sks1160_gitlab)
changed: [10.0.22.173] => (item=xksgr_sks1172_harbor)
changed: [10.0.22.173] => (item=xksgr_sks1195_kcsso)
changed: [10.0.22.173] => (item=xksgr_k8s_core_psql_monitor)
changed: [10.0.22.173] => (item=xksgr_k8s_core_psql_backup)

TASK [postgresql-users : Grant roles to users] ********************************************************************************************************************************************************************
changed: [10.0.22.173] => (item=xksgr_k8s_core_psql_monitor)

TASK [postgresql-databases : Make sure the PostgreSQL databases are present] **************************************************************************************************************************************
changed: [10.0.22.173] => (item={'db': 'k8s_core_gitlab_prod', 'encoding': 'UTF8', 'lc_collate': 'en_US.UTF-8', 'lc_ctype': 'en_US.UTF-8', 'owner': 'xksgr_sks1160_gitlab'})
changed: [10.0.22.173] => (item={'db': 'k8s_core_harbor_prod', 'encoding': 'UTF8', 'lc_collate': 'en_US.UTF-8', 'lc_ctype': 'en_US.UTF-8', 'owner': 'xksgr_sks1172_harbor'})
changed: [10.0.22.173] => (item={'db': 'k8s_core_keycloak_prod', 'encoding': 'UTF8', 'lc_collate': 'en_US.UTF-8', 'lc_ctype': 'en_US.UTF-8', 'owner': 'xksgr_sks1195_kcsso'})
changed: [10.0.22.173] => (item={'db': 'gramic_test', 'encoding': 'UTF8', 'lc_collate': 'en_US.UTF-8', 'lc_ctype': 'en_US.UTF-8', 'owner': 'postgres'})

TASK [postgresql-extensions : Add extensions to the databases] ****************************************************************************************************************************************************
changed: [10.0.22.173] => (item={'ext': 'pgstattuple', 'db': 'postgres'})
changed: [10.0.22.173] => (item={'ext': 'pgstattuple', 'db': 'k8s_core_gitlab_prod'})
changed: [10.0.22.173] => (item={'ext': 'pgstattuple', 'db': 'k8s_core_harbor_prod'})
changed: [10.0.22.173] => (item={'ext': 'pgstattuple', 'db': 'k8s_core_keycloak_prod'})
changed: [10.0.22.173] => (item={'ext': 'pgstattuple', 'db': 'gramic_test'})

TASK [pgbouncer/config : Ensure config directory "/etc/pgbouncer" exist] ******************************************************************************************************************************************
ok: [10.0.22.173]
ok: [10.0.22.174]
ok: [10.0.22.175]

TASK [pgbouncer/config : Update pgbouncer.ini] ********************************************************************************************************************************************************************
ok: [10.0.22.173] => (item=pgbouncer)
ok: [10.0.22.175] => (item=pgbouncer)
ok: [10.0.22.174] => (item=pgbouncer)

TASK [pgbouncer/config : Check if 'user_search' function exists] **************************************************************************************************************************************************
ok: [10.0.22.173]

TASK [pgbouncer/config : Create 'user_search' function for pgbouncer 'auth_query' option] *************************************************************************************************************************
changed: [10.0.22.173]

TASK [deploy-finish : Get postgresql users list] ******************************************************************************************************************************************************************
ok: [10.0.22.173]

TASK [deploy-finish : PostgreSQL list of users] *******************************************************************************************************************************************************************
ok: [10.0.22.173] => {
    "users_result.stdout_lines": [
        "                                      List of roles",
        "          Role name          |                         Attributes                         ",
        "-----------------------------+------------------------------------------------------------",
        " pgbouncer                   | ",
        " postgres                    | Superuser, Create role, Create DB, Replication, Bypass RLS",
        " replicator                  | Replication",
        " xksgr_k8s_core_psql_backup  | Superuser",
        " xksgr_k8s_core_psql_monitor | ",
        " xksgr_sks1160_gitlab        | Superuser",
        " xksgr_sks1172_harbor        | Superuser",
        " xksgr_sks1195_kcsso         | Superuser"
    ]
}

TASK [deploy-finish : Get postgresql database list] ***************************************************************************************************************************************************************
ok: [10.0.22.173]

TASK [deploy-finish : PostgreSQL list of databases] ***************************************************************************************************************************************************************
ok: [10.0.22.173] => {
    "dbs_result.stdout_lines": [
        "          name          |        owner         | encoding |   collate   |    ctype    |  size   | tablespace ",
        "------------------------+----------------------+----------+-------------+-------------+---------+------------",
        " gramic_test            | postgres             | UTF8     | en_US.UTF-8 | en_US.UTF-8 | 7492 kB | pg_default",
        " k8s_core_gitlab_prod   | xksgr_sks1160_gitlab | UTF8     | en_US.UTF-8 | en_US.UTF-8 | 7492 kB | pg_default",
        " k8s_core_harbor_prod   | xksgr_sks1172_harbor | UTF8     | en_US.UTF-8 | en_US.UTF-8 | 7492 kB | pg_default",
        " k8s_core_keycloak_prod | xksgr_sks1195_kcsso  | UTF8     | en_US.UTF-8 | en_US.UTF-8 | 7492 kB | pg_default",
        " postgres               | postgres             | UTF8     | en_US.UTF-8 | en_US.UTF-8 | 7492 kB | pg_default",
        "(5 rows)"
    ]
}

TASK [deploy-finish : Check postgresql cluster health] ************************************************************************************************************************************************************
ok: [10.0.22.173]

TASK [deploy-finish : PostgreSQL Cluster health] ******************************************************************************************************************************************************************
ok: [10.0.22.173] => {
    "patronictl_result.stdout_lines": [
        "+ Cluster: k8s-core-psql (7367246950910762117) ----+----+-----------+",
        "| Member  | Host        | Role         | State     | TL | Lag in MB |",
        "+---------+-------------+--------------+-----------+----+-----------+",
        "| sks1263 | 10.0.22.173 | Leader       | running   |  1 |           |",
        "| sks1264 | 10.0.22.174 | Sync Standby | streaming |  1 |         0 |",
        "| sks1265 | 10.0.22.175 | Sync Standby | streaming |  1 |         0 |",
        "+---------+-------------+--------------+-----------+----+-----------+"
    ]
}

TASK [deploy-finish : PostgreSQL Cluster connection info] *********************************************************************************************************************************************************
ok: [10.0.22.173] => {
    "msg": [
        "+------------------------------------------------+",
        "address (VIP) 10.0.22.178",
        "port 5000 (read/write) master",
        "port 5001 (read only)  all replicas",
        "port 5002 (read only)  synchronous replica only",
        "port 5003 (read only)  asynchronous replicas only",
        "+------------------------------------------------+"
    ]
}

PLAY RECAP ********************************************************************************************************************************************************************************************************
10.0.22.170                : ok=28   changed=11   unreachable=0    failed=0    skipped=56   rescued=0    ignored=0
10.0.22.171                : ok=28   changed=11   unreachable=0    failed=0    skipped=49   rescued=0    ignored=0
10.0.22.172                : ok=28   changed=11   unreachable=0    failed=0    skipped=49   rescued=0    ignored=0
10.0.22.173                : ok=95   changed=52   unreachable=0    failed=0    skipped=308  rescued=0    ignored=1
10.0.22.174                : ok=81   changed=46   unreachable=0    failed=0    skipped=294  rescued=0    ignored=1
10.0.22.175                : ok=81   changed=46   unreachable=0    failed=0    skipped=294  rescued=0    ignored=1
10.0.22.176                : ok=39   changed=20   unreachable=0    failed=0    skipped=89   rescued=0    ignored=0
10.0.22.177                : ok=39   changed=20   unreachable=0    failed=0    skipped=83   rescued=0    ignored=0

    \end{lstlisting}

    \subsection{Maintenance}
    Das \texttt{Inventory} bleibt gleich.\\
    Geändert wurde nur das \texttt{main.yml}:
    \lstset{style=gra_codestyle}
    \begin{lstlisting}[language=yaml, caption=Testsystem - Anhang - Maintenance - main.yml,captionpos=b,label={lst:testsystem-maintenance-main.yml},breaklines=true]
---
# ---------------------------------------------------------------------
# Proxy variables (optional) for download packages using a proxy server
proxy_env:
  http_proxy: http://sproxy.sivc.first-it.ch:8080
  https_proxy: http://sproxy.sivc.first-it.ch:8080

# Cluster variables
#cluster_vip: ""  # IP address for client access to the databases in the cluster (optional).
cluster_vip: "10.0.22.178"  # IP address for client access to the databases in the cluster (optional).
vip_interface: "{{ ansible_default_ipv4.interface }}"  # interface name (e.g., "ens32").
# Note: VIP-based solutions such as keepalived or vip-manager may not function correctly in cloud environments like AWS.

#patroni_cluster_name: "postgres-cluster"  # the cluster name (must be unique for each cluster)
patroni_cluster_name: "k8s-core-psql"  # the cluster name (must be unique for each cluster)
patroni_install_version: "3.3.0"  # or 'latest'

patroni_superuser_username: "postgres"
patroni_superuser_password: "<Password Secure / Safe>"
patroni_replication_username: "replicator"
patroni_replication_password: "<Password Secure / Safe>"

#synchronous_mode: false  # or 'true' for enable synchronous database replication
synchronous_mode: true  # or 'true' for enable synchronous database replication
synchronous_mode_strict: false  # if 'true' then block all client writes to the master, when a synchronous replica is not available
#synchronous_node_count: 1  # number of synchronous standby databases
synchronous_node_count: 2  # number of synchronous standby databases

# Load Balancing
with_haproxy_load_balancing: true  # or 'true' if you want to install and configure the load-balancing
haproxy_listen_port:
  master: 5000
  replicas: 5001
  replicas_sync: 5002
  replicas_async: 5003
# The following ('_direct') ports are used for direct connections to the PostgreSQL database,
# bypassing the PgBouncer connection pool (if 'pgbouncer_install' is 'true').
# Uncomment the relevant lines if you need to set up direct connections.
  stats: 7000
haproxy_maxconn:
  global: 100000
  master: 10000
  replica: 10000
haproxy_timeout:
  client: "60m"
  server: "60m"

# keepalived (if 'cluster_vip' is specified and 'with_haproxy_load_balancing' is 'true')
keepalived_virtual_router_id: "{{ cluster_vip.split('.')[3] | int }}" # The last octet of 'cluster_vip' IP address is used by default.
# virtual_router_id - must be unique in the network (available values are 0..255).

# vip-manager (if 'cluster_vip' is specified and 'with_haproxy_load_balancing' is 'false')
vip_manager_version: "2.4.0"  # version to install
vip_manager_conf: "/etc/patroni/vip-manager.yml"
vip_manager_interval: "1000"  # time (in milliseconds) after which vip-manager wakes up and checks if it needs to register or release ip addresses.
vip_manager_iface: "{{ vip_interface }}"  # interface to which the virtual ip will be added
vip_manager_ip: "{{ cluster_vip }}"  # the virtual ip address to manage
vip_manager_mask: "24"  # netmask for the virtual ip


# DCS (Distributed Consensus Store)
dcs_exists: false  # or 'true' if you don't want to deploy a new etcd cluster
dcs_type: "etcd"  # or 'consul'

# if dcs_type: "etcd" and dcs_exists: false
etcd_version: "3.5.11"  # version for deploy etcd cluster
etcd_data_dir: "/var/lib/etcd"
etcd_cluster_name: "etcd-{{ patroni_cluster_name }}"  # ETCD_INITIAL_CLUSTER_TOKEN

# if dcs_type: "etcd" and dcs_exists: true
patroni_etcd_hosts: []  # list of servers of an existing etcd cluster
patroni_etcd_namespace: "service"  # (optional) etcd namespace (prefix)
patroni_etcd_username: "" # (optional) username for etcd authentication
patroni_etcd_password: "" # (optional) password for etcd authentication
patroni_etcd_protocol: "" # (optional) http or https, if not specified http is used

# if dcs_type: "consul"
consul_version: "1.15.8"
consul_config_path: "/etc/consul"
consul_configd_path: "{{ consul_config_path }}/conf.d"
consul_data_path: "/var/lib/consul"
consul_domain: "consul"  # Consul domain name
consul_datacenter: "dc1"  # Datacenter label (can be specified for each host in the inventory)
consul_disable_update_check: true  # Disables automatic checking for security bulletins and new version releases
consul_enable_script_checks: true  # This controls whether health checks that execute scripts are enabled on this agent
consul_enable_local_script_checks: true  # Enable them when they are defined in the local configuration files
consul_ui: false  # Enable the consul UI?
consul_syslog_enable: true  # Enable logging to syslog
consul_iface: "{{ ansible_default_ipv4.interface }}"  # specify the interface name with a Private IP (ex. "enp7s0")
# TLS
# You can enable TLS encryption by dropping a CA certificate, server certificate, and server key in roles/consul/files/
consul_tls_enable: false
consul_tls_ca_crt: "ca.crt"
consul_tls_server_crt: "server.crt"
consul_tls_server_key: "server.key"
# DNS
consul_recursors: []  # List of upstream DNS servers
consul_dnsmasq_enable: true  # Enable DNS forwarding with Dnsmasq
consul_dnsmasq_cache: 0  # dnsmasq cache-size (0 - disable caching)
consul_dnsmasq_servers: "{{ nameservers }}" # Upstream DNS servers used by dnsmasq
consul_join: []  # List of LAN servers of an existing consul cluster, to join.

# https://developer.hashicorp.com/consul/docs/discovery/services
consul_services:
  - name: "{{ patroni_cluster_name }}"
    id: "{{ patroni_cluster_name }}-master"
    tags: ['master', 'primary']
    port: "{{ pgbouncer_listen_port }}"  # or "{{ postgresql_port }}" if pgbouncer_install: false
    checks:
      - { http: "http://{{ inventory_hostname }}:{{ patroni_restapi_port }}/primary", interval: "2s" }
      - { args: ["systemctl", "status", "pgbouncer"], interval: "5s" }  # comment out this check if pgbouncer_install: false
  - name: "{{ patroni_cluster_name }}"
    id: "{{ patroni_cluster_name }}-replica"
    tags: ['replica']
    port: "{{ pgbouncer_listen_port }}"
    checks:
      - { http: "http://{{ inventory_hostname }}:{{ patroni_restapi_port }}/replica?lag={{ patroni_maximum_lag_on_replica }}", interval: "2s" }
      - { args: ["systemctl", "status", "pgbouncer"], interval: "5s" }

# PostgreSQL variables
postgresql_version: "16"
# postgresql_data_dir: see vars/Debian.yml or vars/RedHat.yml
postgresql_listen_addr: "0.0.0.0" # Listen on all interfaces. Or use "{{ inventory_hostname }},127.0.0.1" to listen on a specific IP address.
postgresql_port: "5432"
postgresql_encoding: "UTF8"  # for bootstrap only (initdb)
postgresql_locale: "en_US.UTF-8"  # for bootstrap only (initdb)
postgresql_data_checksums: true  # for bootstrap only (initdb)
postgresql_password_encryption_algorithm: "scram-sha-256"  # or "md5" if your clients do not work with passwords encrypted with SCRAM-SHA-256

# (optional) list of users to be created (if not already exists)
postgresql_users:
  - { name: "{{ pgbouncer_auth_username }}", password: "{{ pgbouncer_auth_password }}", flags: "LOGIN", role: "" }
  - { name: "xksgr_sks1160_gitlab", password: "<Password Secure / Safe>", flags: "SUPERUSER" }
  - { name: "xksgr_sks1172_harbor", password: "<Password Secure / Safe>", flags: "SUPERUSER" }
  - { name: "xksgr_sks1195_kcsso", password: "<Password Secure / Safe>", flags: "SUPERUSER" }
  - { name: "xksgr_k8s_core_psql_monitor", password: "<Password Secure / Safe>", flags: "LOGIN", role: "pg_monitor" }
  - { name: "xksgr_k8s_core_psql_backup", password: "<Password Secure / Safe>", flags: "SUPERUSER" }
# (optional) list of databases to be created (if not already exists)
#postgresql_databases: []
postgresql_databases:
  - { db: "k8s_core_gitlab_prod", encoding: "UTF8", lc_collate: "en_US.UTF-8", lc_ctype: "en_US.UTF-8", owner: "xksgr_sks1160_gitlab" }
  - { db: "k8s_core_harbor_prod", encoding: "UTF8", lc_collate: "en_US.UTF-8", lc_ctype: "en_US.UTF-8", owner: "xksgr_sks1172_harbor" }
  - { db: "k8s_core_keycloak_prod", encoding: "UTF8", lc_collate: "en_US.UTF-8", lc_ctype: "en_US.UTF-8", owner: "xksgr_sks1195_kcsso" }
  - { db: "gramic_test", encoding: "UTF8", lc_collate: "en_US.UTF-8", lc_ctype: "en_US.UTF-8", owner: "postgres" }

# (optional) list of schemas to be created (if not already exists)
postgresql_schemas: []
#  - { schema: "myschema", db: "mydatabase", owner: "mydb-user" }

# (optional) list of database extensions to be created (if not already exists)
#postgresql_extensions: []
postgresql_extensions:
  - { ext: "pgstattuple", db: "postgres" }
  - { ext: "pgstattuple", db: "k8s_core_gitlab_prod" }
  - { ext: "pgstattuple", db: "k8s_core_harbor_prod" }
  - { ext: "pgstattuple", db: "k8s_core_keycloak_prod" }
  - { ext: "pgstattuple", db: "gramic_test" }

# postgresql parameters to bootstrap dcs (are parameters for example)
postgresql_parameters:
  - { option: "max_connections", value: "500" }
  - { option: "superuser_reserved_connections", value: "5" }
  - { option: "password_encryption", value: "{{ postgresql_password_encryption_algorithm }}" }
  - { option: "max_locks_per_transaction", value: "512" }
  - { option: "max_prepared_transactions", value: "0" }
  - { option: "huge_pages", value: "try" }  # or "on" if you set "vm_nr_hugepages" in kernel parameters
  - { option: "shared_buffers", value: "{{ (ansible_memtotal_mb * 0.25) | int }}MB" }  # by default, 25% of RAM
  - { option: "effective_cache_size", value: "{{ (ansible_memtotal_mb * 0.75) | int }}MB" }  # by default, 75% of RAM
  - { option: "work_mem", value: "128MB" }  # please change this value
  - { option: "maintenance_work_mem", value: "256MB" }  # please change this value
  - { option: "checkpoint_timeout", value: "15min" }
  - { option: "checkpoint_completion_target", value: "0.9" }
  - { option: "min_wal_size", value: "2GB" }
  - { option: "max_wal_size", value: "8GB" }  # or 16GB/32GB
  - { option: "wal_buffers", value: "32MB" }
  - { option: "default_statistics_target", value: "1000" }
  - { option: "seq_page_cost", value: "1" }
  - { option: "random_page_cost", value: "1.1" }  # or "4" for HDDs with slower random access
  - { option: "effective_io_concurrency", value: "200" }  # or "2" for traditional HDDs with lower I/O parallelism
  - { option: "synchronous_commit", value: "on" }  # or 'off' if you can you lose single transactions in case of a crash
  - { option: "autovacuum", value: "on" }  # never turn off the autovacuum!
  - { option: "autovacuum_max_workers", value: "5" }
  - { option: "autovacuum_vacuum_scale_factor", value: "0.01" }  # or 0.005/0.001
  - { option: "autovacuum_analyze_scale_factor", value: "0.01" }
  - { option: "autovacuum_vacuum_cost_limit", value: "500" }  # or 1000/5000
  - { option: "autovacuum_vacuum_cost_delay", value: "2" }
  - { option: "autovacuum_naptime", value: "1s" }
  - { option: "max_files_per_process", value: "4096" }
  - { option: "archive_mode", value: "on" }
  - { option: "archive_timeout", value: "1800s" }
  - { option: "archive_command", value: "cd ." }  # not doing anything yet with WAL-s
  - { option: "wal_level", value: "logical" }
  - { option: "wal_keep_size", value: "2GB" }
  - { option: "max_wal_senders", value: "10" }
  - { option: "max_replication_slots", value: "10" }
  - { option: "hot_standby", value: "on" }
  - { option: "wal_log_hints", value: "on" }
  - { option: "wal_compression", value: "on" }
  - { option: "shared_preload_libraries", value: "pg_stat_statements,auto_explain" }
  - { option: "pg_stat_statements.max", value: "10000" }
  - { option: "pg_stat_statements.track", value: "all" }
  - { option: "pg_stat_statements.track_utility", value: "false" }
  - { option: "pg_stat_statements.save", value: "true" }
  - { option: "auto_explain.log_min_duration", value: "10s" }  # enable auto_explain for 10-second logging threshold. Decrease this value if necessary
  - { option: "auto_explain.log_analyze", value: "true" }
  - { option: "auto_explain.log_buffers", value: "true" }
  - { option: "auto_explain.log_timing", value: "false" }
  - { option: "auto_explain.log_triggers", value: "true" }
  - { option: "auto_explain.log_verbose", value: "true" }
  - { option: "auto_explain.log_nested_statements", value: "true" }
  - { option: "auto_explain.sample_rate", value: "0.01" }  # enable auto_explain for 1% of queries logging threshold
  - { option: "track_io_timing", value: "on" }
  - { option: "log_lock_waits", value: "on" }
  - { option: "log_temp_files", value: "0" }
  - { option: "track_activities", value: "on" }
  - { option: "track_activity_query_size", value: "4096" }
  - { option: "track_counts", value: "on" }
  - { option: "track_functions", value: "all" }
  - { option: "log_checkpoints", value: "on" }
  - { option: "logging_collector", value: "on" }
  - { option: "log_truncate_on_rotation", value: "on" }
  - { option: "log_rotation_age", value: "1d" }
  - { option: "log_rotation_size", value: "0" }
  - { option: "log_line_prefix", value: "'%t [%p-%l] %r %q%u@%d '" }
  - { option: "log_filename", value: "postgresql-%a.log" }
  - { option: "log_directory", value: "{{ postgresql_log_dir }}" }
  - { option: "hot_standby_feedback", value: "on" }  # allows feedback from a hot standby to the primary that will avoid query conflicts
  - { option: "max_standby_streaming_delay", value: "30s" }
  - { option: "wal_receiver_status_interval", value: "10s" }
  - { option: "idle_in_transaction_session_timeout", value: "10min" }  # reduce this timeout if possible
  - { option: "jit", value: "off" }
  - { option: "max_worker_processes", value: "24" }
  - { option: "max_parallel_workers", value: "8" }
  - { option: "max_parallel_workers_per_gather", value: "2" }
  - { option: "max_parallel_maintenance_workers", value: "2" }
  - { option: "tcp_keepalives_count", value: "10" }
  - { option: "tcp_keepalives_idle", value: "300" }
  - { option: "tcp_keepalives_interval", value: "30" }

# Set this variable to 'true' if you want the cluster to be automatically restarted
# after changing the 'postgresql_parameters' variable that requires a restart in the 'config_pgcluster.yml' playbook.
# By default, the cluster will not be automatically restarted.
#pending_restart: false
pending_restart: true # gramic, maintenance test

# specify additional hosts that will be added to the pg_hba.conf
postgresql_pg_hba:
  - { type: "local", database: "all", user: "{{ patroni_superuser_username }}", address: "", method: "trust" }
  - { type: "local", database: "all", user: "{{ pgbouncer_auth_username }}", address: "", method: "trust" } # required for pgbouncer auth_user
  - { type: "local", database: "replication", user: "all", address: "", method: "trust" }
  - { type: "host", database: "replication", user: "all", address: "127.0.0.1/32", method: "trust" }
  - { type: "host", database: "replication", user: "all", address: "::1/128", method: "trust" }
  - { type: "host", database: "replication", user: "{{ patroni_replication_username }}", address: "10.0.22.173/24", method: "{{ postgresql_password_encryption_algorithm }}" }
  - { type: "host", database: "replication", user: "{{ patroni_replication_username }}", address: "10.0.22.174/24", method: "{{ postgresql_password_encryption_algorithm }}" }
  - { type: "host", database: "replication", user: "{{ patroni_replication_username }}", address: "10.0.22.175/24", method: "{{ postgresql_password_encryption_algorithm }}" }
  - { type: "host", database: "replication", user: "{{ patroni_replication_username }}", address: "10.0.28.16/24", method: "{{ postgresql_password_encryption_algorithm }}" } # gramic, maintenance test
  - { type: "local", database: "all", user: "all", address: "", method: "{{ postgresql_password_encryption_algorithm }}" }
  - { type: "host", database: "all", user: "all", address: "127.0.0.1/32", method: "{{ postgresql_password_encryption_algorithm }}" }
  - { type: "host", database: "all", user: "all", address: "::1/128", method: "{{ postgresql_password_encryption_algorithm }}" }
  - { type: "host", database: "k8s_core_gitlab_prod", user: "xksgr_sks1160_gitlab", address: "10.0.20.88/24", method: "{{ postgresql_password_encryption_algorithm }}" }
  - { type: "host", database: "k8s_core_harbor_prod", user: "xksgr_sks1172_harbor", address: "10.0.20.92/24", method: "{{ postgresql_password_encryption_algorithm }}" }
  - { type: "host", database: "k8s_core_keycloak_prod", user: "xksgr_sks1195_kcsso", address: "10.0.20.98/24", method: "{{ postgresql_password_encryption_algorithm }}" }
  - { type: "host", database: "all", user: "{{ patroni_superuser_username }}", address: "10.0.20.63/24", method: "{{ postgresql_password_encryption_algorithm }}" }
  - { type: "host", database: "all", user: "{{ patroni_superuser_username }}", address: "10.0.20.43/24", method: "{{ postgresql_password_encryption_algorithm }}" }
  - { type: "host", database: "all", user: "{{ patroni_superuser_username }}", address: "10.0.20.77/24", method: "{{ postgresql_password_encryption_algorithm }}" }
  - { type: "host", database: "all", user: "{{ patroni_superuser_username }}", address: "10.0.28.16/24", method: "{{ postgresql_password_encryption_algorithm }}" } # gramic, maintenance test

# list of lines that Patroni will use to generate pg_ident.conf
postgresql_pg_ident: []

# the password file (~/.pgpass)
postgresql_pgpass:
  - "localhost:{{ postgresql_port }}:*:{{ patroni_superuser_username }}:{{ patroni_superuser_password }}"
  - "{{ inventory_hostname }}:{{ postgresql_port }}:*:{{ patroni_superuser_username }}:{{ patroni_superuser_password }}"
  - "*:{{ pgbouncer_listen_port }}:*:{{ patroni_superuser_username }}:{{ patroni_superuser_password }}"
  - "*:{{ postgresql_port }}:*:{{ patroni_replication_username }}:{{ patroni_replication_password }}"
  - "10.0.20.88:5432:k8s_core_gitlab_prod:xksgr_sks1160_gitlab:<Password Secure / Safe>"
  - "10.0.20.92:5432:k8s_core_harbor_prod:xksgr_sks1172_harbor:<Password Secure / Safe>"
  - "10.0.20.98:5432:k8s_core_keycloak_prod:xksgr_sks1195_kcsso:<Password Secure / Safe>"

# PgBouncer parameters
pgbouncer_install: true  # or 'false' if you do not want to install and configure the pgbouncer service
pgbouncer_processes: 1  # Number of pgbouncer processes to be used. Multiple processes use the so_reuseport option for better performance.
pgbouncer_conf_dir: "/etc/pgbouncer"
pgbouncer_log_dir: "/var/log/pgbouncer"
pgbouncer_listen_addr: "0.0.0.0" # Listen on all interfaces. Or use "{{ inventory_hostname }}" to listen on a specific IP address.
pgbouncer_listen_port: 6432
pgbouncer_max_client_conn: 10000
pgbouncer_max_db_connections: 1000
pgbouncer_max_prepared_statements: 1024
pgbouncer_default_pool_size: 20
pgbouncer_query_wait_timeout: 120
pgbouncer_default_pool_mode: "session"
pgbouncer_admin_users: "{{ patroni_superuser_username }}"  # comma-separated list of users, who are allowed to change settings
pgbouncer_stats_users: "{{ patroni_superuser_username }}"  # comma-separated list of users who are just allowed to use SHOW command
pgbouncer_ignore_startup_parameters: "extra_float_digits,geqo,search_path"
pgbouncer_auth_type: "{{ postgresql_password_encryption_algorithm }}"
pgbouncer_auth_user: true # or 'false' if you want to manage the list of users for authentication in the database via userlist.txt
pgbouncer_auth_username: pgbouncer # user who can query the database via the user_search function
pgbouncer_auth_password: "<Password Secure / Safe>"
pgbouncer_auth_dbname: "postgres"
pgbouncer_client_tls_sslmode: "disable"
pgbouncer_client_tls_key_file: ""
pgbouncer_client_tls_cert_file: ""
pgbouncer_client_tls_ca_file: ""
pgbouncer_client_tls_protocols: "secure" # allowed values: tlsv1.0, tlsv1.1, tlsv1.2, tlsv1.3, all, secure (tlsv1.2,tlsv1.3)
pgbouncer_client_tls_ciphers: "default" # allowed values: default, secure, fast, normal, all (not recommended)

pgbouncer_pools:
  - { name: "postgres", dbname: "postgres", pool_parameters: "" }
  - { name: "k8s_core_gitlab_prod", dbname: "k8s_core_gitlab_prod", pool_parameters: "" }
  - { name: "k8s_core_harbor_prod", dbname: "k8s_core_harbor_prod", pool_parameters: "" }
  - { name: "k8s_core_keycloak_prod", dbname: "k8s_core_keycloak_prod", pool_parameters: "" }
  - { name: "gramic_test", dbname: "gramic_test", pool_parameters: "" }

# Extended variables (optional)
#patroni_restapi_listen_addr: "0.0.0.0" # Listen on all interfaces. Or use "{{ inventory_hostname }}" to listen on a specific IP address.
patroni_restapi_listen_addr: "{{ inventory_hostname }}" # gramic, maintenance test
patroni_restapi_port: 8008
patroni_ttl: 30
patroni_loop_wait: 10
patroni_retry_timeout: 10
patroni_master_start_timeout: 300
patroni_maximum_lag_on_failover: 1048576 # (1MB) the maximum bytes a follower may lag to be able to participate in leader election.
patroni_maximum_lag_on_replica: "100MB" # the maximum of lag that replica can be in order to be available for read-only queries.

# https://patroni.readthedocs.io/en/latest/yaml_configuration.html#postgresql
patroni_callbacks: []

# https://patroni.readthedocs.io/en/latest/replica_bootstrap.html#standby-cluster
# Requirements:
# 1. the cluster name for Standby Cluster must be unique ('patroni_cluster_name' variable)
# 2. the IP addresses (or network) of the Standby Cluster servers must be added to the pg_hba.conf of the Main Cluster ('postgresql_pg_hba' variable).
patroni_standby_cluster:
  host: ""  # an address of remote master
  port: "5432"  # a port of remote master
#  primary_slot_name: ""  # which slot on the remote master to use for replication (optional)
#  restore_command: ""  # command to restore WAL records from the remote master to standby leader (optional)
#  recovery_min_apply_delay: ""  # how long to wait before actually apply WAL records on a standby leader (optional)

# Permanent replication slots.
# These slots will be preserved during switchover/failover.
# https://patroni.readthedocs.io/en/latest/dynamic_configuration.html
patroni_slots: []
#  - slot: "logical_replication_slot" # the name of the permanent replication slot.
#    type: "logical" # the type of slot. Could be 'physical' or 'logical' (if the slot is logical, you have to define 'database' and 'plugin').
#    plugin: "pgoutput" # the plugin name for the logical slot.
#    database: "postgres" # the database name where logical slots should be created.
#  - slot: "test_logical_replication_slot"
#    type: "logical"
#    plugin: "pgoutput"
#    database: "test"

patroni_log_destination: stderr  # or 'logfile'
# if patroni_log_destination: logfile
patroni_log_dir: /var/log/patroni
patroni_log_level: info
patroni_log_traceback_level: error
patroni_log_format: "%(asctime)s %(levelname)s: %(message)s"
patroni_log_dateformat: ""
patroni_log_max_queue_size: 1000
patroni_log_file_num: 4
patroni_log_file_size: 25000000  # bytes
patroni_log_loggers_patroni_postmaster: warning
patroni_log_loggers_urllib3: warning  # or 'debug'

patroni_watchdog_mode: automatic  # or 'off', 'required'
patroni_watchdog_device: /dev/watchdog

patroni_postgresql_use_pg_rewind: true  # or 'false'
# try to use pg_rewind on the former leader when it joins cluster as a replica.

patroni_remove_data_directory_on_rewind_failure: false  # or 'true' (if use_pg_rewind: 'true')
# avoid removing the data directory on an unsuccessful rewind
# if 'true', Patroni will remove the PostgreSQL data directory and recreate the replica.

patroni_remove_data_directory_on_diverged_timelines: false  # or 'true'
# if 'true', Patroni will remove the PostgreSQL data directory and recreate the replica
# if it notices that timelines are diverging and the former master can not start streaming from the new master.

# https://patroni.readthedocs.io/en/latest/replica_bootstrap.html#bootstrap
patroni_cluster_bootstrap_method: "initdb"  # or "wal-g", "pgbackrest", "pg_probackup"

# https://patroni.readthedocs.io/en/latest/replica_bootstrap.html#building-replicas
patroni_create_replica_methods:
#  - pgbackrest
#  - wal_g
#  - pg_probackup
  - basebackup

pgbackrest:
  - { option: "command", value: "/usr/bin/pgbackrest --stanza={{ pgbackrest_stanza }} --delta restore" }
  - { option: "keep_data", value: "True" }
  - { option: "no_params", value: "True" }
wal_g:
  - { option: "command", value: "wal-g backup-fetch {{ postgresql_data_dir }} LATEST" }
  - { option: "no_params", value: "True" }
basebackup:
  - { option: "max-rate", value: "100M" }
  - { option: "checkpoint", value: "fast" }
#  - { option: "waldir", value: "{{ postgresql_wal_dir }}" }
pg_probackup:
  - { option: "command", value: "{{ pg_probackup_restore_command }}" }
  - { option: "no_params", value: "true" }

# "restore_command" written to recovery.conf when configuring follower (create replica)
postgresql_restore_command: ""

# pg_probackup
pg_probackup_install: false  # or 'true'
pg_probackup_install_from_postgrespro_repo: true  # or 'false'
pg_probackup_version: "{{ postgresql_version }}"
pg_probackup_instance: "pg_probackup_instance_name"
pg_probackup_dir: "/mnt/backup_dir"
pg_probackup_threads: "4"
pg_probackup_add_keys: "--recovery-target=latest --skip-external-dirs --no-validate"
# Ensure there is a space at the beginning of each part to prevent commands from concatenating.
pg_probackup_command_parts:
  - "pg_probackup-{{ pg_probackup_version }}"
  - " restore -B {{ pg_probackup_dir }}"
  - " --instance {{ pg_probackup_instance }}"
  - " -j {{ pg_probackup_threads }}"
  - " {{ pg_probackup_add_keys }}"
pg_probackup_restore_command: "{{ pg_probackup_command_parts | join('') }}"
pg_probackup_patroni_cluster_bootstrap_command: "{{ pg_probackup_command_parts | join('') }}"

# WAL-G
wal_g_install: false  # or 'true'
wal_g_version: "3.0.0"
wal_g_json:  # config https://github.com/wal-g/wal-g#configuration
  - { option: "AWS_ACCESS_KEY_ID", value: "{{ AWS_ACCESS_KEY_ID | default('') }}" }  # define values or pass via --extra-vars
  - { option: "AWS_SECRET_ACCESS_KEY", value: "{{ AWS_SECRET_ACCESS_KEY | default('') }}" }  # define values or pass via --extra-vars
  - { option: "WALG_S3_PREFIX", value: "{{ WALG_S3_PREFIX | default('') }}" } # define values or pass via --extra-vars
  - { option: "WALG_COMPRESSION_METHOD", value: "brotli" }  # or "lz4", "lzma", "zstd"
  - { option: "WALG_DELTA_MAX_STEPS", value: "6" }  # determines how many delta backups can be between full backups
  - { option: "PGDATA", value: "{{ postgresql_data_dir }}" }
  - { option: "PGHOST", value: "{{ postgresql_unix_socket_dir }}" }
  - { option: "PGPORT", value: "{{ postgresql_port }}" }
  - { option: "PGUSER", value: "{{ patroni_superuser_username }}" }
wal_g_archive_command: "wal-g wal-push %p"
wal_g_patroni_cluster_bootstrap_command: "wal-g backup-fetch {{ postgresql_data_dir }} LATEST"

# Define job_parts outside of wal_g_cron_jobs
# Ensure there is a space at the beginning of each part to prevent commands from concatenating.
wal_g_backup_command:
  - "[ $(curl -s -o /dev/null -w '%{http_code}' http://{{ inventory_hostname }}:{{ patroni_restapi_port }}) = '200' ]"
  - " && wal-g backup-push {{ postgresql_data_dir }} > {{ postgresql_log_dir }}/walg_backup.log 2>&1"
wal_g_delete_command:
  - "[ $(curl -s -o /dev/null -w '%{http_code}' http://{{ inventory_hostname }}:{{ patroni_restapi_port }}) = '200' ]"
  - " && wal-g delete retain FULL 4 --confirm > {{ postgresql_log_dir }}/walg_delete.log 2>&1"

wal_g_cron_jobs:
  - name: "WAL-G: Create daily backup"
    user: "postgres"
    file: /etc/cron.d/walg
    minute: "30"
    hour: "3"
    day: "*"
    month: "*"
    weekday: "*"
    job: "{{ wal_g_backup_command | join('') }}"
  - name: "WAL-G: Delete old backups" # retain 4 full backups (adjust according to your company's backup retention policy)
    user: "postgres"
    file: /etc/cron.d/walg
    minute: "30"
    hour: "6"
    day: "*"
    month: "*"
    weekday: "*"
    job: "{{ wal_g_delete_command | join('') }}"

# pgBackRest
pgbackrest_install: false  # or 'true'
pgbackrest_install_from_pgdg_repo: false  # or 'false'
pgbackrest_stanza: "{{ patroni_cluster_name }}"  # specify your --stanza
pgbackrest_repo_type: "posix"  # or "s3", "gcs", "azure"
pgbackrest_repo_host: ""  # dedicated repository host (optional)
pgbackrest_repo_user: "postgres"
pgbackrest_conf_file: "/etc/pgbackrest/pgbackrest.conf"
# config https://pgbackrest.org/configuration.html
pgbackrest_conf:
  global:  # [global] section
    - { option: "log-level-file", value: "detail" }
    - { option: "log-path", value: "/var/log/pgbackrest" }
    - { option: "repo1-type", value: "{{ pgbackrest_repo_type | lower }}" }
    - { option: "repo1-path", value: "/var/lib/pgbackrest" }
    - { option: "repo1-retention-full", value: "4" }
    - { option: "repo1-retention-archive", value: "4" }
    - { option: "start-fast", value: "y" }
    - { option: "stop-auto", value: "y" }
    - { option: "resume", value: "n" }
    - { option: "link-all", value: "y" }
    - { option: "spool-path", value: "/var/spool/pgbackrest" }
    - { option: "archive-async", value: "y" } # Enables asynchronous WAL archiving (details: https://pgbackrest.org/user-guide.html#async-archiving)
    - { option: "archive-get-queue-max", value: "1GiB" }
  stanza:  # [stanza_name] section
    - { option: "process-max", value: "4" }
    - { option: "log-level-console", value: "info" }
    - { option: "recovery-option", value: "recovery_target_action=promote" }
    - { option: "pg1-socket-path", value: "{{ postgresql_unix_socket_dir }}" }
    - { option: "pg1-path", value: "{{ postgresql_data_dir }}" }
#    - { option: "", value: "" }
# (optional) dedicated backup server config (if "repo_host" is set)
pgbackrest_server_conf:
  global:
    - { option: "log-level-file", value: "detail" }
    - { option: "log-level-console", value: "info" }
    - { option: "log-path", value: "/var/log/pgbackrest" }
    - { option: "repo1-type", value: "{{ pgbackrest_repo_type | lower }}" }
    - { option: "repo1-path", value: "/var/lib/pgbackrest" }
    - { option: "repo1-retention-full", value: "4" }
    - { option: "repo1-retention-archive", value: "4" }
    - { option: "repo1-bundle", value: "y" }
    - { option: "repo1-block", value: "y" }
    - { option: "start-fast", value: "y" }
    - { option: "stop-auto", value: "y" }
    - { option: "resume", value: "n" }
    - { option: "link-all", value: "y" }
    - { option: "archive-check", value: "y" }
    - { option: "archive-copy", value: "n" }
    - { option: "backup-standby", value: "y" }
#    - { option: "", value: "" }
# the stanza section will be generated automatically

pgbackrest_archive_command: "pgbackrest --stanza={{ pgbackrest_stanza }} archive-push %p"

pgbackrest_patroni_cluster_restore_command:
  '/usr/bin/pgbackrest --stanza={{ pgbackrest_stanza }} --delta restore'  # restore from latest backup
#  '/usr/bin/pgbackrest --stanza={{ pgbackrest_stanza }} --type=time "--target=2020-06-01 11:00:00+03" --delta restore'  # Point-in-Time Recovery (example)

# By default, the cron jobs is created on the database server.
# If 'repo_host' is defined, the cron jobs will be created on the pgbackrest server.
pgbackrest_cron_jobs:
  - name: "pgBackRest: Full Backup"
    file: "/etc/cron.d/pgbackrest-{{ patroni_cluster_name }}"
    user: "postgres"
    minute: "30"
    hour: "6"
    day: "*"
    month: "*"
    weekday: "0"
    job: "pgbackrest --type=full --stanza={{ pgbackrest_stanza }} backup"
    # job: "if [ $(psql -tAXc 'select pg_is_in_recovery()') = 'f' ]; then pgbackrest --type=full --stanza={{ pgbackrest_stanza }} backup; fi"
  - name: "pgBackRest: Diff Backup"
    file: "/etc/cron.d/pgbackrest-{{ patroni_cluster_name }}"
    user: "postgres"
    minute: "30"
    hour: "6"
    day: "*"
    month: "*"
    weekday: "1-6"
    job: "pgbackrest --type=diff --stanza={{ pgbackrest_stanza }} backup"
    # job: "if [ $(psql -tAXc 'select pg_is_in_recovery()') = 'f' ]; then pgbackrest --type=diff --stanza={{ pgbackrest_stanza }} backup; fi"

# PITR mode (if patroni_cluster_bootstrap_method: "pgbackrest" or "wal-g"):
# 1) The database cluster directory will be cleaned (for "wal-g") or overwritten (for "pgbackrest" --delta restore).
# 2) And also the patroni cluster "{{ patroni_cluster_name }}" will be removed from the DCS (if exist) before recovery.

disable_archive_command: true  # or 'false' to not disable archive_command after restore
keep_patroni_dynamic_json: true  # or 'false' to remove patroni.dynamic.json after restore (if exists)


# Netdata - https://github.com/netdata/netdata
netdata_install: false  # or 'true' for install Netdata on postgresql cluster nodes (with kickstart.sh)
netdata_install_options: "--stable-channel --disable-telemetry --dont-wait"
netdata_conf:
  web_bind_to: "*"
  # https://learn.netdata.cloud/docs/store/change-metrics-storage
  memory_mode: "dbengine"  # The long-term metrics storage with efficient RAM and disk usage.
  page_cache_size: 64  # Determines the amount of RAM in MiB that is dedicated to caching Netdata metric values.
  dbengine_disk_space: 1024  # Determines the amount of disk space in MiB that is dedicated to storing Netdata metric values.

...

    \end{lstlisting}
    Nun muss ebenfalls das Playbook ausgeführt werden:
    \lstset{style=gra_codestyle}
    \begin{lstlisting}[language=bash, caption=Testsystem - Anhang - Maintenance - config\_pgcluster.yml,captionpos=b,label={lst:config-appendix-pgcluster},breaklines=true]
ansible-playbook config_pgcluster.yml
    \end{lstlisting}
    \subsection{Prequenteries - Erweiterungstests}
    Auf dem \texttt{sks9016} muss zuerst der apt-Proxy gesetzt werden:
    \lstset{style=gra_codestyle}
    \begin{lstlisting}[language=bash, caption=Testsystem - sks9016 - apt-Proxy Settings,captionpos=b,label={lst:testsystem-installation-sks9016-apt-proxy-settings},breaklines=true]
# nano /etc/apt/apt.conf.d/proxy.conf
Acauire::http::Proxy "http://sproxy.sivc.first-it.ch:8080";
Acauire::https::Proxy "http://sproxy.sivc.first-it.ch:8080";
Acquire::http::proxy::foreman.ksgr.ch "DIRECT";
    \end{lstlisting}
    Auch der Proxy muss gesetzt werden:
    Als nächstes muss der Proxy gesetzt werden:
    \lstset{style=gra_codestyle}
    \begin{lstlisting}[language=bash, caption=Testsystem - sks9016 - Proxy Settings,captionpos=b,label={lst:testsystem-installation-sks9016-proxy-settings},breaklines=true]
# nano /etc/profile.d/proxy.sh
export https_proxy=http://sproxy.sivc.first-it.ch:8080
export HTTPS_PROXY=http://sproxy.sivc.first-it.ch:8080
export http_proxy=http://sproxy.sivc.first-it.ch:8080
export HTTP_PROXY=http://sproxy.sivc.first-it.ch:8080
export no_proxy=localhost,127.0.0.0/8,::1,10.0.0.0/8,172.16.0.0/12,192.168.0.0/16
export NO_PROXY=localhost,127.0.0.0/8,::1,10.0.0.0/8,172.16.0.0/12,192.168.0.0/16
# source /etc/profile.d/proxy.sh
    \end{lstlisting}
    Firewall war keine auf dem Host installiert
    \subsection{Haproxy erweitern}
    Zuerst musste das \texttt{Inventory} um den entsprechenden Node erweitert werden:
    \lstset{style=gra_codestyle}
    \begin{lstlisting}[language=bash, caption=Testsystem - Deployment - inventory,captionpos=b,label={lst:testsystem-deployment-inventory},breaklines=true]

    \end{lstlisting}
    \subsection{Patroni Node}
\end{flushleft}