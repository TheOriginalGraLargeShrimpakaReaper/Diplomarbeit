%! Author = itgramic
%! Date = 24.01.24

% Preamble
\section{minikube}
\subsection{Vorbereitung}
\subsubsection{Proxy-Settings}
\subsubsection{Debian Update}
\subsubsection{Podman}
Das KSGR arbeitet bei seinen k8s Systemen mit Podman und nicht mit Docker.
Daher wird auch für die Evaluierungs-Enviroments mit Podman gearbeitet.
Die Installation muss wie folgt durchgeführt werden:
\lstset{style=gra_codestyle}
\begin{lstlisting}[language=bash, caption=Podman installieren,captionpos=b,label={lst:podman-install},breaklines=true]
sudo apt-get -y install podman
\end{lstlisting}
\subsection{Installation}
\lstset{style=gra_codestyle}
\begin{lstlisting}[language=bash, caption=minikube installieren,captionpos=b,label={lst:minikube-install},breaklines=true]
curl -LO https://storage.googleapis.com/minikube/releases/latest/minikube_latest_amd64.deb
sudo dpkg -i minikube_latest_amd64.deb
\end{lstlisting}
\subsection{minikube starten}
minikube und Podman können nicht ohne weiteres gestartet werden.
Es gibt einen alten Bug, der immer noch auftritt\cite{8HWWRGUF}.
\begin{lstlisting}[language=bash, caption=minikube start issue,captionpos=b,label={lst:minikube-start-issue},breaklines=true]
itgramic@sks1183:~$ minikube start --driver=podman --container-runtime=cri-o
  minikube v1.32.0 on Debian 12.4
  Using the podman driver based on user configuration
  Using Podman driver with root privileges
  Starting control plane node minikube in cluster minikube
  Pulling base image ...
  Downloading Kubernetes v1.28.3 preload ...
    > preloaded-images-k8s-v18-v1...:  436.67 MiB / 436.67 MiB  100.00% 19.22 M
    > gcr.io/k8s-minikube/kicbase...:  453.90 MiB / 453.90 MiB  100.00% 12.52 M
E0125 13:48:43.294095  260427 cache.go:189] Error downloading kic artifacts:  not yet implemented, see issue #8426
  Creating podman container (CPUs=2, Memory=3900MB) ...
  StartHost failed, but will try again: creating host: create: creating: setting up container node: preparing volume for minikube container: sudo -n podman run --rm --name minikube-preload-sidecar --label created_by.minikube.sigs.k8s.io=true --label name.minikube.sigs.k8s.io=minikube --entrypoint /usr/bin/test -v minikube:/var gcr.io/k8s-minikube/kicbase:v0.0.42 -d /var/lib: exit status 125
stdout:

stderr:
Trying to pull gcr.io/k8s-minikube/kicbase:v0.0.42...
time="2024-01-25T13:48:44+01:00" level=warning msg="Failed, retrying in 1s ... (1/3). Error: initializing source docker://gcr.io/k8s-minikube/kicbase:v0.0.42: pinging container registry gcr.io: Get \"https://gcr.io/v2/\": dial tcp 142.251.18.82:443: connect: connection refused"
time="2024-01-25T13:48:45+01:00" level=warning msg="Failed, retrying in 1s ... (2/3). Error: initializing source docker://gcr.io/k8s-minikube/kicbase:v0.0.42: pinging container registry gcr.io: Get \"https://gcr.io/v2/\": dial tcp 142.251.18.82:443: connect: connection refused"
time="2024-01-25T13:48:46+01:00" level=warning msg="Failed, retrying in 1s ... (3/3). Error: initializing source docker://gcr.io/k8s-minikube/kicbase:v0.0.42: pinging container registry gcr.io: Get \"https://gcr.io/v2/\": dial tcp 142.251.18.82:443: connect: connection refused"
Error: initializing source docker://gcr.io/k8s-minikube/kicbase:v0.0.42: pinging container registry gcr.io: Get "https://gcr.io/v2/": dial tcp 142.251.18.82:443: connect: connection refused

  Restarting existing podman container for "minikube" ...
  Failed to start podman container. Running "minikube delete" may fix it: driver start: start: sudo -n podman start --cgroup-manager cgroupfs minikube: exit status 125
stdout:

stderr:
Error: no container with name or ID "minikube" found: no such container


x  Exiting due to GUEST_PROVISION: error provisioning guest: Failed to start host: driver start: start: sudo -n podman start --cgroup-manager cgroupfs minikube: exit status 125
stdout:

stderr:
Error: no container with name or ID "minikube" found: no such container
\end{lstlisting}

Eine Workaround besteht darin, zuerst den Driver herunterzuladen.
\lstset{style=gra_codestyle}
\begin{lstlisting}[language=bash, caption=minikube / Podman driver download,captionpos=b,label={lst:minikube-podman-driver-download},breaklines=true]
minikube start --driver=podman --download-only
\end{lstlisting}

Danach muss das entsprechende File geladen werden:
\lstset{style=gra_codestyle}
\begin{lstlisting}[language=bash, caption=minikube / Podman kicbase prüfen,captionpos=b,label={lst:minikube-podman-get-kicbase},breaklines=true]
minikube start --help | grep kicbase
\end{lstlisting}

Diese muss danach heruntergeladen werden:
\lstset{style=gra_codestyle}
\begin{lstlisting}[language=bash, caption=minikube / Podman kicbase laden,captionpos=b,label={lst:minikube-podman-load-kicbase},breaklines=true]
sudo podman load < ~/.minikube/cache/kic/amd64/kicbase_v0.0.42\@sha256_d35ac07dfda971cabee05e0deca8aeac772f885a5348e1a0c0b0a36db20fcfc0.tar
\end{lstlisting}

Jetzt sollte sich minikube mit Podman starten lassen:
\lstset{style=gra_codestyle}
\begin{lstlisting}[language=bash, caption=minikube / Podman Starten,captionpos=b,label={lst:minikube-start-podman},breaklines=true]
minikube start --driver=podman
\end{lstlisting}

\lstset{style=gra_codestyle}
\begin{lstlisting}[language=bash, caption=minikube / Podman gestartet,captionpos=b,label={lst:minikube-started-podman},breaklines=true]
itgramic@sks1183:~$ minikube start --driver=podman
  minikube v1.32.0 on Debian 12.4
  Using the podman driver based on existing profile
  Starting control plane node minikube in cluster minikube
  Pulling base image ...
E0125 14:10:09.406266  261745 cache.go:189] Error downloading kic artifacts:  not yet implemented, see issue #8426
  Creating podman container (CPUs=2, Memory=3900MB) ...
  Found network options:
    * HTTP_PROXY=http://sproxy.sivc.first-it.ch:8080
!  You appear to be using a proxy, but your NO_PROXY environment does not include the minikube IP (192.168.49.2).
  Please see https://minikube.sigs.k8s.io/docs/handbook/vpn_and_proxy/ for more details
    * HTTPS_PROXY=http://sproxy.sivc.first-it.ch:8080
    * NO_PROXY=localhost,127.0.0.0/8,::1,10.0.0.0/8
    * HTTP_PROXY=http://sproxy.sivc.first-it.ch:8080
    * HTTPS_PROXY=http://sproxy.sivc.first-it.ch:8080
    * NO_PROXY=localhost,127.0.0.0/8,::1,10.0.0.0/8
!  This container is having trouble accessing https://registry.k8s.io
  To pull new external images, you may need to configure a proxy: https://minikube.sigs.k8s.io/docs/reference/networking/proxy/
  Preparing Kubernetes v1.28.3 on Docker 24.0.7 ...
    * env HTTP_PROXY=http://sproxy.sivc.first-it.ch:8080
    * env HTTPS_PROXY=http://sproxy.sivc.first-it.ch:8080
    * env NO_PROXY=localhost,127.0.0.0/8,::1,10.0.0.0/8
    * Generating certificates and keys ...
    * Booting up control plane ...
    * Configuring RBAC rules ...
  Configuring bridge CNI (Container Networking Interface) ...
  Verifying Kubernetes components...
    * Using image gcr.io/k8s-minikube/storage-provisioner:v5
  Enabled addons: storage-provisioner, default-storageclass
  kubectl not found. If you need it, try: 'minikube kubectl -- get pods -A'
  Done! kubectl is now configured to use "minikube" cluster and "default" namespace by default
\end{lstlisting}


