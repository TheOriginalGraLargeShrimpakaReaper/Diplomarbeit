%! Author = ibw
%! Date = 10.11.23

% Preamble
\subsection{pgpool-II}
\subsubsection{\Gls{PostgreSQL Cluster} Installation}
\textbf{\Gls{PostgreSQL} Package Repository in Debian einbinden}
\lstset{style=gra_codestyle}
\begin{lstlisting}[language=bash, caption=\gls{openSUSE} - Netzwerk Settings,captionpos=b,label={lst:opensuse-network-setting},breaklines=true]
linux-4fi4:~ # ip a
1: lo: <LOOPBACK,UP,LOWER_UP> mtu 65536 qdisc noqueue state UNKNOWN group default qlen 1
    link/loopback 00:00:00:00:00:00 brd 00:00:00:00:00:00
    inet 127.0.0.1/8 scope host lo
       valid_lft forever preferred_lft forever
    inet6 ::1/128 scope host
       valid_lft forever preferred_lft forever
2: eth0: <BROADCAST,MULTICAST,UP,LOWER_UP> mtu 1500 qdisc pfifo_fast state UP group default qlen 1000
    link/ether 00:50:56:33:d8:e4 brd ff:ff:ff:ff:ff:ff
    inet 172.23.150.140/20 brd 172.23.159.255 scope global eth0
       valid_lft forever preferred_lft forever
    inet6 fe80::250:56ff:fe33:d8e4/64 scope link
       valid_lft forever preferred_lft forever
3: eth1: <BROADCAST,MULTICAST,UP,LOWER_UP> mtu 1500 qdisc pfifo_fast state UP group default qlen 1000
    link/ether 00:0c:29:e0:2f:ec brd ff:ff:ff:ff:ff:ff
    inet 192.168.210.1/24 brd 192.168.210.255 scope global eth1
       valid_lft forever preferred_lft forever
    inet6 fe80::20c:29ff:fee0:2fec/64 scope link
       valid_lft forever preferred_lft forever
\end{lstlisting}