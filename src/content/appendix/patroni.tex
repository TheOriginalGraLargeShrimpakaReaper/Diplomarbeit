%! Author = gra
%! Date = 05.04.24

% Preamble
\begin{flushleft}
    \section{Patroni}
    \subsection{Prerequisites}
\end{flushleft}
\begin{flushleft}
    Zuerst muss der Proxy gesetzt werden:
    \lstset{style=gra_codestyle}
\begin{lstlisting}[language=bash, caption=Patroni - Proxy Settings,captionpos=b,label={lst:patroni-proxy-settings},breaklines=true]
# sks1232 / sks1233 / sks1234
# Proxy setzen
# nano /etc/profile.d/proxy.sh
export https_proxy=http://sproxy.sivc.first-it.ch:8080
export HTTPS_PROXY=http://sproxy.sivc.first-it.ch:8080
export http_proxy=http://sproxy.sivc.first-it.ch:8080
export HTTP_PROXY=http://sproxy.sivc.first-it.ch:8080
export no_proxy=localhost,127.0.0.0/8,::1,10.0.0.0/8,172.16.0.0/12,192.168.0.0/16
export NO_PROXY=localhost,127.0.0.0/8,::1,10.0.0.0/8,172.16.0.0/12,192.168.0.0/16
# source /etc/profile.d/proxy.sh
\end{lstlisting}
\end{flushleft}
\begin{flushleft}
    Damit das PostgreSQL-Repository eingebunden werden kann,\\
    muss dem apt-Proxy gesetzt werden.\\
    Da via \Gls{Foreman} Installiert wurde, muss dieser ausgenommen werden:
    \lstset{style=gra_codestyle}
\begin{lstlisting}[language=bash, caption=Patroni - apt-Proxy Settings,captionpos=b,label={lst:patroni-apt-proxy-settings},breaklines=true]
# sks1232 / sks1233 / sks1234
# apt-Proxy setzen
# nano /etc/apt/apt.conf.d/proxy.conf
Acauire::http::Proxy "http://sproxy.sivc.first-it.ch:8080";
Acauire::https::Proxy "http://sproxy.sivc.first-it.ch:8080";
Acquire::http::proxy::foreman.ksgr.ch "DIRECT";
\end{lstlisting}
\end{flushleft}
\begin{flushleft}
    Im nächsten Schritt kann das PostgreSQL-Repository eingebunden werden.\\
    \begin{warning}
    Achtung, die von PostgreSQL beschriebene Variante wurde in Debian 10 als Deprecated gesetzt,
    mit Debian 13 wird diese Repository-Integration einen Fehler werden.
    \end{warning}    \lstset{style=gra_codestyle}
\begin{lstlisting}[language=bash, caption=Patroni - PostgreSQL einbinden,captionpos=b,label={lst:patroni-include-repository},breaklines=true]
# sks1232 / sks1233 / sks1234
# PostgreSQL Repository einbinden
sudo sh -c 'echo "deb https://apt.postgresql.org/pub/repos/apt $(lsb_release -cs)-pgdg main" > /etc/apt/sources.list.d/pgdg.list'
wget --quiet -O - https://www.postgresql.org/media/keys/ACCC4CF8.asc | sudo apt-key add -

# Ausloggen und wieder einloggen
apt update
\end{lstlisting}
\end{flushleft}
\begin{flushleft}
    \gls{etcd} wird als nächstes Installiert.\\
    Hierzu muss zuerst das Repository von \Gls{GitHub} heruntergeladen werden:
\end{flushleft}
\begin{flushleft}
\end{flushleft}
\begin{flushleft}
\end{flushleft}
\begin{flushleft}
\end{flushleft}
\begin{flushleft}
    \subsection{Installation}
\end{flushleft}
\begin{flushleft}
\end{flushleft}
\begin{flushleft}
    \subsection{Konfiguration}
\end{flushleft}
\begin{flushleft}
\end{flushleft}