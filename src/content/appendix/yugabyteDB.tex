%! Author = ibw
%! Date = 13.12.23

% Preamble
\section{yugabyteDB}
%\subsubsection{minikube}
%
%\subsubsection{yugabyteDB Konfiguration}
\subsection{Prerequisites}
\subsubsection{Prometheus Daten}
\subsubsection{StorageClass setzen}
Zuerst muss die StorageClass und das PersistentVolume gesetzt werden:
\lstset{style=gra_codestyle}
\begin{lstlisting}[language=yaml, caption=yugabyteDB - StorageClass setzen,captionpos=b,label={lst:yugabytedb-storageclass-yaml},breaklines=true]
apiVersion: storage.k8s.io/v1
kind: StorageClass
metadata:
  name: yb-storage
provisioner: rancher.io/local-path
parameters:
  nodePath: /var/local-path-provisioner
volumeBindingMode: WaitForFirstConsumer
reclaimPolicy: Delete
---
apiVersion: v1
kind: PersistentVolume
metadata:
  name: yb-storage-pv
  labels:
    type: local
spec:
  accessModes:
    - ReadWriteOnce
  capacity:
    storage: 3Gi
  storageClassName: "yb-storage"
  hostPath:
    path: /var/local-path-provisioner
\end{lstlisting}

\lstset{style=gra_codestyle}
\begin{lstlisting}[language=bash, caption=yugabyteDB - StorageClass / PersistentVolume aktivieren,captionpos=b,label={lst:yugabytedb-storageclass-apply},breaklines=true]
gramic@cks4040:~$ kubectl apply -f /home/gramic/PycharmProjects/rke2_settings/yugabytedb/yugabytedb/storageclass.yaml
storageclass.storage.k8s.io/yb-storage unchanged
persistentvolume/yb-storage-pv created
\end{lstlisting}

\subsection{Installation}

\subsection{Rekonfiguration mit 300GiB Storage}