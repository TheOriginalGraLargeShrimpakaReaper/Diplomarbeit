%! Author = ibw
%! Date = 13.12.23

% Preamble
\section{yugabyteDB}
%\subsubsection{minikube}
%
%\subsubsection{yugabyteDB Konfiguration}
\subsection{Prerequisites}
\subsubsection{Prometheus Daten}
\subsubsection{StorageClass setzen}
Zuerst muss die StorageClass und das PersistentVolume gesetzt werden:
\lstset{style=gra_codestyle}
\begin{lstlisting}[language=yaml, caption=yugabyteDB - StorageClass setzen,captionpos=b,label={lst:yugabytedb-storageclass-yaml},breaklines=true]
apiVersion: storage.k8s.io/v1
kind: StorageClass
metadata:
  name: yb-storage
provisioner: rancher.io/local-path
parameters:
  nodePath: /var/local-path-provisioner
volumeBindingMode: WaitForFirstConsumer
reclaimPolicy: Delete
---
apiVersion: v1
kind: PersistentVolume
metadata:
  name: yb-storage-pv
  labels:
    type: local
spec:
  accessModes:
    - ReadWriteOnce
  capacity:
    storage: 3Gi
  storageClassName: "yb-storage"
  hostPath:
    path: /var/local-path-provisioner
\end{lstlisting}

\lstset{style=gra_codestyle}
\begin{lstlisting}[language=bash, caption=yugabyteDB - StorageClass / PersistentVolume aktivieren,captionpos=b,label={lst:yugabytedb-storageclass-apply},breaklines=true]
gramic@cks4040:~$ kubectl apply -f /home/gramic/PycharmProjects/rke2_settings/yugabytedb/yugabytedb/storageclass.yaml
storageclass.storage.k8s.io/yb-storage unchanged
persistentvolume/yb-storage-pv created
\end{lstlisting}

\subsection{Installation}
Zuerst muss ein Namespace erstellt werden:
\lstset{style=gra_codestyle}
\begin{lstlisting}[language=bash, caption=yugabyteDB - Namespace,captionpos=b,label={lst:yugabytedb-namespace},breaklines=true]
kubectl create namespace yb-platform
\end{lstlisting}

\lstset{style=gra_codestyle}
\begin{lstlisting}[language=yaml, caption=yugabyteDB - Helm Chart Manifest,captionpos=b,label={lst:yugabytedb-values.yaml},breaklines=true]
# Default values for yugabyte.
# This is a YAML-formatted file.
# Declare variables to be passed into your templates.
Component: "yugabytedb"

fullnameOverride: ""
nameOverride: ""

Image:
  repository: "yugabytedb/yugabyte"
  tag: 2.20.2.1-b3
  pullPolicy: IfNotPresent
  pullSecretName: ""

storage:
  ephemeral: false  # will not allocate PVs when true
  master:
    count: 1
    size: 3Gi
    storageClass: "yb-storage"
  tserver:
    count: 1
    size: 3Gi
    storageClass: "yb-storage"

resource:
  master:
    requests:
      cpu: "1"
      memory: 2Gi
    limits:
      cpu: "1"
      ## Ensure the 'memory' value is strictly in 'Gi' or 'G' format. Deviating from these formats
      ## may result in setting an incorrect value for the 'memory_limit_hard_bytes' flag.
      ## Avoid using floating numbers for the numeric part of 'memory'. Doing so may lead to
      ## the 'memory_limit_hard_bytes' being set to 0, as the function expects integer values.
      memory: 2Gi
  tserver:
    requests:
      cpu: "1"
      memory: 4Gi
    limits:
      cpu: "1"
      ## Ensure the 'memory' value is strictly in 'Gi' or 'G' format. Deviating from these formats
      ## may result in setting an incorrect value for the 'memory_limit_hard_bytes' flag.
      ## Avoid using floating numbers for the numeric part of 'memory'. Doing so may lead to
      ## the 'memory_limit_hard_bytes' being set to 0, as the function expects integer values.
      memory: 4Gi

replicas:
  master: 3
  tserver: 3
  ## Used to set replication factor when isMultiAz is set to true
  totalMasters: 3

partition:
  master: 0
  tserver: 0

updateStrategy:
  type: RollingUpdate

# Used in Multi-AZ setup
masterAddresses: ""

isMultiAz: false
AZ: ""

# Disable the YSQL
disableYsql: false

tls:
  # Set to true to enable the TLS.
  enabled: false
  nodeToNode: true
  clientToServer: true
  # Set to false to disallow any service with unencrypted communication from joining this cluster
  insecure: false
  # Set enabled to true to use cert-manager instead of providing your own rootCA
  certManager:
    enabled: false
    # Will create own ca certificate and issuer when set to true
    bootstrapSelfsigned: true
    # Use ClusterIssuer when set to true, otherwise use Issuer
    useClusterIssuer: false
    # Name of ClusterIssuer to use when useClusterIssuer is true
    clusterIssuer: cluster-ca
    # Name of Issuer to use when useClusterIssuer is false
    issuer: yugabyte-ca
    certificates:
      # The lifetime before cert-manager will issue a new certificate.
      # The re-issued certificates will not be automatically reloaded by the service.
      # It is necessary to provide some external means of restarting the pods.
      duration: 2160h # 90d
      renewBefore: 360h # 15d
      algorithm: RSA # ECDSA or RSA
      # Can be 2048, 4096 or 8192 for RSA
      # Or 256, 384 or 521 for ECDSA
      keySize: 2048

  ## When certManager.enabled=false, rootCA.cert and rootCA.key are used to generate TLS certs.
  ## When certManager.enabled=true and boostrapSelfsigned=true, rootCA is ignored.
  ## When certManager.enabled=true and bootstrapSelfsigned=false, only rootCA.cert is used
  ## to verify TLS certs generated and signed by the external provider.
  rootCA:
    cert: "LS0tLS1CRUdJTiBDRVJUSUZJQ0FURS0tLS0tCk1JSUM2VENDQWRHZ0F3SUJBZ0lCQVRBTkJna3Foa2lHOXcwQkFRc0ZBREFXTVJRd0VnWURWUVFERXd0WmRXZGgKWW5sMFpTQkVRakFlRncweE9UQXlNRGd3TURRd01qSmFGdzB5T1RBeU1EVXdNRFF3TWpKYU1CWXhGREFTQmdOVgpCQU1UQzFsMVoyRmllWFJsSUVSQ01JSUJJakFOQmdrcWhraUc5dzBCQVFFRkFBT0NBUThBTUlJQkNnS0NBUUVBCnVOMWF1aWc4b2pVMHM0OXF3QXhrT2FCaHkwcTlyaVg2akVyZWJyTHJOWDJOeHdWQmNVcWJkUlhVc3VZNS96RUQKUC9CZTNkcTFuMm9EQ2ZGVEwweGkyNFdNZExRcnJBMndCdzFtNHM1WmQzcEJ1U04yWHJkVVhkeUx6dUxlczJNbgovckJxcWRscXp6LzAyTk9TOE9SVFZCUVRTQTBSOFNMQ1RjSGxMQmRkMmdxZ1ZmemVXRlVObXhWQ2EwcHA5UENuCmpUamJJRzhJWkh5dnBkTyt3aURQM1Y1a1ZEaTkvbEtUaGUzcTFOeDg5VUNFcnRJa1pjSkYvWEs3aE90MU1sOXMKWDYzb2lVMTE1Q2svbGFGRjR6dWgrZk9VenpOVXRXeTc2RE92cm5pVGlaU0tQZDBBODNNa2l2N2VHaDVkV3owWgpsKzJ2a3dkZHJaRzVlaHhvbGhGS3pRSURBUUFCbzBJd1FEQU9CZ05WSFE4QkFmOEVCQU1DQXFRd0hRWURWUjBsCkJCWXdGQVlJS3dZQkJRVUhBd0VHQ0NzR0FRVUZCd01DTUE4R0ExVWRFd0VCL3dRRk1BTUJBZjh3RFFZSktvWkkKaHZjTkFRRUxCUUFEZ2dFQkFEQjVRbmlYd1ptdk52eG5VbS9sTTVFbms3VmhTUzRUZldIMHY4Q0srZWZMSVBTbwpVTkdLNXU5UzNEUWlvaU9SN1Vmc2YrRnk1QXljMmNUY1M2UXBxTCt0V1QrU1VITXNJNk9oQ05pQ1gvQjNKWERPCkd2R0RIQzBVOHo3aWJTcW5zQ2Rid05kajAyM0lwMHVqNE9DVHJ3azZjd0RBeXlwVWkwN2tkd28xYWJIWExqTnAKamVQMkwrY0hkc2dKM1N4WWpkK1kvei9IdmFrZG1RZDJTL1l2V0R3aU1SRDkrYmZXWkJVRHo3Y0QyQkxEVmU0aAp1bkFaK3NyelR2Sjd5dkVodzlHSDFyajd4Qm9VNjB5SUUrYSszK2xWSEs4WnBSV0NXMnh2eWNrYXJSKytPS2NKClFsL04wWExqNWJRUDVoUzdhOTdhQktTamNqY3E5VzNGcnhJa2tKST0KLS0tLS1FTkQgQ0VSVElGSUNBVEUtLS0tLQo="
    key: "LS0tLS1CRUdJTiBSU0EgUFJJVkFURSBLRVktLS0tLQpNSUlFcEFJQkFBS0NBUUVBdU4xYXVpZzhvalUwczQ5cXdBeGtPYUJoeTBxOXJpWDZqRXJlYnJMck5YMk54d1ZCCmNVcWJkUlhVc3VZNS96RURQL0JlM2RxMW4yb0RDZkZUTDB4aTI0V01kTFFyckEyd0J3MW00czVaZDNwQnVTTjIKWHJkVVhkeUx6dUxlczJNbi9yQnFxZGxxenovMDJOT1M4T1JUVkJRVFNBMFI4U0xDVGNIbExCZGQyZ3FnVmZ6ZQpXRlVObXhWQ2EwcHA5UENualRqYklHOElaSHl2cGRPK3dpRFAzVjVrVkRpOS9sS1RoZTNxMU54ODlVQ0VydElrClpjSkYvWEs3aE90MU1sOXNYNjNvaVUxMTVDay9sYUZGNHp1aCtmT1V6ek5VdFd5NzZET3ZybmlUaVpTS1BkMEEKODNNa2l2N2VHaDVkV3owWmwrMnZrd2RkclpHNWVoeG9saEZLelFJREFRQUJBb0lCQUJsdW1tU3gxR1djWER1Mwpwei8wZEhWWkV4c2NsU3U0SGRmZkZPcTF3cFlCUjlmeGFTZGsxQzR2YXF1UjhMaWl6WWVtVWViRGgraitkSnlSCmpwZ2JNaDV4S1BtRkw5empwU3ZUTkN4UHB3OUF5bm5sM3dyNHZhcU1CTS9aZGpuSGttRC9kQzBadEEvL0JIZ3YKNHk4d3VpWCsvUWdVaER0Z1JNcmR1ZUZ1OVlKaFo5UE9jYXkzSkkzMFhEYjdJSS9vNFNhYnhTcFI3bTg5WjY0NwpUb3hsOEhTSzl0SUQxbkl1bHVpTmx1dHI1RzdDdE93WTBSc2N5dmZ2elg4a1d2akpLZVJVbmhMSCtXVFZOaExICjdZc0tMNmlLa1NkckMzeWVPWnV4R0pEbVdrZVgxTzNPRUVGYkc4TjVEaGNqL0lXbDh1dGt3LzYwTEthNHBCS2cKTXhtNEx3RUNnWUVBNnlPRkhNY2pncHYxLzlHZC8yb3c2YmZKcTFjM1dqQkV2cnM2ZXNyMzgrU3UvdVFneXJNcAo5V01oZElpb2dYZjVlNjV5ZlIzYVBXcjJJdWMxZ0RUNlYycDZFR2h0NysyQkF1YkIzczloZisycVNRY1lkS3pmCnJOTDdKalE4ZEVGZWdYd041cHhKOTRTTVFZNEI4Qm9hOHNJWTd3TzU4dHpVMjZoclVnanFXQ1VDZ1lFQXlVUUIKNzViWlh6MGJ5cEc5NjNwYVp0bGlJY0cvUk1XMnVPOE9rVFNYSGdDSjBob25uRm5IMGZOc1pGTHdFWEtnTTRORworU3ZNbWtUekE5eVVSMHpIMFJ4UW44L1YzVWZLT2k5RktFeWx6NzNiRkV6ZW1QSEppQm12NWQ4ZTlOenZmU0E0CkdpRTYrYnFyV3VVWWRoRWlYTnY1SFNPZ3I4bUx1TzJDbGlmNTg0a0NnWUFlZzlDTmlJWmlOODAzOHNNWFYzZWIKalI5ZDNnYXY3SjJ2UnVyeTdvNDVGNDlpUXNiQ3AzZWxnY1RnczY5eWhkaFpwYXp6OGNEVndhREpyTW16cHF4cQpWY1liaFFIblppSWM5MGRubS9BaVF2eWJWNUZqNnQ5b05VVWtreGpaV1haalJXOGtZMW55QmtDUmJWVnhER0k4CjZOV0ZoeTFGaUVVVGNJcms3WVZFQlFLQmdRREpHTVIrYWRFamtlRlUwNjVadkZUYmN0VFVPY3dzb1Foalc2akkKZVMyTThxakNYeE80NnhQMnVTeFNTWFJKV3FpckQ3NDRkUVRvRjRCaEdXS21veGI3M3pqSGxWaHcwcXhDMnJ4VQorZENxODE0VXVJR3BlOTBMdWU3QTFlRU9kRHB1WVdUczVzc1FmdTE3MG5CUWQrcEhzaHNFZkhhdmJjZkhyTGpQCjQzMmhVUUtCZ1FDZ3hMZG5Pd2JMaHZLVkhhdTdPVXQxbGpUT240SnB5bHpnb3hFRXpzaDhDK0ZKUUQ1bkFxZXEKZUpWSkNCd2VkallBSDR6MUV3cHJjWnJIN3IyUTBqT2ZFallwU1dkZGxXaWh4OTNYODZ0aG83UzJuUlYrN1hNcQpPVW9ZcVZ1WGlGMWdMM1NGeHZqMHhxV3l0d0NPTW5DZGFCb0M0Tkw3enJtL0lZOEUwSkw2MkE9PQotLS0tLUVORCBSU0EgUFJJVkFURSBLRVktLS0tLQo="
  ## When tls.certManager.enabled=false
  ## nodeCert and clientCert will be used only when rootCA.key is empty.
  ## Will be ignored and genSignedCert will be used to generate
  ## node and client certs if rootCA.key is provided.
  ## cert and key are base64 encoded content of certificate and key.
  nodeCert:
    cert: ""
    key: ""
  clientCert:
    cert: ""
    key: ""

gflags:
  master:
    default_memory_limit_to_ram_ratio: 0.85
  tserver: {}
#   use_cassandra_authentication: false

PodManagementPolicy: Parallel

enableLoadBalancer: true

ybc:
  enabled: false
  ## https://kubernetes.io/docs/concepts/configuration/manage-resources-containers/#resource-requests-and-limits-of-pod-and-container
  ## Use the above link to learn more about Kubernetes resources configuration.
  # resources:
  #   requests:
  #     cpu: "1"
  #     memory: 1Gi
  #   limits:
  #     cpu: "1"
  #     memory: 1Gi

ybCleanup: {}
  ## https://kubernetes.io/docs/concepts/configuration/manage-resources-containers/#resource-requests-and-limits-of-pod-and-container
  ## Use the above link to learn more about Kubernetes resources configuration.
  # resources:
  #   requests:
  #     cpu: "1"
  #     memory: 1Gi
  #   limits:
  #     cpu: "1"
  #     memory: 1Gi

domainName: "cluster.local"

serviceEndpoints:
  - name: "yb-master-ui"
    type: LoadBalancer
    annotations: {}
    clusterIP: ""
    ## Sets the Service's externalTrafficPolicy
    externalTrafficPolicy: ""
    app: "yb-master"
    loadBalancerIP: ""
    ports:
      http-ui: "7000"

  - name: "yb-tserver-service"
    type: LoadBalancer
    annotations:
      metallb.universe.tf/loadBalancerIPs: 10.0.20.106
    clusterIP: ""
    ## Sets the Service's externalTrafficPolicy
    externalTrafficPolicy: ""
    app: "yb-tserver"
    loadBalancerIP: ""
    ports:
      tcp-yql-port: "9042"
      tcp-yedis-port: "6379"
      tcp-ysql-port: "5433"

Services:
  - name: "yb-masters"
    label: "yb-master"
    skipHealthChecks: false
    memory_limit_to_ram_ratio: 0.85
    ports:
      http-ui: "7000"
      tcp-rpc-port: "7100"

  - name: "yb-tservers"
    label: "yb-tserver"
    skipHealthChecks: false
    ports:
      http-ui: "9000"
      tcp-rpc-port: "9100"
      tcp-yql-port: "9042"
      tcp-yedis-port: "6379"
      tcp-ysql-port: "5433"
      http-ycql-met: "12000"
      http-yedis-met: "11000"
      http-ysql-met: "13000"
      grpc-ybc-port: "18018"


## Should be set to true only if Istio is being used. This also adds
## the Istio sidecar injection labels to the pods.
## TODO: remove this once
## https://github.com/yugabyte/yugabyte-db/issues/5641 is fixed.
##
istioCompatibility:
  enabled: false

## Settings required when using multicluster environment.
multicluster:
  ## Creates a ClusterIP service for each yb-master and yb-tserver
  ## pod.
  createServicePerPod: false
  ## creates a ClusterIP service whos name does not have release name
  ## in it. A common service across different clusters for automatic
  ## failover. Useful when using new naming style.
  createCommonTserverService: false

  ## Enable it to deploy YugabyteDB in a multi-cluster services enabled
  ## Kubernetes cluster (KEP-1645). This will create ServiceExport.
  ## GKE Ref - https://cloud.google.com/kubernetes-engine/docs/how-to/multi-cluster-services#registering_a_service_for_export
  ## You can use this gist for the reference to deploy the YugabyteDB in a multi-cluster scenario.
  ## Gist - https://gist.github.com/baba230896/78cc9bb6f4ba0b3d0e611cd49ed201bf
  createServiceExports: false

  ## Mandatory variable when createServiceExports is set to true.
  ## Use: In case of GKE, you need to pass GKE Hub Membership Name.
  ## GKE Ref - https://cloud.google.com/kubernetes-engine/docs/how-to/multi-cluster-services#enabling
  kubernetesClusterId: ""

  ## mcsApiVersion is used for the MCS resources created by the
  ## chart. Set to net.gke.io/v1 when using GKE MCS.
  mcsApiVersion: "multicluster.x-k8s.io/v1alpha1"

serviceMonitor:
  ## If true, two ServiceMonitor CRs are created. One for yb-master
  ## and one for yb-tserver
  ## https://github.com/coreos/prometheus-operator/blob/master/Documentation/api.md#servicemonitor
  ##
  enabled: false
  ## interval is the default scrape_interval for all the endpoints
  interval: 30s
  ## extraLabels can be used to add labels to the ServiceMonitors
  ## being created
  extraLabels: {}
    # release: prom

  ## Configurations of ServiceMonitor for yb-master
  master:
    enabled: true
    port: "http-ui"
    interval: ""
    path: "/prometheus-metrics"

  ## Configurations of ServiceMonitor for yb-tserver
  tserver:
    enabled: true
    port: "http-ui"
    interval: ""
    path: "/prometheus-metrics"
  ycql:
    enabled: true
    port: "http-ycql-met"
    interval: ""
    path: "/prometheus-metrics"
  ysql:
    enabled: true
    port: "http-ysql-met"
    interval: ""
    path: "/prometheus-metrics"
  yedis:
    enabled: true
    port: "http-yedis-met"
    interval: ""
    path: "/prometheus-metrics"

  commonMetricRelabelings:
  # https://git.io/JJW5p
  # Save the name of the metric so we can group_by since we cannot by __name__ directly...
  - sourceLabels: ["__name__"]
    regex: "(.*)"
    targetLabel: "saved_name"
    replacement: "$1"
  # The following basically retrofit the handler_latency_* metrics to label format.
  - sourceLabels: ["__name__"]
    regex: "handler_latency_(yb_[^_]*)_([^_]*)_([^_]*)(.*)"
    targetLabel: "server_type"
    replacement: "$1"
  - sourceLabels: ["__name__"]
    regex: "handler_latency_(yb_[^_]*)_([^_]*)_([^_]*)(.*)"
    targetLabel: "service_type"
    replacement: "$2"
  - sourceLabels: ["__name__"]
    regex: "handler_latency_(yb_[^_]*)_([^_]*)_([^_]*)(_sum|_count)?"
    targetLabel: "service_method"
    replacement: "$3"
  - sourceLabels: ["__name__"]
    regex: "handler_latency_(yb_[^_]*)_([^_]*)_([^_]*)(_sum|_count)?"
    targetLabel: "__name__"
    replacement: "rpc_latency$4"

resources: {}

nodeSelector: {}

affinity: {}

statefulSetAnnotations: {}

networkAnnotation: {}

commonLabels: {}

master:
  ## Ref: https://kubernetes.io/docs/reference/generated/kubernetes-api/v1.22/#affinity-v1-core
  ## This might override the default affinity from service.yaml
  # To successfully merge, we need to follow rules for merging nodeSelectorTerms that kubernentes
  # has. Each new node selector term is ORed together, and each match expression or match field in
  # a single selector is ANDed together.
  # This means, if a pod needs to be scheduled on a label 'custom_label_1' with a value
  # 'custom_value_1', we need to add this 'subterm' to each of our pre-defined node affinity
  # terms.
  #
  # Pod anti affinity is a simpler merge. Each term is applied separately, and the weight is tracked.
  # The pod that achieves the highest weight is selected.
  ## Example.
  # affinity:
  #   podAntiAffinity:
  #     requiredDuringSchedulingIgnoredDuringExecution:
  #     - labelSelector:
  #         matchExpressions:
  #         - key: app
  #           operator: In
  #           values:
  #           - "yb-master"
  #       topologyKey: kubernetes.io/hostname
  #
  # For further examples, see examples/yugabyte/affinity_overrides.yaml
  affinity: {}

  ## Extra environment variables passed to the Master pods.
  ## Ref: https://kubernetes.io/docs/reference/generated/kubernetes-api/v1.22/#envvar-v1-core
  ## Example:
  # extraEnv:
  # - name: NODE_IP
  #   valueFrom:
  #     fieldRef:
  #       fieldPath: status.hostIP
  extraEnv: []

  # secretEnv variables are used to expose secrets data as env variables in the master pod.
  # TODO Add namespace also to support copying secrets from other namespace.
  # secretEnv:
  # - name: MYSQL_LDAP_PASSWORD
  #   valueFrom:
  #     secretKeyRef:
  #       name: secretName
  #       key: password
  secretEnv: []

  ## Annotations to be added to the Master pods.
  podAnnotations: {}

  ## Labels to be added to the Master pods.
  podLabels: {}

  ## Tolerations to be added to the Master pods.
  ## Ref: https://kubernetes.io/docs/reference/generated/kubernetes-api/v1.22/#toleration-v1-core
  ## Example:
  # tolerations:
  # - key: dedicated
  #   operator: Equal
  #   value: experimental
  #   effect: NoSchedule
  tolerations: []

  ## Extra volumes
  ## extraVolumesMounts are mandatory for each extraVolumes.
  ## Ref: https://kubernetes.io/docs/reference/generated/kubernetes-api/v1.22/#volume-v1-core
  ## Example:
  # extraVolumes:
  # - name: custom-nfs-vol
  #   persistentVolumeClaim:
  #     claimName: some-nfs-claim
  extraVolumes: []

  ## Extra volume mounts
  ## Ref: https://kubernetes.io/docs/reference/generated/kubernetes-api/v1.22/#volumemount-v1-core
  ## Example:
  # extraVolumeMounts:
  # - name: custom-nfs-vol
  #   mountPath: /home/yugabyte/nfs-backup
  extraVolumeMounts: []

  ## Set service account for master DB pods. The service account
  ## should exist in the namespace where the master DB pods are brought up.
  serviceAccount: ""

  ## Memory limit hard % (between 1-100) of the memory limit.
  memoryLimitHardPercentage: 85


tserver:
  ## Ref: https://kubernetes.io/docs/reference/generated/kubernetes-api/v1.22/#affinity-v1-core
  ## This might override the default affinity from service.yaml
  # To successfully merge, we need to follow rules for merging nodeSelectorTerms that kubernentes
  # has. Each new node selector term is ORed together, and each match expression or match field in
  # a single selector is ANDed together.
  # This means, if a pod needs to be scheduled on a label 'custom_label_1' with a value
  # 'custom_value_1', we need to add this 'subterm' to each of our pre-defined node affinity
  # terms.
  #
  # Pod anti affinity is a simpler merge. Each term is applied separately, and the weight is tracked.
  # The pod that achieves the highest weight is selected.
  ## Example.
  # affinity:
  #   podAntiAffinity:
  #     requiredDuringSchedulingIgnoredDuringExecution:
  #     - labelSelector:
  #         matchExpressions:
  #         - key: app
  #           operator: In
  #           values:
  #           - "yb-tserver"
  #       topologyKey: kubernetes.io/hostname
  # For further examples, see examples/yugabyte/affinity_overrides.yaml
  affinity: {}

  ## Extra environment variables passed to the TServer pods.
  ## Ref: https://kubernetes.io/docs/reference/generated/kubernetes-api/v1.22/#envvar-v1-core
  ## Example:
  # extraEnv:
  # - name: NODE_IP
  #   valueFrom:
  #     fieldRef:
  #       fieldPath: status.hostIP
  extraEnv: []

  ## secretEnv variables are used to expose secrets data as env variables in the tserver pods.
  ## If namespace field is not specified we assume that user already
  ## created the secret in the same namespace as DB pods.
  ## Example
  # secretEnv:
  # - name: MYSQL_LDAP_PASSWORD
  #   valueFrom:
  #     secretKeyRef:
  #       name: secretName
  #       namespace: my-other-namespace-with-ldap-secret
  #       key: password
  secretEnv: []

  ## Annotations to be added to the TServer pods.
  podAnnotations: {}

  ## Labels to be added to the TServer pods.
  podLabels: {}

  ## Tolerations to be added to the TServer pods.
  ## Ref: https://kubernetes.io/docs/reference/generated/kubernetes-api/v1.22/#toleration-v1-core
  ## Example:
  # tolerations:
  # - key: dedicated
  #   operator: Equal
  #   value: experimental
  #   effect: NoSchedule
  tolerations: []

  ## Sets the --server_broadcast_addresses flag on the TServer, no
  ## preflight checks are done for this address. You might need to add
  ## `use_private_ip: cloud` to the gflags.master and gflags.tserver.
  serverBroadcastAddress: ""

  ## Extra volumes
  ## extraVolumesMounts are mandatory for each extraVolumes.
  ## Ref: https://kubernetes.io/docs/reference/generated/kubernetes-api/v1.22/#volume-v1-core
  ## Example:
  # extraVolumes:
  # - name: custom-nfs-vol
  #   persistentVolumeClaim:
  #     claimName: some-nfs-claim
  extraVolumes: []

  ## Extra volume mounts
  ## Ref: https://kubernetes.io/docs/reference/generated/kubernetes-api/v1.22/#volumemount-v1-core
  ## Example:
  # extraVolumeMounts:
  # - name: custom-nfs-vol
  #   path: /home/yugabyte/nfs-backup
  extraVolumeMounts: []

  ## Set service account for tserver DB pods. The service account
  ## should exist in the namespace where the tserver DB pods are brought up.
  serviceAccount: ""

  ## Memory limit hard % (between 1-100) of the memory limit.
  memoryLimitHardPercentage: 85


helm2Legacy: false

ip_version_support: "v4_only" # v4_only, v6_only are the only supported values at the moment

# For more https://docs.yugabyte.com/latest/reference/configuration/yugabyted/#environment-variables
authCredentials:
  ysql:
    user: "yadmin"
    password: "TES2&Daggerfall"
    database: ""
  ycql:
    user: ""
    password: ""
    keyspace: ""

oldNamingStyle: true

preflight:
  # Set to true to skip disk IO check, DNS address resolution, and
  # port bind checks
  skipAll: false
  # Set to true to skip port bind checks
  skipBind: false

  ## Set to true to skip ulimit verification
  ## SkipAll has higher priority
  skipUlimit: false

## Pod securityContext
## Ref: https://kubernetes.io/docs/reference/kubernetes-api/workload-resources/pod-v1/#security-context
## The following configuration runs YB-Master and YB-TServer as a non-root user
podSecurityContext:
  enabled: false
  ## Mark it false, if you want to stop the non root user validation
  runAsNonRoot: true
  fsGroup: 10001
  runAsUser: 10001
  runAsGroup: 10001

## Added to handle old universe which has volume annotations
## K8s universe <= 2.5 to >= 2.6
legacyVolumeClaimAnnotations: false
\end{lstlisting}

Die Installation erfolgt dann wie folgt:
\lstset{style=gra_codestyle}
\begin{lstlisting}[language=bash, caption=yugabyteDB - Installation,captionpos=b,label={lst:yugabytedb-values-apply},breaklines=true]
helm install yw-test yugabytedb/yugabyte --version 2.19.3 -n yb-platform -f /home/gramic/PycharmProjects/rke2_settings/yugabytedb/yugabytedb/values.yaml
\end{lstlisting}

\subsection{Rekonfiguration mit 300GiB Storage}