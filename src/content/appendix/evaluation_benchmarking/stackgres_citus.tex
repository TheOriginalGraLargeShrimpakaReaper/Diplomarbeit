%! Author = gramic
%! Date = 25.04.24

% Preamble
\begin{flushleft}
    \subsection{StackGres - Citus}
    \label{subsec:stackgres_citus_benchmarking_commands}
    Pro Lauf werden erst die Daten initialisiert werden.\\
    Wichtig ist dabei, dass keine Tablespaces mitgegeben werden da diese nicht existieren.\\
    Anschliessend werden die Benchmarks an sich ausgeführt.\\
    Die Resultate werden weggeschrieben.
\lstset{style=gra_codestyle}
\begin{lstlisting}[language=bash, caption=StackGres-Citus - Benchmarking-Commands,captionpos=b,label={lst:stackgres_citus-benchmarking-commands},breaklines=true]
###################
#   1. Lauf       #
#   ca. 5GiB      #
###################
# Init
pgbench --host=10.0.20.106 --port=5432 --initialize --scale=400 --foreign-keys --fillfactor=100 --username=dtgvpf  --username=postgres pgbench_eval_bench

# Benchmarking mixed
pgbench -h 10.0.20.106 -p 5432 -c 10 -C -j 4 -v -t 10 -U postgres pgbench_eval_bench > /home/itgramic/1_1_stackgresmixed_benchmark.txt

# Benchmarking dql
pgbench -h 10.0.20.106 -p 5432 -c 10 -C -j 4 -v -t 10 -S -U postgres pgbench_eval_bench > /home/itgramic/1_1_stackgresdql_benchmark.txt

###################
#   2. Lauf       #
#   ca. 15GiB     #
###################
# Init
pgbench --host=10.0.20.106 --port=5432 --initialize --scale=1200 --foreign-keys --fillfactor=100 --username=dtgvpf  --username=postgres pgbench_eval_bench

# Benchmarking mixed
pgbench -h 10.0.20.106 -p 5432 -c 50 -C -j 4 -v -t 50 -U postgres pgbench_eval_bench > /home/itgramic/2_1_stackgres_mixed_benchmark.txt
# Benchmarking dql
pgbench -h 10.0.20.106 -p 5432 -c 50 -C -j 4 -v -t 50 -S -U postgres pgbench_eval_bench > /home/itgramic/2_1_stackgres_dql_benchmark.txt

###################
#   3. Lauf       #
#   ca. 50GiB     #
###################
pgbench --host=10.0.20.106 --port=5432 --initialize --scale=3999 --foreign-keys --fillfactor=100 --username=dtgvpf  --username=postgres pgbench_eval_bench

# Benchmarking mixed
pgbench -h 10.0.20.106 -p 5432 -c 50 -C -j 4 -v -t 100 -U postgres pgbench_eval_bench > /home/itgramic/3_1_stackgres_mixed_benchmark.txt
# Benchmarking dql
pgbench -h 10.0.20.106 -p 5432 -c 50 -C -j 4 -v -t 100 -S -U postgres pgbench_eval_bench > /home/itgramic/3_1_stackgres_dql_benchmark.txt


###################
#   4. Lauf       #
#   ca. 250GiB    #
###################
pgbench --host=10.0.20.106 --port=5432 --initialize --scale=16784 --foreign-keys --fillfactor=100 --username=dtgvpf  --username=postgres pgbench_eval_bench

# Benchmarking mixed
pgbench -h 10.0.20.106 -p 5432 -c 25 -C -j 4 -v -t 280 -U postgres pgbench_eval_bench > /home/itgramic/4_1_stackgresmixed_benchmark.txt
# Benchmarking dql
pgbench -h 10.0.20.106 -p 5432 -c 25 -C -j 4 -v -t 280 -S -U postgres pgbench_eval_bench > /home/itgramic/4_1_stackgresdql_benchmark.txt
\end{lstlisting}
\end{flushleft}
\begin{flushleft}
    Die grösse der Tabellen lässt sich wie folgt auslesen:
\lstset{style=gra_codestyle}
\begin{lstlisting}[language=sql, caption=StackGres-Citus - Benchmarking - Table Size SQL,captionpos=b,label={lst:stackgres_citus-benchmarking-table-size-sql},breaklines=true]
SELECT * FROM citus_tables;
\end{lstlisting}
    Das Resultat ist aber an sich zu gross, da bei den Reference Tables alle Shards und die Coordinators zusammengerechnet werden.\\
    Die korrekte Grösse muss wie folgt kalkuliert werden:\\
    \(\mathlarger{realsize = \mathlarger{\frac{size}{[coordinator_{syncInstances} + (shard_{clusters} \times shard_{instancesPerCluster})]}}}\)
\end{flushleft}