\begin{table}[H]

\resizebox{\columnwidth}{!}{%

\begin{tabular}{rllll}
\toprule
Nr. & Anforderung & Beschreibung & System & Muss / Kann \\
\midrule
1 & Systemvielfallt & \begin{tabular}[c]{@{}l@{}}Es muss mindestens eine Monolitisches und mindestens 2 zwei Distributed SQL Cluster ermittelt werden\end{tabular} & Beides & MUSS \\
2 & Synergien & \begin{tabular}[c]{@{}l@{}}Skripte und APIs des Monolithisches Systems müssen auch in einem Distributed SQL System verwendet werden können\end{tabular} & Beides & MUSS \\
3 & Failover & \begin{tabular}[c]{@{}l@{}}Das Clustersystem muss bei einem Nodeausfall automatisch auf einen anderen Node umstellt\end{tabular} & Beides & MUSS \\
4 & Failover & \begin{tabular}[c]{@{}l@{}}Beim Failover dürfen bestehende Connections nicht getrennt werden oder sofort Wiederhergestellt werden\end{tabular} & Beides & MUSS \\
5 & Failover-Geschwindigkeit & \begin{tabular}[c]{@{}l@{}}Das umstellen auf den nächsten Node muss so schnell ausgefühgrt werden,\\das ein Disconnect mittels Client-Konfiguration (Timeout) verhindert wird.\end{tabular} & Beides & MUSS \\
6 & Switchover & \begin{tabular}[c]{@{}l@{}}Das System muss ein Skript oder eine API liefern,\\welche einen geordeten Switchover auf einen anderen Node erlaubt\end{tabular} & Beides & MUSS \\
7 & Switchover & \begin{tabular}[c]{@{}l@{}}Beim Switchover dürfen bestehende Connections nicht getrennt werden oder sofort Wiederhergestellt werden\end{tabular} & Beides & MUSS \\
8 & Switchover-Geschwindigkeit & \begin{tabular}[c]{@{}l@{}}Das umstellen auf den nächsten Node muss so schnell ausgefühgrt werden,\\das ein Disconnect mittels Client-Konfiguration (Timeout) verhindert wird.\end{tabular} & Beides & MUSS \\
9 & Restore & \begin{tabular}[c]{@{}l@{}}Das Clustersystem muss ein Skript oder eine API liefern,\\welche das einfache und ggf. automatisierte Restoren eines oder mehreren Nodes ermöglichen\end{tabular} & Beides & MUSS \\
10 & Restore & \begin{tabular}[c]{@{}l@{}}Beim Wiederherstellen des Ursprungszustands darf es zu keinem Datenverlust kommen\end{tabular} & Beides & MUSS \\
11 & Restore & \begin{tabular}[c]{@{}l@{}}Bei der Wiederherstellung einzelner Nodes darf es zu keinen Unterbrechungen auf den Applikationen kommen\end{tabular} & Beides & MUSS \\
12 & Restore-Geschwindigkeit & \begin{tabular}[c]{@{}l@{}}Das Wiederherstellen des Ursprungszustands muss\\innert weniger Stunden für alle Datenbanken aus dem\\Backup Wiederhergestellt und im Clustersystem Synchronisiert werden\end{tabular} & Beides & MUSS \\
13 & Replikation & \begin{tabular}[c]{@{}l@{}}Es muss eine Synchrone Replikation sichergestellt werden\end{tabular} & Monolitisch & MUSS \\
14 & Replikation & \begin{tabular}[c]{@{}l@{}}Die Replikation muss sicherstellen, das es bei einem Failover/Switchover zu keinem Fehler kommt\end{tabular} & Monolitisch & MUSS \\
15 & Sharding & \begin{tabular}[c]{@{}l@{}}Die Datenkonsistenz und Datenintegrität auf den Shards muss sichergestellt werden\end{tabular} & Distributed SQL & MUSS \\
16 & Sharding & \begin{tabular}[c]{@{}l@{}}Die Synchronisation der Shards muss sicherstellen, dass es zu keinem Datenverlust kommt\end{tabular} & Distributed SQL & MUSS \\
17 & Quorum & \begin{tabular}[c]{@{}l@{}}Das Clustersystem muss über ein Quorum-System besitzen\end{tabular} & Beides & MUSS \\
18 & Quorum & \begin{tabular}[c]{@{}l@{}}Das Quorum des Clustersystems muss robust genug sein, um eine Split-Brain-Situation zu verhindern\end{tabular} & Beides & MUSS \\
19 & Connection & \begin{tabular}[c]{@{}l@{}}Das Clustersystem muss sicherstellen,\\dass eine Applikation ohne Entwicklungsaufwand mittels dem PostgreSQL Wired Connector zugreifen kann\end{tabular} & Beides & MUSS \\
20 & Management-API & \begin{tabular}[c]{@{}l@{}}Das Clustersystem muss Skripte oder eine API liefern,\\mit dem das System zu konfigurieren, verwalten oder überwachen zu können.\\Zudem müssen mit geringen Arbeitsaufwand\\damit Nodes hinzugefügt oder entfernt werden können\end{tabular} & Beides & MUSS \\
21 & Management-API & \begin{tabular}[c]{@{}l@{}}Es müssen gängige Standards für Authentifizierung und Autorisierung mitgebracht werden\end{tabular} & Beides & MUSS \\
22 & Backup & \begin{tabular}[c]{@{}l@{}}Backups müssen mittels PostgreSQL Standards angezogen werden\end{tabular} & Beides & MUSS \\
23 & Backup Restore & \begin{tabular}[c]{@{}l@{}}Backups müssen mittels PostgreSQL Standards restored werden können\end{tabular} & Beides & MUSS \\
24 & Housekeeping - Log Rotation & \begin{tabular}[c]{@{}l@{}}Das Clustersystem muss die möglichkeit zur Log Rotation bieten\end{tabular} & Beides & MUSS \\
25 & Self Heahling & \begin{tabular}[c]{@{}l@{}}Das Clustersystem muss im Fehlerfall Nodes selber wiederherstellen können\end{tabular} & Beides & KANN \\
26 & Monitoring - Node Failure & \begin{tabular}[c]{@{}l@{}}Läuft ein Node auf einen Fehler,\\muss das Clustersystem dies erkennen und Melden resp.\\eine Schnittstelle liefern die abgefragt werden kann\end{tabular} & Beides & MUSS \\
27 & Maintenance Quality & \begin{tabular}[c]{@{}l@{}}Da die meisten PostgreSQL HA Lösungen Open-Source sind,\\muss sichergestellt werden,\\dass die gewählte Lösung auch aktiv gepflegt wird.\\Als Basis dienen hier Informationen wie z.B. GitHub Insights.\end{tabular} & Beides & MUSS \\
\bottomrule
\end{tabular}
}
\caption{Anforderungskatalog} \label{anforderungskatalog}
\end{table}
