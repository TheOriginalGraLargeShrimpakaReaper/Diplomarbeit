\begin{longtable}[H]{rlllll}

\toprule
Nr. & Anforderung & Bezeichnung & Beschreibung & System & Muss / Kann \\
\midrule
\endfirsthead
\caption[]{Anforderungskatalog} \\
\toprule
Nr. & Anforderung & Bezeichnung & Beschreibung & System & Muss / Kann \\
\midrule
\endhead
\midrule
\multicolumn{6}{r}{Continued on next page} \\
\midrule
\endfoot
\bottomrule
\endlastfoot
1 & Systemvielfallt &  & \begin{tabular}[c]{@{}l@{}}Es muss mindestens eine Monolitisches und mindestens 2 zwei Distributed SQL Cluster ermittelt werden\end{tabular} & Beides & MUSS \\
2 & Synergien &  & \begin{tabular}[c]{@{}l@{}}Skripte und APIs des Monolithisches Systems müssen auch in einem Distributed SQL System verwendet werden können\end{tabular} & Beides & MUSS \\
3 & Failover & Automatismus & \begin{tabular}[c]{@{}l@{}}Das Clustersystem muss bei einem Nodeausfall automatisch auf einen anderen Node umstellt\end{tabular} & Beides & MUSS \\
4 & Failover & Connection - Stabilität & \begin{tabular}[c]{@{}l@{}}Beim Failover dürfen bestehende Connections nicht getrennt werden oder sofort Wiederhergestellt werden\end{tabular} & Beides & MUSS \\
5 & Failover & Geschwindigkeit & \begin{tabular}[c]{@{}l@{}}Das umstellen auf den nächsten Node muss so schnell ausgefühgrt werden,\\das ein Disconnect mittels Client-Konfiguration (Timeout) verhindert wird.\end{tabular} & Beides & MUSS \\
6 & Switchover & Skript / API & \begin{tabular}[c]{@{}l@{}}Das System muss ein Skript oder eine API liefern,\\welche einen geordeten Switchover auf einen anderen Node erlaubt\end{tabular} & Beides & MUSS \\
7 & Switchover & Connection - Stabilität & \begin{tabular}[c]{@{}l@{}}Beim Switchover dürfen bestehende Connections nicht getrennt werden oder sofort Wiederhergestellt werden\end{tabular} & Beides & MUSS \\
8 & Switchover & Geschwindigkeit & \begin{tabular}[c]{@{}l@{}}Das umstellen auf den nächsten Node muss so schnell ausgefühgrt werden,\\das ein Disconnect mittels Client-Konfiguration (Timeout) verhindert wird.\end{tabular} & Beides & MUSS \\
9 & Restore & Skript / API & \begin{tabular}[c]{@{}l@{}}Das Clustersystem muss ein Skript oder eine API liefern,\\welche das einfache und ggf. automatisierte Restoren eines oder mehreren Nodes ermöglichen\end{tabular} & Beides & MUSS \\
10 & Restore & Datensicherheit & \begin{tabular}[c]{@{}l@{}}Beim Wiederherstellen des Ursprungszustands darf es zu keinem Datenverlust kommen\end{tabular} & Beides & MUSS \\
11 & Restore & Connection - Stabilität & \begin{tabular}[c]{@{}l@{}}Bei der Wiederherstellung einzelner Nodes darf es zu keinen Unterbrechungen auf den Applikationen kommen\end{tabular} & Beides & MUSS \\
12 & Restore & Geschwindigkeit & \begin{tabular}[c]{@{}l@{}}Das Wiederherstellen des Ursprungszustands muss\\innert weniger Stunden für alle Datenbanken aus dem\\Backup Wiederhergestellt und im Clustersystem Synchronisiert werden\end{tabular} & Beides & MUSS \\
13 & Replikation & Synchrone Replikation & \begin{tabular}[c]{@{}l@{}}Es muss eine Synchrone Replikation sichergestellt werden\end{tabular} & Monolitisch & MUSS \\
14 & Replikation & Failover / Switchover Garantie & \begin{tabular}[c]{@{}l@{}}Die Replikation muss sicherstellen, das es bei einem Failover/Switchover zu keinem Fehler kommt\end{tabular} & Monolitisch & MUSS \\
15 & Replikation & Throughput & \begin{tabular}[c]{@{}l@{}}Beschreibt, wie viele Transaktionen pro Zeiteinheit vom Primary an die Replikas gesendet und Commited werden.\\Dieser Wert ist bei Synchroner Replikation entscheidend da Commits auf allen Replicas abgesetzt sein müssen.\end{tabular} & Beides & MUSS \\
16 & Sharding & Datenschutz- und integrität & \begin{tabular}[c]{@{}l@{}}Die Datenkonsistenz und Datenintegrität auf den Shards muss sichergestellt werden\end{tabular} & Distributed SQL & MUSS \\
17 & Sharding & Schutz vor Datenverlust & \begin{tabular}[c]{@{}l@{}}Die Synchronisation der Shards muss sicherstellen, dass es zu keinem Datenverlust kommt\end{tabular} & Distributed SQL & MUSS \\
18 & Quorum & Quorum-System vorhanden & \begin{tabular}[c]{@{}l@{}}Das Clustersystem muss über ein Quorum-System besitzen\end{tabular} & Beides & MUSS \\
19 & Quorum & Robhustheit & \begin{tabular}[c]{@{}l@{}}Das Quorum des Clustersystems muss robust genug sein, um eine Split-Brain-Situation zu verhindern\end{tabular} & Beides & MUSS \\
20 & Connection &  & \begin{tabular}[c]{@{}l@{}}Das Clustersystem muss sicherstellen,\\dass eine Applikation ohne Entwicklungsaufwand mittels dem PostgreSQL Wired Connector zugreifen kann\end{tabular} & Beides & MUSS \\
21 & Management-API & Management-API vorhanden & \begin{tabular}[c]{@{}l@{}}Das Clustersystem muss Skripte oder eine API liefern,\\mit dem das System zu konfigurieren, verwalten oder überwachen zu können.\\Zudem müssen mit geringen Arbeitsaufwand\\damit Nodes hinzugefügt oder entfernt werden können\end{tabular} & Beides & MUSS \\
22 & Management-API & Authentifizierung \& Autorisierung & \begin{tabular}[c]{@{}l@{}}Es müssen gängige Standards für Authentifizierung und Autorisierung mitgebracht werden\end{tabular} & Beides & MUSS \\
23 & Management-API & Aufwand & \begin{tabular}[c]{@{}l@{}}Der Aufwand,\\der benötigt wird um die DB zu verwalten,\\Nodes hinzuzufügen oder zu entfernen usw. muss gegeneinander verglichen werden.\end{tabular} & Beides & MUSS \\
24 & Backup & Backup mit PostgreSQL Standards & \begin{tabular}[c]{@{}l@{}}Backups müssen mittels PostgreSQL Standards angezogen werden\end{tabular} & Beides & MUSS \\
25 & Backup & Restore mit PostgreSQL Standanrds & \begin{tabular}[c]{@{}l@{}}Backups müssen mittels PostgreSQL Standards restored werden können\end{tabular} & Beides & MUSS \\
26 & Housekeeping - Log Rotation &  & \begin{tabular}[c]{@{}l@{}}Das Clustersystem muss die möglichkeit zur Log Rotation bieten\end{tabular} & Beides & MUSS \\
27 & Self Heahling &  & \begin{tabular}[c]{@{}l@{}}Das Clustersystem muss im Fehlerfall Nodes selber wiederherstellen können\end{tabular} & Beides & KANN \\
28 & Monitoring - Node Failure &  & \begin{tabular}[c]{@{}l@{}}Läuft ein Node auf einen Fehler,\\muss das Clustersystem dies erkennen und Melden resp.\\eine Schnittstelle liefern die abgefragt werden kann\end{tabular} & Beides & MUSS \\
29 & Maintenance Quality &  & \begin{tabular}[c]{@{}l@{}}Da die meisten PostgreSQL HA Lösungen Open-Source sind,\\muss sichergestellt werden,\\dass die gewählte Lösung auch aktiv gepflegt wird.\\Als Basis dienen hier Informationen wie z.B. GitHub Insights.\end{tabular} & Beides & MUSS \\
30 & Performance & tps - Read-Only & \begin{tabular}[c]{@{}l@{}}Die Transaktionsrate (transactions per second / tps) für DQL Transaktionen\end{tabular} & Beides & MUSS \\
31 & Performance & tps - Read-Writes & \begin{tabular}[c]{@{}l@{}}Die Transaktionsrate (transactions per second / tps) für DML Transaktionen\end{tabular} & Beides & MUSS \\
32 & Performance & Ø Latenz - Read-Only & \begin{tabular}[c]{@{}l@{}}Die Latenzzeit bei DQL Transaktionen\end{tabular} & Beides & MUSS \\
33 & Performance & Ø Latenz - Read-Write & \begin{tabular}[c]{@{}l@{}}Die Latenzzeit bei DML Transaktionen\end{tabular} & Beides & MUSS \\
\caption{Anforderungskatalog} \label{anforderungskatalog}
\end{longtable}
