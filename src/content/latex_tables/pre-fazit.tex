\begin{table}[H]

\resizebox{\columnwidth}{!}{%

\begin{tabular}{ll}
\toprule
Woche & Beschreibung / Event / Problem \\
\midrule
KW10 & \begin{tabular}[c]{@{}l@{}}Vier ganze Tage war ich in Thalwil für die Oracle Multitenant-Schulung für das ExaCC Projekt (Ablösung HP-UX).\\Am Freiutag war ich ebenfalls fast den ganzen Tag dran.\\Weitere Termine werden folgen, das Risiko durch das Projekt tritt langsam ein.\end{tabular} \\
KW11 & \begin{tabular}[c]{@{}l@{}}Projekt Zeitlich im Verzug.\\Nebst dem HP-UX Ablösungsprojekt schlagen auch diverse Betriebsthemen ein.\\Die analyse der PostgreSQL HA Cluster nimmt ebenfalls mehr Zeit in Anspruch, als erwartet.\end{tabular} \\
KW12 & \begin{tabular}[c]{@{}l@{}}- HP-UX Probleme am Montag.\\  Backups sind über das Weekend nicht durchgelaufen.\\  Die ganze Montagsplanung wurde über den Haufen geworfen.\\- Besprechung bezüglich Backup.\\ Veeam Kasten steht noch nicht zur Verfügung.\end{tabular} \\
KW12 & \begin{tabular}[c]{@{}l@{}}- Mittwochvormittag in Zürich, am Nachmittag Probleme mit dfs-Shares.\\  So wenig Zeit.\\- Mit Norman Termin für nächste Woche Fachgespräch organisiert.\\ Freue mich darauf.\end{tabular} \\
KW12 & \begin{tabular}[c]{@{}l@{}}- Alle Gängigen PostgreSQL HA Lösungen dokumentiert. Aufwand für Die Dokumentation weit grösser als erwartet.\end{tabular} \\
KW13 & \begin{tabular}[c]{@{}l@{}}- YugabyteDB entpuppt sich als recht fordernd.\\Es benötigt eine \guillemotleft private container registry\guillemotright, mir fehlt die Expertise dazu.\\- Der Aufbau der Projektplanung entpuppt sich begrenzt nutzbar.\\Das erstellen der Evaluationsinfrast\end{tabular} \\
KW13 & \begin{tabular}[c]{@{}l@{}}- Das Problem mit dem \guillemotleft private container registry\guillemotright rührte daher,\\ dass das YugabyteDB Anywhere (Repository yugaware) verwendet wurde.\\Kurz ein Schock, dass YugabyteDB ausgeschieden ist.\\Später bemerkte ich, dass man das Repository yugabyte verwendet muss.\end{tabular} \\
KW13 & \begin{tabular}[c]{@{}l@{}}- Bereits jetzt viel gerlernt über Kubernetes, Ranger (rke2) und Helm\end{tabular} \\
\bottomrule
\end{tabular}
}
\caption{Kommentare - Anmerkung} \label{project_comments}
\end{table}
