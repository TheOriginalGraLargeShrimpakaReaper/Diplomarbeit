\begin{longtable}[H]{ll}
 \hdashline
\toprule
Woche & Beschreibung / Event / Problem \\ \hdashline
\midrule
\endfirsthead
\caption[]{Kommentare - Anmerkung} \\ \hdashline
\toprule
Woche & Beschreibung / Event / Problem \\ \hdashline
\midrule
\endhead
\midrule
\multicolumn{2}{r}{Continued on next page} \\ \hdashline
\midrule
\endfoot
\bottomrule
\endlastfoot
KW10 & \begin{tabular}[c]{@{}l@{}}Vier ganze Tage war ich in Thalwil für die Oracle Multitenant-Schulung für das ExaCC Projekt (Ablösung HP-UX).\\Am Freiutag war ich ebenfalls fast den ganzen Tag dran.\\Weitere Termine werden folgen, das Risiko durch das Projekt tritt langsam ein.\end{tabular} \\ \hdashline
KW11 & \begin{tabular}[c]{@{}l@{}}Projekt Zeitlich im Verzug.\\Nebst dem HP-UX Ablösungsprojekt schlagen auch diverse Betriebsthemen ein.\\Die analyse der PostgreSQL HA Cluster nimmt ebenfalls mehr Zeit in Anspruch, als erwartet.\end{tabular} \\ \hdashline
KW12 & \begin{tabular}[c]{@{}l@{}}- HP-UX Probleme am Montag.\\  Backups sind über das Weekend nicht durchgelaufen.\\  Die ganze Montagsplanung wurde über den Haufen geworfen.\\- Besprechung bezüglich Backup.\\ Veeam Kasten steht noch nicht zur Verfügung.\end{tabular} \\ \hdashline
KW12 & \begin{tabular}[c]{@{}l@{}}- Mittwochvormittag in Zürich, am Nachmittag Probleme mit dfs-Shares.\\  So wenig Zeit.\\- Mit Norman Termin für nächste Woche Fachgespräch organisiert.\\ Freue mich darauf.\end{tabular} \\ \hdashline
KW12 & \begin{tabular}[c]{@{}l@{}}- Alle Gängigen PostgreSQL HA Lösungen dokumentiert. Aufwand für Die Dokumentation weit grösser als erwartet.\end{tabular} \\ \hdashline
KW13 & \begin{tabular}[c]{@{}l@{}}- YugabyteDB entpuppt sich als recht fordernd.\\Es benötigt eine \guillemotleft private container registry\guillemotright, mir fehlt die Expertise dazu.\\- Der Aufbau der Projektplanung entpuppt sich begrenzt nutzbar.\\Das erstellen der Evaluationsinfrast\end{tabular} \\ \hdashline
KW13 & \begin{tabular}[c]{@{}l@{}}- Das Problem mit dem \guillemotleft private container registry\guillemotright rührte daher,\\ dass das YugabyteDB Anywhere (Repository yugaware) verwendet wurde.\\Kurz ein Schock, dass YugabyteDB ausgeschieden ist.\end{tabular} \\ \hdashline
KW13 & \begin{tabular}[c]{@{}l@{}}\\Später bemerkte ich, dass man das Repo yugabytedb auswählen muss.\end{tabular} \\ \hdashline
KW13 & \begin{tabular}[c]{@{}l@{}}- MetalLB benötigt zwingend L2Advertisement,\\damit Linux die Kommunikation von aussen nach innen leiten kann.\end{tabular} \\ \hdashline
KW13 & \begin{tabular}[c]{@{}l@{}}- Bereits jetzt viel  über Kubernetes, Ranger (rke2) und Helm gelernt.\\- Benchmarking lässt sich nicht automatisieren,\\die Tools sind zu gut abgesichert.\\Ungeplanter Mehraufwand wegen manuellem Ausführen.\end{tabular} \\ \hdashline
KW14 & \begin{tabular}[c]{@{}l@{}}HP-UX Probleme und ExaCC Ablöseprojekt bremste stark aus.\\StackGres Extension verursachte Probleme.\end{tabular} \\ \hdashline
KW15 & \begin{tabular}[c]{@{}l@{}}Viele Termine diese Woche.\\StackGres Extension Problem gelöst.\\Patroni macht weiterhin probleme mit etcd-Server\end{tabular} \\ \hdashline
KW16 & \begin{tabular}[c]{@{}l@{}}local-path-provisioner machte nochmals Probleme.\\Die ganze Zeit ohne Node-Annotation gearbeitet.\\Danach konnten auch die letzten Benchmarks gemacht werden.\end{tabular} \\ \hdashline
KW17 & \begin{tabular}[c]{@{}l@{}}Gegenüberstellung fertiggestellt.\\Variantenentscheid getroffen.\\Dokumentation die letzte Zeit vernachlässigt, das rächt sich nun.\\Grossen Change Dienstag auf Mittwoch, mehr oder weniger K.O.\end{tabular} \\ \hdashline
\caption{Kommentare - Anmerkung} \label{project_comments}
\end{longtable}
