\begin{table}[H]

\resizebox{\columnwidth}{!}{%

\begin{tabular}{rllll}
\toprule
Nr. & Ziel & Priorität & Ziel Umgesetzt / Erreicht & Begründung / Beschreibung \\ \hdashline
\midrule
1 & Evaluation & Hoch & Ziel erreicht & \begin{tabular}[c]{@{}l@{}}Es wurden drei Methoden genauer betrachtet und installiert\\Am Ende der Evaluation wird ein System ausgewählt,\\welches als Testsystem installiert wurde.\end{tabular} \\ \hdashline
2 & Testsystem & Hoch & Ziel erreicht & \begin{tabular}[c]{@{}l@{}}Das Testsystem wurde mit vitabaks/postgresql\_cluster,\\einem \Gls{Ansible}-Playbook für Patroni,\\zuverlässig und stabil deployt.\\Der Cluster lässt sich schnell und mit wenig Aufwand erweitern.\end{tabular} \\ \hdashline
3 & Automatisierter Failover & Hoch & Ziel grösstenteils erreicht & \begin{tabular}[c]{@{}l@{}}Patroni ist zusammen mit \Gls{HAProxy} in der Lage,\\den \Gls{Failover} Timeout tief zu halten. So tief, dass eine entsprechend konfiugurierte Applikation die Verbindung aufrecht erhalten könnte.\\Da vitabaks/postgresql\_cluster den \Gls{Connection Pooler} nicht auf dem Gls{HAProxy} Layer installiert sodern auf dem Patroni Layer,\\ist der Cluster selber nicht in der Lage die Connections in jedem Fall aufrecht zu erhalten. \end{tabular} \\ \hdashline
4 & Automatisierter Failover Restore & Hoch & Ziel erreicht & \begin{tabular}[c]{@{}l@{}}Patroni fährt einen Node, der ausgefallen ist, selbständig nach\end{tabular} \\ \hdashline
5 & Monitoring - Cluster Healthcheck & Mittel & Ziel erreicht & \begin{tabular}[c]{@{}l@{}}\Gls{PRTG} überwacht die Standard Vitalparameter wie CPU Load, Load Average, Memory Usage, Disk Usage, Network Traffic und die Uptime.\\Zusätzlich wurde ein separater Sensor erstellt,\\der den Cluster Status, die Gesundheit des gesamten Systems und das Replication Lag überwacht.\end{tabular} \\ \hdashline
6 & \Gls{AUTOVACUUM} - Parameter verwalten & Mittel & Ziel erreicht & \begin{tabular}[c]{@{}l@{}}Mit einem \Gls{Kubernetes} CornJob wird täglich der Parameter \texttt{autovacuum\_vacuum\_scale\_factor} berechnet.\end{tabular} \\ \hdashline
7 & SQL optimierungen - Indizes tracken und verwalten & Mittel & Nicht umgesetzt & \begin{tabular}[c]{@{}l@{}}Es fehlt zurzeit das konkrete Wissen über den \Gls{PostgreSQL} Explain Plan,\\also jenes Hilfsmittel mit dem SQL Statements analysiert werden.\\Ohne dieses Wissen lässt sich schwer ermitteln, ob es einen Index auf einer Tabelle benötigt.\\Nicht benötigte Indizes können zwar ermittelt werden,\\allerdings kann sich der Ausführungsplan eines SQL Statements nach dem Löschen stark verändern.\\Das Risiko einen Index ohne Explaining Plan zu löschen, ist zu gross.\end{tabular} \\ \hdashline
8 & Maintenance - Indizes säubern & Hoch & Ziel erreicht & \begin{tabular}[c]{@{}l@{}}Aufgeblähte Tabellen werden erkannt und vakuumiert.\\Anschliessend werden die Indizes einer vakuumierten Tabelle neu aufgebaut.\end{tabular} \\ \hdashline
9 & Housekeeping - Log Rotation & Hoch & Ziel erreicht & \begin{tabular}[c]{@{}l@{}}vitabaks/postgresql\_cluster und Patroni selbst bieten konfigurationensmöglichkeiten für Log Rotationen.\\Da die Umsetzung des Gls{SIEM} Projekts erst gestartet ist, wurden nur rudimentäre einstellungen vorgenommmen.\end{tabular} \\ \hdashline
10 & User Management - Monitoring & Tief & Nicht umgesetzt & \begin{tabular}[c]{@{}l@{}}Ziel wurde nicht verfolgt da es nur eine tiefe Priorität hatte.\end{tabular} \\ \hdashline
11 & Evaluationsziel & Hoch & Ziel erreicht & \begin{tabular}[c]{@{}l@{}}Mit Patroni wurde eine Variante evaluiert und mit vitabaks/postgresql\_cluster auch die Installationsmethode.\end{tabular} \\ \hdashline
12 & Installationsziel & Hoch & Ziel erreicht & \begin{tabular}[c]{@{}l@{}}vitabaks/postgresql\_cluster erreicht Ziel und 3 und 4\end{tabular} \\ \hdashline
13 & Testziele & Hoch & Ziel grösstenteils erreicht & \begin{tabular}[c]{@{}l@{}}Das erste Teilziel lässt sich in einem \Gls{Quorum}-System schlicht nicht umsetzen.\\Die restlichen Teilziele wurden erreicht.\end{tabular} \\ \hdashline
\bottomrule
\end{tabular}
}
\caption{Result - Zielüberprüfung} \label{result_goal_review}
\end{table}
