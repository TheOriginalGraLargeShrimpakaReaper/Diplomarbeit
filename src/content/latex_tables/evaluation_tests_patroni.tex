\begin{table}[H]

\resizebox{\columnwidth}{!}{%

\begin{tabular}{lrllll}
\toprule
Art & Test Case Nr. & Test Case & Erwartetes Ergebnis & Eingetretenes Ergebnis & Begründung \\
\midrule
Failover & 1 & Automatismus & \begin{tabular}[c]{@{}l@{}}Wird der Primary Server vom Netz genommen,\\führt Patroni einen Failover auf einen Replika-Node\end{tabular} & \begin{tabular}[c]{@{}l@{}}Eingetroffen\end{tabular} & \begin{tabular}[c]{@{}l@{}}\end{tabular} \\
Failover & 2 & Connection-Stabilität & \begin{tabular}[c]{@{}l@{}}Bestehende Connections dürfen nicht getrennt werden.\end{tabular} & \begin{tabular}[c]{@{}l@{}}Nicht eingetroffen\end{tabular} & \begin{tabular}[c]{@{}l@{}}Connection-Stabilität kann nur hergestellt werden,\\wenn entweder die Applikation dazu in der Lage ist\\oder man einen Connection-Pooler wie pgBouncer einsetzt.\\Es wurde aber keiner eingesetzt.\end{tabular} \\
Failover & 3 & Geschwindigkeit & \begin{tabular}[c]{@{}l@{}}Der Failover muss so schnell stattfinden,\\dass offene Connections nicht wegen eines Timeouts geschlossen werden.\end{tabular} & \begin{tabular}[c]{@{}l@{}}Bedingt eingetroffen\end{tabular} & \begin{tabular}[c]{@{}l@{}}Auch hier hängt die stabilität an den Settings der Applikation\\und oder einem Connection-Pooler.\end{tabular} \\
Switchover & 4 & Skript / API & \begin{tabular}[c]{@{}l@{}}Mit der Patroni REST-API wird der Switchover ausgeführt\end{tabular} & \begin{tabular}[c]{@{}l@{}}Eingetroffen\end{tabular} & \begin{tabular}[c]{@{}l@{}}\end{tabular} \\
Switchover & 5 & Skript / API & \begin{tabular}[c]{@{}l@{}}Mit dem Patroni Commandset wird er Switchover ausgeführt\end{tabular} & \begin{tabular}[c]{@{}l@{}}Eingetroffen\end{tabular} & \begin{tabular}[c]{@{}l@{}}\end{tabular} \\
Switchover & 6 & Connection-Stabilität & \begin{tabular}[c]{@{}l@{}}Bestehende Connections dürfen nicht getrennt werden.\end{tabular} & \begin{tabular}[c]{@{}l@{}}Eingetroffen\end{tabular} & \begin{tabular}[c]{@{}l@{}}\end{tabular} \\
Switchover & 7 & Geschwindigkeit & \begin{tabular}[c]{@{}l@{}}Der Switchover muss so schnell stattfinden,\\dass offene Connections nicht wegen eines Timeouts geschlossen werden.\end{tabular} & \begin{tabular}[c]{@{}l@{}}Nicht eingetroffen\end{tabular} & \begin{tabular}[c]{@{}l@{}}Connection-Stabilität kann nur hergestellt werden,\\wenn entweder die Applikation dazu in der Lage ist\\oder man einen Connection-Pooler wie pgBouncer einsetzt.\\Es wurde aber keiner eingesetzt.\end{tabular} \\
Restore & 9 & Skript / API & \begin{tabular}[c]{@{}l@{}}Mit der Patroni REST-API wird der Primary-Node Wiederhergestellt\end{tabular} & \begin{tabular}[c]{@{}l@{}}Eingetroffen\end{tabular} & \begin{tabular}[c]{@{}l@{}}\end{tabular} \\
Restore & 10 & Skript / API & \begin{tabular}[c]{@{}l@{}}Mit dem Patroni Commandset der Primary-Node Wiederhergestellt\end{tabular} & \begin{tabular}[c]{@{}l@{}}Eingetroffen\end{tabular} & \begin{tabular}[c]{@{}l@{}}\end{tabular} \\
Restore & 11 & Skript / API & \begin{tabular}[c]{@{}l@{}}Mit der Patroni REST-API wird ein Replika-Node Wiederhergestellt\end{tabular} & \begin{tabular}[c]{@{}l@{}}Eingetroffen\end{tabular} & \begin{tabular}[c]{@{}l@{}}\end{tabular} \\
Restore & 12 & Skript / API & \begin{tabular}[c]{@{}l@{}}Mit dem Patroni Commandset ein Replika-Node Wiederhergestellt\end{tabular} & \begin{tabular}[c]{@{}l@{}}Eingetroffen\end{tabular} & \begin{tabular}[c]{@{}l@{}}\end{tabular} \\
Restore & 13 & Datensicherheit & \begin{tabular}[c]{@{}l@{}}Beim Restore des Primary-Nodes dürfen keine Daten, \\die seit dem Failover gechrieben wurden,\\darf es zu keinem Datenverlust kommen\end{tabular} & \begin{tabular}[c]{@{}l@{}}Eingetroffen\end{tabular} & \begin{tabular}[c]{@{}l@{}}\end{tabular} \\
Restore & 14 & Connection-Stabilität & \begin{tabular}[c]{@{}l@{}}Beim Restore des Primary-Nodes dürfen keine Connections geschlossen werden.\end{tabular} & \begin{tabular}[c]{@{}l@{}}Nicht eingetroffen\end{tabular} & \begin{tabular}[c]{@{}l@{}}Connection-Stabilität kann nur hergestellt werden,\\wenn entweder die Applikation dazu in der Lage ist\\oder man einen Connection-Pooler wie pgBouncer einsetzt.\\Es wurde aber keiner eingesetzt.\end{tabular} \\
\bottomrule
\end{tabular}
}
\caption{Testresultate Evaluation Patroni} \label{evaluation_tests_patroni}
\end{table}
