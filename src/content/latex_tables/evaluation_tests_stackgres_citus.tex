\begin{table}[H]

\resizebox{\columnwidth}{!}{%

\begin{tabular}{lrllll}
\toprule
Art & Test Case Nr. & Test Case & Erwartetes Ergebnis & Eingetretenes Ergebnis & Begründung \\ \hdashline
\midrule
Failover & 1 & Automatismus & \begin{tabular}[c]{@{}l@{}}Wird der Primary Server vom Netz genommen,\\führt Patroni einen Failover auf einen Replika-Node.\end{tabular} & \begin{tabular}[c]{@{}l@{}}Eingetroffen\end{tabular} & \begin{tabular}[c]{@{}l@{}}\end{tabular} \\ \hdashline
Failover & 2 & Connection-Stabilität & \begin{tabular}[c]{@{}l@{}}Bestehende Connections dürfen nicht getrennt werden.\end{tabular} & \begin{tabular}[c]{@{}l@{}}Nicht eingetroffen\end{tabular} & \begin{tabular}[c]{@{}l@{}}Keine.\\StackGres setzt envoy ein.\\Offensichtlich nicht ber einen ganzen Cluster.\end{tabular} \\ \hdashline
Failover & 3 & Geschwindigkeit & \begin{tabular}[c]{@{}l@{}}Der Failover muss so schnell stattfinden,\\dass offene Connections nicht\\wegen eines Timeouts geschlossen werden.\end{tabular} & \begin{tabular}[c]{@{}l@{}}Nicht eingetroffen\end{tabular} & \begin{tabular}[c]{@{}l@{}}Keine.\\StackGres setzt envoy ein.\\Offensichtlich nicht ber einen ganzen Cluster.\end{tabular} \\ \hdashline
Sharding & 4 & Datenkonsistenz und Datenintegrität & \begin{tabular}[c]{@{}l@{}}Daten sind Konsistent und Inetger.\end{tabular} & \begin{tabular}[c]{@{}l@{}}Eingetroffen\end{tabular} & \begin{tabular}[c]{@{}l@{}}\end{tabular} \\ \hdashline
Sharding & 5 & Schutz vor Datenverlust & \begin{tabular}[c]{@{}l@{}}Die Daten müssen Konsistent und schnell auf die Shards verteilt werden.\end{tabular} & \begin{tabular}[c]{@{}l@{}}Eingtetroffen\end{tabular} & \begin{tabular}[c]{@{}l@{}}\end{tabular} \\ \hdashline
Self Healing & 6 & Node stellt sich selber wieder her & \begin{tabular}[c]{@{}l@{}}Shard Node wird automatisch synchronisiert.\end{tabular} & \begin{tabular}[c]{@{}l@{}}Eingtetroffen\end{tabular} & \begin{tabular}[c]{@{}l@{}}\end{tabular} \\ \hdashline
Self Healing & 7 & Leader wird automatisch gesetzt & \begin{tabular}[c]{@{}l@{}}Leader wird entweder beibehalten\\oder wird neu gesetzt wenn ein Node zurückkehrt.\end{tabular} & \begin{tabular}[c]{@{}l@{}}Eingetroffen\end{tabular} & \begin{tabular}[c]{@{}l@{}}\end{tabular} \\ \hdashline
\bottomrule
\end{tabular}
}
\caption{Testresultate Evaluation StackGres - Citus} \label{evaluation_tests_stackgres_citus}
\end{table}
