\begin{table}[H]

\resizebox{\columnwidth}{!}{%

\begin{tabular}{rllllllll}
\toprule
Fachgespräch & Datum & Fachexperte & Nebenexperte & Studenten & Fragen & Antworten & Sonstige Themen & Bemerkungen \\
\midrule
1 & 14.02.2024 & Norman Süsstrunk & - & \begin{tabular}[c]{@{}l@{}}Michael Graber\\Curdin Roffler\end{tabular} & \begin{tabular}[c]{@{}l@{}}- Darf eine Vorauswahl stattfinden, um den Aufwand zur reduzieren?\end{tabular} & \begin{tabular}[c]{@{}l@{}}- Eine Vorauswahl ist Sinnvoll und in diesem Rahmen fast zwingend Notwendig,\\  da sonst viel zuviel Zeit investiert werden müsste\end{tabular} & \begin{tabular}[c]{@{}l@{}}- Vorstellung Norman Süsstrunk, Curdin Roffler und Michael Graber\\- Kontaktdaten shared\\- Bei Fragen jederzeit an Norman wenden\\- Norman braucht aber mindestens 1. Woche vorlaufzeit\\- Norman wird sich spätestens zur Halbzeit melden.\\- Norman wird sic\end{tabular} & \begin{tabular}[c]{@{}l@{}}- Es wurden zwar für alle Studenten von Norman Süsstrunk Zoom-Räume bereitgestellt,\\  aus effizienzgründen nahmen Curdin Roffler und ich beide am selben Meeting teil\end{tabular} \\
2 & 26.03.2024 & Norman Süsstrunk & - & \begin{tabular}[c]{@{}l@{}}Michael Graber\end{tabular} & \begin{tabular}[c]{@{}l@{}}- Muss das Protokoll des Fachgesprächs jeweils Zeitnah freigegeben werden?\\- Hat Norman ggf. noch vorschläge zu PostgreSQL Clustern gefunden?\\- Soll ich die Gewichtung mit 100 Punkten machen oder 1000?\\  Im Moment haben diverse Punkte eine sehr kleine Punktzahl\\- Soll die Disposition in den Anhang?\\  Diese ist 50 Seiten lang\end{tabular} & \begin{tabular}[c]{@{}l@{}}\end{tabular} & \begin{tabular}[c]{@{}l@{}}- Protokoll genehmigen\end{tabular} & \begin{tabular}[c]{@{}l@{}}\end{tabular} \\
\bottomrule
\end{tabular}
}
\caption{Fachgespräche - Protokoll} \label{expert_discussions_full_list}
\end{table}
