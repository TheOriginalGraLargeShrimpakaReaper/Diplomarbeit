\begin{table}[H]

\resizebox{\columnwidth}{!}{%

\begin{tabular}{lllrllllll}
\toprule
Datum & Von & Bis & Dauer [h] & Phase & Subphase & Tätigkeit & Bemerkung & Schwierigkeit & Lösungen \\
\midrule
21.02.2024 & 15:00 & 16:00 & 1.0 & Evaluation & Anorderungskatalog & \begin{tabular}[c]{@{}l@{}}Anorderungskatalog erarbeiten\end{tabular} & \begin{tabular}[c]{@{}l@{}}\end{tabular} & \begin{tabular}[c]{@{}l@{}}\end{tabular} & \begin{tabular}[c]{@{}l@{}}\end{tabular} \\
22.02.2024 & 16:00 & 17:30 & 1.5 & Evaluation & Anorderungskatalog & \begin{tabular}[c]{@{}l@{}}Anorderungskatalog erarbeiten\end{tabular} & \begin{tabular}[c]{@{}l@{}}\end{tabular} & \begin{tabular}[c]{@{}l@{}}\end{tabular} & \begin{tabular}[c]{@{}l@{}}\end{tabular} \\
27.02.2024 & 10:00 & 11:30 & 1.5 & Dokumentation & - & \begin{tabular}[c]{@{}l@{}}Dokumentation erweitern\end{tabular} & \begin{tabular}[c]{@{}l@{}}\end{tabular} & \begin{tabular}[c]{@{}l@{}}\end{tabular} & \begin{tabular}[c]{@{}l@{}}\end{tabular} \\
27.02.2024 & 13:00 & 16:00 & 3.0 & Dokumentation & - & \begin{tabular}[c]{@{}l@{}}Dokumentation erweitern\end{tabular} & \begin{tabular}[c]{@{}l@{}}\end{tabular} & \begin{tabular}[c]{@{}l@{}}Viele LaTEX Tabellen.\end{tabular} & \begin{tabular}[c]{@{}l@{}}Generator mit python pandas gebaut für alle möglichen Tabellen.\\Inkl. Aggregation und Pivot-Mechaniken\end{tabular} \\
28.02.2024 & 09:00 & 11:00 & 2.0 & Dokumentation & - & \begin{tabular}[c]{@{}l@{}}Dokumentation erweitern\end{tabular} & \begin{tabular}[c]{@{}l@{}}\end{tabular} & \begin{tabular}[c]{@{}l@{}}Viele LaTEX Tabellen.\end{tabular} & \begin{tabular}[c]{@{}l@{}}Generator mit python pandas gebaut für alle möglichen Tabellen.\\Inkl. Aggregation und Pivot-Mechaniken\end{tabular} \\
01.03.2024 & 07:00 & 09:00 & 2.0 & Dokumentation & - & \begin{tabular}[c]{@{}l@{}}Dokumentation Exkurs Architektur\end{tabular} & \begin{tabular}[c]{@{}l@{}}Um Entscheidungen Transparent zu machen,\\müssen Grundlegende Konzepte aufgezeigt werden.\\Nicht alle Konzepte wie z.B. Distributed SQL sind bekannt resp. das zusammenspiel mit Kubernetes.\end{tabular} & \begin{tabular}[c]{@{}l@{}}Konzepte wie Distributed SQL sind nicht einfach zu erklären.\end{tabular} & \begin{tabular}[c]{@{}l@{}}\end{tabular} \\
08.03.2024 & 07:00 & 09:00 & 2.0 & Evaluation & Anorderungskatalog & \begin{tabular}[c]{@{}l@{}}Anorderungskatalog erarbeiten\end{tabular} & \begin{tabular}[c]{@{}l@{}}\end{tabular} & \begin{tabular}[c]{@{}l@{}}\end{tabular} & \begin{tabular}[c]{@{}l@{}}\end{tabular} \\
11.03.2024 & 07:00 & 11:30 & 4.5 & Evaluation & Analyse PostgreSQL HA Cluster Lösungen & \begin{tabular}[c]{@{}l@{}}Informationen Sammeln\end{tabular} & \begin{tabular}[c]{@{}l@{}}pgpool II\end{tabular} & \begin{tabular}[c]{@{}l@{}}pgpool II hat kein GitHub Repository.\\Das macht es unmöglich, diese Lösung mit all den anderen zu vergleichen.\end{tabular} & \begin{tabular}[c]{@{}l@{}}pgpool II fällt somit direkt aus der betrachtung raus,\\da kein vergleich möglich ist.\end{tabular} \\
11.03.2024 & 12:00 & 13:30 & 1.5 & Dokumentation & - & \begin{tabular}[c]{@{}l@{}}Dokumentation erweitern\end{tabular} & \begin{tabular}[c]{@{}l@{}}\end{tabular} & \begin{tabular}[c]{@{}l@{}}\end{tabular} & \begin{tabular}[c]{@{}l@{}}\end{tabular} \\
11.03.2024 & 16:45 & 17:30 & 0.5 & Dokumentation & - & \begin{tabular}[c]{@{}l@{}}Dokumentation Stakeholder\end{tabular} & \begin{tabular}[c]{@{}l@{}}\end{tabular} & \begin{tabular}[c]{@{}l@{}}\end{tabular} & \begin{tabular}[c]{@{}l@{}}\end{tabular} \\
13.03.2024 & 17:45 & 19:45 & 2.0 & Evaluation & Analyse PostgreSQL HA Cluster Lösungen & \begin{tabular}[c]{@{}l@{}}Stackgres und Citus analysieren\end{tabular} & \begin{tabular}[c]{@{}l@{}}Citus row-based-sharding\end{tabular} & \begin{tabular}[c]{@{}l@{}}Citus Dokumentation stark Textlastig.\\Wenig Abbildungen, vieles muss selber gezeichnet werden.\end{tabular} & \begin{tabular}[c]{@{}l@{}}\end{tabular} \\
14.03.2024 & 19:45 & 20:45 & 1.0 & Evaluation & Analyse PostgreSQL HA Cluster Lösungen & \begin{tabular}[c]{@{}l@{}}\end{tabular} & \begin{tabular}[c]{@{}l@{}}Citus row-based-sharding\end{tabular} & \begin{tabular}[c]{@{}l@{}}\end{tabular} & \begin{tabular}[c]{@{}l@{}}\end{tabular} \\
14.03.2024 & 20:45 & 21:30 & 0.8 & Dokumentation & - & \begin{tabular}[c]{@{}l@{}}Citus row-based-sharding Dokumentieren\end{tabular} & \begin{tabular}[c]{@{}l@{}}\end{tabular} & \begin{tabular}[c]{@{}l@{}}\end{tabular} & \begin{tabular}[c]{@{}l@{}}\end{tabular} \\
\bottomrule
\end{tabular}
}
\caption{Arbeitsrapport} \label{arbeitsrapport}
\end{table}
