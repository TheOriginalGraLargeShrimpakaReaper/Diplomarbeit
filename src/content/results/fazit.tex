%! Author = gramic
%! Date = 12.05.24

% Preamble
\begin{flushleft}
    \section{Persönliches Fazit}
    Aus heutiger Sicht würde ich die Ziele weiter reduzieren.\\
    Die Evaluation alleine hat wesentlich mehr Zeit in Anspruch genommen, als geplant.\\
    Heute würde ich auf die Umsetzung eines Testsystems verzichten und stattdessen ein oder zwei Systeme mehr evaluieren und installieren.\\
    Generell habe ich wohl unterschätzt, wie Tief der Dive in die Konzepte und Architekturen gehen wird.
\end{flushleft}
\begin{flushleft}
    Den gesamten Zeitplan aber auch die Dokumentenstruktur würde ich heute anders aufbauen.\\
    Die Evaluation hätte wesentlich mehr Raum und auch den Maintenance-Tools stünde eigene Arbeitspakete zur Verfügung.\\
    Das Projektmanagement hat keine eigenen Arbeitspakete erhalten und musste daher immer an anderer Stelle abgebucht werden.\\
    Ein weiterer Punkt den ich anders machen würde, gerade mit einer etwas ausgedehnteren Evaluationsphase wäre mehr Zeit für das Projektmanagement vorhanden gewesen.
\end{flushleft}
\begin{flushleft}
    Die Diplomarbeit zwang mich aus meiner Komfortzone herauszutreten.\\
    Ich habe ungleich mehr DevOps und System Engineering Arbeiten als DBA gemacht, von denen gab es schlussendlich nur eine Handvoll.
\end{flushleft}
\begin{flushleft}
    Erkenntnisse und Lehren habe ich auf zwei Perspektiven gewonnen.\\
    Einmal aus der Technischen Sicht.\\
    So konnte ich mit \gls{rke2} von Grund auf ein \Gls{Kubernetes} System aufbauen und mit \gls{local-path-provisioner} sowie \Gls{MetalLB} Sinnvoll erweitern.\\
    Zudem durfte ich tiefe Einblicke in neue und zukunftsfähige Konzepte wie den Distributed SQL Systemen gewinnen.\\
    Als DBA gehörte ich eher zu einer Konservativen IT-Sparte, die nach wie vor stark auf Monolithischer Systeme setzt und neuen Ansätzen eher Misstrauisch gegenübersteht.\\
    Durch die Diplomarbeit habe ich meinen Horizont um verteilte Systeme und dem CloudNative Ansatz erweitert.\\
    Kommt hinzu, dass es für die Evaluation sehr reizvoll war, ein falsch konfiguriertes System einfach zu löschen und angepasst neu zu deployen.\\
    Bei Patroni, dass ein monolithisches System war, brauchte es immer einen Snapshot Restore.
\end{flushleft}
\begin{flushleft}
    Auf der anderen Seite habe ich auch aus der Sicht der Herangehensweise eine wichtige Lektion gelernt.
    Beim Aufbau von \gls{rke2} sowie dem Umgang mit \gls{local-path-provisioner} und \Gls{MetalLB} hatte ich mit diversen Problemen zu kämpfen,\\
    doch gelang es mir, durch intensive Recherche jeweils Lösungen für diese Probleme zu finden.\\
    Beim Aufbau des Patroni Evaluationssystems brachte dieser Zeitraubende Ansatz keinen Erfolg mehr.\\
    Daher war ich gezwungen, einen Schritt zurückzutreten und das System auf das absolut notwendige Minimum zu vereinfachen.\\
    Mit nur noch einem \gls{etcd} Node, der auf einem anderen Server installiert war, gelang der Aufbau des Evaluationssystems.\\
    Diesen Ansatz nahm ich mit Erfolg mit, als ich bei den Maintenance-Tools wieder auf herausforderungen traf.
\end{flushleft}
\begin{flushleft}
    Auf einen Satz reduziert, dürfte daher die wichtigste Erkenntnis aus der Diplomarbeit folgende sein:\\
    \guillemotleft Im Zweifelsfall geht man besser einen Schritt zurück und nimmt soviel Komplexität aus einem System wie möglich!\guillemotright
\end{flushleft}