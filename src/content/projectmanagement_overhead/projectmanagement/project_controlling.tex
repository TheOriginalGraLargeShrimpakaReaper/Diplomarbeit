%! Author = gramic
%! Date = 29.02.24

% Preamble
%\begin{landscape}
\begin{flushleft}
\clearpage
\KOMAoptions{paper=A3,paper=landscape,pagesize,DIV=20}
\recalctypearea
\pagestyle{headings}
%    \clearpage
%    \KOMAoptions{paper=A3,paper=landscape,pagesize,DIV=20}
%    \recalctypearea
    \chapter{Projektmanagement}
    \section{Projektplanung}
    \subsection{Projektcontrolling}
    \begin{table}[H]

\resizebox{\columnwidth}{!}{%

\begin{tabular}{lllrrr}
\toprule
 & Phase & Subphase & Dauer [h] & Geplante Dauer [h] & Verbleibende Zeit [h] \\
\midrule
0 & Dokumentation & - & 11.2 & 80 & 68.8 \\
1 & Evaluation & Analyse PostgreSQL HA Cluster Lösungen & 7.5 & 16 & 8.5 \\
2 & Evaluation & Anorderungskatalog & 4.5 & 16 & 11.5 \\
3 & Evaluation & Vorbereitung Benchmarking & 0.0 & 4 & 4.0 \\
\bottomrule
\end{tabular}
}
\caption{Projektcontrolling} \label{projektcontrolling}
\end{table}

\end{flushleft}
\clearpage
\KOMAoptions{paper=A4,paper=portrait,pagesize}
\recalctypearea
\begin{flushleft}
%\clearpage
%\KOMAoptions{paper=A4,paper=portrait,pagesize}
%\recalctypearea
%\pagestyle{headings}
%%\begin{flushleft}
%%    \begin{figure}[H]
%%        \centering
%%        \includegraphics[width=0.75\linewidth]{source/pandas_data_chart_plotter/projektcontrolling}
%%        \caption{Projektcontrolling}
%%        \label{fig:projektcontrolling}
%%    \end{figure}
%    \begin{figure}[H]
%        \centering
%        \includegraphics[width=0.75\linewidth]{source/pandas_data_chart_plotter/projektcontrolling_dauer}
%        \caption{Projektcontrolling - Dauer}
%        \label{fig:projektcontrolling_dauer}
%    \end{figure}
%\end{flushleft}
%\begin{flushleft}
%\clearpage
%\KOMAoptions{paper=A4,paper=portrait,pagesize}
%\recalctypearea
%    \begin{figure}[H]
%        \centering
%        \includegraphics[width=0.75\linewidth]{source/pandas_data_chart_plotter/projektcontrolling_geplant}
%        \caption{Projektcontrolling - Geplante Dauer}
%        \label{fig:projektcontrolling_geplant}
%    \end{figure}
%\end{flushleft}
%\clearpage
%\end{landscape}