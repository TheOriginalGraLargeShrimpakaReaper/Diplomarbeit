%! Author = itgramic
%! Date = 09.10.23

% Preamble
%\makeglossaries
\makenoidxglossaries
\newglossaryentry{HP-UX}
{
        name=HP-UX,
        description={Dieses UNIX-Derivat ist ein abkömmling von System III, System V R3 und System V R4 und wurde von HP zum ersten Mal 1982 veröffentlicht.}
}
\newglossaryentry{UNIX}
{
        name=UNIX,
        description={Die erste Version von UNIX wurde im Jahr 1969 in den Bell Labs entwickelt und übernahm viele Komponenten aus dem gescheiterten Multics-Projekt.
        Aus dem Ursprünglichen UNIX enstanden im Laufe der Zeit viele offene und Proprioritäre Derivate deren Einfluss weit über die Welt der Informatik reicht.}
}
\newglossaryentry{PostgreSQL}
{
        name=PostgreSQL,
        description={Die OpenSource Datenbank PostgreSQL wurde in Form von POSTGRES zum ersten Mal 1986 von der University of California at Berkeley veröffentlicht.
und zählt zu den beliebstesten OpenSource Datenbanken.
Zudem besteht in vielen bereichen eine gewisse ähnlichkeit zu Oracles Oracle Database.}
}
\newglossaryentry{Linux}
{
        name=Linux,
        description={Linux ist ein Open-Source Betriebssystem, welches von Linus Torvalds 1991 in seiner frühesten Form entwickelt wurde und lose vom UNIX Derivat MINIX inspiert war.
        Linux besteht heute aus einer enorm grossen Anzahl an Distributionen und läuft auf einer grossen Anzahl von Plattformen.}
}
\newglossaryentry{Debian}
{
        name=Debian,
        description={Debian gehört nebst Slackware Linux zu den ältesten Linux Distribution die noch immer gepflegt und eingesetzt werden.
        Sie wurde im August 1993 gestartet und brachte im Laufe der Zeit einige der beliebstesten Distributionen wie Ubuntu hervor.}
}
\newglossaryentry{RHEL}
{
        name=RedHat Enterpise Linux (RHEL),
        description={RHEL wurde in seiner Ursprüglichen Form Red Hat Linux (RHL) bis in den Oktober 1994 zurück, wobei die erste Version von RHEL wie es heute existiert im Jahr 2002 erfolgte.
        RHEL ist auf lange Wartungszyklen von fünf Jahren und grosskunden ausgelegt.
        Ohne entsprechenden Supportvertrag kann keine ISO-Datei bezogen werden.
        Somit hebt sich RHEL stark von aderen Linux Distributionen ab.}
}
\newglossaryentry{Rocky Linux}
{
        name=Rocky Linux,
        description={Rocky Linux basierte auf der offen zugänglichen Linux Distribution CentOS welche RHEL Binärkompatibel war und gilt als inoffizieller Nachfolger von CentOS.}
}
\newglossaryentry{Oracle Linux}
{
        name=Oracle Linux,
        description={Oracle Linux ist eine RHEL-Distribution der Firma Oracle und ist mit RHL Binärkompatibel.
        Sie wird primär für den Betrieb von Oracle Datenbanken verwendet und komnt auf den Oracle Eigenenen Appliances ODA und Exadata zum Einsatz.
        Für den Zweck als DB Plattform kann ein für Oracle Datenbanken optimimierter Kernel verwendet werden.
        Zu Oracle Linux kann ein kostenpflichtiger Support bezogen werden, allerdings ist die Distribution anders als RHEL auch ohne Lizenz erhältlich.}
}
\newglossaryentry{MINIX}
{
        name=MINIX,
        description={Ist ein freies unixoides Betriebssystem des Infromatikers Andrew S. Tanenbaum welches zum ersten Mal 1987 Entwickelt wurde und als Inspiration von Linux diente.
        MINIX wurde in der Lehre eingesetzt und wird in der Intel Active Management Technology von Intel-Chips verwendet und gehört damit zu den am meisten eingesetzten Betriebssystemen überhaupt.}
}
\newglossaryentry{Oracle Database}
{
        name=Oracle Database,
        description={Die erste verfügbare Version der Oracle Datenbank kam im Jahr 1979 mit Version 2 (statt Version 1) heraus, damals allerdings nur mit den Basisfunktionen.
        Im Laufe der Zeit wuchs der Funktionsumfang sehr stark an, die Grundlage des Client-Server-Designs kam erstmals im Jahr 1985 mit Version auf den Markt und hat sich im Prinzip bis heute gehalten.
        Mit der mit Version 8/8i 1997 erschienen Optimizer und mit der Version 9i 2001 erschienenn Flashback-Funktionalität (die ein schnelles Online Recovery sowie einen Blick in die Vergangheit ermöglichen) konnte Oracle sich stark von der Konkurenz absetzen.
        Heute gilt die Datenbank als erste Wahl, wenn es um Hochverfügbare Systeme, hohe Perforamce oder grosse Datenmengen geht.}
}
\newglossaryentry{MySQL}
{
        name=MySQL,
        description={Die Datenbank MySQL wurde Ursprünglich als reine Relationale Open-Source Datenbank von Firma MySQL AB 1994 Entwickelt.
        Der Name My leitet sich vom Namen My der Tochter des Mitbegründers Michael Widenius ab.
        Als Sun Microsystem 2008 MySQL übernahm, hielt sich die Option frei, bei einem Kauf von Sun Microsyszem durch Oracle gründen zu dürfen.
        Seit Oracle Sun Microsystem 2010 gekauft hat, wurden immer mehr Funktionalitäten von der Community Edition zu der Enterprise Edition verschoben worden.
        Aus diesem Grund hat heute der MySQL Fork MariaDB MySQL mehrheitlich aus allen Linux Distributionen als Standard Datenbank verdrängt.}
}
\newglossaryentry{MariaDB}
{
        name=MariaDB,
        description={MariaDB ist ein MySQL Fork des ehemaligen MySQL Mitbegründers Michael Widenius, wobei sich der Name Maria aus dem VOrnamen einer seiner Töchter ableitet.
        NAch dem Fork 2009 blieb MariaDB für eine Zeitlang sehr ähnlich mit MySQL und behielt ein ähnliches Versionierungsschema bei.
        Dies änderte sich 2012 wo dann direkt mit der Version 10 weitergefahren wurde.
        Beide Datenbanken entfernen sich im Lauf eder Zeit immer mehr voneinander undf sind nicht mehr in jdem Fall kompatibel oder beliebig austauschbar.
        Auf den Linux Distributionen tratt MariaDB die Nachfolge von MySQL als Standard Datenbank an.}
}
\newglossaryentry{Zabbix}
{
        name=Zabbix,
        description={Das 2001 veröffentlichte Open-Source Monitoring System Zabbix gilt zwar als Netzwerk-Monitoring System, allerdings kann heute nahezu jedes IT-System damit überwacht werden.
        Zabbrix speichert die Metriken und nicht die Auswertungen, das heisst, solange die Daten vorhanden sind können Grafiken zu jedem Zeitpunkt generiert worden.
        Zabbix ist grundsätzlich Open-Source, man kann allerdings Supportverträge Abschliessen.}
}
\newglossaryentry{PRTG}
{
        name=PRTG,
        description={Das Monitoring System Paessler Router Traffic Grapher der Firma Paessler wurde 2003 zum erstmals veröffentlicht und war ebenfalls als Netzwerkmonitoring System konzipiert.
        Wie bei Zabbix lässt sich heute damit ebenfalls fast jedes IT-System damit Überwachen.
        Reichen die Zahlreich vorhanden Standard Sensoren nicht, können eigene Sensoren geschrieben werden.
        PRTG ist nicht Open-Source, man bezahlt anhand gewisser Sensor Packages.}
}
\newglossaryentry{Foreman}
{
        name=Foreman,
        description={Foreman ist ein Lifecycle Management und Provisioning System für Virtuelle und Physische Server.
        Ab Version 6 basierte der Red Hat Satellite auf Foreman}
}
\newglossaryentry{VMware vSphere}
{
        name=VMware vSphere,
        description={Die vSphere® ist ein Typ-1 Hypervisor der Firma VMware® der eine reihe leistungsstarker Tools und Funktionen mitbringt.}
}
\newglossaryentry{Kubernetes}
{
        name=Kubernetes,
        description={Kubernetes, oder k8s, ist eine Open-Source Containerplattform die ursprünglich von Google 2014 für die Bereitstellung und Orchestrierung von Containern entwickelt wurde aber 2015 an eine Tochter Foundation der Linux Foundation gespendet.
        Kubernetes kommt aus dem Griechischen und bedeutet Steuermann.}
}
\newglossaryentry{PostgreSQL Failure}
{
        name=PostgreSQL Failure,
        description={Ein PostgreSQL Failure, der zu einem Failover führt, entsteht dann, wenn ein \Gls{PostgreSQL Cluster} nicht mehr auf Anfragen reagieren kann wegen Netzwerkproblemen oder weil interne Prozesse hängen.
        Ein weiterer Grund kann sein, dass das Betriebssystem nicht mehr reagiert oder abgestürzt ist.}
}
\newglossaryentry{PostgreSQL Cluster}
{
        name=PostgreSQL Cluster,
        description={Ein PostgreSQL Cluster entspricht einer Instanz bei MS SQL oder einer Container Database wei Oracle.}
}
\newglossaryentry{PostgreSQL HA Cluster}
{
        name=PostgreSQL HA Cluster,
        description={Der HA Cluster des \Gls{PostgreSQL Cluster}s}
}
\newglossaryentry{GitLab}
{
        name=GitLab,
        description={GitLab ist ein \Gls{Git} basierendes System für die Versionierung und bietet dabei auch noch Dienste für CI/CD.
        GitLab kann sowohl als Online Dienst als auch als On-premises Service konsumiert werden\cite{MPSC6ELK}.}
}
\newglossaryentry{CI/CD}
{
        name=CI/CD,
        description={Continuous Integration/Continuous Delivery bedeutet,
        das Anpassungen kontinuirlich in die Entwicklungsumgebungen integriert und
        auf die Zielplattformen verteilt werden\cite{I65F7WAQ}.}
}
\newglossaryentry{Harbor}
{
        name=Harbor,
        description={Harbor ist ein Open-Source-Tool zur Registrierung von Richtlinien rollenbasierten Zugriffssteuerung\cite{PV6GD72X}.
Harbor wird beim KSGR zur Verwaltung der \Gls{Kubernetes}-Plattform verwendet.}
}
\newglossaryentry{Failover}
{
        name=Failover,
        description={In einem Fehlerfall wird in einem HA-System meist ein Primary Node auf den Secondary ungeplant geswitched.}
}
\newglossaryentry{Switchover}
{
        name=Switchover,
        description={In einem Maintenance-Fall in einem HA-System meist ein Primary Node auf den Secondary geplant geswitched.}
}
\newglossaryentry{SAN}
{
        name=SAN,
        description={Ein Storage Area Network ist ein dediziertes Netzwerk aus Storage Komponenten.
        SAN Systeme bieten redundante Pools an Speicher.
        Die Physischen Festplatten werden zu Virtuellen Lunes, also logischen Einheiten, zusammengefasst.
        Dies werden nach aussen den Konsumenten präsentiert\cite{ZRRXBFRA,7ZTCYW5G,JWVC9B7L}}
}
\newglossaryentry{SIEM}
{
        name=SIEM,
        description={Ein sammelt Daten aus verschieden Netzwerkkomponenten oder Geräten von Agents oder Logs.
        Diese Daten werden permanent analysiert und mit einem definierten Regelwerk gegengeprüft.
        Ziel ist es, verdächtige Events zu erkennen und einem Angriff zuvorzukommen oder ihn möglichst früh zu unterbinden\cite{78JPTB5R}.}
}
\newglossaryentry{Redis}
{
        name=Redis,
        description={Redis ist eine \Gls{Key-Value-orientierte} \Gls{NoSQL} In-Memory-Datenbank, dh.
        die Daten liegen Primär im Memory und nicht auf dem Storage\cite{57XLMIRR}.
Redis wurde 2009 zum ersten Mal veröffentlicht.}
}
\newglossaryentry{NoSQL}
{
        name=NoSQL,
        description={NoSQL steht für Not only SQL. Das heisst, Relationale Datenbanken haben komponenten wie Dokumentendatenbanken,
Graphendatenbanken, \Gls{Key-Value-Datenbank}en und Spaltenorientiert Datenbanken.
Viele der grossen Datenbanklösungen wie \Gls{Oracle Database} oder \Gls{Microsoft SQL Server} sind NoSQL Datenbanken resp.
bieten diese option an.}
}
\newglossaryentry{MongoDB}
{
        name=MongoDB,
        description={MongoDB ist eine dokumentenorientierte \Gls{NoSQL}-Datenbank, die zurm ersten Mal 2007 veröffentlicht wurde\cite{YCYFBRLE}.}
}
\newglossaryentry{Elasticsearch}
{
        name=Elasticsearch,
        description={Elasticsearch ist eine 2010 veröffentlichte Open-Source Suchmaschine die auf Basis von JSON-Dokumenten und einer \Gls{NoSQL}-Datenbank arbeitet\cite{LUHWDIWV}.}
}
\newglossaryentry{IBM DB2}
{
        name=IBM DB2,
        description={IBM DB2 ist eine Relationale Datenbank\cite{DJX54K3M} deren Vorläufer System-R von IBM zwischen 1975 und 1979 entwickelt wurde.
DB2 selber wurde 1983 von IBM veröffentlicht.}
}
\newglossaryentry{SQLite}
{
        name=SQLite,
        description={SQLite ist eine Relationale Embedded Datenbank welche seit 2000 existiert.
Sie verzichtet auf eine Client-Server-Architektur und kann in vielen Frameworks eingebunden werden\cite{JCLXWZSR}.}
}
\newglossaryentry{Snowflake}
{
        name=Snowflake,
        description={Snowflake ist eine Big Data Plattform die Data Warehousing, Data Lakes, Data Engineering und Data Science in einem Service vereint.
Die Daten werden in eigenen internen Relationalen und \Gls{NoSQL}-Datenbanken gespeichert\cite{QCM8CD5A,7VWNV2V4}}
}
\newglossaryentry{Microsoft Access}
{
        name=Microsoft Access,
        description={Access wurde 1992 veröffentlicht und ist Entwicklungsumgebung, Front- und Backend-Software und Relationale Datenbank in einem\cite{44L9YDSR}.}
}
\newglossaryentry{Cassandra}
{
        name=Cassandra,
        description={Cassandra ist eine Spaltenorganisierte \Gls{NoSQL}-Datenbank die 2008 veröffentlicht\cite{KA934RSV} wurde.}
}
\newglossaryentry{Splunk}
{
        name=Splunk,
        description={Splunk ist Big Data Plattform, Monitoring- und Security-Tool in einem\cite{GUPY8F7E, PH3HQUCR}.
}
}
\newglossaryentry{Microsoft Azure SQL Database}
{
        name=Microsoft Azure SQL Database,
        description={Microsoft Azure SQL Database oder auch Azure SQL ist eine Relationale Datenbank die von Microsoft für die Azure Cloud optimiert 2010 Entwickelt wurde\cite{QVZZTCG6}.}
}
\newglossaryentry{OLAP}
{
        name=OLAP,
        description={Eine Online Analytical Processing, kurz OLAP, ist eine Multirelationale resp.
        Multidimensionale Datenbanklösung.
Sie wird oft in Form eines Datenwürfels erklärt, kann aber auf verschiedene Arten umgesetzt werden\cite{W5LMN5ZM, 5D3IPPGJ}.
OLAP-Systeme bieten eine Hochperformante Analyse grosser Datenmengen und sind oftmals zentraler Teil eines Data-Warehouses.}
}
\newglossaryentry{SWOT}
{
        name=SWOT-Analyse,
        description={Eine SWOT-Analyse soll die Stärken (Strengths), Schwächen (Weaknesses), Chancen (Opportunities) und Risiken (Threads) für ein Unternehmen oder ein Projekt aufzueigen.
Anhand einer SWOT-Analyse werden i.d.R. anschliessend Strategien abgeleitet um mit den Stärken und Chancen die Schwächen und Risiken abzufangen oder anzumildern.}
}
\newglossaryentry{AUTOVACUUM}
{
        name=AUTOVACUUM,
        description={Der AUTOVACUUM Job räumt die Tablespaces und Data Files innerhalb von PostgreSQL sowie auf dem Filesystem nach Lösch- und Manipulations-Transaktionen auf,
aktualisiert Datenbank interne Statistiken und verhindert Datenverlust von selten genutzten Datensätzen\cite{9EUWGEF8}.}
}
\newglossaryentry{Microsoft SQL Server}
{
        name=Microsoft SQL Server,
        description={MS SQL Server ist das RDBMS von Microsoft\cite{6LRCXMLC}.
Nebst Microsoft Windows und Windows Server lässt es sich seit Version 2014 ebenfalls auf \Gls{Linux} Betreiben.
        In der Wirtschaft ist die primäre Plattform aber Windows Server.}
}
\newglossaryentry{Ansible}
{
        name=Ansible,
        description={Ansible ist ein Open-Soure Automatisierungstool zur Provisionierung, Konfiguration, Deployment und Orchestrierung.
Ansible verbindet sich auf auf die Zielgeräte und führt dort die Hinterlegten Module aus. Oft werden die verschieden Aufgaben in einem Skript, in einem sogenannten Playbook geschrieben werden\cite{7SPK583Y}.}
}
\newglossaryentry{Terraform}
{
        name=Terraform,
        description={Terraform ist ein Werkzeug für die Verwaltung von Infrastruktur mit Software zu steuern, sogenanntes Infrastructure as Code.
        Terraform wird sehr oft dafür benutzt um Container- und Cloudinfrastruktur ansteuern und verwalten zu können\cite{FMIBZY6N,U29WWCXR}.}
}
\newglossaryentry{RDBMS}
{
        name=RDBMS,
        description={Ein RDBMS ist ein Datenbankmanagementsystem für eine Relationale Datenbank.
        Relationale Datenbanken sind Tabellenorgansierte Datenmodelle die auf Relationen aufbauen, deren Schematas sich Normalisieren lassen.
        Dabei müssen Relationale Datenbanken müssen dabei auch Mengenoperationen, Selektion, Projektion und Joins erfüllen um als Relationale Datenbanken zu gelten\cite{Z9WAAQ2U}.}
}
\newglossaryentry{Git}
{
        name=Git,
        description={Git ist eine Versionierungssoftware und bietet die Möglichkeit, Repositories erstellen zu können.
        Die Repositories sind dabei nicht zentral sondern dezentral organsiert und arbeiten daher mit Working Copies von Repositories\cite{33C32L9G, SCE846MP}.}
}
\newglossaryentry{DBMS}
{
        name=DBMS,
        description={Ein Database Management System regelt und organisiert die Datenbasis einer Datenbank\cite{8XWD67EM}.}
}
\newglossaryentry{Split-brain}
{
        name=Split-brain,
        description={
%        \externaldocument{./content/implementation/evaluation/evaluation}
        Im Kapitel \autoref{chap:Split-brain} werden die ursachen und folgenden eines Split-brains genauer besprochen.
        }
}
\newglossaryentry{Quorum}
{
        name=Quorum,
        description={In verteilten Systemen resp. Cluster muss sichergestellt werden, das bei einem Ausfall oder ein Netzwerktrennung zwischen den Nodes es zu keiner \Gls{Split-brain}-Situation kommt.
        Hierzu wird i.d.R. ein Quorum verwendet.
        I.d.R. wird jener Teil des Quorums zum Primary oder alleinigen Node, der mit der die Mehrheit aller Nodes vereint. Daraus ergeben sich bestimmte grössen, mit 5 Nodes braucht es 3 Nodes um aktiv zu bleiben und mit 3 Nodes deren 2.
        Bei diesen Konstelationen wird daher darauf geachtet, eine ungerade Anzahl Nodes im Cluster zu halten um keine Pat-Situation zu provozieren.
%        \externaldocument{content/implementation/evaluation/evaluation}
        Im Kapitel \autoref{subsubsec:quorum} wird genauer auf die Mechanik eines Quorums eingegangen.
        }
}
\newglossaryentry{openSUSE}
{
        name=openSUSE,
        description={openSUSE ist eine Linux Distribution die aus den Distributionen Slackware Linux, jurix und S.u.S.E. Linux 4.2 heraus entstand.
        Die Distirbution wird heute von der SUSE S.A. betreut.
        Das OS ist Open-Source wobei es Enterprise Editionen für Server und Desktop sowie Tools für das Lifecycle Management und Provisioning gibt,
        die in Konkurenz zu Red Hat Satellite und deren Open-Source Basis \Gls{Foreman} stehen.}
}
\newglossaryentry{VRRP}
{
        name=VRRP,
        description={VRRP }
}
\newglossaryentry{keepalived}
{
        name=keepalived,
        description={keepalived nutzt \Gls{VRRP} um eine leichtgewichtige Lösung für ein HA-\Gls{Failover} zu realsieren.
        keepalived benötigt dazu keinen dritten Node, also einen \Gls{Quorum}-Node.
        Wenn die definierte sekundärseite keine Antwort mehr von der primären Seite nach einer definierten Anzahl versuchen in einem bestimmten Interval mehr bekommt,
        oder ein per Skript definiertes Event auf der primären Seite eintrifft, wird ein \Gls{Failover} auf die sekundäre Seite ausgeführt.
        Je nach Konfiguration kann der Restore auf die primäre Seite eingeleitet werden wenn diese wieder verfügbar ist oder der Restore unterbunden werden\cite{5IP362SV,ZW4PA3EQ}.}
}
\newglossaryentry{PKI}
{
        name=PKI,
        description={}
}
\newglossaryentry{HAProxy}
{
        name=HAProxy,
        description={HAProxy \cite{U6F2DWTQ}}
}
\newglossaryentry{etcd}
{
        name=etcd,
        description={etcd ist \cite{8A4R4E2D}}
}
\newglossaryentry{Transaktion}
{
        name=Transaktion,
        description={Eine Transaktion ist beinhaltet Schreib-, Lese-, Mutatations- oder Löschoperatione auf Daten.}
}
\newglossaryentry{Key-Value-Datenbank}
{
        name=Key-Value-Datenbank,
        description={Eine Key-Value-Datenbank ist ein Typ der\Gls{NoSQL} Datenbanken.
        Diese Datenbanken haben einen Primary Key und oft mindestens einen Sort Key.
        Key-Value-Datenbanken können auch Objekte mit Subitems resp. Referenzen dazu speichern.
        Eine bekannte Key-Value-Datenbank ist \Gls{Redis}}
}
\newglossaryentry{Key-Value-orientierte}
{
        name=Key-Value-orientierte,
        description={Siehe \Gls{Key-Value-Datenbank}}
}
\newglossaryentry{Key-Value-Store}
{
        name=Key-Value-Store,
        description={Siehe \Gls{Key-Value-Datenbank}}
}
\newglossaryentry{DCS}
{
        name=DCS,
        description={Der DSC ist eine Kernkomponente von Patroni \cite{LVMUNS8P}.
        Realisiert wird der DCS bei Patroni mit \Gls{etcd}.}
}
\newglossaryentry{State Machine}
{
        name=State Machine,
        description={}
}