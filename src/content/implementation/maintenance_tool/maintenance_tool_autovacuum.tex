%! Author = gramic
%! Date = 10.05.24

% Preamble
\begin{flushleft}
    \subsubsection{Maintenance-Tool - AUTOVACUUM}
    \paragraph{Ziel und Zweck}
    Folgende beiden \Gls{AUTOVACUUM}-Parameter sind entscheidend für die definition, wann \Gls{AUTOVACUUM} startet:
    \begin{description}
        \item \textbf{autovacuum\_vacuum\_threshold}\hfill \\Mindestanzahl toter Tupels die es in einer Tabelle braucht,\\damit \Gls{AUTOVACUUM} startet.\\Der Default liegt bei 50 toten Tuples.
        \item \textbf{autovacuum\_vacuum\_scale\_factor}\hfill \\
    \end{description}

    Eine der zentralsten \Gls{AUTOVACUUM}-Parameter ist \texttt{autovacuum\_vacuum\_scale\_factor}.\\
    Er definiert, wie
    \paragraph{Funktionsweise}
\end{flushleft}