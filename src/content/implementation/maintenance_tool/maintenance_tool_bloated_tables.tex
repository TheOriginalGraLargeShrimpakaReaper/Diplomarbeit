%! Author = gramic
%! Date = 10.05.24

% Preamble
\begin{flushleft}
    \subsubsection{Maintenance-Tool - Bloated Tables / Indices}
    \paragraph{Ziel und Zweck}
    \Gls{PostgreSQL} hat mit \Gls{AUTOVACUUM} eine Funktionalität, welche Tabellen und Indizes aufräumt und bereinigt.\\
    Allerdings gelten die Parameter für den ganzen \Gls{PostgreSQL Cluster}, einzelne Tabellen werden zwangsläufig nicht optimal vakuumiert.\\
\end{flushleft}
\begin{flushleft}
    Daher war ein Ziel, dass Bloated Tables und Indices von einem Maintenance Skript erkannt und bereinigt werden.
\end{flushleft}
\begin{flushleft}
    \paragraph{Funktionsweise}
    Der Maintenance-Job soll anhand der
    Auf allen Datenbanken muss die Extension \texttt{pgstattuple} installiert werden.\\
    Die Extension sammelt und stellt Informationen über Tabellen, Indizes, Tupels usw., zur verfügung.
\end{flushleft}
\begin{flushleft}
    Mit Python werden alle Tabellen, welche bei \texttt{pgstattuple} mehr als einen konfigurationen Wert, aktuell 1.5\%, gelistet.\\
    Das SQL Statement hierzu, ist wie folgt aufgebaut:
    \lstset{style=gra_codestyle}
    \begin{lstlisting}[language=sql, caption=Maintenance-Tool - List - Maintenance-Tool - Bloated Tables / Indices,captionpos=b,label={lst:maintenannce-tool-list-bloated-tables},breaklines=true]
select
    relname as table,
    (pgstattuple(oid)).dead_tuple_percent
from pg_class
where
    relkind = 'r' and
    (pgstattuple(oid)).dead_tuple_percent > %s
order by dead_tuple_percent desc;
    \end{lstlisting}
    Anschliessend werden die entsprechenden Tabellen mittels VACUUM-Befehl manuell vakuumiert:
    \lstset{style=gra_codestyle}
    \begin{lstlisting}[language=sql, caption=Maintenance-Tool - VACUUM - Maintenance-Tool - Bloated Tables / Indices,captionpos=b,label={lst:maintenannce-tool-vacuum-bloated-tables},breaklines=true]
vacuum <table>;
    \end{lstlisting}
    \begin{warning}
        Ein \texttt{VACUUM} kann nicht in einer normalen Transaktion abgesetzt werden!\\
        Bei \texttt{psycopg2} muss der Transaktions-Isolations Level auf \texttt{ISOLATION\_LEVEL\_AUTOCOMMIT} gesetzt werden.\\
        Anders lässt sich das SQL nicht absetzen.
    \end{warning}
    Im Nachgang werden alle Indizes für die ganze Tabelle neu aufgebaut (\texttt{REINDEX}):
    \lstset{style=gra_codestyle}
    \begin{lstlisting}[language=sql, caption=Maintenance-Tool - REINDEX - Maintenance-Tool - Bloated Tables / Indices,captionpos=b,label={lst:maintenannce-tool-reindex-bloated-tables},breaklines=true]
reindex table <table>;
    \end{lstlisting}
\end{flushleft}
\begin{flushleft}
    Wichtig ist eine vorangehende Prüfung, ob ein \Gls{AUTOVACUUM}-Job am laufen ist.\\
    Trifft dies zu, darf das Skript in diesem Lauf nicht ausgeführt werden.\\
    Überprüft kann dies mit folgendem SQL werden:
    \lstset{style=gra_codestyle}
    \begin{lstlisting}[language=sql, caption=Maintenance-Tool - Check AUTOVACUUM - Maintenance-Tool - Bloated Tables / Indices,captionpos=b,label={lst:maintenannce-tool-check-autovacuum-bloated-tables},breaklines=true]
select count(*) as active from pg_stat_activity where query like 'autovacuum:%';
    \end{lstlisting}
\end{flushleft}
\begin{flushleft}
    Dem Microservice-Gedanken entsprechend, soll es pro Datenbank einen Job geben,\\
    der die Tabellen auf Bloated Tables untersucht und diese bereinigt.
\end{flushleft}