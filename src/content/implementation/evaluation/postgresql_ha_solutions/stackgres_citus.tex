\subsubsection{Stackgres mit Citus}
\begin{flushleft} 
    Stackgres ist eine PostgreSQL Implementation die dafür vorgesehenen ist, in einem Kubernetes Cluster betrieben zu werden.
\end{flushleft} 
\begin{flushleft}
    An sich wäre Stackgres nur eine Implementation von Patroni in Kubernetes inkl. Load Balancer.\\
    Nun kommt das Citus-Plugin ins spiel, welches aus einer jeden Monolithischen, klassischen PostgreSQL Installation eine Distributed SQL Umgebung macht.////
    Citus wiederum ist in den Microsoft Konzern eingebettet
\end{flushleft}
\begin{flushleft}
    \paragraph{Architektur}
    \begin{flushleft}
        \subparagraph{Citus Coordinator und Workers}
        Citus arbeitet mit einem Coordinator-Node, der jedes Query analysiert und an einen Worker-Node weitergibt.
        \begin{figure}[H]
            \centering
            \includegraphics[width=0.75\linewidth]{source/implementation/evaluation/postgresql_ha_solutions/stackgres/citus_coordinator_worker}
            \caption{Citus - Coordinator und Workers}
            \label{fig:citus_coordinator_worker}
        \end{figure}
    \end{flushleft}
    \begin{flushleft}
        \subparagraph{Citus Sharding}
        Citus bietet zwei Sharding-Modelle an.
        \begin{flushleft}
            \textbf{Row-based sharding}
            Beim diesen sharding werden Tabellen anhand einer Distribution Column aufgeteilt. \cite{2Y5FA36C, FDUUL9IM}
            \begin{figure}[H]
                \centering
                \includegraphics[width=0.8\linewidth]{source/implementation/evaluation/postgresql_ha_solutions/stackgres/citus_row-based-sharding}
                \caption{Citus - Row-Based-Sharding}
                \label{fig:citus_row-based-sharding}
            \end{figure}
        \end{flushleft}
        \begin{flushleft}
            \textbf{Schema-based sharding}
            \begin{figure}[H]
                \centering
                \includegraphics[width=0.8\linewidth]{source/implementation/evaluation/postgresql_ha_solutions/stackgres/citus_schema-based-sharding}
                \caption{Citus - Schema-Based-Sharding}
                \label{fig:citus_schema-based-sharding}
            \end{figure}
        \end{flushleft}
        \begin{flushleft}
            \textbf{Schlussfolgerung}
            Beide Sharding-Methoden haben eine grosse Schwäche.
            In Version 7.2 konnte noch ein Replikationsfaktor angegeben werden\cite{Citus},
%            Sie sind nicht vollständig ACID-Konform (\autoref{subsubsec:acid}) da Datenverlust entstehen kann, wenn ein Node wegfällt.
            Sie sind nicht vollständig ACID-Konform (\autoref{chap:acid}) da Datenverlust entstehen kann, wenn ein Node wegfällt.
            Die Shards müssen mit entsprechenden mit Replikation gesichert werden\cite{4GDXA49I}.

            Dies muss aber bei der evaluation mittels Tests noch bestätigt werden.
        \end{flushleft}
    \end{flushleft}
\end{flushleft}
\begin{flushleft}
    \paragraph{Maintenance}
    Bei Stackgres gab es im letzten Monat keine wirkliche Bewegung:
    \begin{figure}[H]
        \centering
        \includegraphics[width=0.75\linewidth]{source/implementation/evaluation/postgresql_ha_solutions/insights/stackgres_citus/pulse_ongres_stackgres}
        \caption{Stackgres - Pulse}
        \label{fig:pulse_ongres_stackgres}
    \end{figure}
    Anders sieht es bei Citus aus, die Firma die mittlerweile zu Microsoft gehört, schliesst Issues rasch und hat eine verhältnissmässig hohe Requstrate:
    \begin{figure}[H]
        \centering
        \includegraphics[width=0.75\linewidth]{source/implementation/evaluation/postgresql_ha_solutions/insights/stackgres_citus/pulse_citusdata_citus}
        \caption{Citus - Pulse}
        \label{fig:pulse_citusdata_citus}
    \end{figure}

    Bei Stackgres wird sehr viel Code hinzugefügt oder gelöscht, beim älteren Citus wurden weniger änderungen verzeichnet:
    \begin{figure}[H]
        \centering
        \includegraphics[width=0.75\linewidth]{source/implementation/evaluation/postgresql_ha_solutions/insights/stackgres_citus/code_frequency_ongres_stackgres}
        \caption{Stackgres - Code Frequency}
        \label{fig:code_frequency_ongres_stackgres}
    \end{figure}
    \begin{figure}[H]
        \centering
        \includegraphics[width=0.75\linewidth]{source/implementation/evaluation/postgresql_ha_solutions/insights/stackgres_citus/code_frequency_citusdata_citus}
        \caption{Citus - Code Frequency}
        \label{fig:code_frequency_citusdata_citus}
    \end{figure}

    Citus legt einen hohen Stellenwert auf die Community-Standars, Stackgres selbst schneidet hier nur Mittelmässig ab:
    \begin{figure}[H]
        \centering
        \includegraphics[width=0.75\linewidth]{source/implementation/evaluation/postgresql_ha_solutions/insights/stackgres_citus/stackgres_community_standards}
        \caption{Stackgres - Community Standards}
        \label{fig:stackgres_community_standards}
    \end{figure}
    \begin{figure}[H]
        \centering
        \includegraphics[width=0.75\linewidth]{source/implementation/evaluation/postgresql_ha_solutions/insights/stackgres_citus/citus_community_standards}
        \caption{Citus - Community Standards}
        \label{fig:citus_community_standards}
    \end{figure}

    Die Stackgres Constributors pflegen aktiv Additions ein, löschen Regelmässig und Commiten ebenfalls auf die main-Branch.
    Citus, dessen Repository länger Commited wird, hat weniger bewegung auf die main-Branch.
    \begin{figure}[H]
        \centering
        \includegraphics[width=0.75\linewidth]{source/implementation/evaluation/postgresql_ha_solutions/insights/stackgres_citus/contributors_commits_ongres_stackgres}
        \caption{Stackgres - Contributors Commits}
        \label{fig:contributors_commits_ongres_stackgres}
    \end{figure}
    \begin{figure}[H]
        \centering
        \includegraphics[width=0.75\linewidth]{source/implementation/evaluation/postgresql_ha_solutions/insights/stackgres_citus/contributors_deletations_ongres_stackgres}
        \caption{Stackgres - Contributors Deletations}
        \label{fig:contributors_deletations_ongres_stackgres}
    \end{figure}
    \begin{figure}[H]
        \centering
        \includegraphics[width=0.75\linewidth]{source/implementation/evaluation/postgresql_ha_solutions/insights/stackgres_citus/contributors_addition_ongres_stackgres}
        \caption{Stackgres - Contributors Additions}
        \label{fig:contributors_addition_ongres_stackgres}
    \end{figure}
    \begin{figure}[H]
        \centering
        \includegraphics[width=0.75\linewidth]{source/implementation/evaluation/postgresql_ha_solutions/insights/stackgres_citus/contributors_commits_citusdata_citus}
        \caption{Citus - Contributors Commits}
        \label{fig:contributors_commits_citusdata_citus}
    \end{figure}
    \begin{figure}[H]
        \centering
        \includegraphics[width=0.75\linewidth]{source/implementation/evaluation/postgresql_ha_solutions/insights/stackgres_citus/contributors_deletations_citusdata_citus}
        \caption{Citus - Contributors Deletations}
        \label{fig:contributors_deletations_citusdata_citus}
    \end{figure}
    \begin{figure}[H]
        \centering
        \includegraphics[width=0.75\linewidth]{source/implementation/evaluation/postgresql_ha_solutions/insights/stackgres_citus/contributors_additions_citusdata_citus}
        \caption{Citus - Contributors Additions}
        \label{fig:contributors_additions_citusdata_citus}
    \end{figure}

    Gerade Ende Januar gab es bei Stackgres eine grössere Anzahl Commits, anhand der statistik wird ersichtlich, dass i.d.R. einmal pro Monat grössere Mengen an Commits eingespielt werden.
    Bei Citus gibt es ebenfalls Regelmässig grössere Mengen an Commits, allerdings scheint bei citusdata mehr mit kürzeren Sprints gearbeitet zu werden als bei ongres denn die Commits sind Regelmässiger:
    \begin{figure}[H]
        \centering
        \includegraphics[width=0.75\linewidth]{source/implementation/evaluation/postgresql_ha_solutions/insights/stackgres_citus/commit_activity_ongres_stackgres}
        \caption{Stackgres - Commit Activity}
        \label{fig:commit_activity_ongres_stackgres}
    \end{figure}
    \begin{figure}[H]
        \centering
        \includegraphics[width=0.75\linewidth]{source/implementation/evaluation/postgresql_ha_solutions/insights/stackgres_citus/commit_activity_citusdata_citus}
        \caption{Citus - Commit Activity}
        \label{fig:commit_activity_citusdata_citus}
    \end{figure}

    In letzter Zeit haben nur ongres, der Entwickler von Stackgres, als auch citusdata, grössere Commits auf das Repository gefahren.
    Andere grössere Entwickler wie EnterpriseDB sind abwesend.
    \begin{figure}[H]
        \centering
        \includegraphics[width=0.75\linewidth]{source/implementation/evaluation/postgresql_ha_solutions/insights/stackgres_citus/network_graph_ongres_stackgres}
        \caption{Stackgres - Network Graph}
        \label{fig:network_graph_ongres_stackgres}
    \end{figure}
    \begin{figure}[H]
        \centering
        \includegraphics[width=0.75\linewidth]{source/implementation/evaluation/postgresql_ha_solutions/insights/stackgres_citus/network_graph_citusdata_citus}
        \caption{Citus - Network Graph}
        \label{fig:network_graph_citusdata_citus}
    \end{figure}

\end{flushleft}