%! Author = itgramic
%! Date = 05.12.23

% Preamble
\begin{flushleft}
    \subsubsection{KSGR Lösung}
    Das Kantonsspital Graubünden hat prüft mit \gls{keepalived} ob die primäre Seite erreichbar und betriebsbereit ist.
    Trifft dies nicht mehr zu, wird ein \Gls{Failover} durchgeführt \cite{NLF2IDBZ}.
    Ist die primäre Seite wieder verfügbar, wird ein Restore auf die primäre Seite gefahren.
\end{flushleft}
\begin{flushleft}
    Beim Restore wird ein komplettes Backup der sekundären Seite auf die primäre Seite übertragen.
    Ursache ist, dass die normalerweise für den Datenrestore benötigten \Gls{PostgreSQL} Board Mttel nur für eine relativ kurze Zeit eingesetzt werden können,
    ehe die Differenzen zwischen den beiden Seiten zu gross werden.
    Bei kleinen Datenbanken wie jene für \Gls{Harbor} und \Gls{GitLab} ist die Zeit die hierfür benötigt wird, nicht relevant.
    Sind die Datenbanken auf dem \Gls{PostgreSQL Cluster} jedoch grösser, kann der Restore mehrere Minuten dauern.
\end{flushleft}