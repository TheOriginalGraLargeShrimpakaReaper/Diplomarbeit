%! Author = itgramic
%! Date = 05.12.23

% Preamble
\begin{flushleft}
    \subsubsection{CloudNativePG}
    CloudNativePG ist eine Containerlösung für PostgreSQL auf Kubernetes.
\end{flushleft}
\begin{flushleft}
    \paragraph{Core-Features}

\end{flushleft}
\begin{flushleft}
    \paragraph{Replikation}
    CloudNativePG bietet die üblichen PostgreSQL Replikaionen an.
\end{flushleft}
\begin{flushleft}
    \paragraph{Proxy}
    CloudNativePG benötigt keinen zusätzlichen Proxy.
\end{flushleft}
\begin{flushleft}
    \paragraph{Pooling}
    CloudNativePG unterstützt pgBouncer als Pooler.
\end{flushleft}
\begin{flushleft}
    \paragraph{API / Skripte}
    CloudNativePG bietet eine API zum Monitoren und Verwalten von Backups, Clustern und dem System selbst\cite{L7PXKAUY}.
\end{flushleft}
\begin{flushleft}
    \paragraph{Architektur}
    Kubernetes regelt die Zugriffe mittels eines entsprechenden Services in die Nodes auf den Pods:
    \begin{figure}[H]
        \centering
        \includegraphics[width=0.75\linewidth]{source/implementation/evaluation/postgresql_ha_solutions/cloudnativepg/k8s-pg-architecture}
        \caption{CloudNativePG - Kubernetes - PostgreSQL}
        \label{fig:k8s-pg-architecture}
    \end{figure}
\end{flushleft}
\begin{flushleft}
    Dabei werden die Read-write workloads an den Primary Node gesendet:
    \begin{figure}[H]
        \centering
        \includegraphics[width=0.75\linewidth]{source/implementation/evaluation/postgresql_ha_solutions/cloudnativepg/cloudnativepg-architecture-rw}
        \caption{CloudNativePG - Kubernetes - Read-write workloads}
        \label{fig:cloudnativepg-architecture-rw}
    \end{figure}
\end{flushleft}
\begin{flushleft}
    Read-only workloads werden mit Round robin an die Replikas zugewiesen:
    \begin{figure}[H]
        \centering
        \includegraphics[width=0.75\linewidth]{source/implementation/evaluation/postgresql_ha_solutions/cloudnativepg/cloudnativepg-architecture-read-only}
        \caption{CloudNativePG - Kubernetes - Read-only workloads}
        \label{fig:cloudnativepg-architecture-read-only}
    \end{figure}
\end{flushleft}
\begin{flushleft}
    Es könnten auch Lösungen mit Designated Kubernetes-Clustern in einem anderen RZ oder einer anderen Geo-Region relaisiert werden.
\end{flushleft}
\begin{flushleft}
    \paragraph{Maintenance}
\end{flushleft}