%! Author = itgramic
%! Date = 05.12.23

% Preamble
\subsubsection{pgpool-II}
pgpool-II ist eine Middleware die zwischen einem \Gls{PostgreSQL Cluster} und einem PostgreSQL Client gesetzt wird.
pgpool-II bietet folgende Funktionen\cite{EXVNLICT,3XWCD3KX}:
\\\textbf{High Availability}\\
pgpool-II bietet einen automatic \Gls{Failover} genannten Service an, den Watchdog.
Dieser schwenkt auf einen Standby-Server und entfernt den Defekten Server.
Um false positive Events und Split-brains zu verhindern setzt pgpool-II auf einen eigens entwickelten \Gls{Quorum}-Algorithmus.
\\\textbf{Connection Pooling}\\
Bestehende Connections werden wiederverwendet um die Anzahl gleichzeitig offener Connections zu reduzieren.
Der Pool wird dabei anhand von Username, Database, Protocol und weiteren Verbindungsparametern zugeordnet.
\\\textbf{Replikation}\\
Nebst dem Standard \Gls{PostgreSQL} bietet pgpool-II sein eigenes Replikationssystem an.
\\\textbf{Load Balancing}\\
Ähnlich wie Oracle Active Data Guard \cite{6294443C} bietet auch pgpool-II die Möglichkeit, SELECT-Queries und Backup-Jobs auf die Secondary-Nodes umzuleiten um den Primary Node zu entlasten.
\\\textbf{Limiting Exceeding Connections}\\
Die Anzahl an concurrent Connections, also gleichzeitiger Verbindungen, ist bei \Gls{PostgreSQL} begrenzt (Systemparameter wird dabei vom DBA gesetzt).
pgpool-II speichert alle Connections, die über dem Limit sind, in einer Queue und somit nicht sofort fehlerhaft abgelehnt.
\\\textbf{Watchdog}\\
Der Watchdog koordiniert mehrere pgpool-II Nodes und verhindert ein Split-brain.
\\\textbf{In Memory Query Caching}\\
pgpool-II speichert SELECT-Queries in einem Cache und verwendet die ResultSets wieder, wenn eine identische Abfrage eingeht.
\\\textbf{Online Recovery}\\
