%! Author = itgramic
%! Date = 05.12.23

% Preamble
\subsubsection{pgpool-II}
\begin{flushleft}
    pgpool-II ist eine Middleware die zwischen einem \Gls{PostgreSQL Cluster} und einem PostgreSQL Client gesetzt wird.
\end{flushleft}
\begin{flushleft}
    \paragraph{Core-Features}
    pgpool-II bietet folgende Core-Feature\cite{3XWCD3KX}:
    \begin{itemize}
        \item Watchdog für Automatischer Failover
        \item Eigener \Gls{Quorum}-Algorithmus
        \item Integrierter Pooler
        \item Eigenes Replikationssystem
        \item Integriertes Load Balancing
        \item Limiting Exceeding Connections, also queuen von Connections
        \item In Memory Query Caching
        \item Online Node Recovery / Resynchronisation
    \end{itemize}
\end{flushleft}
\begin{flushleft}
    \paragraph{Replikation}
    pgpool-II bietet ein eigenes Replikationssystem an.
\end{flushleft}
\begin{flushleft}
    Es besteht allerdings die Möglichkeit, die PostgreSQL Standardreplikationen zu verwenden.
\end{flushleft}
\begin{flushleft}
    \paragraph{Proxy}
    pgpool-II hat bereits einen intergrierten Load Balancer.\\
    Einen zusätzlichen Proxy wird daher nicht benötigt.
\end{flushleft}
\begin{flushleft}
    \paragraph{Pooling}
    pgpool-II bietet ebenfalls von Haus aus einen Pooler.
\end{flushleft}
\begin{flushleft}
    \paragraph{API / Skripte}
    pgpool-II bietet mit \texttt{pgpool} ein eigenes Command- und Toolset an.\\
    Mit dem CLI-Tool \texttt{pcp\_command} bietet pgpool-II zudem über eine abgesicherte API,\\
    die auch via Netzwerk erreichbar ist.
\end{flushleft}
\begin{flushleft}
    \paragraph{Maintenance}
    pgpool-II hat kein GitLab- oder GitHub-Repository.
    Metriken wie die GitHub Insights, welche aufschluss über den Zustand des Projekts geben, finden sich nicht.
\end{flushleft}
\begin{flushleft}
    Die Dokumentation wird zwar aktuell gehalten, ist aber sehr minimalistisch gehalten.\\
    Sie bietet nur wenig Informationen zum Aufbau und Architektur von pgpool-II.
\end{flushleft}

