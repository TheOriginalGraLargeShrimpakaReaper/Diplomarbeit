%! Author = itgramic
%! Date = 05.12.23

% Preamble
\subsubsection{Quorum}
\label{chap:Quorum}
\begin{flushleft}
    Ein Quorum-System soll die Integrität und Konsistenz in einem Datenbank-Cluster sicherstellen.
    Dabei gilt zu beachten, das nicht eine beliebige Anzahl an Nodes hinzugefügt werden können.
    Auch hat das Hinzufügen von Nodes immer eine einbusse an Performance zur Folge, besonders dann, wenn eine Synchrone Replikation gewählt wird und auf jedes Commitmend von den Replica-Nodes gewartet werden muss.

    \begin{description}
        \item \textbf{Quorum}\hfill \\Die Mehrheit der Server, die einen funktionierenden Betrieb gewährleisten können, ohne eine \Gls{Split-brain}Situation zu erzeugen.
        Die Formel ist gemeinhin \(n/2 + 1\)
        \item \textbf{Throughput}\hfill \\Beschreibt, wie sich die Anzahl Nodes auf die Schreibgeschwindigkeit der Commitments auf die restlichen Nodes auswirkt.\\Die verdopplung der Server halbiert i.d.R. den Throughput.
        \item \textbf{Fehlertoleranz}\hfill \\Beschreibt, wie viele Nodes ausfallen können, damit der Cluster noch Arbeitsfähig ist.\\Wobei eine erhöhung der Nodes von 3 auf 4 die Fehlertoleranz nicht erhöht da nun eine \Gls{Split-brain}-Situation entstehen kann.
    \end{description}
    %\begin{landscape}
    %\begin{table}[]
    %\resizebox{\columnwidth}{!}{%
    %\begin{tabular}{@{}llll@{}}
    %\toprule
    %\textbf{Anzahl Nodes} & \textbf{Quorum} & \textbf{Fehlertoleranz} & \textbf{Representative Throughput} \\ \midrule
    %1                     & 1               & 0                                               & 100                                \\
    %2                     & 2               & 0                                               & 85                                 \\
    %3                     & 2               & 1                                               & 82                                 \\
    %4                     & 3               & 1                                               & 57                                 \\
    %5                     & 3               & 2                                               & 48                                 \\
    %6                     & 4               & 2                                               & 41                                 \\
    %7                     & 4               & 3                                               & 36                                 \\ \bottomrule
    %\end{tabular}%
    %}
    %\caption{Quorum Beispiele}
    %\label{tab:quorum-beispiele}
    %\end{table}
    %\end{landscape}
    %\subsubsection{Split-brain}
    %\label{chap:Split-brain}
    Hier ein Beispiel wie sie in den Artikeln \cite{UMIGLCCI, YDS7DTYM, V4XLXN7W} beschrieben werden.
    Es zeigt auf, ab wie vielen Nodes die Fehlertoleranz erhöht wird und wie sich der Representative Throughput verhält.
%\end{flushleft}
%\begin{flushleft}
    \begin{table}[H]
    \resizebox{\columnwidth}{!}{%
    \begin{tabular}{@{}llll@{}}
    \toprule
    \textbf{Anzahl Nodes} & \textbf{Quorum} & \textbf{Fehlertoleranz} & \textbf{Representative Throughput} \\ \midrule
    1                     & 1               & 0                       & 100                                \\
    2                     & 2               & 0                       & 85                                 \\
    3                     & 2               & 1                       & 82                                 \\
    4                     & 3               & 1                       & 57                                 \\
    5                     & 3               & 2                       & 48                                 \\
    6                     & 4               & 2                       & 41                                 \\
    7                     & 4               & 3                       & 36                                 \\ \bottomrule
    \end{tabular}%
    }
    \caption{Quorum Beispiele}
    \label{tab:quorum-beispiele}
    \end{table}
\end{flushleft}