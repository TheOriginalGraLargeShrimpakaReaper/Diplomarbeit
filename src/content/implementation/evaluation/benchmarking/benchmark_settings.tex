%! Author = gramic
%! Date = 22.03.24

% Preamble
\begin{flushleft}
    \subsubsection{Benchmark Settings}
    Das Mass aller Dinge ist die Zabbix-DB.\\
    Sie wird vorerst die grössten Zugriffszahlen, das höchste Datenwachstum und die meisten Transaktionen erzeugen.\\
    Es werden alle Switches sowie der grösste Teil der Router erfasst, es sind im Moment etwas mehr als 32'000 Items erfasst.\\
    Ein Item kann ein Gerät, ein Port oder mehrere States pro Port sein:
    \begin{figure}[H]
        \centering
        \includegraphics[width=0.8\linewidth]{source/implementation/evaluation/benchmarking/sks0970_zabbix_system_information}
        \caption{Benchmark Settings - Zabbix - Systeminformationen}
        \label{fig:sks0970_zabbix_system_information}
    \end{figure}
    Pro Sekunde werden ca.
    950 Datenpunkte abgeholt.
\end{flushleft}
\begin{flushleft}
    Da der grossteil der Netzwerksysteme aber erfasst sind, wird die Anzahl Items nicht mehr stark anwachsen.
\end{flushleft}
\begin{flushleft}
    Auf die Datenbank wird sehr stark zugegriffen.
    Es werden bis zu 23 Connections pro Sekunde ausgeführt:
    \begin{figure}[H]
        \centering
        \includegraphics[width=0.8\linewidth]{source/implementation/evaluation/benchmarking/sks0970_zabbix_mariadb_connections_per_second_graph}
        \caption{Benchmark Settings - Zabbix - Connections per Seconds}
        \label{fig:sks0970_zabbix_mariadb_connections_per_second_graph}
    \end{figure}
\end{flushleft}
\begin{flushleft}
    Pro Sekunde wurden bisher bis zu über 7'000 Queries ausgeführt.
    Dies schliesst Abfragen von Stored Programs ein:
    \begin{figure}[H]
        \centering
        \includegraphics[width=0.8\linewidth]{source/implementation/evaluation/benchmarking/sks0970_zabbix_mariadb_queries_per_second_graph}
        \caption{Benchmark Settings - Zabbix - Queries per Seconds}
        \label{fig:sks0970_zabbix_mariadb_queries_per_second_graph}
    \end{figure}
    Reine Client anfragen waren nichtsdestotrotz über 4'000 Queries pro Sekunde:
    \begin{figure}[H]
        \centering
        \includegraphics[width=0.8\linewidth]{source/implementation/evaluation/benchmarking/sks0970_zabbix_mariadb_questions_per_second_graph}
        \caption{Benchmark Settings - Zabbix - Client Queries per Seconds}
        \label{fig:sks0970_zabbix_mariadb_questions_per_second_graph}
    \end{figure}
\end{flushleft}
\begin{flushleft}
    Auch das wachstum ist beachtlich.
    Die DB startete mit 180GiB und ist zurzeit bei rund 232GiB, war aber schon bei 238GiB:
    \begin{figure}[H]
        \centering
        \includegraphics[width=0.8\linewidth]{source/implementation/evaluation/benchmarking/sks0970_zabbix_mariadb_size_graph}
        \caption{Benchmark Settings - Zabbix - DB Size}
        \label{fig:sks0970_zabbix_mariadb_size_graph}
    \end{figure}
\end{flushleft}
\begin{flushleft}
    Nun kommen noch die restlichen, kleineren DBs hinzu.
    Heisst, für den Mixed Benchmark (DML und DQL \Gls{Transaktion}en) werden folgende Werte und Parameter gesetzt:
    \begin{table}[H]

\resizebox{\columnwidth}{!}{%

\begin{tabular}{lllllll}
\toprule
Typ & Parameter & pgbench-Parameter & 1. Lauf & 2. Lauf & 3. Lauf & 4. Lauf \\
\midrule
mixed & Datenbank &  & pgbench\_eval\_bench & pgbench\_eval\_bench & pgbench\_eval\_bench & pgbench\_eval\_bench \\
mixed & DB-Grösse &  & 5GiB & 15GiB & 50GiB & 250GiB \\
mixed & 1. Iteration Lauf ignorieren &  & ja & ja & ja & ja \\
mixed & Select only & -S & nein & nein & nein & nein \\
mixed & Iterationen pro Lauf &  & 4 & 4 & 4 & 4 \\
mixed & Vacuum & -v & ja & ja & ja & ja \\
mixed & Separate Connects & -C & ja & ja & ja & ja \\
mixed & Client count & -c & 10 & 50 & 100 & 1000 \\
mixed & Anzahl Transaktionen pro Client & -t & 10 & 50 & 50 & 7 \\
mixed & Anzahl Transaktionen Total &  & 100 & 2500 & 5000 & 7000 \\
mixed & Anzahl Worker Threads & -j & 4 & 4 & 4 & 4 \\
\bottomrule
\end{tabular}
}
\caption{Benchmark Settings - Mixed Transaktionen} \label{benchmarking_settings_mixed}
\end{table}

\end{flushleft}
\begin{flushleft}
    Für den DQL-Only Benchmark wird mit folgenden Konfigurationen gearbeitet:
    \begin{table}[H]

\resizebox{\columnwidth}{!}{%

\begin{tabular}{lllllll}
\toprule
Typ & Parameter & pgbench-Parameter & 1. Lauf & 2. Lauf & 3. Lauf & 4. Lauf \\
\midrule
dql & Datenbank &  & pgbench\_eval\_bench & pgbench\_eval\_bench & pgbench\_eval\_bench & pgbench\_eval\_bench \\
dql & DB-Grösse &  & 5GiB & 15GiB & 50GiB & 250GiB \\
dql & 1. Iteration Lauf ignorieren &  & ja & ja & ja & ja \\
dql & Select only & -S & ja & ja & ja & ja \\
dql & Iterationen pro Lauf &  & 4 & 4 & 4 & 4 \\
dql & Vacuum & -v & ja & ja & ja & ja \\
dql & Separate Connects & -C & ja & ja & ja & ja \\
dql & Client count & -c & 10 & 50 & 100 & 1000 \\
dql & Anzahl Transaktionen pro Client & -t & 10 & 50 & 1000 & 7000 \\
dql & Anzahl Worker Threads & -j & 4 & 4 & 4 & 4 \\
\bottomrule
\end{tabular}
}
\caption{Benchmark Settings - DQL Transaktionen} \label{benchmarking_settings_dql}
\end{table}

\end{flushleft}