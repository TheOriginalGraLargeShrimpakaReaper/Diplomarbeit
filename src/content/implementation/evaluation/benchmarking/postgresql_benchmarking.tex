%! Author = gramic
%! Date = 11.03.24

% Preamble
\begin{flushleft}
    \subsection{PostgreSQL Benchmarking}
    \subsubsection{pgBench - Basis-Benchmarking}
    PostgreSQL bietet ein Benchmarking-Tool,\cite{TYJFF7AB,VXNYQFTE} mit dem die DB Vermessen werden kann.
\end{flushleft}
\begin{flushleft}
    Damit die Tests aussagekräftig sind, werden mit den Testläufen mehrere Läufe gestartet.
    Der erste Lauf muss dabei ignoriert werden, denn erst dann wird die DB in den Cache geladen.
    Wird dies nicht eingehalten, so wird die ganze Testreihe unbrauchbar.
\end{flushleft}
\begin{flushleft}
    Es gibt einiges zu beachten, wenn PostgreSQL einem Benchmarking unterzogen wird.
    Aus diversen Quellen \cite{768DXXJK, 67TNVGTX, T3VEV98H, VXNYQFTE} sind dies folgende Anforderungen:
     \begin{description}
        \item \textbf{Maintenance}\hfill \\Vor und nach dem Initialisieren des Benchmarks muss \Gls{AUTOVACUUM} gestartet werden.
        \item \textbf{Non-Default Konfiguration}\hfill \\Die PostgreSQL DB sollte nicht mit der Default Konfiguration betrieben werden.\\Die Konfiguration sollte anhand der zu erwartenden Workloads und Sessions konfiguriert werden.
        \item \textbf{Benchmark anpassen}\hfill \\Der Benchmark sollte sich an die zu erwartenden Anzahl Sessions und Anzahl \Gls{Transaktion}en richten.
        \item \textbf{Benchmark Dauer}\hfill \\Die Zeit zwischen den \Gls{Transaktion}en und die Dauer an sich sollten über einen längeren Zeitraum stattfinden.\\Nur so, kann ein echtes verhalten gemessen werden.
        \item \textbf{Störfaktoren}\hfill \\Störfaktoren, etwa Netzwerklatenzen \cite{5VBTYNB7} usw.,
        müssen ebenfalls bemessen werden.\\Nur so kann sichergestellt werden, dass die Umgebung sauber ist.
    \end{description}
\end{flushleft}
\begin{flushleft}
    \subsubsection{Replication Throughput Benchmarking}
    Etwas Komplexer wird es beim Benchmark des Throughput.
    Den nebst den DB-Latenzen usw. kommt nun noch die Netzwerklatenz usw. hinzu \cite{UU5X5NID}.

\end{flushleft}