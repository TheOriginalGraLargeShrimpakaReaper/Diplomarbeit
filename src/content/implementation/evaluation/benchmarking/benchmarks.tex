%! Author = gra
%! Date = 23.03.24

% Preamble
\begin{flushleft}
    \paragraph{yugabyteDB}
    Zuerst wurde für die Annährend 5GiB-DB initialisiert:
\lstset{style=gra_codestyle}
\begin{lstlisting}[language=sql, caption=yugabyteDB - Benchmarking - DB erstellen,captionpos=b,label={lst:yugabytedb-benchmarking-create-db},breaklines=true]
yugabyte=# create database pgbench_eval_bench;
CREATE DATABASE
\end{lstlisting}

    \paragraph{Patroni}
    Als die 250GiB DB getestet wurde, zeigte sich, das die Parameter nicht darauf optimiert waren.


    \subsubsection{Benchmarks}
    Der vergleich zwischen den verschiedenen Varianten.\\
\end{flushleft}
\begin{flushleft}
    \begin{warning}
        \textbf{YugabyteDB}\\
        4GiB Memory für die \texttt{tserver} waren offensichtlich zu knapp bemessen.
        Zumindest wenn die Tabelle 120'000'000 Rows hat und ein mixed Benchmark abgesetzt wird.

        Dies äusserte sich in einem Fehler (Absturz)auf zwei von drei \texttt{tserver}-Nodes sowie einer hohen Anzahl an Fehlern bei den mixed-Benchmarks.
        Daher wurde das Memory auf 8GiB erhöht und die komplette Testreihe erneut gestartet.
        Zudem wurden die Anzahl Fehler ebenfalls in die Benchmark-Auswertung einbezogen.
    \end{warning}
\end{flushleft}
\begin{flushleft}
    Bei den Transaktionen pro Sekunden gilt, je höher der Wert, umso besser das Ergebnis.\\
    Zuerst die Ergebnisse mit den mixed-Transaktionen:
    \begin{figure}[H]
        \centering
        \includegraphics[width=1\linewidth]{source/pandas_data_chart_plotter/tps_mixed}
        \caption{Benchmarks - tps mixed}
        \label{fig:tps_mixed}
    \end{figure}

    Folgend die reinen Select-Transaktionen.
    \begin{figure}[H]
        \centering
        \includegraphics[width=1\linewidth]{source/pandas_data_chart_plotter/tps_dql}
        \caption{Benchmarks - tps dql}
        \label{fig:tps_dql}
    \end{figure}

    Bei der Latenz ist es genau andersrum, je höher der Wert desto schlechter schnitt die Variante ab.\\
    Auch hier zuerst wieder die mixed-Transaktionen:
    \begin{figure}[H]
        \centering
        \includegraphics[width=1\linewidth]{source/pandas_data_chart_plotter/latency_mixed}
        \caption{Benchmarks - latency mixed}
        \label{fig:latency_mixed}
    \end{figure}

    Folgend die Select-Transaktionen:
    \begin{figure}[H]
        \centering
        \includegraphics[width=1\linewidth]{source/pandas_data_chart_plotter/latency_dql}
        \caption{Benchmarks - latency dql}
        \label{fig:latency_dql}
    \end{figure}
\end{flushleft}
\begin{flushleft}
    Ein weiterer Benchmark sind die Fehler, die bei den DML-Transktionen beim mixed-Benchnmark auftreten können.
    \begin{figure}[H]
        \centering
        \subfloat[\centering absolute]{{\includegraphics[width=0.5\linewidth]{source/pandas_data_chart_plotter/pgbench_errors_absolute} }}%
        \qquad
        \subfloat[\centering percentage]{{\includegraphics[width=0.5\linewidth]{source/pandas_data_chart_plotter/pgbench_errors_percentage} }}%
        \caption{Benchmarks - Fehler bei mixed-Transaktionen}
        \label{fig:pgbench_errors}
    \end{figure}
\end{flushleft}
\begin{flushleft}

\end{flushleft}
