%! Author = ibw
%! Date = 09.11.23

% Preamble
\section{Evaluation}
\subsection{Exkurs Architektur}
%! Author = itgramic
%! Date = 05.12.23

% Preamble
\subsubsection{Monolithische vs. verteilte Systeme}
%! Author = itgramic
%! Date = 05.12.23

% Preamble
\clearpage
\subsubsection{High Availability und Replikation}
\begin{flushleft}
    Wenn eine Datenbank HA (High Availability), also hochverfügbar sein soll, braucht es eine primäre und mindestens eine sekundäre- oder \Gls{Failover}-Datenbank.
    Um Datenverlust zu vermeiden, müssen die Daten permanent von der primären auf die sekundäre Datenbank synchronisiert werden, dies nennt man Replikation \cite{D9RDXENY}.
    Dabei wird zwischen den folgenden beiden Replikationen unterschieden:
\end{flushleft}
\begin{flushleft}
    \textbf{Synchrone Replikation}\\
    Wenn bei einer synchronen Replikation eine Transaktion abgesetzt wird, wird der Commit auf der primären Seite erst gesetzt, wenn die Änderung auf der sekundären Seite oder den sekundären Seiten ebenfalls eingetragen und committet ist.
    Bis zu diesem Moment ist die Transaktion nicht als committet.
    
    Dies wird dann zum Problem, wenn keine Verbindung mehr zu mindestens einer sekundären Seite vorhanden ist.
    Zudem wird die synchrone Replikation bei hohen Latenzen zum Bottleneck der Datenbank.
\end{flushleft}
\begin{flushleft}
    \textbf{Asynchrone Replikation}\\
    Bei der asynchronen Replikation wird eine Transaktion erst auf der eigenen primären Seite committet und erst dann an die sekundären Nodes gesendet.
    Besonders bei hohen Latenzen bleibt die Datenbank immer perfomant, allerdings kann es je nach Latenz und genereller Auslastung zu Datenverlusten kommen, wenn es zum \Gls{Failover} kommt.
\end{flushleft}
%! Author = itgramic
%! Date = 05.12.23

% Preamble
\subsubsection{Quorum}
\label{subsubsec:quorum}
\begin{flushleft}
    Ein Quorum-System soll die Integrität und Konsistenz in einem Datenbank-Cluster sicherstellen.
    Dabei gilt zu beachten, das nicht eine beliebige Anzahl an Nodes hinzugefügt werden können.
    Auch hat das Hinzufügen von Nodes immer eine einbusse an Performance zur Folge, besonders dann, wenn eine Synchrone Replikation gewählt wird und auf jedes Commitmend von den Replica-Nodes gewartet werden muss.

    \begin{description}
        \item \textbf{Quorum}\hfill \\Die Mehrheit der Server, die einen funktionierenden Betrieb gewährleisten können, ohne eine \Gls{Split-brain}Situation zu erzeugen.
        Die Formel ist gemeinhin \(n/2 + 1\)
        \item \textbf{Throughput}\hfill \\Beschreibt, wie sich die Anzahl Nodes auf die Schreibgeschwindigkeit der Commitments auf die restlichen Nodes auswirkt.\\Die verdopplung der Server halbiert i.d.R. den Throughput.
        \item \textbf{Fehlertoleranz}\hfill \\Beschreibt, wie viele Nodes ausfallen können, damit der Cluster noch Arbeitsfähig ist.\\Wobei eine erhöhung der Nodes von 3 auf 4 die Fehlertoleranz nicht erhöht da nun eine \Gls{Split-brain}-Situation entstehen kann.
    \end{description}
    %\begin{landscape}
    %\begin{table}[]
    %\resizebox{\columnwidth}{!}{%
    %\begin{tabular}{@{}llll@{}}
    %\toprule
    %\textbf{Anzahl Nodes} & \textbf{Quorum} & \textbf{Fehlertoleranz} & \textbf{Representative Throughput} \\ \midrule
    %1                     & 1               & 0                                               & 100                                \\
    %2                     & 2               & 0                                               & 85                                 \\
    %3                     & 2               & 1                                               & 82                                 \\
    %4                     & 3               & 1                                               & 57                                 \\
    %5                     & 3               & 2                                               & 48                                 \\
    %6                     & 4               & 2                                               & 41                                 \\
    %7                     & 4               & 3                                               & 36                                 \\ \bottomrule
    %\end{tabular}%
    %}
    %\caption{Quorum Beispiele}
    %\label{tab:quorum-beispiele}
    %\end{table}
    %\end{landscape}
    %\subsubsection{Split-brain}
    %\label{chap:Split-brain}
    Hier ein Beispiel wie sie in den Artikeln \cite{UMIGLCCI, YDS7DTYM, V4XLXN7W} beschrieben werden.
    Es zeigt auf, ab wie vielen Nodes die Fehlertoleranz erhöht wird und wie sich der Representative Throughput verhält.
%\end{flushleft}
%\begin{flushleft}
    \begin{table}[H]
    \resizebox{\columnwidth}{!}{%
    \begin{tabular}{@{}llll@{}}
    \toprule
    \textbf{Anzahl Nodes} & \textbf{Quorum} & \textbf{Fehlertoleranz} & \textbf{Representative Throughput} \\ \midrule
    1                     & 1               & 0                       & 100                                \\
    2                     & 2               & 0                       & 85                                 \\
    3                     & 2               & 1                       & 82                                 \\
    4                     & 3               & 1                       & 57                                 \\
    5                     & 3               & 2                       & 48                                 \\
    6                     & 4               & 2                       & 41                                 \\
    7                     & 4               & 3                       & 36                                 \\ \bottomrule
    \end{tabular}%
    }
    \caption{Quorum Beispiele}
    \label{tab:quorum-beispiele}
    \end{table}
\end{flushleft}
%! Author = itgramic
%! Date = 05.12.23

% Preamble
\clearpage
\subsubsection{CAP-Theorem}
Das CAP-Theorem besagt, das nur zwei der drei folgenden drei Merkmale von verteilten Systeme gewährleistet werden können\cite{EE6EQHU2}.
\begin{flushleft}
\textbf{Konsistenz - Consistency}\\
    Die Datenbank ist konsistent, alle Clients sehen gleichzeitig die gleichen Daten unabhängig, auf welchem Node zugegriffen wird.
    Hierzu muss eine Replikation der Daten an alle Nodes stattfinden und der Commit zurückgegeben werden, also eine synchrone Replikation stattfinden.
\end{flushleft}
\begin{flushleft}
\textbf{Verfügbarkeit - Availability}\\
    Jeder Client, der eine Anfrage sendet, muss auch eine Antwort erhalten.
    Unabhängig davon, wie viele Nodes im Cluster noch aktiv ist.
\end{flushleft}
\begin{flushleft}
\textbf{Ausfalltoleranz / Partitionstoleranz - Partition tolerance}\\
    Der Cluster muss auch dann noch funktionsfähig bleiben, wenn es eine beliebige Anzahl von Verbindungsunterbrüchen oder anderen Netzwerkproblemen zwischen den Nodes gibt.
\end{flushleft}
\begin{flushleft}
    Das CAP-Theorem lässt sich am einfachsten mit folgenden Venn-Diagramm visualisieren:
    \begin{figure}[H]
        \centering
        \includegraphics[width=0.5\linewidth]{source/implementation/evaluation/excursus_architecture/cap_theorem}
        \caption{CAP-Theorem}
        \label{fig:cap_theorem}
    \end{figure}
    \Gls{PostgreSQL}, \Gls{Oracle Database} oder \Gls{IBM DB2} präferieren CA, also Konsistenz und Verfügbarkeit.\\
    In dieser Diplomarbeit ist CA somit die Massgabe.
\end{flushleft}
%! Author = itgramic
%! Date = 05.12.23

% Preamble
\subsubsection{Skalierung}
\begin{flushleft}
    Datenbanken müssen skalierbar sein.
    Dabei wird unterschieden zwischen einer vertikalen Skalierung (scale-up) und horizontaler Skalierung (scale-out).
    Bei der vertikalen Skalierung werden den DB-Servern mehr CPU-Cores und Memory sowie zum Teil Storage hinzugefügt, wobei der Storage in jedem Fall wachsen wird.
    Beim horizontalen Skalieren werden weitere DB-Nodes in den Cluster eingehängt\cite{IZSGZLVT}:
    \begin{figure}[H]
        \centering
        \includegraphics[width=1\linewidth]{source/implementation/evaluation/excursus_architecture/Skalierung}
        \caption{Datenbankskalierung}
        \label{fig:Datenbankskalierung}
    \end{figure}

    Bei monolithischen Datenbanken, werden irgendwann die grenzen der horizontalen Skalierung erreicht und man muss wieder vertikal Skalieren, um dem Primary Node genügend Rechnerleistung vorzuhalten.
\end{flushleft}
\subsection{Erheben und Gewichten der Anforderungen}
%! Author = itgramic
%! Date = 29.12.23

% Preamble
\subsubsection{Anforderungen}
\begin{flushleft}
    Das KSGR hat eine Cloud First Strategie.\\
    Das heisst, alle neuen Applikationen und entsprechend deren Datenbanken müssen Cloud Ready bzw.
    Cloud Native sein.
    Um die Voraussetzung dafür zu schaffen, muss auch der PostgreSQL Cluster Cloud Ready sein.
\end{flushleft}
\begin{flushleft}
    Daher müssen zwei von drei genauer evaluierten Lösungen Cloud Native Lösungen sein.
    Wenn der Zeitaufwand reicht, können auch eine Cloud Native und Monolithisches System aufgebaut werden.
\end{flushleft}
\begin{flushleft}
    \begin{landscape}
    \begin{longtable}[H]{rlllll}

\toprule
Nr. & Anforderung & Bezeichnung & Beschreibung & System & Muss / Kann \\
\midrule
\endfirsthead
\caption[]{Anforderungskatalog} \\
\toprule
Nr. & Anforderung & Bezeichnung & Beschreibung & System & Muss / Kann \\
\midrule
\endhead
\midrule
\multicolumn{6}{r}{Continued on next page} \\
\midrule
\endfoot
\bottomrule
\endlastfoot
1 & Systemvielfallt &  & \begin{tabular}[c]{@{}l@{}}Es muss mindestens eine Monolitisches und mindestens 2 zwei Distributed SQL Cluster ermittelt werden\end{tabular} & Beides & MUSS \\
2 & Synergien &  & \begin{tabular}[c]{@{}l@{}}Skripte und APIs des Monolithisches Systems müssen auch in einem Distributed SQL System verwendet werden können\end{tabular} & Beides & MUSS \\
3 & Failover & Automatismus & \begin{tabular}[c]{@{}l@{}}Das Clustersystem muss bei einem Nodeausfall automatisch auf einen anderen Node umstellt\end{tabular} & Beides & MUSS \\
4 & Failover & Connection - Stabilität & \begin{tabular}[c]{@{}l@{}}Beim Failover dürfen bestehende Connections nicht getrennt werden oder sofort Wiederhergestellt werden\end{tabular} & Beides & MUSS \\
5 & Failover & Geschwindigkeit & \begin{tabular}[c]{@{}l@{}}Das umstellen auf den nächsten Node muss so schnell ausgefühgrt werden,\\das ein Disconnect mittels Client-Konfiguration (Timeout) verhindert wird.\end{tabular} & Beides & MUSS \\
6 & Switchover & Skript / API & \begin{tabular}[c]{@{}l@{}}Das System muss ein Skript oder eine API liefern,\\welche einen geordeten Switchover auf einen anderen Node erlaubt\end{tabular} & Beides & MUSS \\
7 & Switchover & Connection - Stabilität & \begin{tabular}[c]{@{}l@{}}Beim Switchover dürfen bestehende Connections nicht getrennt werden oder sofort Wiederhergestellt werden\end{tabular} & Beides & MUSS \\
8 & Switchover & Geschwindigkeit & \begin{tabular}[c]{@{}l@{}}Das umstellen auf den nächsten Node muss so schnell ausgefühgrt werden,\\das ein Disconnect mittels Client-Konfiguration (Timeout) verhindert wird.\end{tabular} & Beides & MUSS \\
9 & Restore & Skript / API & \begin{tabular}[c]{@{}l@{}}Das Clustersystem muss ein Skript oder eine API liefern,\\welche das einfache und ggf. automatisierte Restoren eines oder mehreren Nodes ermöglichen\end{tabular} & Beides & MUSS \\
10 & Restore & Datensicherheit & \begin{tabular}[c]{@{}l@{}}Beim Wiederherstellen des Ursprungszustands darf es zu keinem Datenverlust kommen\end{tabular} & Beides & MUSS \\
11 & Restore & Connection - Stabilität & \begin{tabular}[c]{@{}l@{}}Bei der Wiederherstellung einzelner Nodes darf es zu keinen Unterbrechungen auf den Applikationen kommen\end{tabular} & Beides & MUSS \\
12 & Restore & Geschwindigkeit & \begin{tabular}[c]{@{}l@{}}Das Wiederherstellen des Ursprungszustands muss\\innert weniger Stunden für alle Datenbanken aus dem\\Backup Wiederhergestellt und im Clustersystem Synchronisiert werden\end{tabular} & Beides & MUSS \\
13 & Replikation & Synchrone Replikation & \begin{tabular}[c]{@{}l@{}}Es muss eine Synchrone Replikation sichergestellt werden\end{tabular} & Monolitisch & MUSS \\
14 & Replikation & Failover / Switchover Garantie & \begin{tabular}[c]{@{}l@{}}Die Replikation muss sicherstellen, das es bei einem Failover/Switchover zu keinem Fehler kommt\end{tabular} & Monolitisch & MUSS \\
15 & Replikation & Throughput & \begin{tabular}[c]{@{}l@{}}Beschreibt, wie viele Transaktionen pro Zeiteinheit vom Primary an die Replikas gesendet und Commited werden.\\Dieser Wert ist bei Synchroner Replikation entscheidend da Commits auf allen Replicas abgesetzt sein müssen.\end{tabular} & Beides & MUSS \\
16 & Sharding & Datenschutz- und integrität & \begin{tabular}[c]{@{}l@{}}Die Datenkonsistenz und Datenintegrität auf den Shards muss sichergestellt werden\end{tabular} & Distributed SQL & MUSS \\
17 & Sharding & Schutz vor Datenverlust & \begin{tabular}[c]{@{}l@{}}Die Synchronisation der Shards muss sicherstellen, dass es zu keinem Datenverlust kommt\end{tabular} & Distributed SQL & MUSS \\
18 & Quorum & Quorum-System vorhanden & \begin{tabular}[c]{@{}l@{}}Das Clustersystem muss über ein Quorum-System besitzen\end{tabular} & Beides & MUSS \\
19 & Quorum & Robhustheit & \begin{tabular}[c]{@{}l@{}}Das Quorum des Clustersystems muss robust genug sein, um eine Split-Brain-Situation zu verhindern\end{tabular} & Beides & MUSS \\
20 & Connection &  & \begin{tabular}[c]{@{}l@{}}Das Clustersystem muss sicherstellen,\\dass eine Applikation ohne Entwicklungsaufwand mittels dem PostgreSQL Wired Connector zugreifen kann\end{tabular} & Beides & MUSS \\
21 & Management-API & Management-API vorhanden & \begin{tabular}[c]{@{}l@{}}Das Clustersystem muss Skripte oder eine API liefern,\\mit dem das System zu konfigurieren, verwalten oder überwachen zu können.\\Zudem müssen mit geringen Arbeitsaufwand\\damit Nodes hinzugefügt oder entfernt werden können\end{tabular} & Beides & MUSS \\
22 & Management-API & Authentifizierung \& Autorisierung & \begin{tabular}[c]{@{}l@{}}Es müssen gängige Standards für Authentifizierung und Autorisierung mitgebracht werden\end{tabular} & Beides & MUSS \\
23 & Management-API & Aufwand & \begin{tabular}[c]{@{}l@{}}Der Aufwand,\\der benötigt wird um die DB zu verwalten,\\Nodes hinzuzufügen oder zu entfernen usw. muss gegeneinander verglichen werden.\end{tabular} & Beides & MUSS \\
24 & Backup & Backup mit PostgreSQL Standards & \begin{tabular}[c]{@{}l@{}}Backups müssen mittels PostgreSQL Standards angezogen werden\end{tabular} & Beides & MUSS \\
25 & Backup & Restore mit PostgreSQL Standanrds & \begin{tabular}[c]{@{}l@{}}Backups müssen mittels PostgreSQL Standards restored werden können\end{tabular} & Beides & MUSS \\
26 & Housekeeping - Log Rotation &  & \begin{tabular}[c]{@{}l@{}}Das Clustersystem muss die möglichkeit zur Log Rotation bieten\end{tabular} & Beides & MUSS \\
27 & Self Heahling &  & \begin{tabular}[c]{@{}l@{}}Das Clustersystem muss im Fehlerfall Nodes selber wiederherstellen können\end{tabular} & Beides & KANN \\
28 & Monitoring - Node Failure &  & \begin{tabular}[c]{@{}l@{}}Läuft ein Node auf einen Fehler,\\muss das Clustersystem dies erkennen und Melden resp.\\eine Schnittstelle liefern die abgefragt werden kann\end{tabular} & Beides & MUSS \\
29 & Maintenance Quality &  & \begin{tabular}[c]{@{}l@{}}Da die meisten PostgreSQL HA Lösungen Open-Source sind,\\muss sichergestellt werden,\\dass die gewählte Lösung auch aktiv gepflegt wird.\\Als Basis dienen hier Informationen wie z.B. GitHub Insights.\end{tabular} & Beides & MUSS \\
30 & Performance & tps - Read-Only & \begin{tabular}[c]{@{}l@{}}Die Transaktionsrate (transactions per second / tps) für DQL Transaktionen\end{tabular} & Beides & MUSS \\
31 & Performance & tps - Read-Writes & \begin{tabular}[c]{@{}l@{}}Die Transaktionsrate (transactions per second / tps) für DML Transaktionen\end{tabular} & Beides & MUSS \\
32 & Performance & Ø Latenz - Read-Only & \begin{tabular}[c]{@{}l@{}}Die Latenzzeit bei DQL Transaktionen\end{tabular} & Beides & MUSS \\
33 & Performance & Ø Latenz - Read-Write & \begin{tabular}[c]{@{}l@{}}Die Latenzzeit bei DML Transaktionen\end{tabular} & Beides & MUSS \\
\caption{Anforderungskatalog} \label{anforderungskatalog}
\end{longtable}

    \end{landscape}
\end{flushleft}
%\begin{flushleft}
%    \begin{description}
%        \item \textbf{Kostenrechnung}\hfill \\Für die Kostenberechnung des Zeitaufwands wird im KSGR intern mit \(120CHF/h\) gerechnet.\\Jeder Arbeitstag hat dabei \(8.4h\) und pro Jahr wird mit \(220 Tagen\) gerechnet.
%        \item \textbf{Messung des Zeitaufwands}\hfill \\Der Zeitaufwand in der Evaluationsphase kann nur mit manueller Ausführung gemessen werden, da die Automatisierung nicht in der Evaluationsphase umgesetzt werden kann.\\In die Evaluation einfliessen wird aber die Schätzung, wie viel Aufwand betrieben werden muss um die wichtigsten Tasks automatisieren zu können.
%        \item
%    \end{description}
%
%    Folgende Messgrössen werden gestellt:
%    \begin{description}
%        \item \textbf{Quorum}\hfill \\
%        \item \textbf{Zeitaufwand Quorum erweitern}\hfill \\Bemessen wird, wie lange man braucht um einen neuen Node dem Quorum hinzuzufügen.
%        \item \textbf{Zeitaufwand Failover und Recovery}\hfill \\Bemessen wird, wie lange ein Failover und ein anschliessender Recover auf den normalen Zustand dauert.
%        \item \textbf{Failover Funktionsfähigkeit}\hfill \\Misst, ob der Failover bei korrekter Konfiguration funktionsfähig ist wie er vom entsprechenden System spezifiziert wurde.
%        \item \textbf{Failover Reaktionszeit}\hfill \\Gemessen und bemessen wird, wie lange es im Failoverszenario dauert, bis auf einen Standby-Node umgeschaltet wird und wie lange es dauert bis offene Connections wieder voll funktionsfähig sind.
%        \item \textbf{Recoverydauer}\hfill \\Bemisst, wie lange es nach einem Failover-Szenario dauert, bis der Normalzustand Widerhergestellt werden kann.
%    \end{description}
%
%
%\end{flushleft}
\subsubsection{Stakeholder}
\input{content/latex_tables/stakeholder}
\begin{flushleft}
\end{flushleft}
\subsubsection{Gewichtung}
\begin{flushleft}
    Die Gewichtung wurde mittels einer Präferenzmatrix ermittelt.\\
    Dabei wurden folgende Anforderungen aus übersichtsgründe in Sub-Matrizen aufgeteilt:
    \begin{itemize}
        \item Failover
        \item Switchover
        \item Restore
        \item Replikation
        \item Sharding
        \item Quorum
        \item Management-IP
        \item Backup
        \item Performance
    \end{itemize}

    Die Grundlegende Gewichtung wurde folgendermassen vorgenommen:
    \begin{figure}[H]
        \centering
        \includegraphics[width=1\linewidth]{source/implementation/evaluation/requirements/preference_matrix}
        \caption{Präferenzmatrix}
        \label{fig:preference_matrix}
    \end{figure}
\end{flushleft}
%\begin{flushleft}
%    Die Gewichtung der Failover-Anforderungen setzt sich wie folgt zusammen:
%    \begin{figure}[H]
%        \centering
%        \includegraphics[width=0.75\linewidth]{source/implementation/evaluation/requirements/preference_matrix_failover}
%        \caption{Präferenzmatrix - Failover}
%        \label{fig:preference_matrix_failover}
%    \end{figure}
%\end{flushleft}
%\begin{flushleft}
%    Beim Switchover wurde die Gewichtung wie folgt aufgeteilt:
%    \begin{figure}[H]
%        \centering
%        \includegraphics[width=0.75\linewidth]{source/implementation/evaluation/requirements/preference_matrix_switchover}
%        \caption{Präferenzmatrix - Switchover}
%        \label{fig:preference_matrix_switchover}
%    \end{figure}
%\end{flushleft}
%\begin{flushleft}
%    Die Gewichtung und Aufteilung der Restore-Anforderungen sieht wie folgt aus:
%    \begin{figure}[H]
%        \centering
%        \includegraphics[width=0.75\linewidth]{source/implementation/evaluation/requirements/preference_matrix_restore}
%        \caption{Präferenzmatrix - Restore}
%        \label{fig:preference_matrix_restore}
%    \end{figure}
%\end{flushleft}
%\begin{flushleft}
%    Die Replikationsanforderungen resp.
%    deren Gewichtung ist wie folgt aufgebaut:
%    \begin{figure}[H]
%        \centering
%        \includegraphics[width=0.75\linewidth]{source/implementation/evaluation/requirements/preference_matrix_replication}
%        \caption{Präferenzmatrix - Replikation}
%        \label{fig:preference_matrix_replication}
%    \end{figure}
%\end{flushleft}
%\begin{flushleft}
%    Das Sharding setzt sich aus folgenden Teilen zusammen:
%    \begin{figure}[H]
%        \centering
%        \includegraphics[width=0.75\linewidth]{source/implementation/evaluation/requirements/preference_matrix_sharding}
%        \caption{Präferenzmatrix - Sharding}
%        \label{fig:preference_matrix_sharding}
%    \end{figure}
%\end{flushleft}
%\begin{flushleft}
%    Die Quorum-Anforderung ist folgendermassen zusammengesetzt:
%    \begin{figure}[H]
%        \centering
%        \includegraphics[width=0.75\linewidth]{source/implementation/evaluation/requirements/preference_matrix_quorum}
%        \caption{Präferenzmatrix - Quorum}
%        \label{fig:preference_matrix_quorum}
%    \end{figure}
%\end{flushleft}
%\begin{flushleft}
%    Bei der Management-API gibt es mehrere Sub-Anforderungen:
%    \begin{figure}[H]
%        \centering
%        \includegraphics[width=0.75\linewidth]{source/implementation/evaluation/requirements/preference_matrix_management_api}
%        \caption{Präferenzmatrix - Management-API}
%        \label{fig:preference_matrix_management_api}
%    \end{figure}
%\end{flushleft}
%\begin{flushleft}
%    Anforderungen zum Backup wurden nachfolgend aufgeteilt und gewichtet:
%    \begin{figure}[H]
%        \centering
%        \includegraphics[width=0.75\linewidth]{source/implementation/evaluation/requirements/preference_matrix_backup}
%        \caption{Präferenzmatrix - Backup}
%        \label{fig:preference_matrix_backup}
%    \end{figure}
%\end{flushleft}
%\begin{flushleft}
%    Performance-Benchmarking lässt sich in nachfolgende Teile unterteilen:
%    \begin{figure}[H]
%        \centering
%        \includegraphics[width=1\linewidth]{source/implementation/evaluation/requirements/preference_matrix_performance}
%        \caption{Präferenzmatrix - Performance}
%        \label{fig:preference_matrix_performance}
%    \end{figure}
%\end{flushleft}
\subsection{Testziele erarbeiten}
\subsection{PostgreSQL Benchmarking}
PostgreSQL bietet ein Benchmarking-Tool,\cite{TYJFF7AB,VXNYQFTE} mit dem die DB Vermessen werden kann.
\subsection{Analyse gängiger PostgreSQL HA Cluster Lösungen}
%! Author = itgramic
%! Date = 05.12.23

% Preamble
\subsubsection{\Gls{PostgreSQL} Replikation}
\begin{flushleft}
    PostgreSQL bietet von Haus aus Möglichkeiten, um Replikationen durchzuführen.
    Dabei ist nicht jede gleich gut für jedes Szenario geeignet\cite{FZAHA89U}.
\end{flushleft}
\begin{flushleft}
    \textbf{Shared Disk Failover}\\
\end{flushleft}
\begin{flushleft}
    \textbf{File System (Block Device) Replication}\\
\end{flushleft}
\begin{flushleft}
    \textbf{Write-Ahead Log Shipping}\\
\end{flushleft}
\begin{flushleft}
    \textbf{Logical Replication}\\
\end{flushleft}
\begin{flushleft}
    \textbf{Trigger-Based Primary-Standby Replication}\\
\end{flushleft}
\begin{flushleft}
    \textbf{Data Partitioning}\\
    \textbf{Multiple-Server Parallel Query Execution}\\
\end{flushleft}
\begin{flushleft}
    \textbf{}\\
\end{flushleft}
\begin{flushleft}
    \textbf{}\\
\end{flushleft}
\begin{flushleft}
    \textbf{}\\
\end{flushleft}
\begin{flushleft}
    \textbf{}\\
\end{flushleft}
\begin{flushleft}
    \textbf{}\\
\end{flushleft}
%! Author = itgramic
%! Date = 05.12.23

% Preamble
\begin{flushleft}
    \subsubsection{KSGR Lösung}
    Das Kantonsspital Graubünden hat prüft mit \gls{keepalived} ob die primäre Seite erreichbar und betriebsbereit ist.
    Trifft dies nicht mehr zu, wird ein \Gls{Failover} durchgeführt \cite{NLF2IDBZ}.
    Ist die primäre Seite wieder verfügbar, wird ein Restore auf die primäre Seite gefahren.
\end{flushleft}
\begin{flushleft}
    Beim Restore wird ein komplettes Backup der sekundären Seite auf die primäre Seite übertragen.
    Ursache ist, dass die normalerweise für den Datenrestore benötigten \Gls{PostgreSQL} Board Mttel nur für eine relativ kurze Zeit eingesetzt werden können,
    ehe die Differenzen zwischen den beiden Seiten zu gross werden.
    Bei kleinen Datenbanken wie jene für \Gls{Harbor} und \Gls{GitLab} ist die Zeit die hierfür benötigt wird, nicht relevant.
    Sind die Datenbanken auf dem \Gls{PostgreSQL Cluster} jedoch grösser, kann der Restore mehrere Minuten dauern.
\end{flushleft}
%! Author = itgramic
%! Date = 05.12.23

% Preamble
\subsubsection{pgpool-II}
pgpool-II ist eine Middleware die zwischen einem \Gls{PostgreSQL Cluster} und einem PostgreSQL Client gesetzt wird.
pgpool-II bietet folgende Funktionen\cite{EXVNLICT,3XWCD3KX}:
\\\textbf{High Availability}\\
pgpool-II bietet einen automatic \Gls{Failover} genannten Service an, den Watchdog.
Dieser schwenkt auf einen Standby-Server und entfernt den Defekten Server.
Um false positive Events und Split-brains zu verhindern setzt pgpool-II auf einen eigens entwickelten \Gls{Quorum}-Algorithmus.
\\\textbf{Connection Pooling}\\
Bestehende Connections werden wiederverwendet um die Anzahl gleichzeitig offener Connections zu reduzieren.
Der Pool wird dabei anhand von Username, Database, Protocol und weiteren Verbindungsparametern zugeordnet.
\\\textbf{Replikation}\\
Nebst dem Standard \Gls{PostgreSQL} bietet pgpool-II sein eigenes Replikationssystem an.
\\\textbf{Load Balancing}\\
Ähnlich wie Oracle Active Data Guard \cite{6294443C} bietet auch pgpool-II die Möglichkeit, SELECT-Queries und Backup-Jobs auf die Secondary-Nodes umzuleiten um den Primary Node zu entlasten.
\\\textbf{Limiting Exceeding Connections}\\
Die Anzahl an concurrent Connections, also gleichzeitiger Verbindungen, ist bei \Gls{PostgreSQL} begrenzt (Systemparameter wird dabei vom DBA gesetzt).
pgpool-II speichert alle Connections, die über dem Limit sind, in einer Queue und somit nicht sofort fehlerhaft abgelehnt.
\\\textbf{Watchdog}\\
Der Watchdog koordiniert mehrere pgpool-II Nodes und verhindert ein Split-brain.
\\\textbf{In Memory Query Caching}\\
pgpool-II speichert SELECT-Queries in einem Cache und verwendet die ResultSets wieder, wenn eine identische Abfrage eingeht.
\\\textbf{Online Recovery}\\

%! Author = itgramic
%! Date = 05.12.23

% Preamble
\begin{flushleft}
    \subsubsection{pg\_auto\_failover}
    pg\_auto\_failover ist eine PostgreSQL-Erweiterung, die von der Microsoft Subunternehmen Citus Data entwickelt wird.
\end{flushleft}
\begin{flushleft}
    \paragraph{Core-Features}
    Die wichtigsten Features von pg\_auto\_failover sind:
    \begin{itemize}
        \item API
        \item PostgreSQL Extension, also reines PostgreSQL
        \item State Machine Driven
        \item Replikations-Quorum
        \item Citus kompatibel
        \item Azure VM Support
    \end{itemize}
\end{flushleft}
\begin{flushleft}
    \paragraph{Replikation}
    pg\_auto\_failover bietet die Standard PostgreSQL-Replikationen.
\end{flushleft}
\begin{flushleft}
    \paragraph{Proxy}
    pg\_auto\_failover benötigt einen \Gls{HAProxy}, um Load Balancing betreiben zu können\cite{VYXTI7BS}.
\end{flushleft}
\begin{flushleft}
    \paragraph{API / Skripte}
    pg\_auto\_failover bietet ein eigenes CLI-Tool, \texttt{pg\_autoctl}.
    Dieses stellt Commands für das Einbinden neuer Nodes,\\
    das Managen von Nodes (Maintenance resp. Switchover),\\
    Konfigurieren oder Monitoren des Systems zur Verfügung\cite{4X2AKDB6}.
\end{flushleft}
\begin{flushleft}
    \paragraph{Architektur}
    Die Dokumentation resp. Grafik von pg\_auto\_failover \cite{PZZIZ5RT} zeigt auf, wie der Failover funktioniert:
    \begin{figure}[H]
        \centering
        \includegraphics[width=0.75\linewidth]{source/implementation/evaluation/postgresql_ha_solutions/pg_auto_failover/pg_auto-failover_arch-single-standby}
        \caption{pg\_auto\_failover-Architektur - Single Standby}
        \label{fig:pg_auto-failover_arch-single-standby}
    \end{figure}
    Aber wie die Grafik zeigt, können auch Multi-Nodes können eingebunden werden\cite{4ZKBDG57}:
    \begin{figure}[H]
        \centering
        \includegraphics[width=0.75\linewidth]{source/implementation/evaluation/postgresql_ha_solutions/pg_auto_failover/pg_auto-failover_arch-multi-standby}
        \caption{pg\_auto\_failover-Architektur - Multi-Node Standby}
        \label{fig:pg_auto-failover_arch-multi-standby}
    \end{figure}

    pg\_auto\_failover kann Citus einbinden.
    Allerdings bleibt die Architektur im Kern immer monolithisch.\\
    ie nachfolgende Grafik zeigt die Architektur mit Citus\cite{3FVHLIFE}:
    \begin{figure}[H]
        \centering
        \includegraphics[width=0.75\linewidth]{source/implementation/evaluation/postgresql_ha_solutions/pg_auto_failover/pg_auto-failover_arch-citus}
        \caption{pg\_auto\_failover-Architektur - Citus}
        \label{fig:pg_auto-failover_arch-citus}
    \end{figure}
\end{flushleft}
\begin{flushleft}
    \paragraph{Synergien und Mehrwert}
    pg\_auto\_failover bietet eine Docker-Compose-Integration.\\
    Allerdings ist keine Kubernetes-Integration dokumentiert.
\end{flushleft}
\begin{flushleft}
    Damit bietet pg\_auto\_failover keine Möglichkeit\\
    Synergien zwischen monolithischer Architektur und einer Cloud-Native-Umsetzung auf Kubernetes.\\
    Entsprechend ist kein Mehrwert vorhanden.
\end{flushleft}
%! Author = gra
%! Date = 05.04.24

% Preamble
\begin{flushleft}
    \section{Patroni}
    \subsection{Prerequisites}
\end{flushleft}
\begin{flushleft}
    Zuerst muss der Proxy gesetzt werden:
    \lstset{style=gra_codestyle}
\begin{lstlisting}[language=bash, caption=Patroni - Proxy Settings,captionpos=b,label={lst:patroni-proxy-settings},breaklines=true]
# sks1232 / sks1233 / sks1234
# Proxy setzen
# nano /etc/profile.d/proxy.sh
export https_proxy=http://sproxy.sivc.first-it.ch:8080
export HTTPS_PROXY=http://sproxy.sivc.first-it.ch:8080
export http_proxy=http://sproxy.sivc.first-it.ch:8080
export HTTP_PROXY=http://sproxy.sivc.first-it.ch:8080
export no_proxy=localhost,127.0.0.0/8,::1,10.0.0.0/8,172.16.0.0/12,192.168.0.0/16
export NO_PROXY=localhost,127.0.0.0/8,::1,10.0.0.0/8,172.16.0.0/12,192.168.0.0/16
# source /etc/profile.d/proxy.sh
\end{lstlisting}
\end{flushleft}
\begin{flushleft}
    Damit das PostgreSQL-Repository eingebunden werden kann,\\
    muss dem apt-Proxy gesetzt werden.\\
    Da via \Gls{Foreman} Installiert wurde, muss dieser ausgenommen werden:
    \lstset{style=gra_codestyle}
\begin{lstlisting}[language=bash, caption=Patroni - apt-Proxy Settings,captionpos=b,label={lst:patroni-apt-proxy-settings},breaklines=true]
# sks1232 / sks1233 / sks1234
# apt-Proxy setzen
# nano /etc/apt/apt.conf.d/proxy.conf
Acauire::http::Proxy "http://sproxy.sivc.first-it.ch:8080";
Acauire::https::Proxy "http://sproxy.sivc.first-it.ch:8080";
Acquire::http::proxy::foreman.ksgr.ch "DIRECT";
\end{lstlisting}
\end{flushleft}
\begin{flushleft}
    Im nächsten Schritt kann das PostgreSQL-Repository eingebunden werden.\\
    \begin{warning}
    Achtung, die von PostgreSQL beschriebene Variante wurde in Debian 10 als Deprecated gesetzt,
    mit Debian 13 wird diese Repository-Integration einen Fehler werden.
    \end{warning}    \lstset{style=gra_codestyle}
\begin{lstlisting}[language=bash, caption=Patroni - PostgreSQL einbinden,captionpos=b,label={lst:patroni-include-repository},breaklines=true]
# sks1232 / sks1233 / sks1234
# PostgreSQL Repository einbinden
sudo sh -c 'echo "deb https://apt.postgresql.org/pub/repos/apt $(lsb_release -cs)-pgdg main" > /etc/apt/sources.list.d/pgdg.list'
wget --quiet -O - https://www.postgresql.org/media/keys/ACCC4CF8.asc | sudo apt-key add -

# Ausloggen und wieder einloggen
apt update
\end{lstlisting}
\end{flushleft}
\begin{flushleft}
    \gls{etcd} wird als nächstes Installiert.\\
    Hierzu muss zuerst das Repository von \Gls{GitHub} heruntergeladen werden:
\end{flushleft}
\begin{flushleft}
\end{flushleft}
\begin{flushleft}
\end{flushleft}
\begin{flushleft}
\end{flushleft}
\begin{flushleft}
    \subsection{Installation}
\end{flushleft}
\begin{flushleft}
\end{flushleft}
\begin{flushleft}
    \subsection{Konfiguration}
\end{flushleft}
\begin{flushleft}
\end{flushleft}
%! Author = itgramic
%! Date = 05.12.23

% Preamble
\begin{flushleft}
    \subsubsection{CloudNativePG}
    CloudNativePG ist eine Containerlösung für PostgreSQL auf Kubernetes.\\
    CloudNativePG wurde ursprünglich von EDB entwickelt.
\end{flushleft}
\begin{flushleft}
    \paragraph{Core-Features}
    Die wichtigsten Features von CloudNativePG sind\cite{5ALQPE2U}:
    \begin{itemize}
        \item k8s API integration
        \item Autoamtischer Failover
        \item Self-Healing von Nodes resp. Replikas
        \item Skalierbarkeit (Vertikal, Horizontal bedingt)
        \item Volumne Backup
        \item Object Backup
        \item Rolling PostgreSQL Upgrade / Updates
        \item Pooling mit PgBouncer
        \item Offline und Online Import von bestehenden PostgreSQL DBs
    \end{itemize}
\end{flushleft}
\begin{flushleft}
    \paragraph{Replikation}
    CloudNativePG bietet die üblichen PostgreSQL Replikaionen an.
\end{flushleft}
\begin{flushleft}
    \paragraph{Proxy}
    CloudNativePG benötigt keinen zusätzlichen Proxy.
\end{flushleft}
\begin{flushleft}
    \paragraph{Pooling}
    CloudNativePG unterstützt pgBouncer als Pooler.
\end{flushleft}
\begin{flushleft}
    \paragraph{API / Skripte}
    CloudNativePG bietet eine API zum Monitoren und Verwalten von Backups, Clustern und dem System selbst\cite{LY8V4XQM}.
\end{flushleft}
\begin{flushleft}
    \paragraph{Architektur}
    Kubernetes regelt die Zugriffe mittels eines entsprechenden Services in die Nodes auf den Pods:
    \begin{figure}[H]
        \centering
        \includegraphics[width=0.75\linewidth]{source/implementation/evaluation/postgresql_ha_solutions/cloudnativepg/k8s-pg-architecture}
        \caption{CloudNativePG - Kubernetes - PostgreSQL}
        \label{fig:k8s-pg-architecture}
    \end{figure}
\end{flushleft}
\begin{flushleft}
    Dabei werden die Read-write workloads an den Primary Node gesendet:
    \begin{figure}[H]
        \centering
        \includegraphics[width=0.75\linewidth]{source/implementation/evaluation/postgresql_ha_solutions/cloudnativepg/cloudnativepg-architecture-rw}
        \caption{CloudNativePG - Kubernetes - Read-write workloads}
        \label{fig:cloudnativepg-architecture-rw}
    \end{figure}
\end{flushleft}
\begin{flushleft}
    Read-only workloads werden mit Round robin an die Replicas zugewiesen:
    \begin{figure}[H]
        \centering
        \includegraphics[width=0.75\linewidth]{source/implementation/evaluation/postgresql_ha_solutions/cloudnativepg/cloudnativepg-architecture-read-only}
        \caption{CloudNativePG - Kubernetes - Read-only workloads}
        \label{fig:cloudnativepg-architecture-read-only}
    \end{figure}
\end{flushleft}
\begin{flushleft}
    Es könnten auch Lösungen mit Designated Kubernetes-Clustern in einem anderen RZ oder einer anderen Geo-Region realisiert werden.
\end{flushleft}
\begin{flushleft}
    \paragraph{Maintenance}
    \hyperref[subsec:maintenance_cloudnativepg]{Anhang - Maintenance}

\end{flushleft}
\begin{flushleft}
    \paragraph{Synergien und Mehrwert}
    CloudNativePG bleibt ein Monolithisches System,\\welches aber keine Möglichkeit bietet,\\auch auf einem Normalen Serversetting betrieben zu werden.
\end{flushleft}
\begin{flushleft}
    Daher bietet CloudNativePG weder einen Benefit durch seine Architektur noch mit der Möglichkeit,\\Synergien nutzen zu können.
\end{flushleft}
%! Author = gramic
%! Date = 22.04.24

% Preamble
\begin{flushleft}
    \subsection{YugabyteDB}
    \label{subsec:appendix_testing_yugabytedb}
    Zum einen kann der Fehler irgendwann auftreten.\\
    In diesem Fall wird erst im Log die Fehlermeldung geworfen, dass die Zeitdifferenz zu gross ist:
    \begin{figure}[H]
        \centering
        \includegraphics[width=1\linewidth]{source/appendix/evaluation_testing/yugabytedb_too_big_clock_skew_is_detected}
        \caption{YugabyteDB - Too big clock skew is detected}
        \label{fig:yugabytedb_too_big_clock_skew_is_detected}
    \end{figure}
    Eine Folge ist, dass kein neuer Leader bestimmt werden kann:
    \begin{figure}[H]
        \centering
        \includegraphics[width=1\linewidth]{source/appendix/evaluation_testing/yugabytedb_tablet_leader_lease}
        \caption{YugabyteDB - Tablet Leader - No Lease}
        \label{fig:yugabytedb_tablet_leader_lease}
    \end{figure}
    Als nächstes wird der komplette \texttt{tserver} in einem \texttt{CrashLoopBackOff} fallen:
    \begin{figure}[H]
        \centering
        \includegraphics[width=1\linewidth]{source/appendix/evaluation_testing/yugabytedb_crashloopbackoff}
        \caption{YugabyteDB - CrashLoopBackOff}
        \label{fig:yugabytedb_crashloopbackoff}
    \end{figure}
    Der ganze Cluster an sich aber bleibt arbeitsfähig.
\end{flushleft}
\begin{flushleft}
    Anders sieht es aus, wenn auch \texttt{tmaster}-Nodes von Start weg betroffen sind.\\
    Es werden aber primär nur die Logs überall geschrieben:
    \begin{figure}[H]
        \centering
        \includegraphics[width=1\linewidth]{source/appendix/evaluation_testing/yugabytedb_yb-tmaster-0_sks1184_clock_error}
        \caption{YugabyteDB - Too big clock skew is detected - tmaster}
        \label{fig:yugabytedb_yb-tmaster-0_sks1184_clock_error}
    \end{figure}
    \begin{figure}[H]
        \centering
        \includegraphics[width=1\linewidth]{source/appendix/evaluation_testing/yugabytedb_yb-tserver-1_sks1184_clock_error}
        \caption{YugabyteDB - Too big clock skew is detected - tserver}
        \label{fig:yugabytedb_yb-tserver-1_sks1184_clock_error}
    \end{figure}
%    \begin{figure}[H]
%        \centering
%        \subfloat[yb-tmaster-0]{{\includegraphics[width=0.47\linewidth]{source/appendix/evaluation_testing/yugabytedb_yb-tmaster-0_sks1184_clock_error} }}%
%        \qquad
%        \subfloat[yb-tserver-1]{{\includegraphics[width=0.47\linewidth]{source/appendix/evaluation_testing/yugabytedb_yb-tserver-1_sks1184_clock_error} }}%
%        \caption{YugabyteDB - Too big clock skew is detected - Node}
%        \label{fig:yugabytedb_sks1184_clock_error}
%    \end{figure}
    YugabyteDB erlaubt in so einem Fall keine Zugriffe mehr auf den Cluster.\\
    So wird verhindert, dass der Cluster korrumpiert wird.
\end{flushleft}
%! Author = gramic
%! Date = 22.04.24

% Preamble
\begin{flushleft}
    \subsection{StackGres - Citus}
    Der Node ging down, als der Server \texttt{sks1184} heruntergefahren wurde:
    \begin{figure}[H]
        \centering
        \includegraphics[width=0.5\linewidth]{source/appendix/evaluation_testing/stackgres_node_sks1184_down}
        \caption{StackGres Testing - Node sks1184 down}
        \label{fig:stackgres_node_sks1184_down}
    \end{figure}
    Entsprechend wurden die Pods ebenfalls auf \texttt{terminating} gesetzt:
    \begin{figure}[H]
        \centering
        \includegraphics[width=0.5\linewidth]{source/appendix/evaluation_testing/stackgres_citus_testing_node_down}
        \caption{StackGres Testing - Pods Down}
        \label{fig:stackgres_citus_testing_node_down}
    \end{figure}
    Der Patroni-Leader des Coordinators aber auch die der Shards wurden einem \texttt{Failover} ausgeführt:
    \begin{figure}[H]
        \centering
        \includegraphics[width=0.5\linewidth]{source/appendix/evaluation_testing/stackgres_patroni_failover_overview}
        \caption{StackGres Testing - Patroni Übersicht}
        \label{fig:stackgres_patroni_failover_overview}
    \end{figure}
    Während dieser Zeit, ist die DB immer erreichbar:
    \begin{figure}[H]
        \centering
        \includegraphics[width=0.5\linewidth]{source/appendix/evaluation_testing/stackgres_node_down_access_possible}
        \caption{StackGres Testing - DB Zugriff}
        \label{fig:stackgres_node_down_access_possible}
    \end{figure}
    Allerdings werden längere Transaktionen geschlossen:
    \begin{figure}[H]
        \centering
        \includegraphics[width=0.5\linewidth]{source/appendix/evaluation_testing/stackgres_citus_connection_timeout}
        \caption{StackGres Testing - Connection Timeout}
        \label{fig:stackgres_citus_connection_timeout}
    \end{figure}
\end{flushleft}
%! Author = gramic
%! Date = 11.03.24

% Preamble
\begin{flushleft}
    \subsection{Vorauswahl}
    Folgende Lösungen werden nicht evaluiert, sondern bereits zu Beginn ausgeschieden:
    \begin{table}[H]

\resizebox{\columnwidth}{!}{%

\begin{tabular}{rlll}
\toprule
Nr. & Lösung & Status & Begründung \\
\midrule
1 & KSGR-Lösung & Vorausgeschieden & \begin{tabular}[c]{@{}l@{}}Hat nur einen Standy / Replika-Node.\\Failover Funktioniert nur bei kleineren Datenmengen wirklich in einer vernüftigen Zeit.\end{tabular} \\
2 & pgpool-II & Vorausgeschieden & \begin{tabular}[c]{@{}l@{}}pgpool-II hat kein GitHub-Repository und bietet daher keine vergleichswerte mittels Github Insights.\end{tabular} \\
3 & pg\_auto\_failover & Vorausgeschieden & \begin{tabular}[c]{@{}l@{}}pg\_auto\_failover würde zwar Citus-Support bieten,\\allerdings gibt es keine gut dokumentierte Implementation für Kubernetes.\\Erfüllt daher das Kriterium für die Synergien nicht\end{tabular} \\
4 & CloudNativePG & Vorausgeschieden & \begin{tabular}[c]{@{}l@{}}CloudNativePG ist keine vollständige Cloud Native Lösung.\\Mittels Citus könnte sogar eine Distributed SQL Lösung implementiert werden.\\Die Grundarchitektur bleibt aber Monolithisch mit einem Primary und Replikas.\\Und da kein Benefit in Form von Synergien vorhanden sind,\\fällt CloudNativePG raus.\end{tabular} \\
8 & Citus row-based-sharding & Vorausgeschieden & \begin{tabular}[c]{@{}l@{}}Citus row-based-sharding wäre Hocheffizient\\wenn es um Ressourcenverteilung geht und zudem echtes Sharding.\\Allerdings setzt es anpassungen an den Tabellen der Applikationen voraus.\\Das KSGR ist allerdings kein Softwarehaus\\und kann keine Forks durchführen,\\auch weil viele Applikationen zertifiziert sein müssen.\\Scheitert daher an der Machbarkeit\end{tabular} \\
\bottomrule
\end{tabular}
}
\caption{Vorauswahl - Ausgeschieden} \label{predecision_out}
\end{table}


    Entsprechend werden nur noch nachfolgende Lösungen genauer betrachtet:
    \begin{table}[H]

\resizebox{\columnwidth}{!}{%

\begin{tabular}{rlll}
\toprule
Nr. & Lösung & Status & Begründung \\
\midrule
5 & Patroni & Evaluation & \begin{tabular}[c]{@{}l@{}}Patroni kann als Monolithisches System genutzt werden,\\ist aber auch Kern von Stackgres.\\Die API und Skripte können also in beiden Welten verwendet werden\end{tabular} \\
6 & Stackgres mit Citus & Evaluation & \begin{tabular}[c]{@{}l@{}}Bietet eine einfache und kompakte Möglichkeit für ein Distributed SQL System.\\Da Patroni unter der Haube ist,\\kann die API und sonstige Skripte auch auf einem Monolithischen System eingesetzt werden.\end{tabular} \\
7 & Yugabyte-DB & Evaluation & \begin{tabular}[c]{@{}l@{}}Ist eine reine Distributed SQL Lösung und ist Vollständig Cloud Native.\end{tabular} \\
\bottomrule
\end{tabular}
}
\caption{Vorauswahl - Evaluation} \label{predecision_in}
\end{table}

\end{flushleft}
\subsection{Installation verschiedener Lösungen}
%! Author = itgramic
%! Date = 26.01.24

% Preamble
\subsubsection{Monolitische Umgebung}
\subsubsection{Kubernetes}
Um ein minimales, dem Produktiven möglichst nahes Enviroment für die evaluation zu erhalten, wurde folgendes Setting ausgewählt:
\begin{table}[]
\begin{tabular}{ll}
\textbf{Linux Distribution}                & Debian 11              \\
\textbf{Kubernetes Runtime}                & rke2                   \\
\textbf{Container-Enviroment}              & containerd             \\
\textbf{Container Network Interface (CNI)} & cilium                 \\
\textbf{Cloud Native Storage (CNS)}        & local-path-provisioner
\end{tabular}
\caption{Evaluations settings}
\label{tab:evaluation-setting}
\end{table}
%! Author = itgramic
%! Date = 29.12.23

% Preamble
\begin{flushleft}
    \subsection{Entscheid}
    Anhand der Kosten-Nutzen-Analyse, steht die Entscheidung fest.\\
    Patroni wird mit der Variante \texttt{postgresql\_cluster} als Testsystem aufgebaut.
\end{flushleft}
\begin{flushleft}
    Die Umsetzung der Cloud Native Lösung, in diesem Fall YugabyteDB, muss verschoben werden.\\
    Dem \Gls{Kubernetes} Testsystem des KSGR fehlt zum einen noch eine notwendige interne Komponente zur Freigabe,\\
    zum anderen Fehlen für den sauberen Betrieb die Externen Systeme wie der \Gls{PKI} oder dass \Gls{SIEM} (welches aber in den nächsten Wochen eingeführt wird).
\end{flushleft}
\subsection{Gegenüberstellung der Lösungen}
%! Author = itgramic
%! Date = 29.12.23

% Preamble
\begin{flushleft}
    \subsubsection{Kosten-Nutzen}
    Patroni, in seinen beiden Ausprägungen als normales Patroni und mit dem \texttt{postgresql\_cluster}-Ansible \Gls{GitHub}-Repository\\
    ist klar die beste Variante:
    \begin{figure}[H]
        \centering
        \includegraphics[width=1\linewidth]{source/implementation/evaluation/solution_comparison/nutzwert_analyse}
        \caption{Kosten-Nutzen-Analyse}
        \label{fig:nutzwert_analyse}
    \end{figure}
\end{flushleft}
\begin{flushleft}
    Bei den Kosten zeigt sich, dass die Patroni-Variante \texttt{postgresql\_cluster} klar am günstigsten kommt.\\
    Dies resultiert auf den tiefen Installationskosten und der schnellen Erweiterungen.
\end{flushleft}
\begin{flushleft}
    Da, dass Repository schon einiges an Ansible-Playbooks mitbringt, ist auch der Erweiterungsaufwand gering.\\
    YugabyteDB liegt auf Platz zwei, gefolgt vom klassischen (Vanilla) Patroni.
\end{flushleft}
\begin{flushleft}
    Auf dem letzten Platz landet StackGres - Citus, dies weil die Erstellung eines Clusters sehr viel Zeit benötigt.
    \begin{figure}[H]
        \centering
        \includegraphics[width=1\linewidth]{source/implementation/evaluation/solution_comparison/cost_benefit_ranking}
        \caption{Kosten-Nutzen-Ranking}
        \label{fig:cost_benefit_ranking}
    \end{figure}
\end{flushleft}
\begin{flushleft}
    Daraus ergibt sich folgendes Kosten-Nutzen-Diagramm:
    \begin{figure}[H]
        \centering
        \includegraphics[width=1\linewidth]{source/cost_benefit_diagram/cost_benefit_diagram}
        \caption{Kosten-Nutzen-Diagramm}
        \label{fig:cost_benefit_diagram}
    \end{figure}
    StackGres - Citus scheidet direkt aus, da es nur 77\% der geforderten Punktezahl erreicht.\\
    Aber auch das klassische Patroni scheidet aus,\\
    dies, weil die Kosten im Verhältnis zur besten Variante (\texttt{postgresql\_cluster}) ebenfalls unter der 80\% Marke liegt.
\end{flushleft}
\subsection{Entscheid}