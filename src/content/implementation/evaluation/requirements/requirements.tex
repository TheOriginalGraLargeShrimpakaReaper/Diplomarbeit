%! Author = itgramic
%! Date = 29.12.23

% Preamble
\subsubsection{Anforderungen}
\begin{flushleft}
\end{flushleft}
\begin{flushleft}

    \begin{description}
        \item \textbf{Kostenrechnung}\hfill \\Für die Kostenberechnung des Zeitaufwands wird im KSGR intern mit \(120CHF/h\) gerechnet.\\Jeder Arbeitstag hat dabei \(8.4h\) und pro Jahr wird mit \(220 Tagen\) gerechnet.
        \item \textbf{Messung des Zeitaufwands}\hfill \\Der Zeitaufwand in der Evaluationsphase kann nur mit manueller Ausführung gemessen werden, da die Automatisierung nicht in der Evaluationsphase umgesetzt werden kann.\\In die Evaluation einfliessen wird aber die Schätzung, wie viel Aufwand betrieben werden muss um die wichtigsten Tasks automatisieren zu können.
        \item 
    \end{description}

    Folgende Messgrössen werden gestellt:
    \begin{description}
        \item \textbf{Quorum}\hfill \\
        \item \textbf{Zeitaufwand Quorum erweitern}\hfill \\Bemessen wird, wie lange man braucht um einen neuen Node dem Quorum hinzuzufügen.
        \item \textbf{Zeitaufwand Failover und Recovery}\hfill \\Bemessen wird, wie lange ein Failover und ein anschliessender Recover auf den normalen Zustand dauert.
        \item \textbf{Failover Funktionsfähigkeit}\hfill \\Misst, ob der Failover bei korrekter Konfiguration funktionsfähig ist wie er vom entsprechenden System spezifiziert wurde.
        \item \textbf{Failover Reaktionszeit}\hfill \\Gemessen und bemessen wird, wie lange es im Failoverszenario dauert, bis auf einen Standby-Node umgeschaltet wird und wie lange es dauert bis offene Connections wieder voll funktionsfähig sind.
        \item \textbf{Recoverydauer}\hfill \\Bemisst, wie lange es nach einem Failover-Szenario dauert, bis der Normalzustand Widerhergestellt werden kann.
    \end{description}


\end{flushleft}
\subsubsection{Gewichtung}
\begin{flushleft}
\end{flushleft}