%! Author = gramic
%! Date = 01.05.24

% Preamble
\begin{flushleft}
    \subsubsection{Testfälle}
    \begin{description}
        \item \textbf{Basistests}\hfill \\
        \begin{enumerate}
            \item Es wird eine Verbindung auf die \texttt{postgres}- und \texttt{gramic\_test} DB\\ auf dem Host \texttt{vks0041.ksgr.ch} oder IP \texttt{10.0.22.178} auf die Ports \texttt{5000} und \texttt{5001} hergestellt
            \item Es werden Daten aus den Tabellen beide Datenbanken ausgelesen
            \item Auf die DB \texttt{gramic\_test} wird mit \texttt{pgbench} ein 5GiB-Init auf den Host \texttt{vks0041.ksgr.ch} oder IP \texttt{10.0.22.178} und Port \texttt{5000} ausgeführt
        \end{enumerate}
        \item \textbf{Failover}\hfill \\
        \begin{enumerate}[resume]
            \item Der Server des Primary-Node wird manuell rebooted.
%            \item Während dem Failover müssen Daten via SQL\\eingeführt und ausgelesen werden.
%            \item Während dem Switchover muss ein mixed \texttt{pgbench} abgesetzt werden
        \end{enumerate}
        \item \textbf{Switchover}\hfill \\
        \begin{enumerate}[resume]
            \item Mit dem \texttt{patronictl}-Command wird der Switchover gesetzt
%            \item Während dem Switchover muss ein mixed \texttt{pgbench} abgesetzt werden
        \end{enumerate}
        \item \textbf{Restore}\hfill \\
        \begin{enumerate}[resume]
            \item Der Primary Node Server muss gestoppt werden, ein mixed \texttt{pgbench} abgesetzt werden und der Server wieder gestartet werden.
            \item Mit dem \texttt{patronictl}-Command und Parameter \texttt{reinit} wird der der Node wiederhergestellt\\und abschliessend mittels Switchover wieder als Primary gesetzt.
            \item Mit dem \texttt{patronictl}-Command und Parameter \texttt{reinit} wird der Replica-Node wiederhergestellt
            \item Vor, während und nach dem Restore müssen Tabellen mit Foreign-Key-Constraints und Daten geprüft werden.
%            \item Während dem Restore muss ein \texttt{pgbench} mixed Benchmark abgesetzt werden
        \end{enumerate}
        \item \textbf{Ansible - Deploy}\hfill \\
        \begin{enumerate}[resume]
            \item Mit \Gls{Ansible} kann der Patroni Cluster deployed werden.
        \end{enumerate}
        \item \textbf{Ansible - Maintenance}\hfill \\
        \begin{enumerate}[resume]
            \item Mit \Gls{Ansible} sollen folgende Parameter angepasst werden:
            \begin{itemize}
                \item Das \texttt{pg\_hba.conf} File.\\So dass der \Gls{HAProxy} Node \texttt{10.0.28.16} erweitert werden kann.
                \item Die Patroni REST-API soll für den jeweiligen Host von aussen Ansprechbar sein.\\Entsprechend muss der Eintrag gesetzt werden
            \end{itemize}
        \end{enumerate}
        \item \textbf{Ansible - Patroni Node Extend}\hfill \\
        \begin{enumerate}[resume]
            \item Mit hilfe eines \Gls{Ansible} Playbooks kann ein Patroni Node angehängt werden.
        \end{enumerate}
        \item \textbf{Ansible - HAproxy Node Extend}\hfill \\
        \begin{enumerate}[resume]
            \item Mit hilfe eines \Gls{Ansible} Playbooks kann ein HAproxy Node angehängt werden.
        \end{enumerate}
    \end{description}
    \begin{warning}
        Die beiden Expansionsserver sind nicht in der \texttt{vks0041} \texttt{AD}-Gruppe und werden auch nicht via Palo Alto ans Init7 Netz gerouted.\\
        Entsprechend muss im \texttt{main.yml} der Proxy gesetzt werden.
    \end{warning}
\end{flushleft}