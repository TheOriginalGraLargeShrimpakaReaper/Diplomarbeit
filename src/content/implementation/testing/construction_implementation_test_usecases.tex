%! Author = gramic
%! Date = 01.05.24

% Preamble
\begin{flushleft}
    \subsubsection{Testfälle}
    \begin{description}
        \item \textbf{Failover}\hfill \\
        \begin{enumerate}
            \item Der Server des Primary-Node wird manuell heruntergefahren.
            \item Während dem Failover müssen Daten via SQL\\eingeführt und ausgelesen werden.
            \item Vor dem Failover muss mindestens eine längere Abfrage gestartet werden,\\welche über den Failover läuft.
        \end{enumerate}
        \item \textbf{Switchover}\hfill \\
        \begin{enumerate}[resume]
            \item Mit der REST-API wird der Switchover\\auf einen anderen Nod abgesetzt.
            \item Mit dem \texttt{patronictl}-Command wird der Switchover gesetzt
            \item Während dem Switchover müssen Daten via SQL\\eingeführt und ausgelesen werden.
            \item Vor dem Switchover muss mindestens eine längere Abfrage gestartet werden,\\welche über den Switchover läuft.
        \end{enumerate}
        \item \textbf{Restore}\hfill \\
        \begin{enumerate}[resume]
            \item Mit der REST-API wird der Node erst mit dem \texttt{reinitialize} wiederhergestellt\\und dann mit einem Switchover wieder als Primary gesetzt.
            \item Mit dem \texttt{patronictl}-Commandund Parameter \texttt{reinit} der Node wiederhergestellt\\und abschliessend mittels Switchover wieder als Primary gesetzt.
            \item Mit der REST-API wird der Node mit dem \texttt{reinitialize} wiederhergestellt
            \item Mit dem \texttt{patronictl}-Commandund Parameter \texttt{reinit} der Node wiederhergestellt
            \item Vor, während und nach dem Restore müssen Tabellen mit Foreign-Key-Constraints und Daten geprüft werden.
            \item Während dem Restore muss mindestens eine längere Abfrage gestartet werden und Daten via SQL\\eingeführt und ausgelesen werden.
        \end{enumerate}
        \item \textbf{Ansible - Deploy}\hfill \\
        \begin{enumerate}[resume]
            \item Mit \Gls{Ansible} kann der Patroni Cluster deployed werden.
        \end{enumerate}
        \item \textbf{Ansible - Patroni Node Extend}\hfill \\
        \begin{enumerate}[resume]
            \item Mit hilfe eines \Gls{Ansible} Playbooks kann ein Patroni Node angehängt werden.
        \end{enumerate}
        \item \textbf{Ansible - HAproxy Node Extend}\hfill \\
        \begin{enumerate}[resume]
            \item Mit hilfe eines \Gls{Ansible} Playbooks kann ein HAproxy Node angehängt werden.
        \end{enumerate}
    \end{description}
\end{flushleft}