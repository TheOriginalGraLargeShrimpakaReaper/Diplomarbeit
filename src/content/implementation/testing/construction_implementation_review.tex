%! Author = gramic
%! Date = 01.05.24

% Preamble
\clearpage
\begin{flushleft}
    \subsection{Review und Auswertung}
    \subsubsection{vitabaks/postgresql\_cluster}
    vitabaks/postgresql\_cluster ist sehr schnell und zuverlässig deployt.\\
    Das \Gls{Ansible}-Repository hat allerdings eine Schwachstelle.\\
    Der \Gls{Connection Pooler} ist auf dem Level der Patroni Nodes nicht auf dem Proxy-Level.\\
    Dadurch verliert ein Client die Verbindung, wenn der entsprechende Server nicht mehr erreichbar ist.
\end{flushleft}
\begin{flushleft}
    Auch der \Gls{HAProxy} hat eine Schwäche.\\
    Es handelt sich um einen Layer 7 Proxy, dadurch ist \Gls{HAProxy} nicht in der Lage, SQL Statements zu analysieren.\\
    Wenn also die Read-Only Statements auf die Replikas geleitet werden sollen, muss die Applikation diese auf den entsprechenden Port zugreifen.\\
    Ist die Applikation selber nicht in der Lage, die Lese- und Schreibprozesse an zwei verschiedene Ports zu leiten, geht der gesamte Load auf den Primary Node.
\end{flushleft}
\begin{flushleft}
    \subsubsection{Maintenance-Tool - Bloating Tables}
    Aufgeblähte Tabellen werden zuverlässig erkannt und die Tabelle vakuumiert.\\
    Zudem werden die Indices neu aufgebaut und somit geordnet.
    \subsubsection{Maintenance-Tool - \Gls{AUTOVACUUM}}
    Der Parameter \texttt{autovacuum\_vacuum\_scale\_factor} wird zuverlässig berechnet und abgelegt.
\end{flushleft}