%! Author = gramic
%! Date = 09.05.24

% Preamble
\begin{flushleft}
    \subsection{Installation und Konfiguration PostgreSQL HA Cluster}
    \texttt{vitabacks / postgresql\_cluster} \Gls{GitHub} Repository hat zwei zentrale Komponten.\\
    Das eine ist das \texttt{Inventory}-File, welches die IP-Adressen und Hostnamen der Server beinhaltet.\\
    Dabei ist zu beachten, dass die Host-Einträge überschrieben werden, wenn \texttt{hostname=<hostname>} eingetragen wird.\\
    Was entsprechend zu Problemen mit den DNS-Einträgen, dem Active Directory oder anderen komponenten führen kann.
\end{flushleft}
\begin{flushleft}
    Die zweite zentrale Komponente ist das \texttt{main.yml}-File.\\
    Über dieses YAML-File werden alle Aspekte des Patroni-Clusters einstellen.\\
    Für dieses Testsystem zentrale Konfigurationsabschnitte waren:
    \begin{description}
        \item \textbf{\Gls{DCS}-Typ}\hfill \\Es stehen \gls{etcd} oder \Gls{Consul} zur Auswahl, entsprechend wird \gls{etcd} ausgewählt.\\Es könnten auch bestehende \gls{etcd}-Nodes verwendet werden.
        \item \textbf{Load Balancing}\hfill \\Grundsätzlich funktionierte der Cluster auch ohne Load Balancer.\\Allerdings muss in diesem Fall \Gls{HAProxy} als Load Balancer eingesetzt werden.
        \item \textbf{Virtual IP}\hfill \\Auch kann definiert werden ob eine Virtuelle IP verwendet wird.\\Ohne Load Balancer würde die virtuelle IP dem Primary Patroni Node zugewiesen.
        \item \textbf{\Gls{Connection Pooler}}\hfill \\\texttt{PgBounder} kann aktiviert oder deaktiviert werden, entsprechend wurde dieser aktiviert.\\Die Konfiguration wurde belassen.\\Ein andererer \Gls{Connection Pooler} steht nicht zur verfügung.
        \item \textbf{PostgreSQL - User}\hfill \\Alle notwendigen User können bereits definiert werden.
        \item \textbf{PostgreSQL - Datenbanken}\hfill \\Selbiges gilt für die Datenbanken.\\Folgende Datenbanken wurden eingetragen:
        \begin{itemize}
            \item \textbf{k8s\_core\_gitlab\_prod}\hfill \\Leere DB Hülle für die \Gls{GitLab} Datenbank
            \item \textbf{k8s\_core\_harbor\_prod}\hfill \\Leere DB Hülle für die \Gls{Harbor} Datenbank
            \item \textbf{k8s\_core\_keycloak\_prod}\hfill \\Leere DB Hülle für die \Gls{Keycloak} Datenbank
            \item \textbf{gramic\_test}\hfill \\Diese DB wird nur für die Benchmarking-Tests und als Test-DB für die Maintenance-Skripte verwendet.
        \end{itemize}
        \item \textbf{PostgreSQL - Extensions}\hfill \\Alle Datenbanken benötigen die Extension \texttt{pgstattuple}.\\Diese wird verwendet, um Bloated Tables und Indices zu ermitteln.
        \item \textbf{PostgreSQL - pg\_hba.conf}\hfill \\Ist eine IP oder ein Netzwerk nicht erfasst, kann trotz korrekten Zugangsdaten nicht auf die DB zugegriffen werden.\\Entsprechend müssen die User und Client Adressen erfasst werden.
        \item \textbf{PostgreSQL - .pgpass}\hfill \\Im Passwortfile \texttt{.pgpass} werden die Passwörter zu den Usern erfasst.\\Zusätzlich kann hier der Zugriff ebenfalls auf bestimmte Adressen eingeschränkt werden.\\Als Alternative würde \texttt{postgresql\_pg\_ident} zur Auswahl stehen.
        \item \textbf{Patroni - REST-API}\hfill \\Die REST-API kann auf \texttt{localhost} beschränkt werden oder so eingestellt werden,\\dass die REST-AIP auf die Patroni Adresse oder den Hostnamen hört.
    \end{description}
\end{flushleft}
\begin{flushleft}
    \subsubsection{Vorbedingungen}
    Zuerst mussten auf allen Servern die entsprechenden Ports auf der Firewall freigegeben werden.\\
    Standardmässig setzt \texttt{vitabacks / postgresql\_cluster} folgende offenen Ports voraus:
    \begin{longtable}[H]{rl}

\toprule
Port & Beschreibung \\
\midrule
\endfirsthead
\caption[]{Testsystem - Benötigte Ports} \\
\toprule
Port & Beschreibung \\
\midrule
\endhead
\midrule
\multicolumn{2}{r}{Continued on next page} \\
\midrule
\endfoot
\bottomrule
\endlastfoot
2379 & \gls{etcd} \\
2380 & \gls{etcd} \\
5000 & \Gls{HAProxy} - Read/Write - Primary \\
5001 & \Gls{HAProxy} - Read Only - Replikas \\
5002 & \Gls{HAProxy} - Read Only - Synchronous Replikas \\
5003 & \Gls{HAProxy} - HAproy Stats \\
5432 & PostgreSQL \\
6432 & PgBounder \\
8008 & Patroni REST-API \\
\caption{Testsystem - Benötigte Ports} \label{construction_implementation_used_ports}
\end{longtable}

\end{flushleft}
\begin{flushleft}
    Es reicht nicht, auf den Servern einen Proxy einzustellen.\\
    Entweder es werden Firewall-Rules erstellt, welche die Ziele von \Gls{GitHub} und den \Gls{Debian}-Repositories freigeben,\\
    oder aber der Proxy wird dem \texttt{main.yml} mitgegeben.\\
    Bei den erweiterungstests gng es nicht mehr anders, der Server \texttt{sks9016} ist nur eine Spielwiese für das \Gls{Foreman} Provisioning.\\
    Auch ist dieser Server nicht in der entsprechenden AD-Gruppe der Patroni Test Gruppe, also muss der alte Proxy verwendet werden.\\
    Die Proxy Settings sind zuoberst, wird die Firewall verwendet, wird folgendens eingetragen:
    \lstset{style=gra_codestyle}
    \begin{lstlisting}[language=yaml, caption=main.xyml - No Proxy,captionpos=b,label={lst:main.yml-no-proxy},breaklines=true]
proxy_env: {}
    \end{lstlisting}
    Der Gruppenproxy funktioniert mit folgenden Proxy-Settings:
    \lstset{style=gra_codestyle}
    \begin{lstlisting}[language=yaml, caption=main.xyml - Gruppenproxy,captionpos=b,label={lst:main.yml-webproxy},breaklines=true]
proxy_env:
  http_proxy: "http://xksgr_vks0041_inet:<Password Secure / Safe>f@webproxy.sivc.first-it.ch:9090"
  https_proxy: "http://xksgr_vks0041_inet:<Password Secure / Safe>@webproxy.sivc.first-it.ch:9090"
    \end{lstlisting}
    Der Proxy für alles andere benötigt folgenden Eintrag:
    \lstset{style=gra_codestyle}
    \begin{lstlisting}[language=yaml, caption=main.xyml - SProxy,captionpos=b,label={lst:main.yml-sproxy},breaklines=true]
proxy_env:
  http_proxy: http://sproxy.sivc.first-it.ch:8080
  https_proxy: http://sproxy.sivc.first-it.ch:8080
    \end{lstlisting}
\end{flushleft}
\begin{flushleft}
    \subsubsection{Installation / Deploy - deploy\_pgcluster.yml}
    Die Installation ist simpel.\\
    Nachdem alle Konfigurationen vollzogen sind, muss nur das entsprechende Playbook ausgeführt werden:
    \lstset{style=gra_codestyle}
    \begin{lstlisting}[language=bash, caption=Deploy - deploy\_pgcluster.yml,captionpos=b,label={lst:deploy-pgcluster},breaklines=true]
ansible-playbook deploy_pgcluster.yml
    \end{lstlisting}
    Das reine Deplyoment dauerte jeweils etwas mehr als fünf minuten.\\
    Danach war ein voll funktionsfähiger Patroni Cluster betriebsbereit.
    \subsubsection{Maintenance - config\_pgcluster.yml}
    Nachdem die Anpassungen am \texttt{main.yml} vorgenommen waren, musste nur das entsprechende Playbook ausgeführt werden:
    \lstset{style=gra_codestyle}
    \begin{lstlisting}[language=bash, caption=Maintenance - config\_pgcluster.yml,captionpos=b,label={lst:config-pgcluster},breaklines=true]
ansible-playbook config_pgcluster.yml
    \end{lstlisting}
    \begin{warning}
        Die Nodes, um die der Cluster in den nächsten beiden Schritten erweitert wurde,\\
        durften noch nicht ins \texttt{Inventory} eingetragen werden, ansonsten wären Fehlermeldungen und ein Abbruch des Playbooks die folge gewesen,\\da es die entsprechenden Nodes noch gar nicht gab.\\
        Ebenso musste ein Proxy in das \texttt{main.yml} eingetragen werden, da der Testsserver \texttt{sks9016} nicht der angepassten Firewall-Rule unterstand.
    \end{warning}
\begin{flushleft}
    Zum Testen wurden nachträglich folgende Werte hinzugefügt:
    \begin{itemize}
        \item Der Proxy wurde gesetzt
        \item Der REST-API wurde die Patroni-Adresse mittels \texttt{patroni\_restapi\_listen\_addr: "\{\{ inventory\_hostname \}\}"} hinzugefügt
        \item Das \texttt{pg\_hba.conf} musste einmal um den \texttt{patroni\_replication\_username} Usereintrag auf die virtuelle Tabelle \texttt{replication} erweitert werden (damit dieser Node replizieren darf)\\als auch um den generellen Zugriff erweitert werden (damit \Gls{HAProxy} zugreifen darf).
        \item Zuletzt musste es auch möglich sein, einen Restart vollziehen zu können wenn es notwendig wäre,\\daher wurde \texttt{pending\_restart: true} gesetzt.
    \end{itemize}
\end{flushleft}
    \subsubsection{\Gls{HAProxy} Node hinzufügen - add\_balancer.yml}
    Um die erweiterung des Proxy-Layers vorzunehmen, waren zwei Sachen notwendig.\\
    Zuerst musste das \texttt{Inventory} um den Node erweitert werden, der entsprechend als neuer Node markiert wurde:
    \lstset{style=gra_codestyle}
    \begin{lstlisting}[language=bash, caption=HAProxy Node erweitern - Inventory,captionpos=b,label={lst:add-balancer-inventory},breaklines=true]
[balancers]
10.0.22.176
10.0.22.177
10.0.20.43 new_node=true
    \end{lstlisting}
    Zum Schluss muss nur noch das Playbook ausgeführt werden:
    \lstset{style=gra_codestyle}
    \begin{lstlisting}[language=bash, caption=HAProxy Node erweitern - add\_balancer.yml,captionpos=b,label={lst:add-balancer},breaklines=true]
ansible-playbook add_balancer.yml
    \end{lstlisting}
    \subsubsection{Patroni Node hinzufügen - add\_pgnode.yml}
    Das vorgehen ist gleich wie beim erweitern von \Gls{HAProxy}.\\
    Das \texttt{Inventory} wurde erweitert:
    \lstset{style=gra_codestyle}
    \begin{lstlisting}[language=bash, caption=Patroni Node erweitern - Inventory,captionpos=b,label={lst:add-pgnode-inventory},breaklines=true]
[master]
10.0.22.173 postgresql_exists=false

[replica]
10.0.22.174 postgresql_exists=false
10.0.22.175 postgresql_exists=false
10.0.28.16 postgresql_exists=false new_node=true
    \end{lstlisting}
    Das Deployment erfolgt nun ebenfalls via Playbook:
    \lstset{style=gra_codestyle}
    \begin{lstlisting}[language=bash, caption=Patroni Node erweitern - add\_pgnode.yml,captionpos=b,label={lst:add-pgnode},breaklines=true]
ansible-playbook add_pgnode.yml
    \end{lstlisting}

\end{flushleft}